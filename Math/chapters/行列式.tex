\chapter{行列式}

行列式是线性代数中常用的工具。本章主要介绍$ n $阶行列式的定义、性质及其计算方法。

\section{二阶与三阶行列式}
\subsection{二元线性方程组与二阶行列式}

用\terminologyItem{消元法}解二元线性方程组

\begin{equation} \label{equationBinaryLinearEquationsGeneralForm}
	\left\{
	\begin{array}{r}
		a_{11}x_1 + a_{12}x_2 = b_1, \\
		a_{21}x_1 + a_{22}x_2 = b_2. \\
	\end{array} \right.
\end{equation}

为消去未知数$ x_2 $,以$ a_{22} $和$ a_{12} $分别乘上列方程的两端,然乎两个方程相减,得到

\begin{equation}
	(a_{11}a_{22}-a_{12}a_{21})x_{1} = b_{1}a_{22}-a_{12}b_{2};
\end{equation}

类似地,消去$ x_1 $,得
\begin{equation}
	(a_{11}a_{22}-a_{12}a_{21})x_{2} = a_{11}b_{2}-b_{1}a_{21};
\end{equation}

当$ a_{11}a_{22}-a_{12}a_{21} \ne 0 $时,求得方程组 \ref{equationBinaryLinearEquationsGeneralForm} 的解为:

\begin{equation} \label{equationBinaryLinearEquationsGeneralFormResult}
	\left\{
	\begin{array}{r}
	x_{1} = \dfrac{b_{1}a_{22}-a_{12}b_{2}}{a_{11}a_{22}-a_{12}a_{21}} \\
	\\
	x_{2} = \dfrac{a_{11}b_{2}-b_{1}a_{21}}{a_{11}a_{22}-a_{12}a_{21}}
	
	\end{array} \right.
\end{equation}

\ref{equationBinaryLinearEquationsGeneralFormResult} 式中的分子、分母都是四个数分两对相乘、再相减而得,其中分母$ (a_{11}a_{22}-a_{12}a_{21}) $是由方程组 \ref{equationBinaryLinearEquationsGeneralForm} 的四个系数确定的,把这四个数按它们在方程组 \ref{equationBinaryLinearEquationsGeneralForm} 中的位置,排列成两行两列(横排称\terminologyItem{行}、竖排成\terminologyItem{列})的数表

\begin{equation} \label{equationBinaryLinearEquationsGeneralFormNumberTable}
\begin{array}{cc}
a_{11} & a_{12}\\
a_{21} & a_{22}.
\end{array}
\end{equation}

表达式$ (a_{11}a_{22}-a_{12}a_{21}) $称为数表 \ref{equationBinaryLinearEquationsGeneralFormNumberTable} 所确定的二阶行列式,并记作

\begin{equation} \label{equationDeterminantGeneralForm}
\left | \begin{array}{cc}
a_{11} & a_{12}\\
a_{21} & a_{22}
\end{array} \right |.
\end{equation}

数$ a_{ij} (i=1,2;j=1,2) $称为行列式 \ref{equationDeterminantGeneralForm} 的\terminologyItem{元素}或\terminologyItem{元}。元素$ a_{ij} $的第一个下标$ i $称为\terminologyItem{行标},表明该元素位于第$ i $行;第二个下标$ j $称为\terminologyItem{列标},表明该元素位于第$ j $列。位于第$ i $行、第$ j $列的元素称为行列式 \ref{equationDeterminantGeneralForm} 的 \terminologyItem{ $ (i,j) $元}。

上述二阶行列式的定义,可用对角线法则来记忆。把$ a_{11} $到$ a_{22} $的实连线称为\terminologyItem{主对角线},$ a_{12} $到$ a_{21} $的虚连线称为\terminologyItem{副对角线},于是二阶行列式便是\myKeypoint{主对角线上两元素之积减去副对角线上两元素之积所得的差}。

利用二阶行列式的概念,式 \ref{equationBinaryLinearEquationsGeneralFormResult} 中的$ x_{1} $,$ x_{2} $的分子也可写成二阶行列式,即

\begin{equation} \label{equationDeterminantGeneralFormResult}
	\begin{array}{r}
	b_{1}a_{22}-a_{12}b_{2} = \left | \begin{array}{cc}
	b_{1} & a_{12}\\
	b_{2} & a_{22}
	\end{array} \right | \\
	\\
	a_{11}b_{2}-b_{1}a_{21} = \left | \begin{array}{cc}
		a_{11} & b_{1}\\
		a_{21} & b_{2}
		\end{array} \right |
	\end{array}
\end{equation}

若记

\begin{equation}
	\begin{array}{r}
	D = \left | \begin{array}{cc}
			a_{11} & a_{12}\\
			a_{21} & a_{22}
			\end{array} \right | \\
	\\
	D_{1} = \left | \begin{array}{cc}
		b_{1} & a_{12}\\
		b_{2} & a_{22}
		\end{array} \right | \\
	\\
	D_{2} = \left | \begin{array}{cc}
		a_{11} & b_{1}\\
		a_{21} & b_{2}
		\end{array} \right |
	\end{array}
\end{equation}

那么式 \ref{equationBinaryLinearEquationsGeneralFormResult} 可写成

\begin{equation}
	\begin{array}{r}
	x_{1}= \dfrac{D_{1}}{D_{2}} = \dfrac{\left | \begin{array}{cc}
			b_{1} & a_{12}\\
			b_{2} & a_{22}
			\end{array} \right |}{\left | \begin{array}{cc}
				a_{11} & a_{12}\\
				a_{21} & a_{22}
				\end{array} \right |}\\
	\\
	
	x_{1}= \dfrac{D_{2}}{D_{2}} = \dfrac{\left | \begin{array}{cc}
			a_{11} & b_{1}\\
			a_{21} & b_{2}
			\end{array} \right |}{\left | \begin{array}{cc}
					a_{11} & a_{12}\\
					a_{21} & a_{22}
					\end{array} \right |}
	\end{array}
\end{equation}

注意这里的分母$ D $是由方程组 \ref{equationBinaryLinearEquationsGeneralForm} (第\pageref{equationBinaryLinearEquationsGeneralForm}页)的系数所确定的二阶行列式(称系数行列式),$ x_{1} $的分子$ D_{1} $是用常数项$ b_{1} $、$ b_{2} $替换$ D $中的第一列的元素$ a_{11} $、$ a_{21} $所得的二阶行列式,$ x_{2} $的分子$ D_{2} $是用常数项$ b_{1} $、$ b_{2} $替换$ D $中的第一列的元素$ a_{12} $、$ a_{22} $所得的二阶行列式。

\subsection{三阶行列式}