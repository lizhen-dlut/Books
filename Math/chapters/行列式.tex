\chapter{行列式}

行列式是线性代数中常用的工具。本章主要介绍$ n $阶行列式的定义、性质及其计算方法。

\section{二阶与三阶行列式}

用\terminologyItem{消元法}解二元线性方程组

\begin{equation} \label{equationBinaryLinearEquationsGeneralForm}
	\left\{
	\begin{array}{r}
		a_{11}x_1 + a_{12}x_2 = b_1, \\
		a_{21}x_1 + a_{22}x_2 = b_2. \\
	\end{array} \right.
\end{equation}

为消去未知数$ x_2 $,以$ a_{22} $和$ a_{12} $分别乘上列方程的两端,然乎两个方程相减,得到

\begin{equation}
	(a_{11}a_{22}-a_{12}a_{21})x_{1} = b_{1}a_{22}-a_{12}b_{2};
\end{equation}

类似地,消去$ x_1 $,得
\begin{equation}
	(a_{11}a_{22}-a_{12}a_{21})x_{2} = a_{11}b_{2}-b_{1}a_{21};
\end{equation}

当$ a_{11}a_{22}-a_{12}a_{21} \ne 0 $时,求得方程组 \ref{equationBinaryLinearEquationsGeneralForm} 的解为:

\begin{equation} \label{equationBinaryLinearEquationsGeneralFormResult}
	\left\{
	\begin{array}{r}
	x_{1} = \dfrac{b_{1}a_{22}-a_{12}b_{2}}{a_{11}a_{22}-a_{12}a_{21}} \\
	\\
	x_{2} = \dfrac{a_{11}b_{2}-b_{1}a_{21}}{a_{11}a_{22}-a_{12}a_{21}}
	
	\end{array} \right.
\end{equation}

\ref{equationBinaryLinearEquationsGeneralFormResult} 式中的分子、分母都是四个数分两对相乘、再相减而得,其中分母$ (a_{11}a_{22}-a_{12}a_{21}) $是由方程组 \ref{equationBinaryLinearEquationsGeneralForm} 的四个系数确定的,把这四个数按它们在方程组 \ref{equationBinaryLinearEquationsGeneralForm} 中的位置,排列成两行两列(横排称\terminologyItem{行}、竖排成\terminologyItem{列})的数表

\begin{equation} \label{equationBinaryLinearEquationsGeneralFormNumberTable}
\begin{array}{cc}
a_{11} & a_{12}\\
a_{21} & a_{22}.
\end{array}
\end{equation}

表达式$ (a_{11}a_{22}-a_{12}a_{21}) $称为数表 \ref{equationBinaryLinearEquationsGeneralFormNumberTable} 所确定的二阶行列式,并记作

\begin{equation} 
\left | \begin{array}{cc}
a_{11} & a_{12}\\
a_{21} & a_{22}
\end{array} \right |
\end{equation}

