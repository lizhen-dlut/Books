\chapter{行列式}

行列式是线性代数中常用的工具。本章主要介绍$ n $阶行列式的定义、性质及其计算方法。

\section{二阶与三阶行列式}
\subsection{二元线性方程组与二阶行列式}

用\terminologyItem{消元法}解二元线性方程组

\begin{equation} \label{equationBinaryLinearEquationsGeneralForm}
	\left\{
	\begin{array}{r}
		a_{11}x_1 + a_{12}x_2 = b_1, \\
		a_{21}x_1 + a_{22}x_2 = b_2. \\
	\end{array} \right.
\end{equation}

为消去未知数$ x_2 $,以$ a_{22} $和$ a_{12} $分别乘上列方程的两端,然乎两个方程相减,得到

\begin{equation}
	(a_{11}a_{22}-a_{12}a_{21})x_{1} = b_{1}a_{22}-a_{12}b_{2};
\end{equation}

类似地,消去$ x_1 $,得
\begin{equation}
	(a_{11}a_{22}-a_{12}a_{21})x_{2} = a_{11}b_{2}-b_{1}a_{21};
\end{equation}

当$ a_{11}a_{22}-a_{12}a_{21} \ne 0 $时,求得方程组 \ref{equationBinaryLinearEquationsGeneralForm} 的解为:

\begin{equation} \label{equationBinaryLinearEquationsGeneralFormResult}
	\left\{
	\begin{array}{r}
	x_{1} = \dfrac{b_{1}a_{22}-a_{12}b_{2}}{a_{11}a_{22}-a_{12}a_{21}} \\
	\\
	x_{2} = \dfrac{a_{11}b_{2}-b_{1}a_{21}}{a_{11}a_{22}-a_{12}a_{21}}
	
	\end{array} \right.
\end{equation}

\ref{equationBinaryLinearEquationsGeneralFormResult} 式中的分子、分母都是四个数分两对相乘、再相减而得,其中分母$ (a_{11}a_{22}-a_{12}a_{21}) $是由方程组 \ref{equationBinaryLinearEquationsGeneralForm} 的四个系数确定的,把这四个数按它们在方程组 \ref{equationBinaryLinearEquationsGeneralForm} 中的位置,排列成两行两列(横排称\terminologyItem{行}、竖排成\terminologyItem{列})的数表

\begin{equation} \label{equationBinaryLinearEquationsGeneralFormNumberTable}
\begin{array}{cc}
a_{11} & a_{12}\\
a_{21} & a_{22}.
\end{array}
\end{equation}

表达式$ (a_{11}a_{22}-a_{12}a_{21}) $称为数表 \ref{equationBinaryLinearEquationsGeneralFormNumberTable} 所确定的二阶行列式,并记作

\begin{equation} \label{equationDeterminantGeneralForm}
\left | \begin{array}{cc}
a_{11} & a_{12}\\
a_{21} & a_{22}
\end{array} \right |.
\end{equation}

数$ a_{ij} (i=1,2;j=1,2) $称为行列式 \ref{equationDeterminantGeneralForm} 的\terminologyItem{元素}或\terminologyItem{元}。元素$ a_{ij} $的第一个下标$ i $称为\terminologyItem{行标},表明该元素位于第$ i $行;第二个下标$ j $称为\terminologyItem{列标},表明该元素位于第$ j $列。位于第$ i $行、第$ j $列的元素称为行列式 \ref{equationDeterminantGeneralForm} 的 \terminologyItem{ $ (i,j) $元}。

上述二阶行列式的定义,可用对角线法则来记忆。把$ a_{11} $到$ a_{22} $的实连线称为\terminologyItem{主对角线},$ a_{12} $到$ a_{21} $的虚连线称为\terminologyItem{副对角线},于是二阶行列式便是\myKeypoint{主对角线上两元素之积减去副对角线上两元素之积所得的差}。

利用二阶行列式的概念,式 \ref{equationBinaryLinearEquationsGeneralFormResult} 中的$ x_{1} $,$ x_{2} $的分子也可写成二阶行列式,即

\begin{equation} \label{equationDeterminantGeneralFormResult}
	\begin{array}{r}
	b_{1}a_{22}-a_{12}b_{2} = \left | \begin{array}{cc}
	b_{1} & a_{12}\\
	b_{2} & a_{22}
	\end{array} \right | \\
	\\
	a_{11}b_{2}-b_{1}a_{21} = \left | \begin{array}{cc}
		a_{11} & b_{1}\\
		a_{21} & b_{2}
		\end{array} \right |
	\end{array}
\end{equation}

若记

\begin{equation}
	\begin{array}{r}
	D = \left | \begin{array}{cc}
			a_{11} & a_{12}\\
			a_{21} & a_{22}
			\end{array} \right | \\
	\\
	D_{1} = \left | \begin{array}{cc}
		b_{1} & a_{12}\\
		b_{2} & a_{22}
		\end{array} \right | \\
	\\
	D_{2} = \left | \begin{array}{cc}
		a_{11} & b_{1}\\
		a_{21} & b_{2}
		\end{array} \right |
	\end{array}
\end{equation}

那么式 \ref{equationBinaryLinearEquationsGeneralFormResult} 可写成

\begin{equation}
	\begin{array}{r}
	x_{1}= \dfrac{D_{1}}{D_{2}} = \dfrac{\left | \begin{array}{cc}
			b_{1} & a_{12}\\
			b_{2} & a_{22}
			\end{array} \right |}{\left | \begin{array}{cc}
				a_{11} & a_{12}\\
				a_{21} & a_{22}
				\end{array} \right |}\\
	\\
	
	x_{1}= \dfrac{D_{2}}{D_{2}} = \dfrac{\left | \begin{array}{cc}
			a_{11} & b_{1}\\
			a_{21} & b_{2}
			\end{array} \right |}{\left | \begin{array}{cc}
					a_{11} & a_{12}\\
					a_{21} & a_{22}
					\end{array} \right |}
	\end{array}
\end{equation}

注意这里的分母$ D $是由方程组 \ref{equationBinaryLinearEquationsGeneralForm} (第\pageref{equationBinaryLinearEquationsGeneralForm}页)的系数所确定的二阶行列式(称系数行列式),$ x_{1} $的分子$ D_{1} $是用常数项$ b_{1} $、$ b_{2} $替换$ D $中的第一列的元素$ a_{11} $、$ a_{21} $所得的二阶行列式,$ x_{2} $的分子$ D_{2} $是用常数项$ b_{1} $、$ b_{2} $替换$ D $中的第一列的元素$ a_{12} $、$ a_{22} $所得的二阶行列式。

\subsection{三阶行列式}

\begin{定义}
	设有9个数排成3行3列的数表:
	\begin{equation} \label{三阶数表}
		\begin{array}{ccc}
			a_{11} & a_{12} & a_{13}\\
			a_{21} & a_{22} & a_{23}\\
			a_{31} & a_{32} & a_{33}\\
		\end{array}
	\end{equation}
	记
	\begin{equation} \label{三阶行列式}
	\left |
		\begin{array}{ccc}
		a_{11} & a_{12} & a_{13}\\
		a_{21} & a_{22} & a_{23}\\
		a_{31} & a_{32} & a_{33}\\
		\end{array}\right |
		=a_{11}a_{22}a_{23}+a_{12}a_{23}a_{31}+a_{13}a_{21}a_{32}-a_{13}a_{22}a_{31}-a_{12}a_{21}a_{33}-a_{23}a_{32}a_{11}
	\end{equation}	
	\ref{三阶行列式} 式成为数表 \ref{三阶数表} 所确定的三阶行列式。
\end{定义}

上述定义表明三阶行列式含6项,每项均为不同行、不同列的三个元素的乘积再加上正负号。

\section{全排列和对换}

\subsection{排列及其逆序数}

把$ n $个不同的元素排成一列,叫做这$ n $个元素的\terminologyItem{全排列}(也简称\terminologyItem{排列})。

$ n $个不同元素的所有排列的总数,通常用$ P_{n} $表示,可计算如下:
\begin{enumerate}
	\item 从n个元素中任取一个放在第一个位置上,有n中取法;
	\item 从剩下的$ n-1 $个元素中任取一个放在第二个位置上,有$ n-1 $种取法;
	\item 这样继续下去,直到只剩下一个元素放在第$ n $个位置上,只有1种取法。
\end{enumerate}
于是
\begin{equation}
	P_{n} = n \times (n-1) \times \dots \times 3 \times 2 \times 1 = n!
\end{equation}

例如,用1、2、3三个数字做排列,排列的总数$ P_{3} = 3 \times 2 \times 1 = 6 $,它们是
\begin{center}
	123、231、321、132、213、312。
\end{center}

\begin{定义}
	对于$ n $个\myKeypoint{不同}的元素,先规定各元素之间有一个标准次序(例如$ n $个不同的自然数,可规定由小到大为标准次序),于是在这$ n $个元素的任一排列中,当某一对元素的先后次序与标准次序不同时,就说它构成了1个\terminologyItem{逆序}。一个排列中所有逆序的总数叫做这个\terminologyItem{排列的逆序数}。
\end{定义}

逆序数为奇数的排列叫做奇排列,逆序数为偶数的排列叫做偶排列。

下面来讨论计算排列逆序数的方法。

不失一般性,不妨设$ n $个元素为1到$ n $这$ n $个自然数,并规定由小到大为标准次序。设$ p_{1}p_{2}\dots p_{n} $为这n个自然数的一个排列,考虑元素$ p_{1}\ (i=1,2,\dots,n) $,如果比$ p_{i} $大的且排在$ p_{i} $前面的元素有$ t_{i} $个,就说$ p_{i} $这个元素的逆序数是$ t_{i} $。全体元素的逆序数之总和
\begin{equation}
t = t_{1} + t_{2} + \dots + t_{n} = \sum_{i=1}^{n}t_{i}
\end{equation}
即是这个排列的逆序数。

\subsection{对换}

在排列中,将任意两个元素对调,其余的元素不动,这种做出新排列的手续叫做对换。将相邻两个元素对换,叫做相邻对换。

\begin{定理}
	一个排列中任意两个元素对换,排列改变奇偶性。
\end{定理}

\begin{推论}
	奇排列对换成标准排列的对换次数为奇数,偶排列对换成标准排列的对换次数为偶数。
\end{推论}

\section{$ n $阶行列式的定义}

\begin{定义}
	设有$ n^{2} $个数,排成$ n $行$ n $列的数表
	\begin{equation} \label{n阶数表}
		\begin{array}{cccc}
			a_{11} & a_{12} & \dots & a_{1n}\\
			a_{21} & a_{22} & \dots & a_{2n}\\
			\multicolumn{4}{c}{\dots\dots\dots}\\
			a_{n1} & a_{n2} & \dots & a_{nn}\\
		\end{array}
	\end{equation}
	做出表中位于不同行、不同列的$ n $个数的乘积,并冠以符号$ (-1)^{t} $,得到形如	
	\begin{equation}\label{n阶行列式一般项}
		(-1)^{t}a_{1p_{1}}a_{2p_{2}}\dots a_{np_{n}}
	\end{equation}
	\noindent 的项,其中$ p_{1}p_{2}\dots p_{n} $为自然数1、2、$ \dots $、$ n $的一个排列,$ t $为这个排列的逆序数。由于这样的排列共有$ n! $个,因而形如\ref{n阶行列式一般项}式的项共有$ n! $项。所有这$ n! $项的代数和	
	\begin{equation}
		\sum(-1)^{t}a_{1p_{1}}a_{2p_{2}}\dots a_{np_{n}}
	\end{equation}
	称为$ n $阶行列式,记作
	\begin{equation} \label{n阶行列式}
		D=\left |\begin{array}{cccc}
			a_{11} & a_{12} & \dots & a_{1n}\\
			a_{21} & a_{22} & \dots & a_{2n}\\
			\multicolumn{4}{c}{\dots\dots\dots}\\
			a_{n1} & a_{n2} & \dots & a_{nn}\\
		\end{array}\right |,
	\end{equation}
	简记为$ det(a_{ij}) $,其中数$ a_{ij} $为行列式$ D $的$ (i,j) $元。
\end{定义}

主对角线以下的元素都为0的行列式叫做上三角行列式;主对角线以上的元素都为0的行列式叫做下三角行列式;主对角线以上和以下的元素都为0的行列式叫做对角行列式。

\section{行列式的性质}

\begin{定义}
	记
	
	\begin{equation} \label{装置行列式}
		D=\left |\begin{array}{cccc}
		a_{11} & a_{12} & \dots & a_{1n}\\
		a_{21} & a_{22} & \dots & a_{2n}\\
		\multicolumn{4}{c}{\dots\dots\dots}\\
		a_{n1} & a_{n2} & \dots & a_{nn}\\
		\end{array}\right | \text{,} 
		D^{T}=\left |\begin{array}{cccc}
		a_{11} & a_{21} & \dots & a_{n1}\\
		a_{12} & a_{22} & \dots & a_{n2}\\
		\multicolumn{4}{c}{\dots\dots\dots}\\
		a_{1n} & a_{2n} & \dots & a_{nn}\\
		\end{array}\right | \text{,} 
	\end{equation}
	
	\noindent 行列式$ D^{T} $称行列式$ D $的\terminologyItem{转置行列式}。
\end{定义}

\begin{性质}
	行列式与它的装置行列式相等。
\end{性质}

\begin{性质}
	对换行列式的两行(或两列),行列式变号。
\end{性质}

\begin{推论}
	如果行列式有两行(或两列)完全相同,则此行列式等于零。
\end{推论}

\begin{性质}
	行列式的某一行(某一列)中所有的元素都乘同一数$ k $,等于用数$ k $乘此行列式。
\end{性质}

\begin{推论}
	行列式中某一行(某一列)的所有元素的公因子可以提到行列式记号的外面。
\end{推论}

\begin{性质}
	行列式如果有两行(两列)成比例,则此行列式等于零。
\end{性质}

\begin{性质}
	若行列式的某一行(某一列)的元素都是两数之和,例如第$ i $行的元素都是两数之和:
	
	\begin{equation}
		D=\left |\begin{array}{cccc}
		a_{11} & a_{12} & \dots & a_{1n}\\
		\vdots & \vdots & & \vdots \\
		a_{i1} + a_{i1}^{'} & a_{i2} + a_{i2}^{'} & \dots & a_{in} + a_{in}^{'}\\
		\vdots & \vdots & & \vdots \\
		a_{n1} & a_{n2} & \dots & a_{nn}\\
		\end{array}\right |,
	\end{equation}
	\noindent 则$ D $等于下列两个行列式之和:
	\begin{equation}
		D=\left |\begin{array}{cccc}
		a_{11} & a_{12} & \dots & a_{1n}\\
		\vdots & \vdots & & \vdots \\
		a_{i1} & a_{i2} & \dots & a_{in}\\
		\vdots & \vdots & & \vdots \\
		a_{n1} & a_{n2} & \dots & a_{nn}\\
		\end{array}\right | +
		\left |\begin{array}{cccc}
		a_{11} & a_{12} & \dots & a_{1n}\\
		\vdots & \vdots & & \vdots \\
		a_{i1}^{'} & a_{i2}^{'} & \dots & a_{in}^{'}\\
		\vdots & \vdots & & \vdots \\
		a_{n1} & a_{n2} & \dots & a_{nn}\\
		\end{array}\right | \text{。}
	\end{equation}
\end{性质}

\begin{性质}
	把行列式的某一行(某一列)的各元素乘同一数、然后加到另一行(另一列)对应的元素上,行列式不变。
\end{性质}

\section{行列式按行(列)展开}

一般来说,低阶行列式的计算比高阶行列式的计算要简便,于是,我们自然地考虑用低阶行列式来表示高阶行列式的问题。为此,先引进\terminologyItem{余子式}和\terminologyItem{代数余子式}的概念。

\begin{定义}
	在$ n $阶行列式中,把$ (i,j) $元$ a_{ij} $所在的第$ i $行和第$ j $列划去后,留下的$ n-1 $阶行列式叫做$ (i,j) $元$ a_{ij} $的\terminologyItem{余子式},记作$ M_{ij} $,记
	\begin{equation}
		A_{ij} = (-1)^{i+j}M_{ij}\text{,}
	\end{equation}
	\noindent $ A_{ij} $叫做$ (i,j) $元$ a_{ij} $的\terminologyItem{代数余子式}。
\end{定义}

\begin{引理}
	一个$ n $阶行列式,如果其中第$ i $行所有元素除$ (i,j) $元$ a_{ij} $外都为零,那么这行列式等于$ a_{ij} $与它的代数余子式的乘积,即
	\begin{equation}
		D = a_{ij}A_{ij}\text{。}
	\end{equation}
\end{引理}

\begin{定理}
	行列式等于它的任一行(任一列)的各元素与其对应的代数余子式乘积之和,即
	\begin{equation}
		D = a_{i1}A_{i1} + a_{i2}A_{i2} + \dots + a_{in}A_{in} (i=1,2,\dots,n) 
	\end{equation}
	或
	\begin{equation}
		D = a_{1j}A_{ij} + a_{2j}A_{2j} + \dots + a_{nj}A_{nj} (j=1,2,\dots,n) 
	\end{equation}
\end{定理}

证明范德蒙德(Vandermonde)行列数
\begin{equation}\label{VandermondeDeterminant}
	D_{n}=\left |\begin{array}{cccc}
	1 & 1 & \dots & 1\\
	a_{1}^{1} & a_{2}^{1} & \dots & a_{n}^{1}\\
	a_{1}^{2} & a_{2}^{2} & \dots & a_{n}^{2}\\
	\vdots & \vdots & & \vdots \\
	a_{1}^{n-1} & a_{2}^{n-1} & \dots & a_{n}^{n-1}\\
	\end{array}\right | = \prod_{2 \geqslant i > j \geqslant i}(x_{i} - x_{j})\text{,}
\end{equation}
其中,记号``$ \prod $''表示全体同类因子的乘积。

\begin{推论}
	行列式某一行(某一列)的元素与另一行(另一列)的对应元素的代数余子式乘积之和等于零,即
	\begin{equation}
		a_{i1}A_{j1} + a_{i2}A_{j2} + \dots + a_{in}A_{jn} = 0 \text{,} (i \ne j) 
	\end{equation}
	或
	\begin{equation}
		a_{1i}A_{1j} + a_{2i}A_{2j} + \dots + a_{ni}A_{nj} = 0 \text{,} (i \ne i) 
	\end{equation}
\end{推论}






