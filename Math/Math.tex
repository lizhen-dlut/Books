\documentclass{book}
\usepackage{ctex}

\usepackage{accsupp}

\usepackage{amsmath}
\usepackage{amssymb}
\usepackage{amsthm}

\usepackage{animate}
\usepackage{appendix}
\usepackage{array}%表格相关
\usepackage{bbding}%符号
\usepackage{booktabs}% 三线表
\usepackage{boxedminipage}
\usepackage[format=hang, font=small, textfont=it]{caption}%改变图表标题格式
\usepackage{changepage}
\usepackage{chemformula}
\usepackage{color}%颜色
\usepackage{xcolor} % 颜色

\definecolor{KeywordColor}{HTML}{00000f}
\definecolor{StringColor}{HTML}{a31515}
\definecolor{CommentColor}{HTML}{008000}
\definecolor{VariableColor}{HTML}{008000}

\usepackage{colortbl}%彩色表格

\usepackage{ctexcap}
\usepackage{diagbox}%斜线表格
%\usepackage{dingbat}%符号
%\usepackage{enumitem}%定制列表的标签、尺寸等
\usepackage{esint}
\usepackage{eurosym}
\usepackage{extarrows}
\usepackage{fancyvrb}
\usepackage{fancyhdr}%页眉、页脚定制

\usepackage{float}%提供H位置选项
%\usepackage{fontspec}
\usepackage{footmisc}%脚注定制

\usepackage[left=1in, right=1in, top=1in, bottom=1in]{geometry} % 设计页面尺寸
\usepackage{glossaries} % 词汇表支持
\makeglossaries
\usepackage{graphicx}%图形包
\usepackage{graphpap}
\usepackage{hologo}%符号
\usepackage{ifthen}
\usepackage{imakeidx}%索引支持
\usepackage{lastpage}%获取最后一页的页码
\usepackage{layout}
\usepackage{lettrine} % 首字下沉
\usepackage{lineno}% 行号
\usepackage{listings}% 代码环境

\usepackage{longtable}%处理跨页表格
\usepackage{ltxtable}
\usepackage{makecell}%单独控制表格单元
\usepackage{makeidx}%索引支持
\makeindex
\usepackage{mathdots}
\usepackage{mathrsfs}
\usepackage{mathtools}
\usepackage{mflogo}%符号
\usepackage{multicol} % 分栏控制
\usepackage{multirow}%表格跨行排版
\usepackage{paralist}%段内列表
\usepackage{pdflscape}
\usepackage{pgfornament}
\usepackage{picinpar}%图文绕排
\usepackage[square,super,sort&compress]{natbib} %定制参考文献
\usepackage{pifont}%符号
\usepackage{shapepar}%根据预定义形状排版
\usepackage{subcaption}%子图标排版
\usepackage{svg}
\usepackage{syntonly}%语法检查
%	\syntaxonly
\usepackage{tabularx}%提供X格式说明符

\usepackage{textcomp}%多种符号

\usepackage{tcolorbox}
\tcbuselibrary{listings,skins,breakable}
\usepackage{tikz}%绘图功能
\usetikzlibrary{shapes}
\usepackage{titlesec}%设置标题的对齐方式
%\usepackage{titletoc} % 用于目录
\usepackage[numindex,numbib]{tocbibind}%把目录、参考文献等加入到目录中
\usepackage{tracefnt}
\usepackage{ulem}
\usepackage{verbatim}%原文输出功能
\usepackage{wasysym}
\usepackage{wrapfig}%图文绕排

%\usepackage{xtab} % 长表格处理
\usepackage{xunicode}%Unicode处理
\usepackage{yhmath}

\usepackage{hyperref}
\usepackage{hyperxmp}

\setcounter{secnumdepth}{4}

\newcommand{\terminologyItem}[1]{\textbf{\uwave{#1}}}

\newcommand{\Bioconductor}{\href{http://www.bioconductor.org/}{Bioconductor}}

\newtheorem{定义}{定义}

\newtheorem{定理}{定理}
\newtheorem{推论}{推论}
\newtheorem{引理}{引理}
\newtheorem{性质}{性质}

\newcommand{\SoftwareMenu}[1]{\textit{\textbf{#1}}}
\newcommand{\ProgrammingLanguagePackageName}[1]{\textit{\textbf{#1}}}
\newcommand{\ProgrammingLanguageNamespace}[1]{\textit{\textbf{#1}}}
\newcommand{\ProgrammingLanguageClass}[1]{\textit{\textbf{#1}}}
\newcommand{\ProgrammingLanguageFunction}[1]{\textit{\textbf{#1}}}
\newcommand{\ProgrammingLanguageInterface}[1]{\textit{\textbf{#1}}}
\newcommand{\ProgrammingLanguageVariable}[1]{\textit{\textbf{#1}}}

\newcommand{\doubleQuote}[1]{``#1''}
\newcommand{\myDoubleQuoteEnglish}[1]{\lq\lq#1\rq\rq}
\newcommand{\myDoubleQuoteChinese}[1]{"#1"}
\newcommand{\myHumanGene}[1]{\textbf{\textcolor{blue}{\textit{#1}}}}
\newcommand{\mykeyword}[1]{\textbf{\textcolor{blue}{#1}}}
\newcommand{\myImportantPoint}[1]{\textbf{\uline{#1}}}
\newcommand{\myKeyPoint}[1]{\textbf{\uline{#1}}}
\newcommand{\myWarningPoint}[1]{\textbf{\uline{#1}}}
\newcommand{\mySuggestion}[1]{\textbf{\uline{#1}}}
\newcommand{\myInlineGlossary}[1]{\textbf{\uline{#1}}}
\newcommand{\myProteinName}[1]{\textcolor{red}{#1}}
\newcommand{\myLifeSpan}[2]{ $ #1 \sim #2 $ }

\newcommand{\myStructureVariation}[1]{\texttt{\detokenize{#1}}}

\input{../terminology}
\bibliographystyle{plain} %推荐JabRef工具

\newcommand{\documentTitle}
    {\texorpdfstring{数学学习笔记}{数学学习笔记}}
\newcommand{\documentAuthor}{Roger Young}
\newcommand{\documentKeywords}{高等数学,线性代数,概率论,数理统计}
\newcommand{\documentSubject}{高等数学,线性代数,概率论,数理统计}

\newcommand{\myKeypoint}[1]{\uline{#1}}

\hypersetup{
	pdftitle=\documentTitle,
	pdfauthor=\documentAuthor,
	pdfsubject= \documentSubject,
	pdfkeywords= \documentKeywords,
	pdfstartview=FitH,
	pdfproducer = \documentAuthor,
	pdfcreator = \documentAuthor, 
	pdfcopyright = \documentAuthor, 
	pdflicenseurl = {https://github.com/hiatcg}
}

\begin{document}
\title{\documentTitle}
\author{\documentAuthor}

\maketitle
\tableofcontents

\part{高等数学}

\input{./chapters/函数与极限}
\input{./chapters/导数与微分}
\chapter{微分中值定理与导数的应用}
\input{./chapters/不定积分}
\input{./chapters/定积分}
\input{./chapters/定积分的应用}
\input{./chapters/微分方程}
\input{./chapters/向量代数与空间解析几何}
\input{./chapters/多元函数微分法及其应用}
\input{./chapters/重积分}
\input{./chapters/曲线积分与曲面积分}
\input{./chapters/无穷级数}
\input{./chapters/积分表}

\part{线性代数}
\chapter{行列式}


\chapter{矩阵及其运算}

\section{线性方程组和矩阵}

\subsection{线性方程组}

\begin{定义}
	设有$ n $个未知数$ m $个方程的线性方程组
	\begin{equation} \label{n元线性方程组}
		\left\{
		\begin{array}{l}
			a_{11}x_{1} + a_{12}x_{2} + \dots + a_{1n}x_{n}= b_{1}, \\
			a_{21}x_{1} + a_{22}x_{2} + \dots + a_{2n}x_{n}= b_{2}, \\
			\dots \dots \dots \dots\\
			a_{m1}x_{1} + a_{m2}x_{2} + \dots + a_{mn}x_{n}= b_{m}, \\
		\end{array} \right.
	\end{equation}
	其中$ a_{ij} $是第$ i $个方程的第$ j $个未知数的系数,$ b_{i} $是第$ i $个方程的常数项,$ i=1,2,\dots,m;j=1,2,\dots,n $,当常数项$ b_{1},b_{2},\dots,b_{m} $不全为零时,线性方程组\ref{n元线性方程组}叫做\terminologyItem{n元非齐次线性方程组},当$ b_{1},b_{2},\dots,b_{m} $全为零时,\ref{n元线性方程组}式称为
	\begin{equation} \label{n元齐次线性方程组}
		\left\{
		\begin{array}{l}
			a_{11}x_{1} + a_{12}x_{2} + \dots + a_{1n}x_{n}= 0, \\
			a_{21}x_{1} + a_{22}x_{2} + \dots + a_{2n}x_{n}= 0, \\
			\dots \dots \dots \dots\\
			a_{m1}x_{1} + a_{m2}x_{2} + \dots + a_{mn}x_{n}= 0, \\
		\end{array} \right.
	\end{equation}
	叫做\terminologyItem{n元齐次线性方程组}。
\end{定义}

\input{./chapters/矩阵的初等变换与线性方程组}
\input{./chapters/向量组的线性相关性}
\input{./chapters/相似矩阵及二次型}
\input{./chapters/线性空间与线性变换}

\part{概率论与数理统计}
\input{./chapters/概率论的基本概念}
\input{./chapters/随机变量及其分布}
\input{./chapters/多维随机变量及其分布}
\chapter{随机变量的数字特征}

前面我们介绍了随机变量的分布函数、概率函数和分布律,它们都能完整地描述随机变量,但在某些实际或理论问题中,人们感兴趣于某些能描述随机变量某一种特征的常数。这种由随机变量的分布所确定的、能刻画随机变量某一方面的特征的常数统称为\myconcepts{数字特征},它在理论和实际应用中都很重要。

\section{数学期望}

\begin{definition}
	设离散型随机变量X的分布律为
	\begin{equation}
		P\left\lbrace X = x_k \right\rbrace = p_k\text{,} k = 1, 2, \dots. \notag
	\end{equation}
	若级数
	\begin{equation}
		\sum_{k =1}^{\infty}x_{k}p_{k} \notag
	\end{equation}
	绝对收敛,则称级数$ \displaystyle \sum_{k=1}^{\infty}x_{k}p_{k} $的和为随机变量$ X $的\myconcepts{数学期望},记为$ E(X) $,即
	\begin{equation}
		E(X) = \sum_{k =1}^{\infty}x_{k}p_{k} \notag
	\end{equation}
	
	设连续型随机变量X的概率密度为f(x),若积分
	\begin{equation}
		\int_{-\infty}^{\infty}xf(x)\dx \notag
	\end{equation}
	绝对收敛,则称积分$ \int_{-\infty}^{\infty}xf(x)\dx $的值为随机变量$ X $的\myconcepts{数学期望},记为$ E(X) $,即
	\begin{equation}
		E(X) = \int_{-\infty}^{\infty}xf(x)\dx
	\end{equation}
\end{definition}

数学期望简称\myconcepts{期望},又称为\myconcepts{均值}。

数学期望$ E(X) $完全由随机变量$ X $的概率分布所确定,若$ X $服从某一分布,也称$ E(X) $是这一分布的数学期望。

\begin{theorem}
	设$ Y $是随机变量$ X $的函数:$ Y=g(X) $\myparenthese{g是连续函数}
	\begin{enumerate}
		\item 如果$ X $是离散型随机变量,它的分布律为$ P\left\lbrace X=x_k\right\rbrace = p_k, k = 1, 2, \dots $,若$ \displaystyle \sum_{k=1}^{\infty}g(x_k)p_k $绝对收敛,则有
		\begin{equation}
			E(Y) = E[g(X)] = \sum_{k=1}^{\infty}g(x_k)p_K \text{。}
		\end{equation}
		\item 如果$ X $是连续型随机变量,它的概率密度为$ f(x) $,若$ \displaystyle \int_{\infty}^{\infty}g(x)f(x)\dx $绝对收敛,则有
		\begin{equation}
			E(Y) = E[g(X)] = \int_{\infty}^{\infty}g(x)f(x)\dx \text{。}
		\end{equation}
	\end{enumerate}
\end{theorem}

定理的重要意义在于当我们求$ E(Y) $时,不必算出$ Y $的分布律或概率密度,而只需要利用$ X $的分布律或概率密度就可以了。

\section{方差}

\begin{definition}
	设$ X $是一个随机变量,若$ E\left\lbrace [X-E(X)]^2 \right\rbrace $存在,则称$ E\left\lbrace [X-E(X)]^2 \right\rbrace $为$ X $的方差,记为$ D(X) $或$ Var(X) $,即
	\begin{equation}
		D(X) = Var(X) = E\left\lbrace [X-E(X)]^2 \right\rbrace \text{。}
	\end{equation}
	在应用上,还引入量$ \sqrt{D(X)} $,记为$ \sigma(X) $,称为标准差或均方差。
\end{definition}

\begin{theorem}
	设随机变量$ X $具有数学期望$ E(X)=\mu $,方差$ D(X)=\sigma^2 $,则对于任意整数$ \epsilon $,不等式
	\begin{equation}
		P\left\lbrace |X-\mu|\geqslant\epsilon \right\rbrace \leqslant \frac{\sigma^2}{\epsilon^2}
	\end{equation}
	成立。
	
	这一不等式称为\myconcepts{切比雪夫\myparenthese{Chebyshev}不等式}}。
\end{theorem}

\section{协方差及相关系数}

\begin{definition}
	量$ E\left\lbrace [X-E(X)][Y-E(Y)] \right\rbrace $称为随机变量$ X $与$ Y $的协方差,记为$ Cov(X, Y) $,即
	\begin{equation}
		Cov(X,Y)=E\left\lbrace [X-E(X)][Y-E(Y)] \right\rbrace \text{。}
	\end{equation}
	而
	\begin{equation}
		\rho_{XY} = \frac{Cov(X, Y)}{\sqrt{D(X)}\sqrt{D(Y)}}
	\end{equation}
	称为随机变量$ X $与$ Y $的相关系数。
\end{definition}

\section{矩、协方差矩阵}
\input{./chapters/大数定律及中心极限定理}
\input{./chapters/样本及抽样分布}
\input{./chapters/参数估计}
\input{./chapters/假设检验}
\input{./chapters/方差分析及回归分析}
\input{./chapters/bootstrap方法}
\input{./chapters/随机过程及其统计描述}
\input{./chapters/马尔可夫链}


\chapter*{附录}
\printindex
\printglossaries
\end{document}