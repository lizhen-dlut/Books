\chapter{文件和数据}

在应用程序的开发过程中,数据的读写操作是一项基本、必备的操作。需要熟练掌握相关的技术。

\section{文件与目录}

UWP提供了对文件/目录操作的支持,使用UWP提供的API,可以实现最常见的文件读写操作。

文件读写相关的主要类型,主要分布在\ProgrammingLanguageNamespace{Windows.Storage}命名空间及该命名空间的子命名空间中。

\begin{longtable}{|c|l|}
	\caption{文件/目录操作相关的重要类型} \label{tab:文件/目录操作相关的重要类型}\\
	\hline 
	类型名 & 说明 \\
	\hline 
	\endfirsthead
	\multicolumn{2}{l}{(续表\ref{tab:文件/目录操作相关的重要类型})}\\
	\hline 
	类型名 & 说明 \\
	\hline
	\endhead
	\hline
	\multicolumn{2}{c}{\itshape 接下页$ \dots \dots $}\\
	\endfoot
	\hline
	\endlastfoot
	\ProgrammingLanguageClass{StorageFile} & 表示一个文件实例。 \\ 
	\hline 
	\ProgrammingLanguageClass{StorageFolder} & 表示一个目录。 \\ 
	\hline 
	\ProgrammingLanguageClass{KnownFolders} & 静态类型,可访问几个特殊的目录。 \\ 
	\hline 
	\ProgrammingLanguageClass{FileIO} & 针对StorageFile实例而设计,通过一系列静态方法来实现对文件的读写。 \\ 
	\hline 
	\ProgrammingLanguageClass{PathIO} & 针对表示文件(夹)的字符串而设计,通过一系列静态方法来实现对文件的读写。 \\ 
	\hline 
\end{longtable} 

\section{XML数据处理}

\subsection{XML格式简介}

\subsection{读写XML}

\ProgrammingLanguageNamespace{Windows.Data.Xml.Dom}命名空间下提供了若干类型,可以帮助开发者对XML文档进行读取、写入、筛选等操作。其中\ProgrammingLanguageClass{XmlDocument}类是比较核心的类型,它表示一个XML文档的实例,使用该类可以通过代码来构建XML文档。

在读取XML数据时,可以调用\ProgrammingLanguageFunction{LoadFromUriAsync}(从URI加载)方法或\ProgrammingLanguageFunction{LoadFromFileAsync}(从文件加载)两个静态成员,直接返回\ProgrammingLanguageClass{XmlDocument}对象实例,随后,就可以对XML文档进行读取或编辑操作。如果要创建全新的XML文档,可在实例化\ProgrammingLanguageClass{XmlDocument}类后,调用相应的方法来为文档创建各种类型的节点:

\begin{itemize}
	\item \ProgrammingLanguageFunction{CreateComment}方法:创建XML注释节点,返回类型为\ProgrammingLanguageClass{XmlComment};
	\item \ProgrammingLanguageFunction{CreateElement}方法:创建XML元素,返回类型为\ProgrammingLanguageClass{XmlElement};
	\item \ProgrammingLanguageFunction{CreateTextNode}方法:创建纯文本节点;
	\item \ProgrammingLanguageFunction{CreateAttribute}方法:创建XML元素属性的\doubleQuote{$ name = value $}对,返回类型为\ProgrammingLanguageClass{XmlAttribute};
	\item \ProgrammingLanguageFunction{CreateCDataSection}方法:创建CData节点,可以存储不被XML解析器分析的文本;
\end{itemize}

不管是\ProgrammingLanguageClass{XmlDocument},还是\ProgrammingLanguageClass{XmlComment}、\ProgrammingLanguageClass{XmlElement}等类型的节点,它们都实现了同一个接口\ProgrammingLanguageInterface{IXmlNode},所以只需要调用\ProgrammingLanguageVariable{A}节点的\ProgrammingLanguageFunction{AppendChild}方法,并将\ProgrammingLanguageVariable{B}节点作为参数,就可以将\ProgrammingLanguageVariable{B}节点添加到\ProgrammingLanguageVariable{A}节点的子节点集合中。要从\ProgrammingLanguageVariable{A}节点的子节点集合中删除\ProgrammingLanguageVariable{B}节点,调用\ProgrammingLanguageFunction{RemoveChild}方法即可。

下面通过一个实例来展示如何创建一个XML文档和读取XML文档。

XmlSample页面的XAML代码如下:

\begin{lstlisting}[style=XamlStyle]
<Grid Background="{ThemeResource ApplicationPageBackgroundThemeBrush}">
	<Grid.RowDefinitions>
		<RowDefinition Height="*"/>
		<RowDefinition Height="auto"/>
	</Grid.RowDefinitions>

	<CommandBar Grid.Row="1">
		<AppBarButton x:Name="SaveXml"  Icon="Save" Label="创建" Click="SaveXml_Click"/>
		<AppBarButton x:Name="LoadXml" Icon="Read" Label="读取" Click="LoadXml_Click"/>
	</CommandBar>

	s<TextBlock Name="XmlTextBlock" TextWrapping="Wrap" Margin="5"/>
</Grid>
\end{lstlisting}

可用下面的代码创建并保存XML文档(函数\ProgrammingLanguageFunction{SaveXml\_Click}):

\begin{lstlisting}[style=CSharpStyle]
private async void SaveXml_Click(System.Object sender, RoutedEventArgs e)
{
	//创建XmlDocument实例
	XmlDocument doc = new XmlDocument();
	XmlElement root = doc.CreateElement("Books");
	doc.AppendChild(root);

	XmlElement book = doc.CreateElement("book");
	// 设置属性
	book.SetAttribute("ISBN", "978-0321563842");
	book.SetAttribute("Version", "4th Edition");
	XmlText text = doc.CreateTextNode("The C++ Programming Language");
	book.AppendChild(text);

	root.AppendChild(book);
	// 保持到文件
	StorageFolder local = ApplicationData.Current.LocalFolder;
	StorageFile xmlFile = await local.CreateFileAsync("sample.xml", CreationCollisionOption.ReplaceExisting);
	if (xmlFile != null)
	{
		await doc.SaveToFileAsync(xmlFile);
		await new MessageDialog("File Saved!").ShowAsync();
	}
}
\end{lstlisting}

可用下面的代码读取XML文档(函数\ProgrammingLanguageFunction{LoadXml\_Click}):

\begin{lstlisting}[style=CSharpStyle]
private async void LoadXml_Click(System.Object sender, RoutedEventArgs e)
{
	// 从本地文件中加载XML
	StorageFolder local = ApplicationData.Current.LocalFolder;
	StorageFile xmlFile = await local.GetFileAsync("sample.xml");
	if (xmlFile != null)
	{
		XmlDocument doc = await XmlDocument.LoadFromFileAsync(xmlFile);
		XmlTextBlock.Text = doc.GetXml();
	}
}
\end{lstlisting}

\subsection{相关类简介}

\section{JSON数据处理}

JSON(JavaScript Object Notation)即JavaScript对象表示方法。可用简单的文本来描述JavaScript语言所能识别的对象。

\ProgrammingLanguageNamespace{Windows.Data.Json}命名空间下包含一些让开发者能够轻松访问和处理JSON格式数据的类型。

\begin{itemize}
	\item \ProgrammingLanguageInterface{IjsonValue}接口:对封装JSON值的类型进行界定规范。其中\ProgrammingLanguageFunction{Stringify}方法可以返回某个JSON值的字符串表示形式;
	\item \ProgrammingLanguageClass{JsonValue}类:表示一个简单的JSON值,如字符串、数值等。
	\item \ProgrammingLanguageClass{JsonArray}类:表示JSON中的数值;
	\item \ProgrammingLanguageClass{JsonObject}类:表示一个单独的JSON对象。
\end{itemize}

下面通过一个实例来展示如何创建一个JSON文档和读取JSON文档。

JsonSample页面的XAML代码如下:

\begin{lstlisting}[style=XamlStyle]
<Grid Background="{ThemeResource ApplicationPageBackgroundThemeBrush}">
	<Grid.RowDefinitions>
		<RowDefinition Height="*"/>
		<RowDefinition Height="*"/>
		<RowDefinition Height="auto"/>
	</Grid.RowDefinitions>

	<CommandBar Grid.Row="2">
		<AppBarButton x:Name="SaveJson"  Icon="Save" Label="创建" Click="SaveJson_Click"/>
		<AppBarButton x:Name="LoadJson" Icon="Read" Label="读取" Click="LoadJson_Click"/>
	</CommandBar>

	<TextBlock Name="SaveJsonTextBlock" Grid.Row="0" TextWrapping="Wrap" Margin="5"/>
	<TextBlock Name="LoadJsonTextBlock" Grid.Row="1" TextWrapping="Wrap" Margin="5"/>
</Grid>
\end{lstlisting}

可用下面的代码创建并保存JSON(函数\ProgrammingLanguageFunction{SaveJson\_Click})

\begin{lstlisting}[style=CSharpStyle]
private async void SaveJson_Click(object sender, RoutedEventArgs e)
{
	JsonObject obj = new JsonObject();
	obj["Title"] = JsonValue.CreateStringValue("The C++ Programming Language");
	obj["ISBN"] = JsonValue.CreateStringValue("978-0321563842");
	obj["Version"] = JsonValue.CreateStringValue("4th Edition");
	
	SaveJsonTextBlock.Text = obj.Stringify();
	
	var local = ApplicationData.Current.LocalFolder;
	var jsonFile = await local.CreateFileAsync("sample.json", CreationCollisionOption.ReplaceExisting);
	if (jsonFile != null)
	{
		await FileIO.WriteTextAsync(jsonFile, obj.Stringify());
		await new MessageDialog("File Saved!").ShowAsync();
	}
}
\end{lstlisting}

可用下面的代码读取JSON文档(函数\ProgrammingLanguageFunction{LoadJson\_Click}):

\begin{lstlisting}[style=CSharpStyle]
private async void LoadJson_Click(object sender, RoutedEventArgs e)
{
	var local = ApplicationData.Current.LocalFolder;
	var jsonFile = await local.GetFileAsync("sample.json");
	if (jsonFile != null)
	{
		LoadJsonTextBlock.Text = await FileIO.ReadTextAsync(jsonFile);
	}
}
\end{lstlisting}
