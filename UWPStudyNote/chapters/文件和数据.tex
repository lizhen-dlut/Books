\chapter{文件和数据}

\section{XML数据处理}

\subsection{XML格式简介}

\subsection{读写XML}

\ProgrammingLanguageNamespace{Windows.Data.Xml.Dom}命名空间下提供了若干类型,可以帮助开发者对XML文档进行读取、写入、筛选等操作。其中\ProgrammingLanguageClass{XmlDocument}类是比较核心的类型,它表示一个XML文档的实例,使用该类可以通过代码来构建XML文档。

在读取XML数据时,可以调用\ProgrammingLanguageFunction{LoadFromUriAsync}(从URI加载)方法或\ProgrammingLanguageFunction{LoadFromFileAsync}(从文件加载)两个静态成员,直接返回\ProgrammingLanguageClass{XmlDocument}对象实例,随后,就可以对XML文档进行读取或编辑操作。如果要创建全新的XML文档,可在实例化\ProgrammingLanguageClass{XmlDocument}类后,调用相应的方法来为文档创建各种类型的节点:

\begin{itemize}
	\item \ProgrammingLanguageFunction{CreateComment}方法:创建XML注释节点,返回类型为\ProgrammingLanguageClass{XmlComment};
	\item \ProgrammingLanguageFunction{CreateElement}方法:创建XML元素,返回类型为\ProgrammingLanguageClass{XmlElement};
	\item \ProgrammingLanguageFunction{CreateTextNode}方法:创建纯文本节点;
	\item \ProgrammingLanguageFunction{CreateAttribute}方法:创建XML元素属性的\doubleQuote{$ name = value $}对,返回类型为\ProgrammingLanguageClass{XmlAttribute};
	\item \ProgrammingLanguageFunction{CreateCDataSection}方法:创建CData节点,可以存储不被XML解析器分析的文本;
\end{itemize}

不管是\ProgrammingLanguageClass{XmlDocument},还是\ProgrammingLanguageClass{XmlComment}、\ProgrammingLanguageClass{XmlElement}等类型的节点,它们都实现了同一个接口\ProgrammingLanguageInterface{IXmlNode},所以只需要调用\ProgrammingLanguageVariable{A}节点的\ProgrammingLanguageFunction{AppendChild}方法,并将\ProgrammingLanguageVariable{B}节点作为参数,就可以将\ProgrammingLanguageVariable{B}节点添加到\ProgrammingLanguageVariable{A}节点的子节点集合中。要从\ProgrammingLanguageVariable{A}节点的子节点集合中删除\ProgrammingLanguageVariable{B}节点,调用\ProgrammingLanguageFunction{RemoveChild}方法即可。

下面通过一个实例来展示如何创建一个XML文档和读取XML文档。

\begin{lstlisting}[style=XamlStyle]
<Grid Background="{ThemeResource ApplicationPageBackgroundThemeBrush}">
	<Grid.RowDefinitions>
		<RowDefinition Height="*"/>
		<RowDefinition Height="auto"/>
	</Grid.RowDefinitions>

	<CommandBar Grid.Row="1">
		<AppBarButton Icon="Save" Label="创建"/>
		<AppBarButton Icon="Read" Label="读取"/>
	</CommandBar>

	<TextBlock Name="XmlTextBlock" TextWrapping="Wrap" Margin="5"/>
</Grid>
\end{lstlisting}
