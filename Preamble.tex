\usepackage{ctex}
\usepackage{accsupp}
	\newcommand{\emptyaccsupp}[1]{\BeginAccSupp{ActualText={}}#1\EndAccSupp{}}
\usepackage{amsmath}
\usepackage{amssymb}
\usepackage{amsthm}
	
\usepackage{animate}
\usepackage{appendix}
\usepackage{array}%表格相关
\usepackage{bbding}%符号
\usepackage{booktabs}% 三线表
\usepackage{boxedminipage}
\usepackage[format=hang, font=small, textfont=it]{caption}%改变图表标题格式
\usepackage{changepage}
\usepackage{chemformula}
\usepackage{color}%颜色
\usepackage{colortbl}%彩色表格

\usepackage{ctexcap}
\usepackage{diagbox}%斜线表格
%\usepackage{dingbat}%符号
%\usepackage{enumitem}%定制列表的标签、尺寸等
\usepackage{esint}
\usepackage{eurosym}
\usepackage{extarrows}
\usepackage{fancyvrb}
\usepackage{fancyhdr}%页眉、页脚定制

\usepackage{float}%提供H位置选项
\usepackage{fontspec}
\usepackage{footmisc}%脚注定制

\usepackage[left=1in, right=1in, top=1in, bottom=1in]{geometry} % 设计页面尺寸
\usepackage{glossaries} % 词汇表支持
	\makeglossaries
\usepackage{graphicx}%图形包
\usepackage{graphpap}
\usepackage{hologo}%符号
\usepackage{ifthen}
\usepackage{imakeidx}%索引支持
\usepackage{lastpage}%获取最后一页的页码
\usepackage{layout}
\usepackage{lettrine} % 首字下沉
\usepackage{lineno}% 行号
\usepackage{listings}% 代码环境

\lstdefinestyle{CSharpStyle}{
	belowcaptionskip=1\baselineskip,
	breaklines=true,
	tabsize=4,
	numbers=left,numberstyle=\tiny, numbersep=10pt,
	frame=single,
	xleftmargin=\parindent,
	language=C,
	showstringspaces=false,
	basicstyle=\footnotesize\ttfamily,
	keywordstyle=\bfseries\color{green!40!black},
	commentstyle=\itshape\color{purple!40!black},
	identifierstyle=\color{blue},
	stringstyle=\color{orange},
}

\lstdefinestyle{HtmlStyle}{
	belowcaptionskip=1\baselineskip,
	breaklines=true,
	tabsize=4,
	numbers=left,numberstyle=\tiny, numbersep=10pt,
	frame=single,
	xleftmargin=\parindent,
	language=C,
	showstringspaces=false,
	basicstyle=\footnotesize\ttfamily,
	keywordstyle=\bfseries\color{green!40!black},
	commentstyle=\itshape\color{purple!40!black},
	identifierstyle=\color{blue},
	stringstyle=\color{orange},
}


\usepackage{longtable}%处理跨页表格
\usepackage{ltxtable}
\usepackage{makecell}%单独控制表格单元
\usepackage{makeidx}%索引支持
	\makeindex
\usepackage{mathdots}
\usepackage{mathrsfs}
\usepackage{mathtools}
\usepackage{mflogo}%符号
\usepackage{multicol} % 分栏控制
\usepackage{multirow}%表格跨行排版
\usepackage{paralist}%段内列表
\usepackage{pdflscape}
\usepackage{pgfornament}
\usepackage{picinpar}%图文绕排
\usepackage[square,super,sort&compress]{natbib} %定制参考文献
\usepackage{pifont}%符号
\usepackage{shapepar}%根据预定义形状排版
\usepackage{subcaption}%子图标排版
\usepackage{svg}
\usepackage{syntonly}%语法检查
%	\syntaxonly
\usepackage{tabularx}%提供X格式说明符

\usepackage{textcomp}%多种符号

\usepackage{tcolorbox}
	\tcbuselibrary{listings,skins,breakable}
\usepackage{tikz}%绘图功能
	\usetikzlibrary{shapes}
\usepackage{titlesec}%设置标题的对齐方式
%\usepackage{titletoc} % 用于目录
\usepackage[numindex,numbib]{tocbibind}%把目录、参考文献等加入到目录中
\usepackage{tracefnt}
\usepackage{ulem}
\usepackage{verbatim}%原文输出功能
\usepackage{wasysym}
\usepackage{wrapfig}%图文绕排
\usepackage{xcolor} % 颜色
%\usepackage{xtab} % 长表格处理
\usepackage{xunicode}%Unicode处理
\usepackage{yhmath}

\newtcbox{\latexCommand}[1][green]{on line,
	before upper=\char`\\, arc=2pt,outer arc=2pt, 
	colback=#1!10!white, colframe=#1!50!black, boxsep=1pt, 
	left=1pt, right=1pt, top=1pt, bottom=1pt,
	boxrule=0pt, bottomrule=1pt, toprule=1pt}
  
\newtcbox{\latexEnvironment}[1][violet!70!cyan]{on line,
	before upper=\ttfamily,arc=0pt,outer arc=0pt,
	colback=#1!10!white,colframe=#1!50!black,
	boxsep=0pt,left=1pt, right=1pt, top=1pt, bottom=1pt,
	boxrule=0pt, bottomrule=1pt, toprule=1pt}

\newtcbox{\latexPackage}[1][orange!70!red]{on line,
	before upper={\rule[-0.2ex]{0pt}{1ex}\ttfamily},
	arc=0.8ex,colback=#1!30!white, colframe=#1!50!black,
	boxsep=0pt, left=1.5pt, right=1.5pt, top=1pt, bottom=1pt, boxrule=1pt}

\newtcblisting{latexCode}{breakable,
	skin=bicolor, listing only, 
	colback=gray!30!white, colbacklower=white, colframe=cyan!75!black,
	left=6mm, top=2pt, bottom=2pt, fontupper=\small,
	% listing style
	listing options={style=tcblatex,
		keywordstyle=\color{blue}, 
		breaklines = true,
		commentstyle=\color{green!50!black},
		numbers=left,
		numberstyle=\tiny\color{red!75!black}\emptyaccsupp,
		emptylines=1,
		escapeinside=``}
}

\newtcblisting{cppCode}[1]{breakable,
	title=#1, fonttitle=\bfseries,
	skin=bicolor, listing only, 
	colback=gray!30!white, colbacklower=white, colframe=cyan!75!black,
	left=6mm, top=2pt, bottom=2pt, fontupper=\small,
	% listing style
	listing options={language=c++,
		keywordstyle=\color{blue}, 
		commentstyle=\color{green!50!black},
		numbers=left,
		numberstyle=\tiny\color{red!75!black}\emptyaccsupp,
		emptylines=1,
		escapeinside=``}
}

%\newtcblisting{csCode}[1]{breakable,
%	title=#1, fonttitle=\bfseries,
%	skin=bicolor, listing only, 
%	colback=gray!30!white, colbacklower=white, colframe=cyan!75!black,
%	left=6mm, top=2pt, bottom=2pt, fontupper=\small,
%	% listing style
%	listing options={language=csh,
%		keywordstyle=\color{blue}, 
%		commentstyle=\color{green!50!black},
%		numbers=left,
%		numberstyle=\tiny\color{red!75!black}\emptyaccsupp,
%		emptylines=1,
%		escapeinside=``}
%}

\newtcblisting{rCode}{breakable,
	skin=bicolor, listing only, 
	colback=gray!30!white, colbacklower=white, colframe=cyan!75!black,
	left=6mm, top=2pt, bottom=2pt, fontupper=\small,
	% listing style
	listing options={language=R,
		keywordstyle=\color{blue}, 
		commentstyle=\color{green!50!black},
		numbers=left,
		numberstyle=\tiny\color{red!75!black}\emptyaccsupp,
		emptylines=1,
		escapeinside=``}
}

\newtcblisting{perlCode}{breakable,
	skin=bicolor, listing only, 
	colback=gray!30!white, colbacklower=white, colframe=cyan!75!black,
	left=6mm, top=2pt, bottom=2pt, fontupper=\small,
	% listing style
	listing options={language=Perl,
		keywordstyle=\color{blue}, 
		commentstyle=\color{green!50!black},
		numbers=left,
		numberstyle=\tiny\color{red!75!black}\emptyaccsupp,
		emptylines=1,
		escapeinside=``}
}

\newtcblisting{pythonCode}{breakable,
	skin=bicolor, listing only, 
	colback=gray!30!white, colbacklower=white, colframe=cyan!75!black,
	left=6mm, top=2pt, bottom=2pt, fontupper=\small,
	% listing style
	listing options={language=Python,
		keywordstyle=\color{blue}, 
		commentstyle=\color{green!50!black},
		numbers=left,
		numberstyle=\tiny\color{red!75!black}\emptyaccsupp,
		emptylines=1,
		escapeinside=``}
}

\newtcblisting{latexCodePreview}{
	skin=bicolor,
	colback=gray!30!white, colbacklower=pink!50!yellow, colframe=cyan!75!black, 
	left=6mm, righthand width=0.4\linewidth, fontupper=\small,
	% listing style
	listing side text,
	listing options={style=tcblatex,
		keywordstyle=\color{blue},
		commentstyle=\color{green!50!black},
		numbers=left,
		numberstyle=\tiny\color{red!75!black}\emptyaccsupp,
		emptylines=1,
		escapeinside=``}
}

\newtcblisting{latexCodePreviewAbove}{
	skin=bicolor, 
	colback=gray!30!white, colbacklower=pink!50!red, colframe=cyan!75!black, 
	left=6mm, righthand width=0.4\linewidth, fontupper=\small,
	% listing style
	listing and text,
	listing options={style=tcblatex,
		keywordstyle=\color{blue},
		commentstyle=\color{green!50!black},
		numbers=left,
		numberstyle=\tiny\color{red!75!black}\emptyaccsupp,
		emptylines=1,
		escapeinside=``}
}

% Tikz Code
\newtcblisting{tikzshow}{tikz lower,
	skin=bicolor,
	colback=gray!30!white, colbacklower=white, colframe=cyan!75!black, 
	left=6mm, righthand width=3.5cm, listing outside text,
	% listing style  
	listing options={style=tcblatex,
		language=TeX,
		basicstyle=\small\ttfamily,
		keywordstyle=\color{blue},
		commentstyle=\color{green!50!black},
		numbers=left,
		numberstyle=\tiny\color{red!75!black}\emptyaccsupp,
		emptylines=1,
		escapeinside=``}
}

\lstdefinelanguage{tikzlang}{
	classoffset=0, % 蓝色的keyword
	morekeywords={begin, end, newcommand, draw, node, 
								coordinate, tikzstyle, foreach},
	keywordstyle=\color{keywordcolor},
	classoffset=1, % 棕色的其他关键字
	morekeywords={tikzpicture,grid,at, thick,thin,very,ultra,
								red,green,yellow,blue,cyan,magenta,black,
								gray,darkgray,lightgray,brown,lime,
								olive,orange,pink,purple,teal,violet,white},
	keywordstyle=\color{envicolor},
	morecomment=[l]{\%},
	morecomment=[s]{/*}{*/},
	morestring=[b]',
	% Escape
	escapeinside=``
}

\lstnewenvironment{tikzcode}[1]
		{\lstset{language=tikzlang, 
			basicstyle=\small\ttfamily,
			breaklines=true,%backgroundcolor=\color{white},
			linewidth=0.7\linewidth,#1}
		}
		{}

\usepackage{hyperref}
\usepackage{hyperxmp}