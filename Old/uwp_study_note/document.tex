\documentclass{ctexbook}
\usepackage[left=2cm, right=2cm]{geometry}

\usepackage{xcolor}
\definecolor{bg}{rgb}{0.95,0.95,0.95}

\usepackage{minted}
\usepackage{hyperref}

\newcommand{\Rplus}{\protect\hspace{-.1em}\protect\raisebox{.35ex}{\smaller{\smaller\textbf{+}}}}
\newcommand{\Cpp}{c++}
\newcommand{\csharp}{C\#}

\newmintedfile[]{news}{
linenos=true,frame=single, bgcolor=bg
}

\begin{document}
	\chapter{网络编程}
	\url{http://www.cnblogs.com/mushroom/p/5079964.html}
	
	\section{概述}
	作为一个Universal Windows Platform (UWP)开发者,如果你尝试使用http与web服务或其他服务端通讯时,有多个API可以选择。 UWP中最常见并推荐使用的HTTP客户端API实现是 \mintinline{csharp}{System.Net.Http.HttpClient} 和 \mintinline{csharp}{Windows.Web.Http.HttpClient} 。
	
	关于这些APIs不同之处,从功能上来说两组APIs是上相等的,那在不同场景下选择哪一个呢,诸如此类的问题。 在这篇文章中,我们会去尝试定位这些问题,理清楚这两组APIs的用途及使用场景。
	
	第一个推荐AIP是 \mintinline{csharp}{System.Net.Http.HttpClient},相比旧的HttpWebRequest API,这个API的目标是提供一个简单的,干净的抽象层,比较灵活的实现http客户端功能。比如,它允许链接自定义处理器,开发者可以拦截每个request和response,去实现自定义逻辑。 在windows8.1之后,所有功能都在.NET下面实现。 在windows10 UWP中这个API实现移到 \mintinline{csharp}{Windows.Web.Http}和 \mintinline{csharp}{WinINet Http}层上。
	
	另外一个推荐API是 \mintinline{csharp}{Windows.Web.Http.HttpClient},增加这个API的主要目是,把不同windows应用开发语言(C\#, VB, C++, JavaScript)下,不同Http APIs合成一个,它支持上述APIs的所有特性。 大多数基础API都是从System.Net.Http派生的,在Windows HTTP基础上实现。
	
	\section{如何选择}
	在UWP中这些HTTP API都是可以使用的,对于开发者来说最大的问题是在APP中应该使用哪一个。其答案取决去几个因素:
	\begin{itemize}
		\item 是否需要结合本地UI收集用户证书,控制HTTP缓存读和写,或者通过指定的ssl客户端证书去做认证? 如果需要认证,那是应使用Windows.Web.Http.HttpClient。在现在的UWP中,Windows.Web.Http提供HTTP设置,它比System.Net.Http API更好的控制这些。 在未来的版本,也会加强支持System.Net.Http在UWP中的特性。
		\item 是否考虑写\textbf{跨平台}的.NET代码(跨UWP/ASP.NET 5/IOS和Android)? 如果需要,那使用System.Net.Http API。它可以让你写的代码复用在其他.Net平台上,比如ASP.Net 5和.NET桌面平台应用。 通过使用Xamarin,这些API在IOS和Android中也得到支持。
	\end{itemize}
	现在就比较好理解为什么会有两个相似APIs了,也了解怎么在二者之间进行选择,下面进一步了解这两个对象模型。
	
	\section{System.Net.Http}
	其HttpClient对象是最顶端的抽象模型,在HTTP协议client-server模型中它表示client这部分。其client能发出多个request请求(用HttpRequestMessage表示)到服务端上,从服务端接收响应(用HttpResponseMessage表示)。用HttpContent基类和它派生出的类,表示对象body和每个request或response的content头部,比如StreamContent,MultipartContent和StringContent。它们表示各种http实体body内容。这些类都会提供ReadAs开头的一组方法,它能从请求或响应实体body中,以字符串形式、字节数组、流形式读取内容。
	
	每一个HttpClient对象下都有一个处理者对象,它表示client下所有与HTTP相关的配置。从概念上来说,可以认为它是client部分下HTTP协议栈的代表。在客户端发送HTTP请求到服务端和传输数据到客户端上,它是非常可靠的。
	
	在System.Net.Http API中默认处理者是HttpClientHandler。当你创建HttpClient对象实例时,会使用默认HTTP stack设置,自动帮你创建一个HttpClientHandler。如果你想修改默认一些设置,比如缓存行为,自动压缩,证书或代理,可以直接创建一个HttpClientHandler实例,修改它的属性,把它当做HttpClient构造函数的参数传入。这样HttpClient对象就会使用我们自定义的处理器,如下:
	
\begin{minted}[frame=single]{csharp}
HttpClientHandler myHandler = new HttpClientHandler(); 
myHandler.AllowAutoRedirect = false; 
HttpClient myClient = new HttpClient(myHandler);
\end{minted}

\subsection{链式处理器}
System.Net.Http.HttpClient API设计中一个重要优势是:能够插入自定义处理器、在HttpClient对象下创建一连串的处理器。例如:构建一个app,它从web服务中请求一些数据。这时就可以自定义逻辑去处理HTTP服务端响应的4xx (客户端错误)和5xx (服务端错误),使用具体的重试步骤,比如尝试不同的端口请求或添加一个用户认证。 还可能会想从业务逻辑部分分离出HTTP相关的工作,它只关心web服务的数据返回。

这就可以使用自定义处理器类来完成,它从DelegatingHandler派生出,例如CustomHandler1,然后创建一个新实例,把它传入HttpClient构造函数。 DelegatingHandler类的InnerHandler属性被用指定下一个处理器,比如,可以添加个新的自定处理器(例CustomHandler2)到处理链上。处理链上最后一个处理者的InnerHandler,可以设置成HttpClientHandler的实例,它将传递请求到系统的HTTP协议栈上。 从概念上来看如下图:

下面是完成这部分的例子代码:

\begin{minted}[frame=single]{csharp}
public class CustomHandler1 : DelegatingHandler
{
    protected async override Task<HttpResponseMessage> SendAsync(
        HttpRequestMessage request, CancellationToken cancellationToken)
    {
            Debug.WriteLine("Processing request in Custom Handler 1");
            HttpResponseMessage response = await base.SendAsync(request,
                cancellationToken);
            Debug.WriteLine("Processing response in Custom Handler 1");
            return response;
        }
}

public class CustomHandler2 : DelegatingHandler
{
    // Similar code as CustomHandler1.
}

public class Foo
{
    public void CreateHttpClientWithChain()
    {
        HttpClientHandler systemHandler = new HttpClientHandler();
        CustomHandler1 myHandler1 = new CustomHandler1();
        CustomHandler2 myHandler2 = new CustomHandler2();

        // Chain the handlers together.
        myHandler1.InnerHandler = myHandler2;
        myHandler2.InnerHandler = systemHandler;

        // Create the client object with the topmost handler in the chain.
        HttpClient myClient = new HttpClient(myHandler1);
    }
}
\end{minted}

说明:

如果你试图发送一个请求到远程服务器端口上,其链上最后的处理器通常是HttpClientHandler,它实际是从系统HTTP协议栈层面发送这个请求或接收这个响应。作为一种选择,可以使用一个模拟处理器,模拟发送请求到服务器上,返回一个伪造的响应,这可以用来单元测试。

在传递请求到内部处理器之前或响应处理器之上,添加一个处理逻辑,能减少性能消耗。这个处理器场景下,最好能避免使用耗时的同步操作。

关于链式处理概念的详细信息,可以看Henrik Nielsen的这篇博客,(注意文章参考的是ASP.NET Web API的API版本。它和本文讨论的.NET framework有一些细微的不同,但在链式处理器上的概念是一样的)

\subsection{Windows.Web.Http}
Windows.Web.Http API的对象模型跟上面描述的System.Net.Http版本非常 ,它也有client entity的概念,一个处理器(在这叫“filter”过滤器),及在client和系统默认过滤器之间选择是否插入自定义逻辑。

其大多数类型是直接类似于System.Net.Http的类型的,如下:

在上面关于System.Net.Http API的链式处理器讨论,也可应用于Windows.Web.Http API,这里你可以创建自定义链式过滤器,传递它们到HttpClient对象的构造函数中。
	
	
	\part{常用数据库说明}


\chapter{cBioPortal数据库简介}
cBioPortal数据库主要针对癌症基因组学(Cancer Genomics)研究,其提供了大规模癌症基因组的可视化、分析和下载功能。其官方网址为:
\centerline{\href{http://www.cbioportal.org/index.do}{http://www.cbioportal.org/index.do}}

\section{API简介}
通过cBioPortal提供的CGDS网络服务(Cancer Genomic Data Server web service),可以通过编程的方式快速获取cBioPortal所有的基因组学数据。CGDS网络服务是基于REST的,其返回的数据是Tab键分隔的文本格式或者XML格式。可以选择的编程语言包括:
\begin{itemize}
	\item Python
	\item R
	\item Perl
	\item Java
	\item MatLab
\end{itemize}

所有的请求通过\href{http://www.cbioportal.org/webservice.do}{http://www.cbioportal.org/webservice.do}提交,请求时需要附上一些必要的参数:
\begin{itemize}
\item cmd:需要执行的操作,可选值包括:
\begin{itemize}
	\item getTypesOfCancer:\href{http://www.cbioportal.org/webservice.do?cmd=getTypesOfCancer}{获取癌症类型。}
	\item getNetwork:\href{http://www.cbioportal.org/webservice.do?cmd=getNetwork}{获取。}
	\item getCancerStudies:\href{http://www.cbioportal.org/webservice.do?cmd=getCancerStudies}{获取癌症研究。}
	\item getGeneticProfiles:\href{http://www.cbioportal.org/webservice.do?cmd=getGeneticProfiles&cancer_study_id=msk_impact_2017}{获取特定癌症研究项目的Genetic Profile信息。}
	\item getProfileData:\href{http://www.cbioportal.org/webservice.do?cmd=getProfileData&case_set_id=msk_impact_2017_all&genetic_profile_id=msk_impact_2017_mutations&gene_list=p53,kras}{获取一个或多个基因的genomic profile数据。}
	\item getCaseLists:\href{http://www.cbioportal.org/webservice.do?cmd=getCaseLists&cancer_study_id=msk_impact_2017}{获取。}
	\item getClinicalData:\href{http://www.cbioportal.org/webservice.do?cmd=getClinicalData&case_set_id=msk_impact_2017_all}{获取样本的临床信息。}
	\item getMutationData:\href{http://www.cbioportal.org/webservice.do?cmd=getMutationData&genetic_profile_id=msk_impact_2017_mutations&gene_list=p53,kras}{获取基因的突变信息。}
\end{itemize}
\item 其他的一些可选参数,该参数跟所执行的cmd相关。
\end{itemize}
比如,我们可以通过以下链接获取cBioPortal中所有的有关癌症的研究项目:
\centerline{\href{http://www.cbioportal.org/webservice.do?cmd=getCancerStudies}{http://www.cbioportal.org/webservice.do?cmd=getCancerStudies}}

对于不同的cmd,其所对应的可选参数也不同。

\subsection{getTypesOfCancer命令}\label{subsec:getTypesOfCancer}
getTypesOfCancer命令用于获取服务器中存储的癌症列表。在调用该命令时,不需要可选参数。其返回的数据包括两列:
\begin{center}
\begin{tabular}{|c|l|}
\hline
$ type\_of\_cancer\_id $ & cBioPortal中唯一表示该癌症的编号。\\
\hline
$ name $ & 癌症名称 \\
\hline
\end{tabular}
\end{center}


\subsection{getCancerStudies命令}\label{subsec:getCancerStudies}
getCancerStudies命令可以用来获取服务器上存储的有关癌症的研究项目的基础数据。在调用该命令时,不需要可选参数。其返回的数据包括三列:
\begin{center}
\begin{tabular}{|c|l|l|}
	\hline
	$ cancer\_study\_id $ & cBioPortal中唯一表示该癌症研究项目的编号。\\
	\hline
	$ name $ & 研究项目的名称 \\
	\hline
	$ description $ & 有关该研究项目的简单说明 \\ 
	\hline
\end{tabular}
\end{center}

\subsection{getGeneticProfiles命令}\label{subsec:getGeneticProfiles}
getGeneticProfiles命令用于获取某个癌症研究项目的所有元数据,包括变异信息、拷贝数信息等。在调用该命令时需要提供一个可选参数:
\begin{center}
	\begin{tabular}{|c|l|}
		\hline
		$ cancer\_study\_id $ & 癌症研究项目的编号。\\
		\hline
	\end{tabular}
\end{center}
其返回的数据包括六列:
\begin{center}
\begin{tabular}{|c|l|l|}
	\hline
	$ genetic\_profile\_id $ & cBioPortal中唯一表示该genetic profile的编号。\\
	\hline
	$ genetic\_profile\_name $ & genetic profile的名称 \\
	\hline
	$ genetic\_profile\_description $ & genetic profile的简单说明 \\ 
	\hline
	$ cancer\_study\_id $ & 癌症研究项目的ID \\ 
	\hline
	$ genetic\_alteration\_type $ & 有关该研究项目的简单说明 \\ 
	\hline
	$ show\_profile\_in\_analysis\_tab $ & 有关该研究项目的简单说明 \\ 
	\hline
\end{tabular}
\end{center}
\subsubsection{举例}
通过链接:

\centerline
{\href{http://www.cbioportal.org/webservice.do?cmd=getGeneticProfiles\&cancer\_study\_id=msk\_impact\_2017}{http://www.cbioportal.org/webservice.do?cmd=getGeneticProfiles\&cancer\_study\_id=msk\_impact\_2017}}可以获取癌症研究项目“MSK-IMPACT”的Genetic Profiles。

\subsection{getCaseLists命令}\label{subsec:getCaseLists}
getCaseLists命令可以返回特定癌症研究项目中的样本信息。在调用该命令时需要提供一个必选参数:
\begin{center}
	\begin{tabular}{|c|l|}
		\hline
		$ cancer\_study\_id $ & 癌症研究项目的编号。\\
		\hline
	\end{tabular}
\end{center}

其返回信息包括五列:
\begin{center}
\begin{tabular}{|c|l|}
	\hline
	$ case\_list\_id $ & \\
	\hline
	$ case\_list\_name $ & \\
	\hline
	$ case\_list\_description $& \\
	\hline
	$ cancer\_study\_id $& \\
	\hline
	$ case\_ids $ &  \\
	\hline
\end{tabular}
\end{center}

\subsection{getProfileData命令}\label{subsec:getProfileData}
	getProfileData命令可以返回一个或多个基因的genomic profile数据。在调用该命令时需要提供三个必选参数:
\begin{center}
	\begin{tabular}{|c|l|}
		\hline
		$ case\_set\_id $ & 由\hyperref[subsec:getCaseLists]{getCaseLists}命令返回的Case Set的ID。\\
		\hline
		$ genetic\_profile\_id $ & 由\hyperref[subsec:getGeneticProfiles]{getGeneticProfiles}命令返回的Genetic Profile。\\
		\hline
		$ gene\_list $ & 基因列表,多个基因之间以英文逗号“,”分隔。\\
		\hline
	\end{tabular}
\end{center}

\subsection{getMutationData命令}\label{subsec:getMutationData}
getMutationData命令可以获取一些额外的信息,比如变异的注释信息。使用该命令时,需要添加以下参数:
\begin{center}
\begin{tabular}{|c|c|l|}
	\hline
	$ genetic\_profile\_id $ & 必选 & 由\hyperref[subsec:getGeneticProfiles]{getGeneticProfiles}命令返回的$ genetic\_profile\_id $。\\
	\hline
	$ case\_set\_id $ & 可选 & 由\hyperref[subsec:getGeneticProfiles]{getGeneticProfiles}命令返回的$ case\_list\_id $。 \\
	\hline
	$ gene\_list $ & 必选 & 以逗号分隔的基因列表(HUGO基因名称或Entrez的基因ID)。\\
	\hline
\end{tabular}
\end{center}

\subsection{getClinicalData 命令}\label{subsec:getClinicalData }
getClinicalData用于获取癌症研究项目中所涉及的样本的基本临床信息。可以获取到的基本信息包括样本编号(CASE\_ID)癌症类型(CANCER\_TYPE)、癌症稍微详细的描述(CANCER\_TYPE\_DETAILED)、所使用的DNA量(DNA\_INPUT)、癌症的转移位置(METASTATIC\_SITE)、Oncotree编号(ONCOTREE\_CODE)、原发灶(PRIMARY\_SITE)、样本类型(SAMPLE\_CLASS)、患者性别(SEX)等。在调用该命令时需要提供一个可选参数:
\begin{center}
\begin{tabular}{|c|l|}
	\hline
	$ case\_set\_id $ & 样本集编号 \\
	\hline
\end{tabular}
\end{center}


\section{cBioPortal使用流程}
cBioPortal数据下载流程:
\begin{enumerate}
	\item 通过“\hyperref[subsec:getCancerStudies]{getCancerStudies}”命令,获取癌症研究项目的基本信息,主要获取$ cancer\_study\_id $值;
	\item 以$ cancer\_study\_id $值为参数,调用“\hyperref[subsec:getCaseLists]{getCaseLists}”命令,获取癌症项目研究的样本集合的 $ case\_list\_id $值。
	\item 以$ case\_list\_id $值作为参数,调用“\hyperref[subsec:getClinicalData]{getClinicalData}”命令,获取癌症项目研究样本的临床信息。
	\item 以$ cancer\_study\_id $值为参数,调用“\hyperref[subsec:getGeneticProfiles]{getGeneticProfiles}”命令,获取癌症项目研究的Genetic Profile的列表。
	\item 以$ genetic\_profile\_id $为参数,$ case\_list\_id $值作为$ case\_set\_id $的参数
\end{enumerate}
%	\chapter{Tips}
	
	UWP(Universal Windows Platform,通用Windows平台)是正快速发展的、相当具有潜力的平台之一,其应用所遵循的设计规范令UWP应用的辨识度很高。本文即将介绍的,就是一些UWP的特色API(包含参考资料),以及一些让你的Win10 UWP看起来更像一个UWP的小技巧。
	
	\begin{enumerate}
		\item 使用UWP图标集
		\item 使用动态磁贴
		\item 使用跳转列表
	\end{enumerate}

\section{使用UWP图标集}

微软为开发者们提供了一整套与UWP风格相适应的图标集,并集合成了一个字体:Segoe MDL2 Assets。这个字体提供了大量的专用字符(其实就是一些Icon)。

一般来讲,只需要创建一个普通的TextBlock,设置字体为Segoe MDL2 Assets,并从字符映射表里找到相应的字符拷贝进内容,就可以正确显示这些Icon。这些Icon可以参考:\href{https://docs.microsoft.com/en-us/windows/uwp/style/segoe-ui-symbol-font}{Segoe MDL2 icons}的说明。

\section{使用动态磁贴}

UWP应用有个独特的展示区:它的磁贴。正确更新动态磁贴内容能给用户提供关键信息,并吸引用户眼球。

更新磁贴有两种方式,一种是按时轮询一个URI来获取磁贴信息,另一种是使用代码在本地更新磁贴。通常来说,后者更灵活,但前者更方便。这里只讲解后者。

本地更新动态磁贴的官方说法,是“推送本地磁贴通知”,这和右下角的弹出式通知一样,属于“通知”一类。使用名称空间 \mintinline{csharp}{Windows.UI.Notifications} 下的TileUpdateManager类提供的 \mintinline{csharp}{CreateTileUpdaterForApplication()} 静态方法,可以获取当前应用磁贴的“更新器(Updater)”。使用这个TileUpdater实例的Update()方法可以依照参数中提供的TileNotification对象来更新磁贴。

TileNotification对象的本质是一段XAML,但是由于微软提供了用于构建磁贴的社区工具包,我们再也不用自己去手动写这种XAML了。引用下面的Nuget包,你就能用很直观的C\#代码来构建一个自适应磁贴。

这里不赘述“自适应磁贴”具体构造方式,可以参见:\href{https://docs.microsoft.com/zh-cn/windows/uwp/controls-and-patterns/tiles-and-notifications-create-adaptive-tiles}{https://docs.microsoft.com/zh-cn/windows/uwp/controls-and-patterns/tiles-and-notifications-create-adaptive-tiles}

\section{使用跳转列表}

对于桌面端和正在构建中的Windows 10 CShell来说,右键菜单中的跳转列表也是UWP的特色之一。正确使用跳转列表能使用户快速访问他们需要的功能、快速抵达他们想到的位置或者快速恢复最近还没做完的工作

跳转列表只被版本号高于Build 10586的Windows 10支持,其核心API位于\mintinline{csharp}{Windows.UI.StartScreen} 名称空间下。使用JumpList类的\mintinline{csharp}{LoadCurrentAsync()}静态方法来取得当前应用的跳转列表实例,对其Items属性(对应跳转列表的各项的集合)作出修改后,调用这个实例的\mintinline{csharp}{SaveAsync()}方法来更新跳转列表。

值得一提的是,Items属性的集合成员类型是JumpListItem,这个类型有几个重要成员,它们分别是:Arguments(对应后面提到的、App的OnLaunch事件中传递的参数值),DisplayName(显示名称),GroupName(所在组的名称)和Logo(一个URI,指明该项的Logo位置)。

响应跳转列表的操作的具体方法是在App的OnLaunch事件处理程序中编写代码。凡是通过跳转列表启动的App,该列表项的Arguments属性就会被传递到这个事件的参数中,以此可以确定用户选择了跳转列表的哪一项。可以参阅此处:
	
\end{document}