\chapter{数据库的使用}
	
	\section{使用SQLite}
	SQLite是开源的独立数据库,其优势在于可以完全独立工作,不需要太多设置。自Windows 10 Anniversary Update (Build 14393)开始,Windows 10 SDK内置了SQLite。
	
	在开发时,出于以下几方面的原因,建议直接采用系统内置的SQLite SDK:
	\begin{itemize}
		\item 可以减小程序的安装包的大小。
		\item 可以加快程序的启动时间,因为SDK版的SQLite很有可能已经在内存中了。
		\item 无需关心SQLite的更新,因为Windows系统会完成该工作。
	\end{itemize}
	
	关于SQLite更多的知识,可以参考:	\url{http://sqlite.org/}
	\subsection{通过\Cpp 使用SQLite}
	在UWP开发的过程中,通过\Cpp 使用SQLite,目前由三种方法(\href{https://docs.microsoft.com/en-us/windows/uwp/data-access/sqlite-databases}{针对于使用\Cpp 的程序,非我的重点!}):
	\begin{enumerate}
		\item 使用SDK中的SQLite。
		\begin{enumerate}
		\item 首先需要引入库:
\begin{minted}{c++}
#include <winsqlite/winsqlite3.h>
\end{minted}
\item 设置项目链接到“winsqlite3.lib”。
\end{enumerate}	
\item 通过App Package来包含SQLite。
\item 在Visual Studio中直接编译SQLite源代码。
	\end{enumerate}

\subsection{通过\csharp 使用SQLite}
\subsubsection{安装SQLite的\csharp 包装器}
SQLite采用C语言开发,在通过\csharp 使用SQLite时需要安装\csharp 包装器。推荐使用“\uline{\textbf{Microsoft.Data.Sqlite}}”:

\begin{enumerate}
	\item 在“Solution Explorer”中右键单击“Reference”,选择“Manage NuGet Packages”。
	\item 在“Installed”选项卡中,确保“Microsoft.NETCore.UniversalWindowsPlatform ”的版本大于5.2.2。
	\item 在“Browse”选项卡中,搜索“Microsoft.Data.Sqlite”,点击“Install”。
\end{enumerate}

当然,更简单的安装方法时:
\begin{minted}[linenos=true,frame=single, bgcolor=bg]{csharp}
Install-Package Microsoft.Data.SQLite
\end{minted}

\subsubsection{引入SQLite命名空间}

在需要使用SQLite的\csharp 代码中需要引入命名空间:
\begin{minted}[linenos=true,frame=single, bgcolor=bg]{csharp}
using Microsoft.Data.Sqlite;
using Microsoft.Data.Sqlite.Internal;
\end{minted}

在App的Constructor中,使用SQLite之前,需要使用:
\begin{minted}[linenos=true,frame=single, bgcolor=bg]{csharp}
SqliteEngine.UseWinSqlite3();
\end{minted}
来确保程序使用SDK版的SQLite。

\subsubsection{创建SQLite数据库}

示例代码如下:

\begin{minted}[linenos=true,frame=single, bgcolor=bg]{csharp}
string conn_string = "Filename=sqliteSample.db"
string create_table_sql = @"
    CREATE TABLE IF NOT EXISTS book (
        id    INTEGER PRIMARY KEY AUTOINCREMENT,
        title TEXT    NOT NULL,
        isbn  TEXT    NOT NULL UNIQUE);
    ";
using (SqliteConnection db_conn = new SqliteConnection(conn_string))
{
    db.Open();
    SqliteCommand createTable = new SqliteCommand(create_table_sql, db_conn);
    try
    {
        createTable.ExecuteReader();
    }
    catch (SqliteException e)
    {
        //Do nothing
    }
    db.Close()
}
\end{minted}

\subsubsection{插入数据}

\begin{minted}[showspaces=false]{csharp}
string conn_string = "Filename=sqliteSample.db"
string book_title = "Molecular Cloning: A Laboratory Manual"
string book_isbn = "978-1-936113-42-2"

using (SqliteConnection db_conn = new SqliteConnection(conn_string))
{
    db.Open();
    SqliteCommand insert_cmd = new SqliteCommand(
        @"INSERT INTO TABLE book (title, isbn) VALUES (@title, @isbn)",
        db_conn);
    insert_cmd.Parameters.AddWithValue("@title", book_title);
    insert_cmd.Parameters.AddWithValue("@isbn", book_isbn);
    try
    {
        insert_cmd.ExecuteReader();
    }
    catch (SqliteException e)
    {
        //Do nothing
    }
    db.Close()
}
\end{minted}

\subsubsection{查询数据}

\begin{minted}[showspaces=false]{csharp}
public class Book {
    public int Id;
    public string Title;
    public string Isbn;
}

List<Book> books = new List<Book>();

string conn_string = "Filename=sqliteSample.db"
string book_title = "Molecular Cloning: A Laboratory Manual"

using (SqliteConnection db_conn = new SqliteConnection(conn_string))
{
    db.Open();
    SqliteCommand select_cmd = new SqliteCommand(
        @"SELECT id, title, isbn FROM book WHERE title = @title",
        db_conn);
    select_cmd.Parameters.AddWithValue("@title", book_title);
    SqliteDataReader query;
    try
    {
        query = select_cmd.ExecuteReader();
    }
    catch (SqliteException e)
    {
        //Do nothing
    }
    while (query.Read())
    {
        books.Add(new Book(
            Id = query.GetInt(0),
            Title = query.GetString(1),
            Isbn = query.GetString(2)
        ));
    }
    db.Close()
}
\end{minted}




\section{Entity Framework Core}
\url{https://docs.microsoft.com/en-us/ef/core/get-started/uwp/getting-started}


\section{MySQL}