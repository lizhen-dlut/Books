\documentclass[11pt,a4paper,oneside]{ctexbook}

\usepackage{amsmath}
\usepackage{amssymb}
\usepackage{array}
\usepackage{booktabs}
\usepackage{calc}
\usepackage{caption}
\usepackage{changepage}

\usepackage{ctex}
\usepackage{ctexcap}
\usepackage{datetime}
% \usepackage{dingbat}

\usepackage{eurosym}
\usepackage{fancyhdr}

\usepackage{fontspec}
\usepackage{geometry}

\usepackage{glossaries}
\makeglossaries

\usepackage{graphicx}
\usepackage{graphpap}

\usepackage{ifthen}

\usepackage{listings}
\usepackage{longtable}

\usepackage{makeidx}
\makeindex

\usepackage{mflogo}
\usepackage{minted}

\usepackage{multicol}

\usepackage{paralist}
\usepackage{pdflscape}

\usepackage{svg}
\usepackage{syntonly}
\usepackage{tabularx}
\usepackage{textcomp}

\usepackage{tracefnt}
\usepackage{ulem}
\usepackage{verbatim}

\usepackage{hyperref}


\bibliographystyle{plain}

\usepackage{minted}

\newcommand{\doctitle}{R语言基础学习}
\newcommand{\docauthor}{Roger Young}

\hypersetup{
	hidelinks,
	pdftitle=\doctitle,
	pdfauthor=\docauthor,
	pdfsubject={学习笔记, R},
	pdfkeywords={学习笔记, R},
	pdfstartview=FitH
}

\title{\doctitle}
\author{\docauthor}

\begin{document}
	\frontmatter
	
	\maketitle
	
	\include{./chapters/preface}
	
	\tableofcontents
	
	% 正文部分
	\mainmatter
	
	\begin{itemize}
		\item R语言开发环境设置
		\item R语言基本语法
		\item R的帮助系统
		\item 基本数据类型及其操作
		\item 内置函数
		\item 面向对象编程
	\end{itemize}

	\chapter{R语言简介}
	\section{R编程环境的安装}
	
	\section{R的帮助系统}
	扎实地掌握R的帮助系统可以在很大程度上简化R使用者的编程工作。R的内置帮助系统提供了很多函数,来协助用户了解R函数的实现细节、使用方法、参考文献以及使用示例等。同时R也提供函数,协助用户了解R包的细节。
	
	\subsection{R函数的帮助系统}
	
	R中的一切基本上都是基于函数的。我们首先来看一下R有关函数的帮助系统。毋庸置疑,最重要的函数当然是\mintinline{R}|help()| 函数了。
	
\begin{minted}{R}
help(topic, package = NULL, lib.loc = NULL,
	verbose = getOption("verbose"),
	try.all.packages = getOption("help.try.all.packages"),
	help_type = getOption("help_type")
)
\end{minted}

	
	
	
	\subsection{R包的帮助系统}
	
	
	
	
	\appendix
	
	\backmatter
	
	\include{./chapters/prologue}
	
	
	% 利用BibTex工具生成参考文献
	\bibliography{./chapters/references}
	
	% 利用makeindex工具生成索引
	\printindex
\end{document}

