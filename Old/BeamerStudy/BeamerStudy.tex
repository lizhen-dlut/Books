\documentclass[11pt]{beamer}
\usepackage{ctex}

\usepackage{amsmath}
\usepackage{amsfonts}
\usepackage{amssymb}
\usepackage{graphicx}

\usetheme{CambridgeUS}

\begin{document}
	\author{张洋}
	\title{``乐孕安''优生优育产品}
	\subtitle{染色体解析:助力新一代试管婴儿技术}
	\logo{}
	\institute{乐土精准医疗}
	\date{\today}
	\subject{产品介绍}
	%\setbeamercovered{transparent}
	%\setbeamertemplate{navigation symbols}{}

\begin{frame}
\titlepage
\end{frame}

\begin{frame}
\frametitle{目\qquad 录}
\tableofcontents
\end{frame}

\section{产品概览}

\begin{frame}
\frametitle{产品概览}
本产品包括两个阶段

\begin{enumerate}
	\item 查因
	
	通过高精度的全基因组测序,大数据的分析,检测样本的染色体异常情况,并给出异常断点的验证策略,零假阳性。
	
	最终给出临床表征的遗传学解释。
	
	\item 助孕
	
	新一代试管婴儿技术,充分利用第一阶段的检测结果“PCR引物对”,完美阻断亲代结构异常向子代遗传。再配合PGS剔除自发突变产生的不良胚胎。
	
	最终提高植入成功率,降低出生缺陷,给家庭一个健康的宝宝。
\end{enumerate}

\end{frame}

\section{查因}

\begin{frame}{``乐孕安''第一阶段}{查因}
\begin{definition}
	A \alert{prime number} is a number that has exactly two divisors.
\end{definition}
\begin{example}
	\begin{itemize}
		\item 2 is prime (two divisors: 1 and 2).
		\pause
		\item 3 is prime (two divisors: 1 and 3).
		\pause
		\item 4 is not prime (\alert{three} divisors: 1, 2, and 4).
	\end{itemize}
\end{example}
\end{frame}

\begin{frame}
\frametitle{There Is No Largest Prime Number}
\framesubtitle{The proof uses \textit{reductio ad absurdum}.}
\begin{theorem}
	There is no largest prime number.
\end{theorem}
\begin{proof}
	\begin{enumerate}
		\item<1-> Suppose $p$ were the largest prime number.
		\item<2-> Let $q$ be the product of the first $p$ numbers.
		\item<3-> Then $q + 1$ is not divisible by any of them.
		\item<1-> But $q + 1$ is greater than $1$, thus divisible by some prime
		number not in the first $p$ numbers.\qedhere
	\end{enumerate}
\end{proof}
\uncover<4->{The proof used \textit{reductio ad absurdum}.}
\end{frame}

\newtheorem{answeredquestions}[theorem]{Answered Questions}
\newtheorem{openquestions}[theorem]{Open Questions}

\begin{frame}
\frametitle{What’s Still To Do?}
\begin{block}{Answered Questions}
	How many primes are there?
\end{block}
\begin{block}{Open Questions}
	Is every even number the sum of two primes?
\end{block}
\end{frame}


\begin{frame}
\frametitle{What’s Still To Do?}
\begin{itemize}
	\item Answered Questions
	\begin{itemize}
		\item How many primes are there?
	\end{itemize}
	\item Open Questions
	\begin{itemize}
		\item Is every even number the sum of two primes?
	\end{itemize}
\end{itemize}
\end{frame}



\begin{frame}
\frametitle{What’s Still To Do?}
\begin{columns}
	\column{.4\textwidth}
	\begin{block}{Answered Questions}
		How many primes are there?
	\end{block}
	\column{.4\textwidth}
	\begin{block}{Open Questions}
		Is every even number the sum of two primes?\cite{Goldbach1742}
	\end{block}
\end{columns}
\end{frame}


\begin{thebibliography}{10}
	\bibitem{Goldbach1742}[Goldbach, 1742]
	Christian Goldbach.
	\newblock A problem we should try to solve before the ISPN ’43 deadline,
	\newblock \emph{Letter to Leonhard Euler}, 1742.
\end{thebibliography}

\begin{frame}[fragile]
\frametitle{An Algorithm For Finding Prime Numbers.}
\begin{verbatim}
int main (void)
{
std::vector<bool> is_prime (100, true);
for (int i = 2; i < 100; i++)
if (is_prime[i])
{
std::cout << i << " ";
for (int j = i; j < 100; is_prime [j] = false, j+=i);
}
return 0;
}
\end{verbatim}
\begin{uncoverenv}<2>
	Note the use of \verb|std::|.
\end{uncoverenv}
\end{frame}

\begin{frame}[fragile]
\frametitle{An Algorithm For Finding Primes Numbers.}
\begin{semiverbatim}
	\uncover<1->{\alert<0>{int main (void)}}
	\uncover<1->{\alert<0>{\{}}
	\uncover<1->{\alert<1>{ \alert<4>{std::}vector<bool> is_prime (100, true);}}
	\uncover<1->{\alert<1>{ for (int i = 2; i < 100; i++)}}
	\uncover<2->{\alert<2>{ if (is_prime[i])}}
	\uncover<2->{\alert<0>{ \{}}
	\uncover<3->{\alert<3>{ \alert<4>{std::}cout << i << " ";}}
	\uncover<3->{\alert<3>{ for (int j = i; j < 100;}}
	\uncover<3->{\alert<3>{ is_prime [j] = false, j+=i);}}
	\uncover<2->{\alert<0>{ \}}}
	\uncover<1->{\alert<0>{ return 0;}}
	\uncover<1->{\alert<0>{\}}}
\end{semiverbatim}
\visible<4->{Note the use of \alert{\texttt{std::}}.}
\end{frame}

\end{document}