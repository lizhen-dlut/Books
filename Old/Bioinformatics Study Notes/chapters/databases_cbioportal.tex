
\chapter{cBioPortal数据库简介}
cBioPortal数据库主要针对癌症基因组学(Cancer Genomics)研究,其提供了大规模癌症基因组的可视化、分析和下载功能。其官方网址为:
\centerline{\href{http://www.cbioportal.org/index.do}{http://www.cbioportal.org/index.do}}

\section{API简介}
通过cBioPortal提供的CGDS网络服务(Cancer Genomic Data Server web service),可以通过编程的方式快速获取cBioPortal所有的基因组学数据。CGDS网络服务是基于REST的,其返回的数据是Tab键分隔的文本格式或者XML格式。可以选择的编程语言包括:
\begin{itemize}
	\item Python
	\item R
	\item Perl
	\item Java
	\item MatLab
\end{itemize}

所有的请求通过\href{http://www.cbioportal.org/webservice.do}{http://www.cbioportal.org/webservice.do}提交,请求时需要附上一些必要的参数:
\begin{itemize}
\item cmd:需要执行的操作,可选值包括:
\begin{itemize}
	\item getTypesOfCancer:\href{http://www.cbioportal.org/webservice.do?cmd=getTypesOfCancer}{获取癌症类型。}
	\item getNetwork:\href{http://www.cbioportal.org/webservice.do?cmd=getNetwork}{获取。}
	\item getCancerStudies:\href{http://www.cbioportal.org/webservice.do?cmd=getCancerStudies}{获取癌症研究。}
	\item getGeneticProfiles:\href{http://www.cbioportal.org/webservice.do?cmd=getGeneticProfiles&cancer_study_id=msk_impact_2017}{获取特定癌症研究项目的Genetic Profile信息。}
	\item getProfileData:\href{http://www.cbioportal.org/webservice.do?cmd=getProfileData&case_set_id=msk_impact_2017_all&genetic_profile_id=msk_impact_2017_mutations&gene_list=p53,kras}{获取一个或多个基因的genomic profile数据。}
	\item getCaseLists:\href{http://www.cbioportal.org/webservice.do?cmd=getCaseLists&cancer_study_id=msk_impact_2017}{获取。}
	\item getClinicalData:\href{http://www.cbioportal.org/webservice.do?cmd=getClinicalData&case_set_id=msk_impact_2017_all}{获取样本的临床信息。}
	\item getMutationData:\href{http://www.cbioportal.org/webservice.do?cmd=getMutationData&genetic_profile_id=msk_impact_2017_mutations&gene_list=p53,kras}{获取基因的突变信息。}
\end{itemize}
\item 其他的一些可选参数,该参数跟所执行的cmd相关。
\end{itemize}
比如,我们可以通过以下链接获取cBioPortal中所有的有关癌症的研究项目:
\centerline{\href{http://www.cbioportal.org/webservice.do?cmd=getCancerStudies}{http://www.cbioportal.org/webservice.do?cmd=getCancerStudies}}

对于不同的cmd,其所对应的可选参数也不同。

\subsection{getTypesOfCancer命令}\label{subsec:getTypesOfCancer}
getTypesOfCancer命令用于获取服务器中存储的癌症列表。在调用该命令时,不需要可选参数。其返回的数据包括两列:
\begin{center}
\begin{tabular}{|c|l|}
\hline
$ type\_of\_cancer\_id $ & cBioPortal中唯一表示该癌症的编号。\\
\hline
$ name $ & 癌症名称 \\
\hline
\end{tabular}
\end{center}


\subsection{getCancerStudies命令}\label{subsec:getCancerStudies}
getCancerStudies命令可以用来获取服务器上存储的有关癌症的研究项目的基础数据。在调用该命令时,不需要可选参数。其返回的数据包括三列:
\begin{center}
\begin{tabular}{|c|l|l|}
	\hline
	$ cancer\_study\_id $ & cBioPortal中唯一表示该癌症研究项目的编号。\\
	\hline
	$ name $ & 研究项目的名称 \\
	\hline
	$ description $ & 有关该研究项目的简单说明 \\ 
	\hline
\end{tabular}
\end{center}

\subsection{getGeneticProfiles命令}\label{subsec:getGeneticProfiles}
getGeneticProfiles命令用于获取某个癌症研究项目的所有元数据,包括变异信息、拷贝数信息等。在调用该命令时需要提供一个可选参数:
\begin{center}
	\begin{tabular}{|c|l|}
		\hline
		$ cancer\_study\_id $ & 癌症研究项目的编号。\\
		\hline
	\end{tabular}
\end{center}
其返回的数据包括六列:
\begin{center}
\begin{tabular}{|c|l|l|}
	\hline
	$ genetic\_profile\_id $ & cBioPortal中唯一表示该genetic profile的编号。\\
	\hline
	$ genetic\_profile\_name $ & genetic profile的名称 \\
	\hline
	$ genetic\_profile\_description $ & genetic profile的简单说明 \\ 
	\hline
	$ cancer\_study\_id $ & 癌症研究项目的ID \\ 
	\hline
	$ genetic\_alteration\_type $ & 有关该研究项目的简单说明 \\ 
	\hline
	$ show\_profile\_in\_analysis\_tab $ & 有关该研究项目的简单说明 \\ 
	\hline
\end{tabular}
\end{center}
\subsubsection{举例}
通过链接:

\centerline
{\href{http://www.cbioportal.org/webservice.do?cmd=getGeneticProfiles\&cancer\_study\_id=msk\_impact\_2017}{http://www.cbioportal.org/webservice.do?cmd=getGeneticProfiles\&cancer\_study\_id=msk\_impact\_2017}}可以获取癌症研究项目“MSK-IMPACT”的Genetic Profiles。

\subsection{getCaseLists命令}\label{subsec:getCaseLists}
getCaseLists命令可以返回特定癌症研究项目中的样本信息。在调用该命令时需要提供一个必选参数:
\begin{center}
	\begin{tabular}{|c|l|}
		\hline
		$ cancer\_study\_id $ & 癌症研究项目的编号。\\
		\hline
	\end{tabular}
\end{center}

其返回信息包括五列:
\begin{center}
\begin{tabular}{|c|l|}
	\hline
	$ case\_list\_id $ & \\
	\hline
	$ case\_list\_name $ & \\
	\hline
	$ case\_list\_description $& \\
	\hline
	$ cancer\_study\_id $& \\
	\hline
	$ case\_ids $ &  \\
	\hline
\end{tabular}
\end{center}

\subsection{getProfileData命令}\label{subsec:getProfileData}
	getProfileData命令可以返回一个或多个基因的genomic profile数据。在调用该命令时需要提供三个必选参数:
\begin{center}
	\begin{tabular}{|c|l|}
		\hline
		$ case\_set\_id $ & 由\hyperref[subsec:getCaseLists]{getCaseLists}命令返回的Case Set的ID。\\
		\hline
		$ genetic\_profile\_id $ & 由\hyperref[subsec:getGeneticProfiles]{getGeneticProfiles}命令返回的Genetic Profile。\\
		\hline
		$ gene\_list $ & 基因列表,多个基因之间以英文逗号“,”分隔。\\
		\hline
	\end{tabular}
\end{center}

\subsection{getMutationData命令}\label{subsec:getMutationData}
getMutationData命令可以获取一些额外的信息,比如变异的注释信息。使用该命令时,需要添加以下参数:
\begin{center}
\begin{tabular}{|c|c|l|}
	\hline
	$ genetic\_profile\_id $ & 必选 & 由\hyperref[subsec:getGeneticProfiles]{getGeneticProfiles}命令返回的$ genetic\_profile\_id $。\\
	\hline
	$ case\_set\_id $ & 可选 & 由\hyperref[subsec:getGeneticProfiles]{getGeneticProfiles}命令返回的$ case\_list\_id $。 \\
	\hline
	$ gene\_list $ & 必选 & 以逗号分隔的基因列表(HUGO基因名称或Entrez的基因ID)。\\
	\hline
\end{tabular}
\end{center}

\subsection{getClinicalData 命令}\label{subsec:getClinicalData }
getClinicalData用于获取癌症研究项目中所涉及的样本的基本临床信息。可以获取到的基本信息包括样本编号(CASE\_ID)癌症类型(CANCER\_TYPE)、癌症稍微详细的描述(CANCER\_TYPE\_DETAILED)、所使用的DNA量(DNA\_INPUT)、癌症的转移位置(METASTATIC\_SITE)、Oncotree编号(ONCOTREE\_CODE)、原发灶(PRIMARY\_SITE)、样本类型(SAMPLE\_CLASS)、患者性别(SEX)等。在调用该命令时需要提供一个可选参数:
\begin{center}
\begin{tabular}{|c|l|}
	\hline
	$ case\_set\_id $ & 样本集编号 \\
	\hline
\end{tabular}
\end{center}


\section{cBioPortal使用流程}
cBioPortal数据下载流程:
\begin{enumerate}
	\item 通过“\hyperref[subsec:getCancerStudies]{getCancerStudies}”命令,获取癌症研究项目的基本信息,主要获取$ cancer\_study\_id $值;
	\item 以$ cancer\_study\_id $值为参数,调用“\hyperref[subsec:getCaseLists]{getCaseLists}”命令,获取癌症项目研究的样本集合的 $ case\_list\_id $值。
	\item 以$ case\_list\_id $值作为参数,调用“\hyperref[subsec:getClinicalData]{getClinicalData}”命令,获取癌症项目研究样本的临床信息。
	\item 以$ cancer\_study\_id $值为参数,调用“\hyperref[subsec:getGeneticProfiles]{getGeneticProfiles}”命令,获取癌症项目研究的Genetic Profile的列表。
	\item 以$ genetic\_profile\_id $为参数,$ case\_list\_id $值作为$ case\_set\_id $的参数
\end{enumerate}