\documentclass{ctexart}
\usepackage[left=1cm,right=1cm,bottom=1.5cm]{geometry}
\usepackage{amsmath}
\usepackage{amssymb} 
\usepackage{array}

\usepackage{enumerate}
\usepackage{enumitem}
\usepackage{fancyhdr}
\usepackage{fontenc} 
\usepackage{graphicx,calc}

\usepackage{ifthen}
\usepackage{inputenc} 
\usepackage{lastpage}
\usepackage{lipsum}
\usepackage{multicol}
\usepackage{multirow}
\usepackage{tikz}
\usetikzlibrary{calc}

\usepackage{hyperref}

\newcommand{\productName}{\uline{``乐孕安''优生优育产品}}
\newcommand{\DocumentTitle}{\kaishu{\huge 乐土精准医疗 \\ \Large 2017年7月18日\productName 培训测试题}}
\newcommand{\doumentAuthor}{张洋}

\hypersetup{
	pdftitle= {\DocumentTitle} ,
	pdfauthor= {\doumentAuthor}, 
	pdfkeywords = {乐孕安, 测试题}, 
	pdfcreator={\doumentAuthor},
	pdfproducer={\doumentAuthor},
	pdfstartview={FitH},
	hidelinks
} 


\newlength{\lengthOptionA}
\newlength{\lengthOptionB}
\newlength{\lengthOptionC}
\newlength{\lengthOptionD}
\newlength{\lhalf}
\newlength{\lquarter}
\newlength{\lmax}
\setlength{\lhalf}{0.5\linewidth}
\setlength{\lquarter}{0.25\linewidth}

\newcommand{\choiceOptions}[4]
{
	\hfill \\[.5pt]%
	\settowidth{\lengthOptionA}{A.~#1~~~}
	\settowidth{\lengthOptionB}{B.~#2~~~}
	\settowidth{\lengthOptionC}{C.~#3~~~}
	\settowidth{\lengthOptionD}{D.~#4~~~}
	
	\ifthenelse{\lengthtest{\lengthOptionA>\lengthOptionB}}
		{\setlength{\lmax}{\lengthOptionA}}
		{\setlength{\lmax}{\lengthOptionB}}
	\ifthenelse{\lengthtest{\lengthOptionC>\lmax}}
		{\setlength{\lmax}{\lengthOptionC}}{}
	\ifthenelse{\lengthtest{\lengthOptionD>\lmax}}
		{\setlength{\lmax}{\lengthOptionD}}{}
	
	\ifthenelse{\lengthtest{\lmax>\lhalf}}
		{\noindent{}A.~#1 \\ B.~#2 \\ C.~#3 \\ D.~#4}
		{
			\ifthenelse{\lengthtest{\lmax>\lquarter}}
				{\noindent\makebox[\lhalf][l]{A.~#1~~~}%
					\makebox[\lhalf][l]{B.~#2~~~}\\\noindent%
					\makebox[\lhalf][l]{C.~#3~~~}%
					\makebox[\lhalf][l]{D.~#4~~~}
				}%
				{\noindent\makebox[\lquarter][l]{A.~#1~~~}%
					\makebox[\lquarter][l]{B.~#2~~~}%
					\makebox[\lquarter][l]{C.~#3~~~}%
					\makebox[\lquarter][l]{D.~#4~~~}
				}
			}
}

\newcommand{\zongfenlana}
{%☆总分栏
	\begin{center}
		\setlength{\tabcolsep}{4mm}%列宽
		\renewcommand{\arraystretch}{1.5}%行高
		\begin{tabular}{*{12}{|c}|}%格式化列样式
			\hline
			题号&一&二&三&四&总分&总分人&复核人\\ \hline
			分数&&&&&&&\\
			\hline
		\end{tabular}
	\end{center}
\vskip-3mm
}

\newcommand{\examineeInfo}
{
	\ifthenelse{\isodd{\value{page}}}
	{
		\begin{tikzpicture}[remember picture,overlay]
		\path 	(current page.south east) coordinate (a0);
		\path 	(current page.north) coordinate (a0);
		\draw 	(a0)[shift={(0,-2)}] node (a1) [rotate=0,fill=gray!0,minimum height=1cm,minimum width=1cm]
		{\kaishu{}\zihao{4}
			区域\raisebox{-2pt}{\rule{35mm}{0.4pt}}%
			部门\raisebox{-2pt}{\rule{35mm}{0.4pt}}%
			姓名\raisebox{-2pt}{\rule{35mm}{0.4pt}}%
			工号\raisebox{-2pt}{\rule{35mm}{0.4pt}}}
		(a1)[shift={(0,-1.5)}] node [rotate=0,fill=gray!0,minimum height=1cm,minimum width=1cm]
		{\kaishu{}\zihao{4}...........................................%
			\raisebox{-0.6ex}{装}............................................%
			\raisebox{-0.6ex}{订}............................................%
			\raisebox{-0.6ex}{线}...........................................};
		\end{tikzpicture}
	}
	{
		\begin{tikzpicture}[remember picture,overlay]
		\path 	(current page.south east) coordinate (a0);
		\path 	(current page.north) coordinate (a0);
		\draw 	(a0)[shift={(0,-2)}] node (a1) [rotate=0,fill=gray!0,minimum height=1cm,minimum width=1cm]
		{\kaishu{}\zihao{4}
			区域\raisebox{-2pt}{\rule{35mm}{0.4pt}}%
			部门\raisebox{-2pt}{\rule{35mm}{0.4pt}}%
			姓名\raisebox{-2pt}{\rule{35mm}{0.4pt}}%
			工号\raisebox{-2pt}{\rule{35mm}{0.4pt}}}
		(a1)[shift={(0,-1.5)}] node [rotate=0,fill=gray!0,minimum height=1cm,minimum width=1cm]
		{\kaishu{}\zihao{4}...........................................%
			\raisebox{-0.6ex}{装}............................................%
			\raisebox{-0.6ex}{订}............................................%
			\raisebox{-0.6ex}{线}...........................................};
		\end{tikzpicture}
	}
}

\newcommand{\markArea}
{
	\setlength{\tabcolsep}{4mm}%列宽
	\renewcommand{\arraystretch}{1}%行高
	\begin{tabular}[c]{*{2}{|c}|}
		\hline
		评卷人&得分\\\hline
		&\\[2mm]
		\hline
	\end{tabular}
}

\pagestyle{fancy}

\lhead{\setlength{\unitlength}{1em}
	\begin{picture}(0,0)
	\put(0,0){\includegraphics[width=6em]{./figures/logo.pdf}}
	\end{picture}}
\chead{\textbf{\productName 培训测试题}} 
\rhead{\textbf{姓名:\TextField[name=name,width=5em,charsize=0pt,align=1]{\mbox{}}}}

\begin{document}
	\begin{Form}[action=mailto:zhangyang@cheerlandgroup.com, encoding=html]

	
	%\examineeInfo
	
	\begin{center}
		{\DocumentTitle}
	\end{center}

	\zongfenlana
	
	\begin{center}
		\begin{tabular}{p{15em}p{15em}p{15em}}
			\multicolumn{1}{c}{ \Submit[name=Submit]{\large 提\qquad 交}} & & \multicolumn{1}{c}{\Reset[name=Reset]{\large 重\qquad  置}}
		\end{tabular}
	\end{center}
	
	\newcommand{\answerField}[1]{\uline{\TextField[name=#1,charsize=0pt,width=4em,align=1]{\mbox{}}}}
	
	\newcommand{\blockAnswer}[1]{答:
		
		\TextField[name=#1,charsize=0pt, height=5cm, width=0.9\textwidth, multiline=true]{\ }}
	
	\newcommand{\glossaryDefinitionAnswer}[1]{
		\TextField[name=#1,width=0.9\textwidth,charsize=0pt,multiline=true]{}}
	
	\newcommand{\fillBlankField}[1]{\TextField[name=#1,width=0.3\textwidth,charsize=0pt]{}}
	
	\begin{enumerate}
		\item[\kaishu{一}]{\makebox[2mm][r]{、}\kaishu{}\markArea{ }选择题(单选、多选)}
		
		\item \productName 所采用的方法是\answerField{1} 。
			\choiceOptions{高深度全集因组测序}{高深度全外显子组测序}{低深度全基因组测序}{低深度全外显子组测序}
			
		\item \productName 测序深度约为\answerField{2} 。
			\choiceOptions{$ 1\times $}{$ 5\times $}{$ 15\times $}{$ 30\times $}
			
		\item \productName 测序数据量至少为\answerField{3} 。
			\choiceOptions{5 Gb}{15 Gb}{45 Gb}{90 Gb}
			
		\item \productName 所针对的人群为\answerField{4} 。
			\choiceOptions{不明原因不孕不育的夫妻}{有反复流产史的夫妻}{出生缺陷儿童家庭}{激素水平异常的夫妻}
			
		\item \productName 目前拥有的营销资料包括\answerField{5} 。
			\choiceOptions{产品宣传册}{示例报告}{演讲PPT}{宣传用易拉宝等}
			
		\item \productName 在查因阶段有下列哪些技术有可比性\answerField{6} 。
			\choiceOptions{核型分析}{FISH}{Array-CGH}{PCR}
		
		\item \productName 在查因阶段可检测(报告)的最小片段长度为\answerField{7} 。
			\choiceOptions{1 Kb}{10 Kb}{100 Kb}{1000 Kb(1M)}
			
		\item \productName 在查因阶段断点精度为\answerField{8} 。
			\choiceOptions{1 bp}{10 bp}{100 bp}{1 kb}
			
		\item \productName 在查因阶段,所采用的测序类型是\answerField{9} 。
			\choiceOptions{Mate Pair Sequencing}{Paired-End Sequencing}{Single-Read Sequencing}{以上都不对}
			
		\item 客户于2017年6月30日(周五)寄出两个需要进行\productName 检测的样本,我司于2017年7月3日(周一)收到样本,实验部门于2017年7月5日(周三)完成质检,则我司必须在\answerField{10} 之前向客户提交``查因''阶段的报告。
			\choiceOptions{2017年8月24日}{2017年8月25日}{2017年8月29日}{2017年9月5日}
			
		\item 相较于市面上\productName 的类似产品,\productName 的优势包括\answerField{11} 。
			\choiceOptions{检测精度高}{可以检测罗氏易位}{更多的关注外显子区域}{可以检测倒位、易位}
			
		
		\item[\kaishu{二}]{\makebox[2mm][r]{、}\kaishu{\markArea{}填空题(请根据实际情况作答,空格无需全部填写)}}
		
		\item 客户进行\productName 查因阶段检测后可以获得的结果包括:
		
		\uline{\fillBlankField{b11}}、\uline{\fillBlankField{b12}}、\uline{\fillBlankField{b13}}。
		
		\item 市面上现有的\productName 类似产品包括有(公司,产品名):
				
		\uline{\fillBlankField{b21}}、\uline{\fillBlankField{b22}}、\uline{\fillBlankField{b23}}。
		
		\item 市面上现有的\productName 类似产品的报价为(产品名,报价):
		
		\uline{\fillBlankField{b31}}、\uline{\fillBlankField{b32}}、\uline{\fillBlankField{b33}}。
		
		
		
		
		\item[\kaishu{三}]{\makebox[2mm][r]{、}\kaishu{}\markArea{名词解释(每小题3分)}}
		
		\item 非整倍体:
		
		\glossaryDefinitionAnswer{aneuploid}
		
		\item 整倍体:
		
		\glossaryDefinitionAnswer{polyploid}
		
		\item 染色体非平衡易位:
		
		\glossaryDefinitionAnswer{unbalancedTranslocation}
		
		\item 染色体平衡易位:
		
		\glossaryDefinitionAnswer{balancedTranslocation}
		
		\item 罗氏易位(罗伯逊易位):
		
		\glossaryDefinitionAnswer{RobertsonianTranslocation }
		
		\newpage
		\item[\kaishu{四}]{\makebox[2mm][r]{、}\kaishu{\markArea{}问答题(每小题10分)}}
		
		\item 在营销过程中,关于\productName 可能遇到的问题有哪些?
		
		\blockAnswer{wendaone}
		
		\item 有关\productName ,在营销过程中,仍需要需要产品团队提供的支持有哪些?
		
		\blockAnswer{wendatwo}
		
		\item 您认为\productName ,需要明确、改进、完善的地方有哪些?
		
		\blockAnswer{wendathree}
		
	\end{enumerate}
\end{Form}
\end{document}