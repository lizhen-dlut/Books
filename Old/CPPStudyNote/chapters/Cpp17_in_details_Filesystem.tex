\section{C++17 in details: Filesystem}
\url{http://www.bfilipek.com/2017/08/cpp17-details-filesystem.html}

Although C++ is an old programming language, its Standard Library misses a few basic things. Features that Java or .NET had for years were/are not available in STL. With C++17 there’s a nice improvement: for example, we now have the standard filesystem! 
Traversing a path, even recursively is so simple now!

Intro
For the last five episodes/articles, I’ve covered most of the language features. Now it’s time for the Standard Library. I’ve planned three posts on that: Filesystem, Parallel STL and Concurrency, Utils.
Maybe I was a bit harsh in the introduction. Although the Standard Library lacks some important features, you could always use Boost with its thousands of sub-libraries and do the work. The C++ Committee and the Community decided that the Boost libraries are so important that some of the systems were merged into the Standard. For example smart pointers (although improved with the move semantics in C++11), regular expressions, and much more.
The similar story happened with the filesystem. Let’s try to understand what’s inside.
The Series
This post is the sixth in the series about C++17 features details.
The plan for the series
Fixes and deprecation
Language clarification
Templates
Attributes
Simplification
Library changes - Filesystem (today)
Library changes - Parallel STL (soon)
Library changes - Utils (soon + 1)
Wrap up, Bonus
Documents & Links
Just to recall:
First of all, if you want to dig into the standard on your own, you can read the latest draft here: 
N4659, 2017-03-21, Working Draft, Standard for Programming Language C++ - the link also appears on the isocpp.org.
And you can also grab my list of concise descriptions of all of the C++17 language features:
Download a free copy of my C++17 Cheat Sheet! 
It’s a one-page reference card, PDF.
Links:
Compiler support: C++ compiler support
The official paper with changes: P0636r0: Changes between C++14 and C++17 DIS
There’s also a talk from Bryce Lelbach: C++Now 2017: C++17 Features
My master C++17 features post: C++17 Features
Jason Turner: C++ Weekly channel, where he covered most (or even all!) of C++17 features.
OK, let’s return to our main topic: working with paths and directories!
Filesystem Overview
I think the Committee made a right choice with this feature. The filesystem library is nothing new, as it’s modeled directly over Boost filesystem, which is available since 2003 (with the version 1.30). There are only a little differences, plus some wording changes. Not to mention, all of this is also based on POSIX.
Thanks to this approach it’s easy to port the code. Moreover, there’s a good chance a lot of developers are already familiar with the library. (Hmmm… so why I am not that dev? :))
The library is located in the <filesystem> header. It uses namespace std::filesystem.
The final paper is P0218R0: Adopt the File System TS for C++17 but there are also others like P0317R1: Directory Entry Caching, PDF: P0430R2–File system library on non-POSIX-like operating systems, P0492R2… All in all, you can find the final spec in the C++17 draft: the “filesystem” section, 30.10.
We have three/four core parts:
The path object
directory_entry
Directory iterators
Plus many supportive functions 
getting information about the path
files manipulation: copy, move, create, symlinks
last write time
permissions
space/filesize
…
Compiler/Library support
Depending on the version of your compiler you might need to use std::experimental::filesystem namespace.
GCC: You have to specify -lstdc++fs when you want filesystem. Implemented in <experimental/filesystem>.
Clang should be ready with Clang 5.0 
https://libcxx.llvm.org/cxx1z_status.html 
Visual Studio: In VS 2017 (2017.2) you still have to use std::experimental namespace, it uses TS implementation. 
See the reference link and also C++17 Features In Visual Studio 2017 Version 15.3 Preview.
Hopefully by the end of the year VS 2017 will fully implement C++17 (and STL)
Examples
All the examples can be found on my Github: github.com/fenbf/articles/cpp17.
I’ve used Visual Studio 2017 Update 2.
Working with the Path object
The core part of the library is the path object. Just pass it a string of the path, and then you have access to lots of useful functions.
For example, let’s examine a path:
namespace fs = std::experimental::filesystem;

fs::path pathToShow(/* ... */);
cout << "exists() = " << fs::exists(pathToShow) << "\n"
     << "root_name() = " << pathToShow.root_name() << "\n"
     << "root_path() = " << pathToShow.root_path() << "\n"
     << "relative_path() = " << pathToShow.relative_path() << "\n"
     << "parent_path() = " << pathToShow.parent_path() << "\n"
     << "filename() = " << pathToShow.filename() << "\n"
     << "stem() = " << pathToShow.stem() << "\n"
     << "extension() = " << pathToShow.extension() << "\n";
Here’s an output for a file path like "C:\Windows\system.ini":
exists() = 1
root_name() = C:
root_path() = C:\
relative_path() = Windows\system.ini
parent_path() = C:\Windows
filename() = system.ini
stem() = system
extension() = .ini
What’s great about the above code?
It’s so simple to use! But there’s more cool stuff:
For example, if you want to iterate over all the elements of the path just write:
int i = 0;    
for (const auto& part : pathToShow)
    cout << "path part: " << i++ << " = " << part << "\n";
The output:
path part: 0 = C:
path part: 1 = \
path part: 2 = Windows
path part: 3 = system.ini
We have several things here:
the path object is implicitly convertible to std::wstring or std::string. So you can just pass a path object into any of the file stream functions.
you can initialize it from a string, const char*, etc. Also, there’s support for string_view, so if you have that object around there’s no need to convert it to string before passing to path. PDF: WG21 P0392
path has begin() and end() (so it’s a kind of a collection!) that allows iterating over every part.
What about composing a path?
We have two options: using append or operator /=, or operator +=.
append - adds a path with a directory separator.
concat - only adds the ‘string’ without any separator.
For example:
fs::path p1("C:\\temp");
p1 /= "user";
p1 /= "data";
cout << p1 << "\n";

fs::path p2("C:\\temp\\");
p2 += "user";
p2 += "data";
cout << p2 << "\n";
output:
C:\temp\user\data
C:\temp\userdata
What can we do more?
Let’s find a file size (using file_size):
uintmax_t ComputeFileSize(const fs::path& pathToCheck)
{
    if (fs::exists(pathToCheck) &&
        fs::is_regular_file(pathToCheck))
    {
        auto err = std::error_code{};
        auto filesize = fs::file_size(pathToCheck, err);
        if (filesize != static_cast<uintmax_t>(-1))
            return filesize;
    }

    return static_cast<uintmax_t>(-1);
}
Or, how to find the last modified time for a file:
auto timeEntry = fs::last_write_time(entry);
time_t cftime = chrono::system_clock::to_time_t(timeEntry);
cout << std::asctime(std::localtime(&cftime));
Isn’t that nice? :)
As an additional information, most of the functions that work on a path have two versions:
One that throws: filesystem_error
Another with error_code (system specific)
Let’s now take a bit more advanced example: how to traverse the directory tree and show its contents?
Traversing a path
We can traverse a path using two available iterators:
directory_iterator
recursive_directory_iterator - iterates recursively, but the order of the visited files/dirs is unspecified, each directory entry is visited only once.
In both iterators the directories . and .. are skipped.
Ok… show me the code:
void DisplayDirTree(const fs::path& pathToShow, int level)
{
    if (fs::exists(pathToShow) && fs::is_directory(pathToShow))
    {
        auto lead = std::string(level * 3, ' ');
        for (const auto& entry : fs::directory_iterator(pathToShow))
        {
            auto filename = entry.path().filename();
            if (fs::is_directory(entry.status()))
            {
                cout << lead << "[+] " << filename << "\n";
                DisplayDirTree(entry, level + 1);
                cout << "\n";
            }
            else if (fs::is_regular_file(entry.status()))
                DisplayFileInfo(entry, lead, filename);
            else
                cout << lead << " [?]" << filename << "\n";
        }
    }
}
The above example uses not a recursive iterator but does the recursion on its own. This is because I’d like to present the files in a nice, tree style order.
We can also start with the root call:
void DisplayDirectoryTree(const fs::path& pathToShow)
{
    DisplayDirectoryTree(pathToShow, 0);
}
The core part is:
for (auto const & entry : fs::directory_iterator(pathToShow))
The code iterates over entries, each entry contains a path object plus some additional data used during the iteration.
Not bad, right?
Of course there’s more stuff you can do with the library:
Create files, move, copy, etc.
Work on symbolic links, hard links
Check and set file flags
Count disk space usage, stats
Today I wanted to give you a general glimpse over the library. As you can see there are more potential topics for the future.
More resources
You might want to read:
Chapter 7, “Working with Files and Streams” - of Modern C++ Programming Cookbook. 
examples like: Working with filesystem paths, Creating, copying, and deleting files and directories, Removing content from a file, Checking the properties of an existing file or directory, searching.
The whole Chapter 10 “”Filesystem” from “C++17 STL Cookbook” 
examples: path normalizer, Implementing a grep-like text search tool, Implementing an automatic file renamer, Implementing a disk usage counter, statistics about file types,Implementing a tool that reduces folder size by substituting duplicates with symlinks
C++17- std::byte and std::filesystem - ModernesCpp.com
How similar are Boost filesystem and the standard C++ filesystem libraries? - Stack Overflow
Summary
I think the filesystem library is an excellent part of the C++ Standard Library. A lot of time I had to use various API to do the same tasks on different platforms. Now, I’ll be able to just use one API that will work for probably 99.9% cases.
The feature is based on Boost, so not only a lot of developers are familiar with the code/concepts, but also it’s proven to work in many existing projects.
And look at my samples: isn’t traversing a directory and working with paths so simple now? I am happy to see that everting can be achieved using std:: prefix and not some strange API :)