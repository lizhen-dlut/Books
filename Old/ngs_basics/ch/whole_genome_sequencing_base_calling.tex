\chapter{预处理和测序质量控制}

\section{Base calling}

Next-generation sequencing platforms are dramatically reducing the cost of DNA sequencing. With these technologies, bases are inferred from light intensity signals, a process commonly referred to as base-calling. Thus, understanding and improving the quality of sequence data generated using these approaches are of high interest.

\subsection{bcl2fastq Conversion Software}

\url{https://support.illumina.com/downloads/bcl2fastq_conversion_software_184.html}

\section{Read quality control}

Next-generation sequencing (NGS) technologies have been widely used in life sciences. However, several kinds of sequencing artifacts, including low-quality reads and contaminating reads, were found to be quite common in raw sequencing data, which compromise downstream analysis. Therefore, quality control (QC) is essential for raw NGS data.

\subsection{Picard}

\url{http://broadinstitute.github.io/picard/}

\subsection{FastQC}

\url{http://www.bioinformatics.babraham.ac.uk/projects/fastqc/}

\subsection{NGS QC Toolkit}

\url{http://nipgr.res.in/ngsqctoolkit.html}

\subsection{SAMtools}

\url{http://www.htslib.org/}

\subsection{FASTX-Toolkit}

\url{http://hannonlab.cshl.edu/fastx_toolkit/}

\subsection{BAMStats}





\section{Error correction}


\section{Duplicate read removal}


\section{Adapter trimming}

\subsection{Trimmomatic}
Trimmomatic是使用java编写的去除Illumina高通量测序数据接头的程序 \cite{Trimmomatic2014bioinformatics}。最新程序可以通过以下链接获取:
\url{http://www.usadellab.org/cms/index.php?page=trimmomatic}

\subsubsection{使用简介}

先按照官方说明,来两段代码吧:

\subparagraph{第一段}

\begin{minted}{bash}
java -jar trimmomatic-0.35.jar PE -phred33 
    input_forward.fq.gz input_reverse.fq.gz 
    output_forward_paired.fq.gz output_forward_unpaired.fq.gz 
    output_reverse_paired.fq.gz output_reverse_unpaired.fq.gz 
    ILLUMINACLIP:TruSeq3-PE.fa:2:30:10 
    LEADING:3 TRAILING:3 SLIDINGWINDOW:4:15 MINLEN:36
\end{minted}

这段代码,对Illumina测序平台双端测序(PE,Paired End)的数据文件input\_forward.fq.gz和input\_reverse.fq.gz进行了以下操作:

\begin{itemize}
	\item 去除测序接头(\mintinline{bash}{ILLUMINACLIP:TruSeq3-PE.fa:2:30:10})
	\item 去除开头的3个或低质量碱基(\mintinline{bash}{LEADING:3})
	\item 去除结尾的3个或低质量碱基(\mintinline{bash}{TRAILING:3})
	\item 以四个碱基为一个窗口,当平均碱基质量低于15时,去除(SLIDINGWINDOW:4:15)
	\item 去除短语36个碱基的Reads(MINLEN:36)
\end{itemize}

其输出文件为:
\begin{itemize}
	\item output\_forward\_paired.fq.gz:经清理后,input\_forward.fq.gz文件剩余的配对的Reads。
	\item output\_forward\_unpaired.fq.gz: 经清理后,input\_forward.fq.gz文件剩余的未配对的Reads。
	\item output\_reverse\_paired.fq.gz:经清理后,input\_reverse.fq.gz文件剩余的配对的Reads。
	\item output\_reverse\_unpaired.fq.gz:经清理后,input\_reverse.fq.gz文件剩余的未配对的Reads。
\end{itemize}

\subparagraph{第二段}

\begin{minted}{bash}
java -jar trimmomatic-0.35.jar SE -phred33 
    input.fq.gz 
    output.fq.gz 
    ILLUMINACLIP:TruSeq3-SE:2:30:10 
    LEADING:3 TRAILING:3 SLIDINGWINDOW:4:15 MINLEN:36
\end{minted}

这一段代码,和上一段代码完成了相同的工作,不过,其输入文件为SE测序数据(input.fq.gz)

\subsubsection{使用说明}

Trimmomatic可以用来处理Illumina双端测序和单端测序产生的FASTQ文件。其中FASTQ文件可以采用Phred-33或Phred-64碱基质量表示方法。Trimmomatic支撑以gzip和bzip2压缩的FASTQ文件。

其支持的命令包括:

\begin{longtable}{|l|p{25em}|}
	\hline
	命令 & 说明 \endhead \hline
	ILLUMINACLIP & 去除Illumina测序平台的接头 \\ \hline
	SLIDINGWINDOW & 以窗口的形式从5端扫描reads。如果平均碱基质量低于给定值,则切断Reads \\ \hline
	MAXINFO & 综合考虑Read长度和碱基质量,以使Read的评分最高 \\ \hline
	LEADING & 去掉开头的几个低于阈值的碱基 \\ \hline
	TRAILING & 去掉结尾的几个低于阈值碱基 \\ \hline
	CROP & 截取开头的指定数目的碱基 \\ \hline
	HEADCROP & 去除开头指定数目的碱基 \\ \hline
	MINLEN & 去掉小于指定长度的reads \\ \hline
	AVGQUAL & 去掉评价碱基质量小于指定值的reads \\ \hline
	TOPHRED33 & 将碱基质量表示方法转化为Phred-33 \\ \hline
	TOPHRED64 & 将碱基质量表示方法转化为Phred-64 \\ \hline
	
\end{longtable}

\begin{minted}{bash}
java -jar <path to trimmomatic jar> SE [-threads <threads>]
    [-phred33 | -phred64] [-trimlog <logFile>] 
    <input> <output> <step 1> ...
\end{minted}

\begin{minted}{bash}
java -classpath <path to trimmomatic jar>
    org.usadellab.trimmomatic.TrimmomaticSE 
    [threads <threads>] [-phred33 | -phred64] [-trimlog <logFile>] 
    <input> <output> <step 1> ... 
\end{minted}

\begin{minted}{bash}
java -jar <path to trimmomatic.jar> PE 
    [-threads <threads] [-phred33 | -phred64] 
    [-trimlog <logFile>] 
    [-basein <inputBase> | <input 1> <input 2>] 
    [-baseout <outputBase> | <unpaired output 1> 
        <paired output 2> <unpaired output 2>
    <step 1> ... 
\end{minted}

\begin{minted}{bash}
java -classpath <path to trimmomatic jar>
    org.usadellab.trimmomatic.TrimmomaticPE 
    [threads <threads>] [-phred33 | -phred64] 
    [-trimlog <logFile>] 
    [-basein <inputBase> | <input 1> <input 2>] 
    [-baseout <outputBase> | <paired output 1> <unpaired output 1>
        <paired output 2> <unpaired output 2>
    <step 1> ... 
\end{minted}

\subsection{cutadapt}

\subsection{NGS QC Toolkit}

\subsection{BCL2FASTQ Conversion Software}

\subsection{FASTX-Toolkit}

\url{http://hannonlab.cshl.edu/fastx_toolkit/}


\section{Read Clustering}


\section{k-mer counting}

\subsection{Jellyfish}


\section{Depth of Coverage}


\section{Variant recalibration}

