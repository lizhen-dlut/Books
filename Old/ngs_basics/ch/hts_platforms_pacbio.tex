\chapter{Pacific Biosciences}

Pacific Biosciences(PacBio\textregistered)公司的测序平台主要是基于单分子实时测序技术(SMRT,Single Molecular Real Time Sequencing)。现在已经商业化的测序平台包括PacBio Sequel和PacBio RS II。

\section{测序原理}



SMRT技术的基本原理是:DNA模板被聚合酶捕获后,四种不用荧光标记的dNTP随机进入检测区域与聚合酶/模板复合体结合。与模板匹配的碱基生成的化学键的时间远远长于其他碱基停留的时间。因此统计不同荧光信号存在的时间长短,即可区分与DNA模板结合的碱基。通过统计四种荧光信号与时间的关系图,即可测定DNA模板序列。

\subsection{ZMWs:Zero-Mode Waveguides}

ZMWs可是光仅仅照亮固定着一个DNA聚合酶/DNA模板复合体的小孔的底部。ZWM是一个直径只有10$ \sim $ 15nm的孔,远小于检测激光的波长(数百纳米),因此,当激光打在SMW底部时,激光无法穿过,而是在ZMW底部发生衍射,只能照亮很小的区域。DNA聚合酶以及DNA聚合酶捕获的DNA模板就被固定在这一区域。只有在这个区域内,碱基携带的荧光基团才能被激活而被检测到,大幅地降低了背景荧光干扰。每个ZWM只固定一个DNA聚合酶。当一个ZMW结合的DNA模板数目不是一时,该ZWM所产生的测序结果会在后续数据分析时被过滤掉,由此保证每个可用的ZMW都是一个单独的DNA合成体系。15万个ZMW聚集在一个芯片上,成为一个SMRT Cell。 PacBio RS II测序仪一个流程内可以同时完成8个SMRT Cell的测序,产生约3.2Gb的数据。

\subsection{phospholinked nucleotides}
SMRT测序的领一个核心技术是荧光基团标记在核苷酸3‘端的磷酸上。在DNA合成过程中,3’端的磷酸键随着DNA链的延伸被切断,标记物被弃去,减少了DNA合成的空间位阻,维持DNA链连续合成,延长了测序读长。

而在第二代测序技术中,荧光基团标记在DNA链的5‘端,在合成过程中,荧光标记物保留在DNA链上,随着DNA链的延伸会产生三维空间阻力,导致DNA链延长到一定程度后出现错读,这是限制二代测序读长的原因之一。

SMRT测序最大限度地保持了聚合酶的活性,更接近天然状态的聚合酶反应体系。在实时监控系统下,DNA链以每秒10个碱基的速度合成,从建库到测序,整个过程可在两天内完成。

\subsection{优点}

\subsubsection{超长读长}

\subsubsection{无需模板扩增}

\subsubsection{直接检测表观修饰位点}

\subsubsection{直接测转录本}

\subsubsection{较高但完全随机的测序错误}



基本原理:DNA聚合酶和模板结合,4色荧光标记4种碱基,经过Watson配对后不同的碱基加入,会发出不同光,根据光的波长与峰值可判断进入的碱基类型。这个DNA聚合酶是实现超长读长的关键之一,读长主要跟酶的活性保持有关,主要受激光对它的损伤的影响。PacBio还在不断优化聚合酶的性能,比如给聚合酶加上免受激光影响的保护基团等,进一步地提高读长,提高测序质量和通量。

和其他基本测序技术一样,在反应管中进行的是大规模平行的多分子反应,怎样在其中进行单分子反应检测?周围有大量的荧光标记的游离碱基,怎样将反应信号与周围游离碱基的强大荧光背景区别出来?

通过一个物理现象解释:ZMW(zero-mode waveguides,零模波导孔)。例如微波炉壁上可看到有很多密集的小孔。小孔直径有考究,如果直径大于微波波长,能量就会穿透面板泄露。如果孔径小于波长,能量不会辐射外部,可起保护作用。 

同理,在一个反应管(SMRTCell:单分子实时反应孔)中有许多这样的圆形纳米小孔,即ZMW(零模波导孔),外径100多纳米,比检测激光波长小(数百纳米),激光从底部打上去后不能穿透小孔进入上方溶液区,能量被限制在一个小范围(体积20X 10-21 L)里,正好足够覆盖需要检测的部分,使得信号仅来自这个小反应区域,孔外过多游离核苷酸单体依然留在黑暗中,将背景降到最低。


单个ZMW底部固定有一个结合了模板DNA的聚合酶,当加入测序反应试剂后,每个碱基配对合成后会发出相应的光并被检测。一个SMRTCell中有15万个ZMW,每个孔中有一个单分子DNA链在高速合成,如众星闪烁。原始检测数据的结果,每合成一个碱基即显示为一个脉冲峰,每分钟>100个碱基的速度,配上高分辨率的光学检测系统,就能实时检进行检测。









关键点之二:荧光标记位点。这是影响测序长度的非常关键的因素。

二代测序都标记在5‘端甲基上,在合成过程中,荧光标记物保留在DNA链上,随DNA链的延伸会产生三维空间阻力导致DNA链延长到一定程度后会出现错读。这是NGS的测序读长仅能达到100多bp到200bp的一个原因。

PacBio平台的碱基荧光标记在3‘端磷酸键。在DNA合成过程中正确的碱基进入时,在3’端磷酸键的标记是会随磷酸键断裂自动被打断,标记物被弃去,亦即合成的DNA链不带荧光标记,和天然的DNA链合成产物一致,可以达到很长的读长。

(笔者疑问:是不是NGS改用5‘端标记,就能实现延长读长?

答:首先,荧光标记在3‘端磷酸键是PacBio的专利。其它公司没法做。荧光标记位点仅仅是影响读长的一个重要因素之一,PacBio的单分子实时测序反应是最接近天然状态的聚合酶反应体系,最大限度地保持了聚合酶的活性。NGS测序反应原理不尽相同,有的是焦磷酸测序反应,除聚合酶外有多种酶参与测序反应, 要兼顾多种酶的活力并容非一件易的事;有的是通过添加保护基团来控制碱基的加入和检测,通过淬灭试剂来消除背景荧光和保护基团,这些都增加了测序反应体系本身的复杂性,此外,NGS每加入一种碱基或一个碱基后都需要清洗步骤清除没有反应的多余反应物及反应产生的次级产物,这都影响了聚合酶的合成进程。)

关键点之三:时空段概念

合成过程中,每次进入一个碱基,原始数据会实时地产生一个脉冲峰,每两个相邻的脉冲峰之间有一定的距离,也就是有一个时间段的概念。这个距离的长短与模板上碱基是否存在修饰有关,如果有碱基修饰,就像开车经过路障时,通过速度会减慢,导致两个相邻峰之间距离加大。根据这个距离的变化,可以判断模板相应位点是否出现碱基修饰,并且结果是实时的。甲基化就是一种主要的碱基修饰,PacBio技术不仅可以提供序列信息,还可提供实时信息了解模板修饰的情况,用于甲基化等碱基修饰研究。

