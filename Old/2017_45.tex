\documentclass{article}
\usepackage{ctex}
\usepackage{picinpar}

\usepackage[top=1in, bottom=1in, left=1.2in, right=1.2in]{geometry}

\usepackage{hyperref}

\title{科技部\ 发展改革委\ 工业和信息化部\ 国家卫生计生委\ 体育总局\ 食品药品监管总局关于印发《“十三五”健康产业科技创新专项规划》的通知}
\author{国科发社〔2017〕149号}

\begin{document}
	
\maketitle

\noindent \textbf{各省、自治区、直辖市及计划单列市科技厅(委、局)、发展改革委、工业和信息化委、卫生计生委(卫生厅、局)、体育局、食品药品监督管理局,新疆生产建设兵团科技局、发展改革委、工业和信息化委、卫生局、体育局、食品药品监督管理局,各有关单位}:

\vspace{0.2in} 

按照\uline{《中华人民共和国国民经济和社会发展第十三个五年规划纲要》}、\uline{《“十三五”国家科技创新规划》}、\uline{《“健康中国2030”规划纲要》}等的总体部署,为加快推进健康产业科技发展,打造经济发展新动能,促进未来经济增长,引领健康服务模式变革,支撑健康中国建设,特制定《“十三五”健康产业科技创新专项规划》。现印发你们,请认真贯彻执行。

\vspace{0.2in} 

\begin{flushright}
	\begin{tabular}{c}
		科 技 部 \\
		发展改革委 \\
		工业和信息化部 \\
		国家卫生计生委 \\
		国家体育总局 \\
		国家食品药品监管总局 \\
		2017年5月26日 
	\end{tabular}
	
\end{flushright}

\newpage
\tableofcontents
\newpage


\vspace{0.5in} 
{\noindent \Huge \centerline{“十三五”\uline{健康产业科技创新专项}规划}}
\vspace{0.5em} 

健康产业是为维护和促进健康的产业,涉及到\uline{生命全周期}和\uline{健康全过程}的产品及服务。健康产业涉及面广、产业链长、综合性强、增长空间大,是未来国民经济的重要支柱产业。

按照\uline{《国家创新驱动发展战略纲要》}、\uline{《“十三五”国家科技创新规划》}、\uline{《“健康中国2030”规划纲要》}、\uline{《关于促进健康服务业发展的若干意见》}、\uline{《关于加快发展体育产业促进体育消费的若干意见》}、\uline{《关于积极推进“互联网+”行动的指导意见》}、\uline{《关于加快发展康复辅助器具产业的若干意见》}和\uline{《关于加快发展养老服务业的若干意见》}等相关文件要求,为统筹布局和加快推进健康产业科技发展,打造经济发展新动能,促进未来经济增长,引领健康服务模式变革,支撑健康中国建设,特制定本规划。

\section{形势与需求}

\subsection{战略意义}

随着经济发展和社会进步,\uline{健康需求越来越成为驱动未来经济增长的“核心驱力”}。当前,以\textbf{创新药物}、\textbf{高端医疗器械}为主体的医药产业持续增长和快速发展,医药产业在重塑未来经济产业格局中的引领性作用和支柱性地位不断增强。同时,\textbf{新一代信息、生物、工程技术与医疗健康领域的深度融合}日趋紧密,\textbf{远程医疗、移动医疗、精准医疗、智慧医疗}等技术蓬勃发展,以\textbf{主动健康}为方向的营养、运动、行为、环境、心理健康技术和产品正推动健康管理、健康养老、全民健身、健康食品、“互联网+”健康、健康旅游等\textbf{健康产业新业态、新模式}蓬勃兴起,健康产业整体发展呈现跨界融合、集群发展、快速放大、强劲增长之势,并在推动医疗服务模式变革,构建院内外连续性服务和医疗健康一体化体系等方面展现出巨大前景,广受科技界、产业界和投资界的关注,成为新一轮国际科技和产业竞争的前沿焦点。我国人口规模大,健康产业需求增长动力强,市场潜力大,发展前景广阔,\textbf{\uline{预计2020年我国健康产业总规模将达到8万亿}},是引领推动我国新经济增长的一个重要方向。同时,健康产业也是发展健康事业的基础和保障,健康产业的创新是健康事业发展的重要推力,实现健康事业与健康产业的协调发展和相互促进是推进健康中国建设的必然要求和战略重点。面临新的历史机遇,以科技为引领,大力发展健康科技产业群和服务新业态,对于打造未来竞争优势,抢占新产业发展的战略高地、推动供给侧改革至关重要。

\subsection{发展趋势}

新一轮科技革命正在孕育兴起,全球创新格局发生重大调整,发达国家大力推进健康产业新技术的发展和应用,国际竞争态势日趋激烈,并呈现出四个重要的创新趋向:

\begin{enumerate}
	\item 生物前沿技术加快突破,\textbf{免疫疗法}、\textbf{3D打印}和\textbf{基因治疗}等新技术快速发展;
	\item 大数据、互联网、物联网、人工智能等新一代信息技术与健康服务加速融合,“互联网+”健康服务新业态快速形成和发展;
	\item 健康相关的科学研究不断深入,引领传统医疗模式向覆盖生命全周期、健康全过程的\textbf{新型健康保障模式}快速转变;
	\item 健康产业链条不断延伸,可穿戴设备、科学健身、健康营养、中医药养生保健等面向个人、家庭的新型健康产品和服务迅速发展。
\end{enumerate}

\subsection{科技发展基础}

我国在发展健康产业的基础前沿研究、平台基地建设、人才团队培养等方面已具备良好基础,科技实力不断提升。“十二五”期间,获得新药证书85件,成功研发了\textbf{埃克替尼}、\textbf{西达本胺}等创新品种,并完成一批大品种药物技术改造升级,\textbf{热毒宁}等中药大品种二次开发取得突出成果,基本满足群众用药需求;突破了抗体大规模细胞培养、中药质量控制与安全性等50余项制约新药研发的瓶颈关键技术,建成各类平台近300个,构建了以平台和技术体系为核心的国家药物创新体系,新药研发能力大幅提升。医疗器械国产化发展取得了长足进步,超导磁体、全数字正电子探测器、磁兼容电极、数字化X射线探测器、单晶超声换能器、计算机断层扫描(CT)X射线管等核心部件研制取得实质性突破; X射线机、超声、生化等基层新“三大件”实现全线技术升级;磁共振成像(MRI)、彩超、CT、\textbf{正电子发射计算机断层显像(PET-CT)}、放疗等高端产品成功实现国产化;脑起搏器、手术机器人、血管内超声等创新产品取得了重大进展;康复辅助器具产业规模持续扩大,假肢矫形器、轮椅及其他助行器、护理床、失禁用品、无障碍设施、助视器(含眼镜)、助听器、个人治疗辅助器具、康复训练设备等新产品不断涌现,具有自主知识产权产品的市场占有率大幅提升;基因检测、干细胞治疗、免疫治疗、重离子放射治疗等新型医疗服务技术加快发展;远程医疗、智慧医疗、智能养老、科学健身等技术迅速突破,为培育健康医疗服务新业态奠定了良好基础。

总体来看,在取得一系列成果的同时,我国健康产业领域的科技创新能力和核心竞争力还亟待提高,\textbf{高端生物医药产品}以仿为主、以进口为主的局面仍未改变,新型诊疗技术转化瓶颈问题仍然突出,优质健康产品和服务供给不足等问题仍然严峻。按照党中央国务院在健康领域深入实施创新驱动发展战略的要求,切实加强健康产业科技创新,是支撑产业发展、培育国民经济新增长点、提升全民健康水平的重大战略任务。

\section{指导思想与基本原则}

\subsection{指导思想}

全面贯彻党的十八大和十八届三中、四中、五中、六中全会精神,深入贯彻习近平总书记系列重要讲话精神,按照\uline{全国科技创新大会}和\uline{全国卫生与健康大会}战略部署,以科技创新为动力,以健康需求为导向,以新技术、新产品、新模式、新业态的创新引领,加快推进新药、医疗器械、健康产品、新型健康服务的创新突破,促进健康医学模式转变和支撑健康产业发展,助力健康中国建设。

\subsection{基本原则}

\begin{enumerate}
	\item 坚持\textbf{创新驱动}。将自主创新作为我国健康产业科技发展的战略基点,强化基础前沿研究,加强核心技术突破,推动创新链和产业链深度融合,提升科技核心竞争力。
	\item 坚持\textbf{高端引领}。按照“\textbf{高点起步,高位切入}”的原则,准确把握全球科技发展方向,坚持以攻占健康产业发展制高点为目标定位,着力主攻引领性技术、高端产品和高技术服务,推动我国健康产业由中低端向中高端迈进。
	\item 坚持\textbf{民生导向}。以提升健康保障水平和改善民生为根本目的,按照“产业与服务并重、创新与示范同步”的原则,确保健康领域先进技术和成果加快落地,满足公众个性化、多样化的医疗健康服务需求。
	\item 坚持\textbf{规范发展}。促进创新激励和应用监管的协调发展,协同构建科学规范的评估标准和准入体系,推进技术应用和规范服务,引领我国健康产业的良性发展和快速发展。
\end{enumerate}

\section{发展目标}

\subsection{总体目标}

以保障全人群、全生命周期的健康需求为核心,重点发展创新药物、医疗器械、健康产品等三类产品,引领发展以“\textbf{\uline{精准化、数字化、智能化、一体化}}”为方向的新型医疗健康服务模式,着力打造科技创新平台、公共服务云平台等支撑平台,构建全链条、竞争力强的产业科技支撑体系,建设一批健康产业专业化园区和综合示范区,培育一批具有国际竞争力的健康产业优势品牌企业,助推健康产业创新发展。

\subsection{具体目标}

\begin{enumerate}
	\item 技术突破。重点突破新药发现、高端医疗器械、个性化健康干预等关键科技问题,攻克10-15项重大关键共性技术,发展20-30项前沿性技术。
	\item 产品开发。重点开发8-10个原创性新药产品、10-20项前沿创新医疗器械、50种高端健康产品。
	\item 产业培育。积极推进新型健康产业培育,引领发展新型医疗健康服务,培育5-10个有国际影响力的健康品牌企业集群,建立10-15个健康产业专业化园区。
\end{enumerate}

\section{重点任务}

\subsection{重点发展三类产品}

\begin{window}[0, r, {\fbox{
			\begin{minipage}{38em}
				\centerline{专栏1:\textbf{重大创新药物}}
				
				\textbf{新药靶标发现}。重点\textbf{\uline{开展大规模结构基因组研究}}、基于片段的药物设计、以及基于靶标结构的药物结构修饰与优化研究,多维度研究药物作用机制、潜在的脱靶效应,发现和选择合适的药物靶点,为原创性药物研发提供治疗靶标。
				
				\textbf{疫苗和抗体}。支持病毒性疫苗、联合疫苗、基因重组蛋白质疫苗、多糖蛋白结合等细菌性疫苗及治疗性疫苗研究;支持抗体偶联药物、双特异性抗体、新靶点抗体及单克隆抗体药物研究。
				
				\textbf{小分子靶向药物}。结合新型生物标志物和药物干预靶标,开展新化学实体(NCE)药物的新颖性、优效性和药代研究,改善药物递送系统的可及性以及药物的安全性,研发新型小分子靶向药物。
				
				\textbf{中药新药}。突出重大疾病和中医优势病种,重点加强源于经典名方、院内制剂、名老中医经验方等中药复方新药以及中药组分或单体新药的研发,创制临床疗效突出、安全性高、质量可控、易于服用的中药新药。
				
				\textbf{制药装备}。以数字化、网络化、智能化为主要发展方向,重点发展原料药机械及设备、制剂机械及设备、饮片(中药)设备、药品包装机械设备、药品检测设备等面向医药产业需求的高端生产设备,提升自主产品市场占有率。
				
			\end{minipage}
	}}, {}]

\subsubsection{重大创新药物}

以严重危害我国人民健康的重大疾病为重点,突破10-15项重大核心关键技术,自主研制30个左右创新性强、科技含量高、市场前景好、拥有自主知识产权的新药,提升制药装备技术水平,提高创新药物的国际竞争力,满足公众用药需求。



\end{window}

\subsubsection{高端医疗器械}

突出解决我国高端医疗器械严重依赖进口、核心部件国产化程度低的问题,重点加强数字诊疗装备、体外诊断产品、高值耗材等重大产品攻关,协同推进检测与计量技术提升、标准体系建设、示范应用推广等工作,打破进口垄断,降低医疗费用,提高产业竞争力,促进我国高端医疗器械行业的跨越发展,推动产业整体向创新驱动发展转型。

\fbox{
\begin{minipage}{30em}
	专栏2:高端医疗器械
	
	数字诊疗装备。以早期、精准、微创诊疗为方向,以严重依赖进口的医学影像诊断和先进治疗的前沿产品为主攻方向,突破新型成像、先进治疗和一体化诊疗等颠覆性技术,重点推进多模态分子成像、新型磁共振成像系统、新型X射线计算机断层成像、新一代超声成像、低剂量X射线成像、复合窥镜成像、新型显微成像、大型放射治疗装备、手术机器人、医用有源植入式装置等产品研发,完善相关标准与检测,加快推进数字诊疗装备国产化、高端化、品牌化,完善创新链、产业链和服务链。
	
	体外诊断产品。突破微流控芯片、单分子检测、自动化核酸检测等关键技术,开发全自动核酸检测系统、高通量液相悬浮芯片、医用生物质谱仪、快速病理诊断系统等重大产品,研发一批重大疾病早期诊断和精确治疗诊断试剂以及适合基层医疗机构的高精度诊断产品,提升我国体外诊断产业的竞争力。
	
	组织器官修复和替代材料及植介入器械。以组织替代、功能修复、智能调控为方向,突破3D打印、人工智能等新技术应用,研发组织工程化产品、植介入医疗器械及配套手术器械、口腔植入及颌面修复材料、血液净化材料及设备等新一代生物医用材料器械,提升产品品质,实现进口替代。
	
	便携式、小型化的移动医疗装备。重点突破远程、移动、智能一体化融合关键技术,开发适用于移动医疗的体征监测、疾病诊断、支持治疗相关设备与诊断软件,研发多种移动环境下的专用医疗设备、应急救治设备和配套的标准规范与质量评价体系,制定移动医疗解决方案和家庭化、个体化医疗健康连续服务综合解决方案。
\end{minipage}
}

\subsubsection{新型健康产品}

围绕健康促进、慢病管理、养老服务等需求,重点发展健康管理、智能康复辅具、健康营养食品、环境健康、科学健身、中医药养生保健等新型健康产品。

\begin{minipage}{20em}
	专栏3:新型健康产品
	
	健康管理产品。重点突破人体生理、心理、行为关键参数移动检测与获取的新原理、新方法,脑、心脏等重要器官新型生物医学传感测量新技术,以及非药物干预相关关键技术,开发适用于人体日常生理、行为、心理以及环境等连续性信息检测健康电子类穿戴式监测、便携式检测、非接触式监控、近人体空间健康信息采集等健康管理产品。
	
	智能康复辅具。围绕功能代偿、生活护理、康复训练等需求,重点突破柔性控制、多信息融合、运动信息解码、外部环境感知等新技术,开发系列智能假肢、智能矫形器、外固定矫正系统、新型电子喉、智能护理机器人、外骨骼助行机器人、智能喂食系统、多模态康复轮椅、智能康复机器人、虚拟现实康复系统、肢体协调动作系统、智能体外精准反搏等康复辅具。
	
	健康营养食品。围绕婴幼儿、孕妇、老年人的健康营养问题,患病人群的医学营养的临床需求,以及特殊环境工作人员的防护需要,重点推动抗衰老产品、膳食补充剂、营养强化食品、功能食品、特殊医学用途食品的研发。
	
	环境健康产品。研发与建筑设备集成的健康数据监测与采集设备、家居环境风险报警设备、室内健康环境自维持控制设备、智能家居服务终端,建立人居健康环境标准体系,支持健康服务入户进家。
	
	科学健身产品。开发运动健康芯片、运动可穿戴设备、评估测试一体化系统、科学健身指导、智慧健身器材和装备、“互联网+”健身服务配套设备等新型产品,促进运动健康,提升全民健身的科技水平。
	
	中医药健康产品。发挥中医药“治未病”的特色优势,重点发展中医健康状态评估及风险预警仪器设备、中医药干预技术产品、传统养生功法产品、以及功能性食品和新型中医诊疗设备。
\end{minipage}


\subsection{引领发展四类服务}

以精准化、数字化、智能化、一体化为方向,重点发展新型诊疗、协同医疗、智慧医疗、主动健康服务,建立覆盖医院、社区、家庭、个体的集成式健康服务模式,推动医学诊疗和健康服务模式变革,引领医养康护一体化、连续性的健康保障体系建设,实现优化资源配置、改善就医模式和强化健康促进的目标。

\subsubsection{以精准化为重点方向的新型诊疗服务}

抢抓生物技术和信息技术融合发展的战略机遇,以恶性肿瘤、心脑血管、代谢性疾病、罕见病等为重点,重点攻克新一代基因测序技术、肿瘤免疫治疗、干细胞与再生医学、生物医学大数据分析等关键技术,建立重大疾病的早期筛查、分子分型、个体化治疗等精准化的应用解决方案和决策支持系统,推动医学诊疗模式变革。

\begin{minipage}{20em}
专栏4:新型诊疗服务

基因测序。研发用于临床DNA序列分析的小型临床测序仪,开发纳米孔测序等新一代测序相关技术,研究与测序仪或测序技术配套的新型相关试剂及校准和质量控制技术。

肿瘤免疫治疗。重点研究嵌合型抗原受体修饰的T细胞(CAR-T)和T细胞受体修饰的T细胞(TCR-T)等基因工程T细胞技术,选择特异性好的靶点和适中亲和力的识别受体等提高其应用的安全性,研究通用型生产技术,研究新型的基因修饰肿瘤细胞疫苗技术,增加免疫细胞的应答,减少免疫逃逸,最终实现对肿瘤细胞的特异性免疫应答。

干细胞与再生医学。深入开展干细胞、生物材料、组织工程、生物人工器官,以及干细胞与疾病发生等方面的应用研究和转化开发,获得能够调控干细胞增殖、分化和功能的关键技术,利用干细胞体内外分化特性,结合智能生物材料、组织工程、胚胎工程,实现神经、肝脏、肾脏、生殖系统等组织器官再造,加快临床应用。

生物医学大数据分析。整合以生命组学数据、临床信息和健康数据为核心的多层次数据,开发用于不同层次数据的快速分析体系,研发海量个人多组学信息管理、注释、可视化与应用系统,构建支持精准医疗的大型知识库系统,开发系列工具,为精准医疗和智慧医疗提供支撑。

个体化治疗。研究基于基因型或分子特征谱的复杂疾病亚型的药效评估方法,开发基于分子影像的疾病疗效评估及预后监测产品,建立规模化的药物疗效与安全性个性化评价、个性化治疗、个性化筛选、耐药鉴定和检测的技术体系,研发个体化治疗方法与制剂,研发药物临床个性化应用方案和伴随诊断方法及试剂盒,在阐明现有药物的药效、毒性、耐药等个性化特征的基础上,对“型”下药,实现个体化精准治疗。
\end{minipage}

\subsubsection{以数字化为方向的协同医疗服务}

以新一代信息技术为支撑,重点突破网络协同、分布式系统、临床决策支持等关键技术,建立基于数据、信息、知识的集成、融合、共享、多学科协同的集成式服务模式,支持分级诊疗、区域协同和整合服务,提高医疗服务供给质量和改善就医模式,破解医疗资源不足和利用不够的难题,提高医疗卫生服务的可及性,推动医疗服务模式变革。

\begin{minipage}{20em}
专栏5:协同医疗服务

远程医疗和移动医疗服务。针对医疗资源配置严重不均衡的问题,积极发展移动医疗技术产品及服务,加快推动远程技术发展与装备集成,开展集成创新和典型应用示范,推动构建覆盖基层的远程和移动医疗健康服务体系建设,优化医疗资源配置,促进医疗服务模式优化升级,改善公众就医难题。

专科协同服务。针对心脑血管疾病、代谢性疾病、肿瘤等重大疾病以及妇产、口腔等常见多发病,突破专科专病信息与流程建模、专病临床路径实现、专科数据集成可视化、专科专病医学知识共享等关键技术,研制专科专病电子病历系统、临床决策支持系统、诊疗业务协同系统等,建立信息联动、整合服务的专科协同诊疗服务模式,有效提升专科及专病的诊疗水平。
\end{minipage}

\subsubsection{以智能化为方向的智慧医疗服务}

围绕健康风险监测、疾病预测预警、疾病诊疗与康复等环节,重点加强医疗卫生健康大数据应用的人工智能前沿技术研究,推动智能辅助诊断、智能临床决策等新模式发展,提高我国医疗大数据资源开发应用水平,缓解医疗资源供给难题,改善供给质量。

\begin{minipage}{20em}
专栏6:智慧医疗服务
智能诊断技术。加快推进医学影像大数据分析、图像处理、人工视觉、模式识别等关键技术突破,发展病灶识别、病理分型、心电图信号判读等自动诊断技术,提高诊断效率和质量。研究医学知识库构建、医学自然语言处理与分析、医学文档语义分析与理解、人工经验学习等关键技术,发展智能全科“机器人”辅助诊断系统。
智能治疗技术。加快增强现实、虚拟现实、计算机图形图像可视化、人工神经网络的深度学习、自然进化和人工免疫等算法、认知计算等关键技术的应用突破,推动治疗规划、外科手术、微创介入、活检穿刺、放疗等技术的智能化发展,提高治疗水平。
智能临床决策支持系统。积极发展医疗健康数据获取技术和建立医疗健康大数据平台,突破基于大数据的医疗效果比较分析技术,研制基于患者相似性比较的个性化诊疗决策支持系统。针对手术、急救、监护等复杂易错的关键医疗过程,推动关键医疗过程的智能监控与优化反馈,加强安全用药智能支持技术研发,降低医疗差错,提高医疗效率和水平。
\end{minipage}

\subsubsection{以主动健康为方向的医疗健康一体化服务}

以主动健康为方向,积极开展个人健康状况的监测、评价、预警和干预的研究,提供连续性疾病和健康管理服务,将医疗健康服务延伸到个人、家庭和社区。

\begin{minipage}{20em}
专栏7:医疗健康一体化服务

个性化健康干预技术。集成穿戴式、移动式健康信息采集终端,以及基于大数据挖掘和云计算的服务平台,构建数字健康服务平台,推进个性化的营养、心理、行为干预服务发展。

科学健身服务技术。加强健身物联网、新型运动训练、大众健身器材、公共体育服务等关键技术发展,开展体质与健康监测大数据开发利用服务和人类表型组数据在体质与健康监测与评价中的应用服务;推动针对肥胖及脂肪肝、糖耐量受损和糖尿病、心脏康复、老年运动障碍等健康问题的运动处方、健身咨询等健康服务新技术、新业态发展。

家庭健康服务技术。聚焦家庭健康危险因素,加强基于居家健康生活、居家养老、家庭病床等家庭服务技术研究,建立完善的家庭健康评价标准体系,实现医疗与健康家庭一体化服务。

养老助残服务技术。以智能服务、功能康复、个性化适配为方向,突破人机交互、神经-机器接口、多信息融合与智能控制等关键技术,建立养老服务科技标准体系和技术解决方案,科学应对人口老龄化。

健康管理服务技术。开发健康管理相关的设备和信息系统,推进重大慢性疾病全程照护和院内外连续性的个性化健康管理和基于专业指导的健康自主管理。开展健康科普信息筛选与评价技术研究,建立重大疾病和慢病合理用药、中医保健产品等科普平台,加快健康产品的知识普及。
\end{minipage}

\subsection{着力打造支撑平台}
加强自主创新能力建设,优化科技资源配置,推进科技资源开放共享和高效利用,基本建成有效支撑健康产业发展科技资源和条件支撑体系,形成比较完善的共享机制和服务体系。

\begin{minipage}{20em}
专栏8:着力打造支撑平台

建设健康产业科技创新平台。推进健康产业科技创新平台建设,鼓励健康产业相关的企业、高等学校、科研院所等机构建立产学研协同的创新平台,集聚创新资源,加快突破前沿关键技术瓶颈,为健康产业科技发展提供支撑。

构建健康产业创新产品评价平台。依托国家临床医学研究中心、检验检测机构等平台,开展基础临床紧密结合的健康科技转化研究,健康产品与服务的评价研究,加强规范评价,引领推动健康产业发展。

搭建健康大数据公共服务云平台。应用物联网、大数据和云计算技术,构建健康大数据公共服务云平台,加快推动健康、医疗、医保等信息的互联互通,为基于专业指导的医疗健康管理和定制化服务提供全过程的健康大数据支撑。
\end{minipage}

\subsection{推动产业集聚发展}

技术创新、模式创新、政策创新相结合,加强技术引领、资源集成、辐射带动、开放创新,建立“政产学研金”结合、产业化导向明晰的创新体系,研发一批具有核心自主知识产权、高附加值的品牌产品,培育一批基础扎实、创新性强的品牌企业,打造健康产业集聚发展的新载体,引领健康产业集聚发展。

\begin{minipage}{20em}
专栏9:推动产业集聚发展

专业化园区。选择一批产业基础好、集聚度高的园区,坚持以市场需求为根本导向,发挥企业在产业集聚中的主要作用,加强政策引导,集聚技术、金融、人才等创新要素,打造若干“政产学研用”紧密协同、资源集聚、政策衔接配套、研究开发和成果转化有机结合的健康产业专业化园区,引领推动健康产业发展。

健康产业综合示范区。选择一批地方政府重视、政策力度大、有条件投入的地区,加强健康技术、产品、服务的融合创新,推进健康产业科技综合示范,提高我国健康产业综合发展能力和国际竞争力。
\end{minipage}

\section{保障措施}
\subsection{强化组织协同}

提升战略高度,加强组织领导,深化部门协同,统筹推进科技引领、产业发展、应用政策全链条的顶层设计,引导新型健康产业快速、规范发展。加强与地方的沟通协调,协同推进健康产业的统筹布局。

\subsection{加强科技部署}

把握健康产业发展新方向,结合国家科技计划改革总体部署,通过“科技创新2030-重大项目”和国家科技重大专项、国家重点研发计划、基地与人才专项等国家科技计划(专项、基金)的实施,着力开展健康产业关键共性技术攻关与重大产品研发,加快推进健康产业领域的科技创新。

\subsection{推动创新创业}

完善投融资环境,充分调动社会资本的积极性,推动健康产业大数据平台建设,构建健康产业创新创业平台;推进健康产业技术创新战略联盟建设,推动“政产学研用”深度融合,促进健康产业的创新创业。

\subsection{引领品牌发展}

出台创新药物和创新医疗器械产品目录,打造一批国产创新品牌;加快推动医疗器械可靠性与工程化技术应用推广,提升国产设备质量;积极探索财政引导支持方式,合理运用服务收费、医保政策,引导使用国产设备;开展创新药物质控标准体系建设、国际多中心临床试验实施规范和临床疗效评价等研究,加速推进创新药物国际化进程;落实国家“一带一路”规划愿景,促进我国健康产品在沿线国家的推广,助力我国健康产业的国际化发展。

\subsection{创新政策环境}

充分调动各方积极性,支持健康产业技术研发与产业化。完善政策制度,制定数据安全、数据共享技术标准和隐私保护政策,促进医疗服务、公共卫生与健康、养生养老、体育健身等领域的数据融合应用,促进新型健康产业集群和融合发展。完善药品、医疗器械、保健食品注册管理与监管制度,探索新型保险付费机制,优化创新环境,加速产业发展。

\end{document}