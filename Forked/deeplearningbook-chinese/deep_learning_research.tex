% !Mode:: "TeX:UTF-8"
\part{深度学习研究}
\label{part:deep_learning_research}

\newpage
本书这一部分描述目前研究社群所追求的、更有远见和更先进的\gls{DL}方法。

在本书的前两部分,我们已经展示了如何解决\gls{supervised_learning}问题,即在给定足够的映射样本的情况下,学习将一个向量映射到另一个。

我们想要解决的问题并不全都属于这个类别。
我们可能希望生成新的样本、或确定一个点的似然性、或处理缺失值以及利用一组大量的未标记样本或相关任务的样本。
当前应用于工业的最先进技术的缺点是我们的学习算法需要大量的监督数据才能实现良好的精度。
在本书这一部分,我们讨论一些推测性的方法,来减少现有模型工作所需的标注数据量,并适用于更广泛的任务。 
实现这些目标通常需要某种形式的\gls{unsupervised}或\gls{semi_supervised}学习。

许多\gls{DL}算法被设计为处理\gls{unsupervised_learning}问题,但不像\gls{DL}已经在很大程度上解决了各种任务的\gls{supervised_learning}问题,没有一个算法能以同样的方式真正解决\gls{unsupervised_learning}问题。
在本书这一部分,我们描述\gls{unsupervised_learning}的现有方法和一些如何在这一领域取得进展的流行思想。

\gls{unsupervised_learning}困难的核心原因是被建模的随机变量的高维度。
这带来了两个不同的挑战:统计挑战和计算挑战。
\emph{统计挑战}与泛化相关:我们可能想要区分的配置数会随着感兴趣的维度数指数增长,并且这快速变得比可能具有的(或者在有限计算资源下使用的)样本数大得多。
与高维分布相关联的\emph{计算挑战}之所以会出现,是因为用于学习或使用训练模型的许多算法(特别是基于估计显式概率函数的算法)涉及难处理的计算量,并且随维数呈指数增长。

使用概率模型,这种计算挑战来自执行难解的\gls{inference}或归一化分布。
\begin{itemize}
 \item \emph{难解的\gls{inference}}:\gls{inference}主要在\chapref{chap:approximate_inference}讨论。
\gls{inference}关于捕获$a$,$b$和$c$上联合分布的模型,给定其他变量$b$的情况下,猜测一些变量$a$的可能值。
为了计算这样的条件概率,我们需要对变量$c$的值求和,以及计算对$a$和$c$的值求和的归一化常数。
 \item \emph{难解的归一化常数(\gls{partition_function})}:\gls{partition_function}主要在\chapref{chap:confronting_the_partition_function}讨论。
归一化概率函数的常数在\gls{inference}(上文)以及学习中出现。
许多概率模型涉及这样的归一化常数。
不幸的是,学习这样的模型通常需要相对于模型参数计算\gls{partition_function}对数的梯度。
该计算通常与计算\gls{partition_function}本身一样难解。
\glsacr{mcmc}(\chapref{chap:monte_carlo_methods})通常用于处理\gls{partition_function}。
不幸的是,当模型分布的模式众多且分离良好时,\glssymbol{mcmc}方法会出现问题,特别是在高维空间中(\secref{sec:the_challenge_of_mixing_between_separated_modes})。
\end{itemize}

面对这些难以处理的计算的一种方法是近似它们,如在本书的第三部分中讨论的,研究者已经提出了许多方法。
这里还讨论另一种有趣的方式是通过设计模型,完全避免这些难以处理的计算,因此不需要这些计算的方法是非常有吸引力的。
近年来,研究者已经提出了数种具有该动机的生成模型。
其中\chapref{chap:deep_generative_models}讨论了各种各样的现代生成式建模方法。

第三部分对于研究者来说是最重要的,研究者想要了解\gls{DL}领域的广度,并将领域推向真正的\gls{AI}。



% !Mode:: "TeX:UTF-8"
% Translator: Tianfan Fu
\chapter{\glsentrytext{linear_factor}}
\label{chap:linear_factor_models}




许多深度学习的研究前沿均涉及构建输入的概率模型$p_{\text{model}}(\Vx)$。
原则上说,给定任何其他变量的情况下,这样的模型可以使用概率推断来预测其环境中的任何变量。
许多这样的模型还具有\gls{latent_variable} $\Vh$,其中$p_{\text{model}}(\Vx) = \SetE_{\Vh}\, p_{\text{model}}(\Vx\mid\Vh)$。
这些\gls{latent_variable}提供了表示数据的另一种方式。
我们在\gls{deep_feedforward_network}和\gls{recurrent_network}中已经发现,基于\gls{latent_variable}的分布式表示继承了\gls{representation_learning}的所有优点。
% 479


在本章中,我们描述了一些基于\gls{latent_variable}的最简单的概率模型:\firstgls{linear_factor}。
这些模型有时被用来作为混合模型的组成模块~\citep{Hinton-nips95,ghahramani96em,Roweis+Saul+Hinton-2002}或者更大的深度概率模型~\citep{tang2012deep}。
同时,也介绍了构建\gls{generative_model}所需的许多基本方法,在此基础上更先进的深度模型也将得到进一步扩展。
% 479


\gls{linear_factor}通过随机线性\gls{decoder}函数来定义,该函数通过对$\Vh$的线性变换以及添加噪声来生成$\Vx$。
% 479


有趣的是,通过这些模型我们能够发现一些符合简单联合分布的解释性因子。
%这些模型很有趣,因为它们使得我们能够发现一些拥有简单联合分布的解释性因子。 
线性\gls{decoder}的简单性使得它们成为了最早被广泛研究的\gls{latent_variable}模型。
% 479


\gls{linear_factor}描述如下的数据生成过程。 
首先,我们从一个分布中抽取解释性因子$\Vh$
\begin{align}
\label{eqn:131}
\RVh \sim p(\Vh),
\end{align}
其中$p(\Vh)$是一个\gls{factorial}分布,满足$p(\Vh) = \prod_{i}^{}p(h_i)$,所以易于从中采样。
接下来,在给定因子的情况下,我们对实值的可观察变量进行采样
\begin{align}
\label{eqn:132}
\Vx = \MW \Vh + \Vb + \text{noise},
\end{align}
其中噪声通常是对角化的(在维度上是独立的)且服从高斯分布。
这在\figref{fig:linear_factors}有具体说明。
% 480

\begin{figure}[!htb]
\ifOpenSource
\centerline{\includegraphics{figure.pdf}}
\else
	\centerline{\includegraphics{Chapter13/figures/linear_factors}}
\fi
\caption{描述\gls{linear_factor}族的\gls{directed_graphical_model},其中我们假设观察到的数据向量$\Vx$是通过独立的\gls{latent}因子$\Vh$的线性组合再加上一定噪声获得的。
不同的模型,比如\gls{PPCA},\gls{FA}或者是~\glssymbol{ICA},都是选择了不同形式的噪声以及先验$p(\Vh)$。}
\label{fig:linear_factors}
\end{figure}


\section{\glsentrytext{PPCA}和\glsentrytext{FA}}
\label{sec:probabilistic_PCA_and_factor_analysis}
% 480

\firstgls{PPCA}、\gls{FA}和其他\gls{linear_factor}是上述等式(\eqnref{eqn:131}和\eqnref{eqn:132})的特殊情况,并且仅在对观测到$\Vx$之前的噪声分布和\gls{latent_variable} $\Vh$先验的选择上有所不同。
% 480

在\firstgls{FA}~\citep{Bartholomew-1987,Basilevsky94}中,\gls{latent_variable}的先验是一个方差为单位矩阵的\gls{gaussian_distribution}
\begin{align}
\label{eqn:133}
\RVh \sim \CalN(\Vh; \mathbf{0},\MI),
\end{align}
同时,假定在给定$\Vh$的条件下观察值$x_i$是\firstgls{conditional_independent}的。
具体来说,我们可以假设噪声是从对角协方差矩阵的高斯分布中抽出的,\gls{covariance_matrix}为$\Vpsi = \text{diag}(\Vsigma^2)$,其中$\Vsigma^2 = [\sigma_1^2,\sigma_2^2,\ldots,\sigma_n^2]^{\top}$表示一个向量,每个元素表示一个变量的方差。
% 480


因此,\gls{latent_variable}的作用是\emph{捕获}不同观测变量$x_i$之间的\emph{依赖关系}。
实际上,可以容易地看出$\Vx$服从\gls{multivariate_normal_distribution},并满足
\begin{align}
\label{eqn:134}
\RVx \sim \CalN(\Vx; \Vb, \MW\MW^{\top}+\Vpsi).
\end{align}
% 480 end



% 481 head
为了将~\glssymbol{PCA}~引入到概率框架中,我们可以对\gls{FA}模型作轻微修改,使条件方差$\sigma_i^2$等于同一个值。
在这种情况下,$\Vx$的协方差简化为$\MW\MW^{\top}+\sigma^2\MI$,这里的$\sigma^2$是一个标量。
由此可以得到条件分布,如下:
\begin{align}
\label{eqn:135}
\RVx \sim \CalN(\Vx; \Vb, \MW\MW^{\top} + \sigma^2\MI ),
\end{align}
或者等价地
\begin{align}
\label{eqn:136}
\RVx = \MW\RVh + \Vb + \sigma\RVz,
\end{align}
其中$\RVz \sim \CalN(\Vz;\mathbf{0},\MI)$是高斯噪声。
之后~\citet{Tipping99}提出了一种迭代的~\glssymbol{EM}~算法来估计参数$\MW$和$\sigma^2$。
% 481


这个\firstgls{PPCA}模型利用了这样一种观察现象:除了一些微小残余的\firstgls{reconstruction_error}(至多为$\sigma^2$),数据中的大多数变化可以由\gls{latent_variable} $\Vh$描述。
通过\citet{Tipping99}的研究我们可以发现,当$\sigma \xrightarrow{} 0$时,\gls{PPCA}~退化为~\glssymbol{PCA}。
在这种情况下,给定$\Vx$情况下$\Vh$的条件期望等于将$\Vx - \Vb$投影到$\MW$的$d$列所生成的空间上,与~\glssymbol{PCA}~一样。
% 481

当$\sigma\xrightarrow{} 0$时, \gls{PPCA}~所定义的密度函数在$d$维的$\MW$的列生成空间周围非常尖锐。
这导致模型会为没有在一个超平面附近聚集的数据分配非常低的概率。
%如果某些数据实际上没有集中在超平面附近,这会导致模型为数据分配非常低的可能性。
% 481
%导致模型会为没有在一个超空间附近聚集的数据分配非常低的概率”

\section{\glsentrytext{ICA}}
\label{sec:independent_component_analysis_ica}
% 481


\firstall{ICA}是最古老的\gls{representation_learning}算法之一~\citep{Herault+Ans-1984,Jutten+Herault-91,Comon94,Hyvarinen-1999,Hyvarinen-2001,Hinton-ICA-2001,Teh-2003}。
它是一种建模线性因子的方法,旨在将观察到的信号分离成许多潜在信号,这些潜在信号通过缩放和叠加可以恢复成观察数据。
这些信号是完全独立的,而不是仅仅彼此不相关\footnote{\secref{sec:expectation_variance_and_covariance}讨论了不相关变量和独立变量之间的差异。}。
% 481


许多不同的具体方法被称为~\glssymbol{ICA}。
与我们本书中描述的其他\gls{generative_model}最相似的~\glssymbol{ICA}~变种~\citep{Pham-et-al-1992}训练了完全参数化的\gls{generative_model}。
\gls{latent}因子$\Vh$的先验$p(\Vh)$,必须由用户提前给出并固定。
接着模型确定性地生成$\Vx = \MW \Vh$。
我们可以通过非线性变化(使用\eqnref{eqn:3.47})来确定$p(\Vx)$。
然后通过一般的方法比如最大化似然进行学习。
% 482 head


这种方法的动机是,通过选择一个独立的$p(\Vh)$,我们可以尽可能恢复接近独立的潜在因子。
这是一种常用的方法,它并不是用来捕捉高级别的抽象因果因子,而是恢复已经混合在一起的低级别信号。
在该设置中,每个训练样本对应一个时刻,每个$x_i$是一个传感器对混合信号的观察值,并且每个$h_i$是单个原始信号的一个估计。
例如,我们可能有$n$个人同时说话。 
如果我们在不同位置放置$n$个不同的麦克风,则~\glssymbol{ICA}~可以检测每个麦克风的音量变化,并且分离信号,使得每个$h_i$仅包含一个人清楚地说话。
这通常用于脑电图的神经科学,这种技术可用于记录源自大脑的电信号。
放置在受试者头部上的许多电极传感器用于测量来自身体的多种电信号。
实验者通常仅对来自大脑的信号感兴趣,但是来自受试者心脏和眼睛的信号强到足以混淆在受试者头皮处的测量结果。
信号到达电极,并且混合在一起,因此为了分离源于心脏与源于大脑的信号,并且将不同脑区域中的信号彼此分离,\glssymbol{ICA}~是必要的。
% 482 mid 


如前所述,\glssymbol{ICA}~存在许多变种。
一些版本在$\Vx$的生成中添加一些噪声,而不是使用确定性的\gls{decoder}。
大多数方法不使用\gls{maximum_likelihood}准则,而是旨在使$\Vh = \MW^{-1}\Vx$的元素彼此独立。
许多准则能够达成这个目标。
\eqnref{eqn:3.47}需要用到$\MW$的行列式,这可能是代价很高且数值不稳定的操作。
\glssymbol{ICA}~的一些变种通过将$\MW$约束为正交来避免这个有问题的操作。
% 482 mid


\glssymbol{ICA}~的所有变种均要求$p(\Vh)$是非高斯的。
这是因为如果$p(\Vh)$是具有高斯分量的独立先验,则$\MW$是不可识别的。
对于许多$\MW$值,我们可以在$p(\Vx)$上获得相同的分布。 
这与其他\gls{linear_factor}有很大的区别,例如\gls{PPCA}~和\gls{FA}通常要求$p(\Vh)$是高斯的,以便使模型上的许多操作具有\gls{closed_form_solution}。
在用户明确指定分布的\gls{maximum_likelihood}方法中,一个典型的选择是使用$p(h_i) = \frac{d}{dh_i}\sigma(h_i)$。
这些非高斯分布的典型选择在$0$附近具有比高斯分布更高的峰值,因此我们也可以看到\gls{ICA}经常用于学习稀疏特征。
% 483 head




按照我们对\gls{generative_model}这个术语的定义,\glssymbol{ICA}~的许多变种不是\gls{generative_model}。
在本书中,\gls{generative_model}可以直接表示$p(\Vx)$,也可以认为是从$p(\Vx)$中抽取样本。
\glssymbol{ICA}~的许多变种仅知道如何在$\Vx$和$\Vh$之间变换,而没有任何表示$p(\Vh)$的方式,因此也无法在$p(\Vx)$上施加分布。
例如,许多~\glssymbol{ICA}~变量旨在增加$\Vh = \MW^{-1}\Vx$的样本峰度,因为高峰度说明了$p(\Vh)$是非高斯的,但这是在没有显式表示$p(\Vh)$的情况下完成的。
这就是为什么~\glssymbol{ICA}~多被用作分离信号的分析工具,而不是用于生成数据或估计其密度。
% 483 head


正如~\glssymbol{PCA}~可以推广到\chapref{chap:autoencoders}中描述的非线性\gls{AE},\glssymbol{ICA}~也可以推广到非线性\gls{generative_model},其中我们使用非线性函数$f$来生成观测数据。
关于非线性~\glssymbol{ICA}~最初的工作可以参考~\citet{hyvarinen1999nonlinear},它和\gls{ensemble_learning}的成功结合可以参见~\citet{roberts2001independent,lappalainen2000nonlinear}。
\glssymbol{ICA}~的另一个非线性扩展是\firstall{NICE}方法~\citep{Dinh-et-al-arxiv2014},这个方法堆叠了一系列可逆变换(在\gls{encoder}阶段),其特性是能高效地计算每个变换的~\gls{jacobian}~行列式。
这使得我们能够精确地计算似然,并且像~\glssymbol{ICA}~一样,\glssymbol{NICE}~尝试将数据变换到具有因子的边缘分布的空间。
由于非线性\gls{encoder}的使用,这种方法更可能成功。%\footnote{译者注:相比于~\glssymbol{ICA}}
因为\gls{encoder}和一个能进行完美逆变换的\gls{decoder}相关联,所以可以直接从模型生成样本(首先从$p(\Vh)$采样,然后使用\gls{decoder})。
% 483


\glssymbol{ICA}~的另一个推广是通过鼓励组内统计依赖关系、抑制组间依赖关系来学习特征组\citep{hyvarinen1999emergence,HyvarinenA2001}。
当相关单元的组被选为不重叠时,这被称为\firstgls{ISA}。
我们还可以向每个\gls{hidden_unit}分配空间坐标,并且空间上相邻的单元组形成一定程度的重叠。
这能够鼓励相邻的单元学习类似的特征。
当应用于自然图像时,这种\firstgls{tICA}方法可以学习Gabor滤波器,从而使得相邻特征具有相似的方向、位置或频率。
在每个区域内出现类似Gabor函数的许多不同相位存在抵消作用,使得在小区域上的\gls{pooling}产生了平移不变性。
% 483 end


\section{\glsentrytext{SFA}}
\label{sec:slow_feature_analysis}
% 484 head


\firstall{SFA}是使用来自时间信号的信息学习不变特征的\gls{linear_factor}~\citep{WisSej2002}。
% 484


\gls{SFA}的想法源于所谓的\firstgls{slow_principle}。
其基本思想是,与场景中起描述作用的单个量度相比,场景的重要特性通常变化得非常缓慢。
例如,在\gls{CV}中,单个像素值可以非常快速地改变。
如果斑马从左到右移动穿过图像并且它的条纹穿过对应的像素时,该像素将迅速从黑色变为白色,并再次恢复成黑色。
通过比较,指示斑马是否在图像中的特征将不发生改变,并且描述斑马位置的特征将缓慢地改变。
因此,我们可能希望将模型\gls{regularize},从而能够学习到那些随时间变化较为缓慢的特征。
% 484


\gls{slow_principle}早于\gls{SFA},并已被应用于各种模型~\citep{Hinton89b,Foldiak89,MobahiCollobertWestonICML2009,Bergstra+Bengio-2009}。
一般来说,我们可以将\gls{slow_principle}应用于可以使用\gls{GD}训练的任何可微分模型。 
为了引入\gls{slow_principle},我们可以向\gls{cost_function}添加以下项
\begin{align}
\label{eqn:137}
\lambda \sum_t L(f(\Vx^{(t+1)}),f(\Vx^{(t)})),
\end{align}
其中$\lambda$是确定慢度\gls{regularization}强度的超参数项,$t$是样本时间序列的索引,$f$是需要\gls{regularize}的\gls{feature_extractor},$L$是测量$f(\Vx^{(t)})$和$f(\Vx^{(t+1)})$之间的距离的\gls{loss_function}。
$L$的一个常见选择是\gls{mean_squared_error}。
% 484


\gls{SFA}是\gls{slow_principle}中一个特别高效的应用。
由于它被应用于线性\gls{feature_extractor},并且可以通过\gls{closed_form_solution}训练,所以它是高效的。
像~\glssymbol{ICA}~的一些变种一样,\glssymbol{SFA}~本身并不是\gls{generative_model},只是在输入空间和特征空间之间定义了一个线性映射,但是没有定义特征空间的先验,因此没有在输入空间上施加分布$p(\Vx)$。
% 484



\glssymbol{SFA}~算法~\citep{WisSej2002}先将$f(\Vx;\theta)$定义为线性变换,然后求解如下优化问题
\begin{align}
	\label{eqn:138}
	\min_{\Vtheta} \SetE_t  (f(\Vx^{(t+1)})_i - f(\Vx^{(t)})_i  )^2
\end{align}
并且满足下面的约束:
\begin{align}
	\label{eqn:139}
	\SetE_t  f(\Vx^{(t)})_i = 0 
\end{align}
以及
\begin{align}
	\label{eqn:1310}
	\SetE_t [ f(\Vx^{(t)})_i^2 ] =1. 
\end{align} % 485
学习特征具有零均值的约束对于使问题具有唯一解是必要的; 
否则我们可以向所有特征值添加一个常数,
并获得具有相等慢度目标值的不同解。
特征具有单位方差的约束对于防止所有特征趋近于$0$的病态解是必要的。
与~\glssymbol{PCA}~类似,\glssymbol{SFA}~特征是有序的,其中学习第一特征是最慢的。
要学习多个特征,我们还必须添加约束
\begin{align}
\label{eqn:1311}
\forall i<j,\ \  \SetE_t [f(\Vx^{(t)})_i  f(\Vx^{(t)})_j] = 0.
\end{align}
这要求学习的特征必须彼此线性去相关。 
没有这个约束,所有学习到的特征将简单地捕获一个最慢的信号。
可以想象使用其他机制,如最小化\gls{reconstruction_error},也可以迫使特征多样化。
但是由于~\glssymbol{SFA}~特征的线性,这种去相关机制只能得到一种简单的解。 
\glssymbol{SFA}~问题可以通过线性代数软件获得\gls{closed_form_solution}。
% 485



在运行~\glssymbol{SFA}~之前,\glssymbol{SFA}~通常通过对$\Vx$使用非线性的基扩充来学习非线性特征。
例如,通常用$\Vx$的二次基扩充来代替原来的$\Vx$,得到一个包含所有$x_ix_j$的向量。
由此,我们可以通过反复地学习一个线性~\glssymbol{SFA}~\gls{feature_extractor},对其输出应用非线性基扩展,然后在该扩展之上学习另一个线性~\glssymbol{SFA}~\gls{feature_extractor}的方式来组合线性~\glssymbol{SFA}~模块从而学习深度非线性慢\gls{feature_extractor}。

% 485


当在自然场景视频的小块空间部分上训练时,使用二次基扩展的~\glssymbol{SFA}~所学习到的特征与V1皮层中那些复杂细胞的特征有许多共同特性~\citep{Berkes-Wiskott-2005}。
当在计算机渲染的3D环境内随机运动的视频上训练时,深度~\glssymbol{SFA}~模型能够学习的特征与大鼠脑中用于导航的神经元学到的特征有许多共同特性~\citep{franzius2007slowness}。
因此从生物学角度上来说~\glssymbol{SFA}~是一个合理的有依据的模型。
% 485



\glssymbol{SFA}~的一个主要优点是,即使在深度非线性条件下,它依然能够在理论上预测~\glssymbol{SFA}~能够学习哪些特征。
为了做出这样的理论预测,必须知道关于配置空间的环境动力(例如,在3D渲染环境中随机运动的例子中,理论分析是从相机位置、速度的概率分布中入手的)。
已知潜在因子如何改变的情况下,我们能够通过理论分析解出表达这些因子的最佳函数。
在实践中,基于模拟数据的实验上,使用深度~\glssymbol{SFA}~似乎能够恢复理论预测的函数。
相比之下,在其他学习算法中,\gls{cost_function}高度依赖于特定像素值,使得难以确定模型将学习到什么特征。
% 486


深度~\glssymbol{SFA}~也已经被用于学习用在\gls{object_recognition}和姿态估计的特征~\citep{Franzius2008}。
到目前为止,\gls{slow_principle}尚未成为任何最先进应用的基础。
究竟是什么因素限制了其性能仍有待研究。
我们推测,或许慢度先验太过强势,并且,最好添加这样一个先验使得当前\gls{time_step}到下一个\gls{time_step}的预测更加容易,而不是加一个先验使得特征近似为一个常数。
对象的位置是一个有用的特征,无论对象的速度是高还是低。 
但\gls{slow_principle}鼓励模型忽略具有高速度的对象的位置。
% 486


\section{\glsentrytext{sparse_coding}}
\label{sec:sparse_coding}
% 486


\firstgls{sparse_coding}~\citep{Olshausen+Field-1996}是一个\gls{linear_factor},已作为一种\gls{unsupervised}特征学习和特征提取机制得到了广泛研究。
严格来说,术语``\gls{sparse_coding}''是指在该模型中推断$\Vh$值的过程,而``稀疏建模''是指设计和学习模型的过程,但是通常这两个概念都可以用术语``\gls{sparse_coding}''描述。
% 486

像大多数其他\gls{linear_factor}一样,它使用了线性的\gls{decoder}加上噪声的方式获得一个$\Vx$的重构,就像\eqnref{eqn:132}描述的一样。
更具体地说,\gls{sparse_coding}模型通常假设线性因子有一个各向同性精度为$\beta$的高斯噪声:
\begin{align}
\label{eqn:1312}
p(\Vx\mid \Vh) = \CalN
(\Vx;\MW\Vh + \Vb ,\frac{1}{\beta}\MI).
\end{align}
% 486


分布$p(\Vh)$通常选取为一个峰值很尖锐且接近$0$的分布~\citep{Olshausen+Field-1996}。
常见的选择包括可分解的Laplace、Cauchy或者可分解的Student-t分布。
例如,以稀疏惩罚系数$\lambda$为参数的Laplace先验可以表示为
\begin{align}
\label{eqn:1313}
p(h_i) = \text{Laplace}(h_i;0,\frac{2}{\lambda}) = \frac{\lambda}{4} \text{e}^{ -\frac{1}{2}\lambda \vert h_i\vert},
\end{align}
相应的,Student-t先验分布可以表示为
\begin{align}
\label{eqn:1314}
p(h_i)\propto \frac{1}{(1+\frac{h_i^2}{\nu})^{\frac{\nu+1}{2}}}.
\end{align}
% 487 head

使用\gls{maximum_likelihood}的方法来训练\gls{sparse_coding}模型是不可行的。
相反,为了在给定编码的情况下更好地重构数据,训练过程在编码数据和训练\gls{decoder}之间交替进行。
稍后在\secref{sec:map_inference_and_sparse_coding}中,这种方法将被进一步证明为是解决\gls{maximum_likelihood}问题的一种通用的近似方法。
% 487

对于诸如~\glssymbol{PCA}~的模型,我们已经看到使用了预测$\Vh$的参数化的\gls{encoder}函数,并且该函数仅包括乘以权重矩阵。
\gls{sparse_coding}中的\gls{encoder}不是参数化的\gls{encoder}。
相反,\gls{encoder}是一个优化算法,在这个优化问题中,我们寻找单个最可能的编码值:
\begin{align}
\label{eqn:1315}
\Vh^* = f(\Vx) = \underset{\Vh}{\arg\max}\  p(\Vh\mid\Vx).
\end{align} % 487
结合\eqnref{eqn:1313}和\eqnref{eqn:1312},我们得到如下的优化问题:
\begin{align}
\label{eqn:1316}
& \underset{\Vh}{\arg\max}\  p(\Vh\mid\Vx) \\
= ~& \underset{\Vh}{\arg\max}\ \log  p(\Vh\mid\Vx)\\
= ~& \underset{\Vh}{\arg\min}\ \lambda \Vert \Vh\Vert_1 + \beta  \Vert \Vx - \MW \Vh\Vert_2^2,
\end{align}
其中,我们扔掉了与$\Vh$无关的项,并除以一个正的缩放因子来简化表达。
% 487

由于在$\Vh$上施加$L^1$范数,这个过程将产生稀疏的$\Vh^*$(详见\secref{sec:l1_regularization})。
% 487


为了训练模型而不仅仅是进行推断,我们交替迭代关于$\Vh$和$\MW$的最小化过程。
在本文中,我们将$\beta$视为超参数。
我们通常将其设置为$1$,因为它在此优化问题的作用与$\lambda$类似,没有必要使用两个超参数。 
原则上,我们还可以将$\beta$作为模型的参数,并学习它。
我们在这里已经放弃了一些不依赖于$\Vh$但依赖于$\beta$的项。
要学习$\beta$,必须包含这些项,否则$\beta$将退化为$0$。
% 487


不是所有的\gls{sparse_coding}方法都显式地构建了一个$p(\Vh)$和一个$p(\Vx\mid\Vh)$。 
通常我们只是对学习一个带有激活值的特征的字典感兴趣,当特征是由这个推断过程提取时,这个激活值通常为$0$。
% 487 end

如果我们从Laplace先验中采样$\Vh$,$\Vh$的元素实际上为$0$是一个零概率事件。
\gls{generative_model}本身并不稀疏,只有\gls{feature_extractor}是稀疏的。
\citet{Goodfeli-et-al-TPAMI-Deep-PrePrint-2013-small}描述了不同模型族中的近似推断,如\gls{ss}\gls{sparse_coding}模型,其中先验的样本通常包含许多真正的$0$。
% 488 head

与非参数\gls{encoder}结合的\gls{sparse_coding}方法原则上可以比任何特定的参数化\gls{encoder}更好地最小化重构误差和对数先验的组合。
另一个优点是\gls{encoder}没有\gls{generalization_error}。
参数化的\gls{encoder}必须泛化地学习如何将$\Vx$映射到$\Vh$。
对于与训练数据差异很大的异常$\Vx$,所学习的参数化\gls{encoder}可能无法找到对应精确重构或稀疏的编码$\Vh$。
对于\gls{sparse_coding}模型的绝大多数形式,推断问题是凸的,优化过程总能找到最优编码(除非出现退化的情况,例如重复的权重向量)。
显然,稀疏和重构成本仍然可以在不熟悉的点上升,但这归因于\gls{decoder}权重中的\gls{generalization_error},而不是\gls{encoder}中的\gls{generalization_error}。
当\gls{sparse_coding}用作分类器的\gls{feature_extractor},而不是使用参数化的函数来预测编码值时,基于优化的\gls{sparse_coding}模型的编码过程中较小的\gls{generalization_error}可以得到更好的\gls{generalization}能力。
\citet{Coates2011b}证明了在\gls{object_recognition}任务中\gls{sparse_coding}特征比基于参数化的\gls{encoder}(线性-\gls{sigmoid}~\gls{AE})的特征拥有更好的泛化能力。
受他们的工作启发,\citet{Goodfeli-et-al-TPAMI-Deep-PrePrint-2013-small}表明一种\gls{sparse_coding}的变体在标签极少(每类20个或更少标签)的情况中比相同情况下的其他\gls{feature_extractor}拥有更好的\gls{generalization}能力。
% 488 



非参数\gls{encoder}的主要缺点是在给定$\Vx$的情况下需要大量的时间来计算$\Vh$,因为非参数方法需要运行迭代算法。
在\chapref{chap:autoencoders}中讲到的参数化\gls{AE}方法仅使用固定数量的层,通常只有一层。
另一个缺点是它不直接通过非参数\gls{encoder}进行\gls{back_propagate},这使得我们很难采用先使用\gls{unsupervised}方式\gls{pretraining}\gls{sparse_coding}模型然后使用\gls{supervised}方式对其进行\gls{fine_tuning}的方法。
允许近似导数的\gls{sparse_coding}模型的修改版本确实存在但未被广泛使用~\citep{Bradley+Bagnell-2009-small}。
% 488  end

像其他\gls{linear_factor}一样,\gls{sparse_coding}经常产生糟糕的样本,如\figref{fig:s3c_samples}所示。
即使当模型能够很好地重构数据并为分类器提供有用的特征时,也会发生这种情况。
这种现象发生的原因是每个单独的特征可以很好地被学习到,但是隐藏编码值的\gls{factorial}先验会导致模型包括每个生成样本中所有特征的随机子集。
这促使人们开发更深的模型,可以在其中最深的编码层施加一个非\gls{factorial}分布, 与此同时也在开发一些复杂的浅度模型。

\begin{figure}[!htb]
\ifOpenSource
\centerline{\includegraphics{figure.pdf}}
\else
    \centerline{\includegraphics[width=\figwidth]{Chapter13/figures/s3c_samples}}
\fi
\caption{\gls{ss}\gls{sparse_coding}模型上在MNIST数据集训练的样例和权重。
\emph{(左)}这个模型中的样本和训练样本相差很大。
第一眼看来,我们可能认为模型拟合得很差。
\emph{(右)}这个模型的权重向量已经学习到了如何表示笔迹,有时候还能写完整的数字。
因此这个模型也学习到了有用的特征。
问题在于特征的\gls{factorial}先验会导致特征子集合随机的组合。
一些这样的子集能够合成可识别的MNIST集上的数字。
这也促进了拥有更强大\gls{latent}编码分布的\gls{generative_model}的发展。
此图经~\citet{Goodfeli-et-al-TPAMI-Deep-PrePrint-2013-small}允许转载。}
\label{fig:s3c_samples}
\end{figure}

%这促进了更深层模型的发展,可以在最深层上施加non-factorial分布,以及开发更复杂的浅层模型。
% 489 head


\section{\glssymbol{PCA}的\glsentrytext{manifold}解释}
\label{sec:manifold_interpretation_of_pca}
% 489 au


\gls{linear_factor},包括~\glssymbol{PCA}~和\gls{FA},可以理解为学习一个\gls{manifold}~\citep{hinton97modelling}。
我们可以将\gls{PPCA}~定义为高概率的薄饼状区域,即一个\gls{gaussian_distribution},沿着某些轴非常窄,就像薄饼沿着其垂直轴非常平坦,但沿着其他轴是细长的,正如薄饼在其水平轴方向是很宽的一样。
\figref{fig:PPCA_pancake}解释了这种现象。
\glssymbol{PCA}~可以理解为将该薄饼与更高维空间中的线性\gls{manifold}对准。
这种解释不仅适用于传统~\glssymbol{PCA},而且适用于学习矩阵$\MW$和$\MV$的任何线性\gls{AE},其目的是使重构的$\Vx$尽可能接近于原始的$\Vx$。
% 489 end

\begin{figure}[!htb]
\ifOpenSource
\centerline{\includegraphics{figure.pdf}}
\else
	\centerline{\includegraphics{Chapter13/figures/PPCA_pancake_color}}
\fi
\caption{平坦的高斯能够描述一个低维\gls{manifold}附近的概率密度。
此图表示了``\gls{manifold}平面''上``馅饼''的上半部分,并且这个平面穿过了馅饼的中心。
正交于\gls{manifold}方向(指向平面外的箭头方向)的方差非常小,可以被视作是``噪声'',其他方向(平面内的箭头)的方差则很大,对应了``信号''以及降维数据的坐标系统。}
\label{fig:PPCA_pancake}
\end{figure}


\gls{encoder}表示为
\begin{align}
\label{eqn:1319}
\Vh  = f(\Vx) = \MW^{\top} (\Vx - \Vmu).
\end{align}% 490 head
\gls{encoder}计算$h$的低维表示。
从\gls{AE}的角度来看,\gls{decoder}负责计算重构:
\begin{align}
\label{eqn:1320}
\hat{\Vx} = g(\Vh) = \Vb + \MV \Vh.
\end{align}
% 490


能够最小化\gls{reconstruction_error}
\begin{align}
\label{eqn:1321}
\SetE[\Vert\Vx - \hat{\Vx}\Vert^2]
\end{align}
的线性\gls{encoder}和\gls{decoder}的选择对应着$\MV = \MW$,${\Vmu} = \Vb = \SetE[\Vx]$, $\MW$的列 形成一组标准正交基,这组基生成的子空间与\gls{covariance}矩阵$\MC$
\begin{align}
\label{eqn:1322}
\MC = \SetE[(\Vx - {\Vmu})(\Vx - {\Vmu})^{\top}]
\end{align}
的主特征向量所生成的子空间相同。
在~\glssymbol{PCA}~中,$\MW$的列是按照对应特征值(其全部是实数和非负数)幅度大小排序所对应的特征向量。
% 490 end

我们还可以发现$\MC$的特征值$\lambda_i$对应了$\Vx$在特征向量$\Vv^{(i)}$方向上的方差。
如果$\Vx\in \SetR^D$,$\Vh\in\SetR^d$并且满足$d<D$,则(给定上述的${\Vmu},\Vb,\MV,\MW$的情况下)最佳的\gls{reconstruction_error}是
\begin{align}
\label{eqn:1323}
\min \SetE[\Vert \Vx - \hat{\Vx} \Vert^2] = \sum_{i=d+1}^{D}\lambda_i.
\end{align}
因此,如果\gls{covariance_matrix}的秩为$d$,则特征值$\lambda_{d+1}$到$\lambda_{D}$都为$0$,并且\gls{reconstruction_error}为$0$。
% 491 head 

此外,我们还可以证明上述解可以通过在给定正交矩阵$\MW$的情况下最大化$\Vh$元素的方差而不是最小化\gls{reconstruction_error}来获得。
% 491 


某种程度上说,\gls{linear_factor}是最简单的\gls{generative_model}和学习数据表示的最简单模型。
许多模型如\gls{linear_classifier}和\gls{linear_regression}模型可以扩展到\gls{deep_feedforward_network},而这些\gls{linear_factor}可以扩展到\gls{AE}网络和深度概率模型,它们可以执行相同任务但具有更强大和更灵活的模型族。
% 491   

% !Mode:: "TeX:UTF-8"
% Translator: Shenjian Zhao
\chapter{\glsentrytext{AE}}
\label{chap:autoencoders}
\firstgls{AE}是\gls{NN}的一种,经过训练后能尝试将输入复制到输出。
\firstgls{AE}内部有一个\gls{hidden_layer} $\Vh$,可以产生\firstgls{code}表示输入。
该网络可以看作由两部分组成:一个由函数$ \Vh = f(\Vx)$表示的\gls{encoder}和一个生成\gls{reconstruction}的\gls{decoder} $\Vr=g(\Vh)$。
\figref{fig:chap14_autoencoder}展示了这种架构。
如果一个\gls{AE}只是简单地学会将处处设置为$g(f(\Vx)) =\Vx$,那么这个\gls{AE}就没什么特别的用处。
相反,我们不应该将\gls{AE}设计成输入到输出完全相等。
这通常需要向\gls{AE}强加一些约束,使它只能近似地复制,并只能复制与训练数据相似的输入。
这些约束强制模型考虑输入数据的哪些部分需要被优先复制,因此它往往能学习到数据的有用特性。


现代\gls{AE}将\gls{encoder}和\gls{decoder}的概念推而广之,将其中的确定函数推广为随机映射$p_{\text{encoder}} (\Vh \mid \Vx)$和$p_{\text{decoder}}(\Vx \mid \Vh)$。


数十年间,\gls{AE}的想法一直是\gls{NN}历史景象的一部分~\citep{Lecun-these87,Bourlard88,hinton1994amd-small}。
传统\gls{AE}被用于\gls{dimensionality_reduction}或特征学习。
近年来,\gls{AE}与\gls{latent_variable}模型理论的联系将\gls{AE}带到了生成式建模的前沿,我们将在\chapref{chap:deep_generative_models}揭示更多细节。
\gls{AE}可以被看作是\gls{feedforward_network}的一个特例,并且可以使用完全相同的技术进行训练,通常使用\gls{minibatch}\gls{GD}法(其中梯度基于\gls{back_propagation}计算)。
不同于一般的\gls{feedforward_network},\gls{AE}也可以使用\firstgls{recirculation}训练\citep{Hinton+McClelland-NIPS1987},这种学习算法基于比较原始输入的激活和\gls{reconstruction}输入的激活。
相比\gls{back_propagation}算法,\gls{recirculation}算法更具生物学意义,但很少用于\gls{ML}应用。

\begin{figure}[!htb]
\ifOpenSource
\centerline{\includegraphics{figure.pdf}}
\else
\centerline{\includegraphics{Chapter14/figures/autoencoder}}
\fi
\caption{\gls{AE}的一般结构,通过内部表示或\gls{code} $\Vh$将输入$\Vx$映射到输出(称为\gls{reconstruction})$\Vr$。
\gls{AE}具有两个组件:\gls{encoder}~$f$(将$\Vx$映射到$\Vh$)和\gls{decoder}~$g$(将$\Vh$映射到$\Vr$)。
}
\label{fig:chap14_autoencoder}
\end{figure}

% -- 493 --

\section{\glsentrytext{undercomplete}\glsentrytext{AE}}
\label{sec:undercomplete_autoencoders}
将输入复制到输出听起来没什么用,但我们通常不关心\gls{decoder}的输出。
相反,我们希望通过训练\gls{AE}对输入进行复制而使$\Vh$获得有用的特性。


从\gls{AE}获得有用特征的一种方法是限制$\Vh$的维度比$\Vx$小,这种\gls{code}维度小于输入维度的\gls{AE}称为\firstgls{undercomplete}\gls{AE}。
学习\gls{undercomplete}的\gls{representation}将强制\gls{AE}捕捉训练数据中最显著的特征。


学习过程可以简单地描述为最小化一个\gls{loss_function} 
\begin{align}
    L(\Vx, g(f(\Vx))),
\end{align}
其中$L$是一个\gls{loss_function},惩罚$g(f(\Vx))$与$\Vx$的差异,如\gls{mean_squared_error}。


当\gls{decoder}是线性的且$L$是\gls{mean_squared_error},\gls{undercomplete}的\gls{AE}会学习出与~\glssymbol{PCA}~相同的生成子空间。
这种情况下,\gls{AE}在训练来执行复制任务的同时学到了训练数据的主元子空间。


因此,拥有非线性\gls{encoder}函数$f$和非线性\gls{decoder}函数$g$的\gls{AE}能够学习出更强大的~\glssymbol{PCA}~非线性推广。
不幸的是,如果\gls{encoder}和\gls{decoder}被赋予过大的\gls{capacity},\gls{AE}会执行复制任务而捕捉不到任何有关数据分布的有用信息。
从理论上说,我们可以设想这样一个\gls{AE},它只有一维\gls{code},但它具有一个非常强大的非线性\gls{encoder},能够将每个训练数据$\Vx^{(i)}$表示为\gls{code}~$i$。
而\gls{decoder}可以学习将这些整数索引映射回特定训练样本的值。
这种特定情形不会在实际情况中发生,但它清楚地说明,如果\gls{AE}的\gls{capacity}太大,那训练来执行复制任务的\gls{AE}可能无法学习到数据集的任何有用信息。

% -- 494 --

\section{正则\glsentrytext{AE}}
\label{sec:regularized_autoencoders}
\gls{code}维数小于输入维数的\gls{undercomplete}\gls{AE}可以学习数据分布最显著的特征。
我们已经知道,如果赋予这类\gls{AE}过大的\gls{capacity},它就不能学到任何有用的信息。


如果隐藏\gls{code}的维数允许与输入相等,或隐藏\gls{code}维数大于输入的\firstgls{overcomplete}情况下,会发生类似的问题。
在这些情况下,即使是线性\gls{encoder}和线性\gls{decoder}也可以学会将输入复制到输出,而学不到任何有关数据分布的有用信息。


理想情况下,根据要建模的数据分布的复杂性,选择合适的\gls{code}维数和\gls{encoder}、\gls{decoder}\gls{capacity},就可以成功训练任意架构的\gls{AE}。
正则\gls{AE}提供这样的能力。
正则\gls{AE}使用的\gls{loss_function}可以鼓励模型学习其他特性(除了将输入复制到输出),而不必限制使用浅层的\gls{encoder}和\gls{decoder}以及小的\gls{code}维数来限制模型的\gls{capacity}。
这些特性包括\gls{sparse}\gls{representation}、\gls{representation}的小导数、以及对噪声或输入缺失的鲁棒性。
即使模型\gls{capacity}大到足以学习一个无意义的恒等函数,非线性且\gls{overcomplete}的正则\gls{AE}仍然能够从数据中学到一些关于数据分布的有用信息。


除了这里所描述的方法(\gls{regularization}\gls{AE}最自然的解释),几乎任何带有\gls{latent_variable}并配有一个\gls{inference}过程(计算给定输入的\gls{latent}表示)的\gls{generative_model},都可以看作是\gls{AE}的一种特殊形式。
强调与\gls{AE}联系的两个生成式建模方法是~\gls{helmholtz_machine}~\citep{Hinton95}的衍生模型,如\gls{VAE}(\secref{sec:variational_autoencoders})和\gls{GSN}(\secref{sec:generative_stochastic_networks})。
这些变种(或衍生)\gls{AE}能够学习出高\gls{capacity}且\gls{overcomplete}的模型,进而发现输入数据中有用的结构信息,并且也无需对模型进行\gls{regularization}。
这些\gls{code}显然是有用的,因为这些模型被训练为近似训练数据的概率分布而不是将输入复制到输出。 % confusing

% -- 495 --

\subsection{\glsentrytext{sparse}\glsentrytext{AE}}
\label{sec:sparse_autoencoders}
\gls{sparse}\gls{AE}简单地在训练时结合\gls{code}层的\gls{sparse}惩罚$\Omega(\Vh)$和\gls{reconstruction_error}:
\begin{align}
    L(\Vx, g(f(\Vx))) + \Omega(\Vh),
\end{align}
其中$g(\Vh)$是\gls{decoder}的输出,通常$\Vh$是\gls{encoder}的输出,即$\Vh = f(\Vx)$。


\gls{sparse}\gls{AE}一般用来学习特征,以便用于像分类这样的任务。
\gls{sparse}\gls{regularization}的\gls{AE}必须反映训练数据集的独特统计特征,而不是简单地充当恒等函数。
以这种方式训练,执行附带\gls{sparse}惩罚的复制任务可以得到能学习有用特征的模型。


我们可以简单地将惩罚项$\Omega(\Vh)$视为加到\gls{feedforward_network}的正则项,这个\gls{feedforward_network}的主要任务是将输入复制到输出(\gls{unsupervised_learning}的目标),并尽可能地根据这些\gls{sparse}特征执行一些\gls{supervised_learning}任务(根据\gls{supervised_learning}的目标)。
不像其它正则项如\gls{weight_decay}——没有直观的贝叶斯解释。
如\secref{sec:maximum_a_posteriori_map_estimation}描述,\gls{weight_decay}和其他正则惩罚可以被解释为一个~\glssymbol{MAP}~近似贝叶斯\gls{inference},\gls{regularization}的惩罚对应于模型参数的先验概率分布。
这种观点认为,\gls{regularization}的最大似然对应最大化$p(\Vtheta \mid \Vx)$, 相当于最大化$\log p(\Vx \mid \Vtheta) + \log p(\Vtheta)$。 $\log p(\Vx \mid \Vtheta)$即通常的数据似然项,参数的对数先验项$\log p(\Vtheta)$则包含了对$\Vtheta$特定值的偏好。
这种观点在\secref{sec:bayesian_statistics}有所描述。
正则\gls{AE}不适用这样的解释是因为正则项取决于数据,因此根据定义上(从文字的正式意义)来说,它不是一个先验。
虽然如此,我们仍可以认为这些正则项隐式地表达了对函数的偏好。

% -- 496 --

我们可以认为整个\gls{sparse}\gls{AE}框架是对带有\gls{latent_variable}的\gls{generative_model}的近似最大似然训练,而不将\gls{sparse}惩罚视为复制任务的\gls{regularization}。
假如我们有一个带有可见变量$\Vx$和\gls{latent_variable} $\Vh$的模型,且具有明确的联合分布$p_{\text{model}}(\Vx,\Vh)=p_{\text{model}}(\Vh)p_{\text{model}} (\Vx \mid \Vh)$。
我们将$p_{\text{model}}(\Vh)$视为模型关于\gls{latent_variable}的先验分布,表示模型看到$\Vx$的信念先验。
这与我们之前使用``先验''的方式不同,之前指分布$p(\Vtheta)$在我们看到数据前就对模型参数的先验进行编码。
对数似然函数可分解为
\begin{align}
\log p_{\text{model}}(\Vx)=\log \sum_{\Vh} p_{\text{model}}(\Vh, \Vx) .
\end{align}
我们可以认为\gls{AE}使用一个高似然值$\Vh$的点估计近似这个总和。
这类似于\gls{sparse_coding}\gls{generative_model}(\secref{sec:sparse_coding}),但$\Vh$是参数\gls{encoder}的输出,而不是从优化结果推断出的最可能的$\Vh$。
从这个角度看,我们根据这个选择的$\Vh$,最大化如下
\begin{align}
\log p_{\text{model}}(\Vh, \Vx)=\log p_{\text{model}}(\Vh) + \log p_{\text{model}}(\Vx \mid \Vh) .
\end{align}
$\log p_{\text{model}}(\Vh) $ 项能被\gls{sparse}诱导。
如\ENNAME{Laplace}先验,
\begin{align}
p_{\text{model}}(h_i) = \frac{\lambda}{2} e^{-\lambda | h_i |},
\end{align}
对应于绝对值\gls{sparse}惩罚。
将对数先验表示为绝对值惩罚,我们得到
\begin{align}
\Omega(\Vh) &= \lambda \sum_{i} | h_i  |,\\ 
-\log p_{\text{model}}(\Vh) &= 
\sum_i (\lambda | h_i | - \log \frac{\lambda}{2}) = \Omega(\Vh) + \text{const},
\end{align}
这里的常数项只跟$\lambda$有关。
通常我们将$\lambda$视为超参数,因此可以丢弃不影响参数学习的常数项。
其他如~\ENNAME{Student-t}~先验也能诱导\gls{sparse}性。
从\gls{sparse}性导致$p_{\text{model}}(\Vh)$学习成近似最大似然的结果看,\gls{sparse}惩罚完全不是一个正则项。
这仅仅影响模型关于\gls{latent_variable}的分布。
这个观点提供了训练\gls{AE}的另一个动机:这是近似训练\gls{generative_model}的一种途径。
这也给出了为什么\gls{AE}学到的特征是有用的另一个解释:它们描述的\gls{latent_variable}可以解释输入。

% -- 497 --

\gls{sparse}\gls{AE}的早期工作~\citep{ranzato-07-small,ranzato-08-small}探讨了各种形式的\gls{sparse}性,并提出了\gls{sparse}惩罚和$\log  Z$项(将最大似然应用到无向概率模型$p(\Vx)=\frac{1}{Z}\tilde{p}(\Vx)$时产生)之间的联系。
这个想法是最小化$\log Z$防止概率模型处处具有高概率,同理强制\gls{sparse}可以防止\gls{AE}处处具有低的\gls{reconstruction_error} 。
这种情况下,这种联系是对通用机制的直观理解而不是数学上的对应。
在数学上更容易解释\gls{sparse}惩罚对应于有向模型$p_{\text{model}}(\Vh)p_{\text{model}}(\Vx \mid \Vh) $中的$\log p_{\text{model}}(\Vh)$。


\citet{Glorot+al-ICML-2011-small}提出了一种在\gls{sparse}(和\gls{denoising})\gls{AE}的$\Vh$中实现\emph{真正为零}的方式。
该想法是使用\gls{ReLU}产生\gls{code}层。
基于将\gls{representation}真正推向零(如绝对值惩罚)的先验,可以间接控制\gls{representation}中零的平均数量。



\subsection{\glsentrytext{DAE}}
\label{sec:sub_denoising_autoencoders}
除了向\gls{cost_function}增加一个惩罚项,我们也可以通过改变\gls{reconstruction_error}项来获得一个能学到有用信息的\gls{AE}。


传统的\gls{AE}最小化以下目标
\begin{align}
    L(\Vx, g(f(\Vx))),
\end{align}
其中$L$是一个\gls{loss_function},惩罚$g(f(\Vx))$与$\Vx$的差异,如它们彼此差异的$L^2$范数。
如果模型被赋予过大的\gls{capacity},$L$仅仅使得$g \circ  f$学成一个恒等函数。


相反,\firstall{DAE}最小化 
\begin{align}
    L(\Vx, g(f(\tilde \Vx))),
\end{align}
其中 $\tilde \Vx$是被某种噪声损坏的$\Vx$的副本。
因此\gls{DAE}必须撤消这些损坏,而不是简单地复制输入。

\citet{Alain+Bengio-ICLR2013-small}和~\citet{Bengio-et-al-NIPS2013-small}指出\gls{denoising}训练过程强制$f$和$g$隐式地学习$p_{\text{data}} (\Vx)$的结构。
因此\gls{DAE}也是一个通过最小化\gls{reconstruction_error}获取有用特性的例子。
这也是将\gls{overcomplete}、高\gls{capacity}的模型用作\gls{AE}的一个例子——只要小心防止这些模型仅仅学习一个恒等函数。
\gls{DAE}将在\secref{sec:denoising_autoencoders}给出更多细节。

% -- 498 --

\subsection{惩罚导数作为正则}
\label{sec:regularizing_by_penalizing_derivatives}
另一正则化\gls{AE}的策略是使用一个类似\gls{sparse}\gls{AE}中的惩罚项$\Omega$,
\begin{align}
    L(\Vx, g(f(\Vx))) + \Omega(\Vh, \Vx),
\end{align}
但$\Omega$的形式不同:
\begin{align}
\Omega(\Vh, \Vx) = \lambda \sum_i \| \nabla_{\Vx}h_i \|^2.
\end{align}


这迫使模型学习一个在$\Vx$变化小时目标也没有太大变化的函数。
因为这个惩罚只对训练数据适用,它迫使\gls{AE}学习可以反映训练数据分布信息的特征。


这样\gls{regularization}的\gls{AE}被称为\firstall{CAE}。
这种方法与\gls{DAE}、\gls{manifold_learning}和概率模型存在一定理论联系。
\gls{CAE}将在\secref{sec:contractive_autoencoders}更详细地描述。


\section{表示能力、层的大小和深度}
\label{sec:representational_power_layer_size_and_depth}
\gls{AE}通常只有单层的\gls{encoder}和\gls{decoder},但这不是必然的。
实际上深度\gls{encoder}和\gls{decoder}能提供更多优势。


回忆\secref{sec:universal_approximation_properties_and_depth},其中提到加深\gls{feedforward_network}有很多优势。
这些优势也同样适用于\gls{AE},因为它也属于\gls{feedforward_network}。
此外,\gls{encoder}和\gls{decoder}各自都是一个\gls{feedforward_network},因此这两个部分也能各自从深度结构中获得好处。


\gls{universal_approximation_theorem}保证至少有一层\gls{hidden_layer}且\gls{hidden_unit}足够多的\gls{feedforward_neural_network}能以任意精度近似任意函数(在很大范围里),这是非平凡深度(至少有一层\gls{hidden_layer})的一个主要优点。
这意味着具有单\gls{hidden_layer}的\gls{AE}在数据域内能表示任意近似数据的恒等函数。
但是,从输入到\gls{code}的映射是浅层的。
这意味这我们不能任意添加约束,比如约束\gls{code}\gls{sparse}。
深度\gls{AE}(\gls{encoder}至少包含一层额外\gls{hidden_layer})在给定足够多的\gls{hidden_unit}的情况下,能以任意精度近似任何从输入到\gls{code}的映射。

% -- 499 --

深度可以指数地降低表示某些函数的计算成本。
深度也能指数地减少学习一些函数所需的训练数据量。
读者可以参考\secref{sec:universal_approximation_properties_and_depth}巩固深度在\gls{feedforward_network}中的优势。


实验中,深度\gls{AE}能比相应的浅层或线性\gls{AE}产生更好的压缩效率\citep{Hinton-Science2006}。

训练深度\gls{AE}的普遍策略是训练一堆浅层的\gls{AE}来贪心地预训练相应的深度架构。
所以即使最终目标是训练深度\gls{AE},我们也经常会遇到浅层\gls{AE}。


\section{随机\glsentrytext{encoder}和\glsentrytext{decoder}}
\label{sec:stochastic_encoders_and_decoders}
\gls{AE}本质上是一个\gls{feedforward_network},可以使用与传统\gls{feedforward_network}相同的\gls{loss_function}和输出单元。


如\secref{sec:other_output_types}中描述,设计\gls{feedforward_network}的输出单元和\gls{loss_function}普遍策略是定义一个输出分布$p(\Vy \mid \Vx) $并最小化负对数似然$-\log p(\Vy \mid \Vx)$。
在这种情况下,$\Vy$是关于目标的向量(如类标)。


在\gls{AE}中,$\Vx$既是输入也是目标。
然而,我们仍然可以使用与之前相同的架构。
给定一个隐藏\gls{code} $\Vh$,我们可以认为\gls{decoder}提供了一个条件分布$p_{\text{model}}(\Vx \mid \Vh)$。
接着我们根据最小化$-\log p_{\text{decoder}}(\Vx \mid \Vh)$来训练\gls{AE}。
\gls{loss_function}的具体形式视$p_{\text{decoder}}$的形式而定。
就传统的\gls{feedforward_network}来说,如果$\Vx$是实值的,那么我们通常使用线性输出单元参数化高斯分布的均值。
在这种情况下,负对数似然对应\gls{mean_squared_error}\gls{criterion}。
类似地,二值$\Vx$对应于一个~\gls{bernoulli_distribution},其参数由~\ENNAME{sigmoid}~输出单元确定的。
而离散的$\Vx$对应~\ENNAME{softmax}~分布,以此类推。
在给定$\Vh$的情况下,为了便于计算概率分布,输出变量通常被视为是条件独立的,但一些技术(如混合密度输出)可以解决输出相关的建模。

% -- 500 --

为了更彻底地与我们之前了解到的\gls{feedforward_network}相区别,我们也可以将\textbf{编码函数}(encoding function)~$f(\Vx)$的概念推广为\textbf{编码分布}(encoding distribution)~$ p_{\text{encoder}}(\Vh \mid \Vx)$, 如\figref{fig:chap14_stochastic-autoencoder}中所示。

\begin{figure}[!htb]
\ifOpenSource
\centerline{\includegraphics{figure.pdf}}
\else
\centerline{\includegraphics{Chapter14/figures/stochastic-autoencoder}}
\fi
\caption{随机\gls{AE}的结构,其中\gls{encoder}和\gls{decoder}包括一些噪声注入,而不是简单的函数。
这意味着可以将它们的输出视为来自分布的采样(对于\gls{encoder}是$p_{\text{encoder}}(\Vh \mid \Vx)$,对于\gls{decoder}是$p_{\text{decoder}}(\Vx\mid \Vh)$)。}
\label{fig:chap14_stochastic-autoencoder}
\end{figure}

任何\gls{latent_variable}模型$p_{\text{model}}(\Vh, \Vx)$定义一个随机\gls{encoder}
\begin{align}
p_{\text{encoder}}(\Vh \mid \Vx) = p_{\text{model}}(\Vh\mid\Vx)
\end{align}
以及一个随机\gls{decoder}
\begin{align}
p_{\text{decoder}}(\Vx \mid \Vh) = p_{\text{model}}(\Vx\mid\Vh).
\end{align}
通常情况下,\gls{encoder}和\gls{decoder}的分布没有必要是与唯一一个联合分布$p_{\text{model}}(\Vx, \Vh)$相容的条件分布。
\citet{Alain-et-al-arxiv2015}指出,在保证足够的\gls{capacity}和样本的情况下,将\gls{encoder}和\gls{decoder}作为\gls{DAE}训练,能使它们渐近地相容。



\section{\glsentrytext{DAE}}
\label{sec:denoising_autoencoders}
\firstall{DAE}是一类接受损坏数据作为输入,并训练来预测原始未被损坏数据作为输出的\gls{AE}。

% -- 501 --

\glssymbol{DAE}~的训练过程如\figref{fig:chap14_DAE}中所示。
我们引入一个损坏过程$C(\tilde{\RVx} \mid \RVx)$,这个条件分布代表给定数据样本$\RVx$产生损坏样本$\tilde \RVx$的概率。
\gls{AE}则根据以下过程,从训练数据对$(\Vx, \tilde \Vx)$中学习\textbf{重构分布}(reconstruction distribution)~$p_{\text{reconstruct}} (\RVx \mid \tilde \RVx)$:
\begin{enumerate}
\item 从训练数据中采一个训练样本$\Vx$。
\item 从$C(\tilde{\RVx} \mid \RVx=\Vx)$采一个损坏样本$\tilde \Vx$。
\item 将$(\Vx, \tilde \Vx)$作为训练样本来估计\gls{AE}的\gls{reconstruction}分布 
$p_{\text{reconstruct}} (\Vx \mid \tilde \Vx) = p_{\text{decoder}}(\Vx \mid\Vh)$,其中$\Vh$是\gls{encoder} $f(\tilde \Vx)$的输出,$p_{\text{decoder}}$根据解码函数$g(\Vh)$定义。
\end{enumerate}
通常我们可以简单地对负对数似然$-\log p_{\text{decoder}} (\Vx \mid \Vh)$进行基于梯度法(如\gls{minibatch}\gls{GD})的近似最小化。
只要\gls{encoder}是确定性的,\gls{DAE}就是一个\gls{feedforward_network},并且可以使用与其他\gls{feedforward_network}完全相同的方式进行训练。

\begin{figure}[!htb]
\ifOpenSource
\centerline{\includegraphics{figure.pdf}}
\else
\centerline{\includegraphics{Chapter14/figures/DAE}}
\fi
\caption{\gls{DAE}\gls{cost_function}的计算图。\gls{DAE}被训练为从损坏的版本$\tilde \Vx$~\gls{reconstruction}干净数据点$\Vx$。
这可以通过最小化损失$L = - \log p_{\text{decoder}} (\Vx \mid \Vh = f(\tilde \Vx))$实现,其中$\tilde \Vx$是样本$\Vx$经过损坏过程$C (\tilde \Vx \mid \Vx)$后得到的损坏版本。
通常, 分布$p_{\text{decoder}}$是因子的分布(平均参数由前馈网络$g$给出)。
}
\label{fig:chap14_DAE}
\end{figure}

因此我们可以认为~\glssymbol{DAE}~是在以下期望下进行\gls{SGD}:
\begin{align}
   - \SetE_{\RVx \sim \hat{p}_{\text{data}}(\RVx)} \SetE_{\tilde{\RVx} \sim C(\tilde{\RVx}\mid\Vx)} \log p_{\text{decoder}}(\Vx \mid \Vh = f(\tilde{\Vx})),
\end{align}
其中$\hat{p}_{\text{data}}(\Vx)$是训练数据的分布。

% -- 502 --

\subsection{\glsentrytext{score}估计}
\label{sec:estimating_the_score}
\gls{score_matching}~\citep{Hyvarinen-2005}是最大似然的代替。
它提供了概率分布的一致估计,促使模型在各个数据点$\Vx$上获得与数据分布相同的\firstgls{score}。
在这种情况下,\gls{score}是一个特定的梯度场:
\begin{align}
 \nabla_{\Vx} \log p(\Vx) .
\end{align}

我们将在\secref{sec:score_matching_and_ratio_matching}中更详细地讨论\gls{score_matching}。
对于现在讨论的\gls{AE},理解学习$\log p_{\text{data}}$的梯度场是学习$p_{\text{data}}$结构的一种方式就足够了。


\glssymbol{DAE}~的训练\gls{criterion}(条件高斯$p(\Vx \mid \Vh)$)能让\gls{AE}学到能估计数据分布得分的向量场$(g(f(\Vx))-\Vx)$ ,这是~\glssymbol{DAE}~的一个重要特性。
具体如\figref{fig:chap14_denoising_task}所示。

\begin{figure}[!htb]
\ifOpenSource
\centerline{\includegraphics{figure.pdf}}
\else
\centerline{\includegraphics{Chapter14/figures/denoising_task}}
\fi
\caption{\gls{DAE}被训练为将损坏的数据点$\tilde \Vx$映射回原始数据点$\Vx$。
我们将训练样本$\Vx$表示为位于低维\gls{manifold}(粗黑线)附近的红叉。
我们用灰色圆圈表示等概率的损坏过程$C(\tilde \Vx \mid \Vx)$。
灰色箭头演示了如何将一个训练样本转换为经过此损坏过程的样本。
当训练\gls{DAE}最小化平方误差$\| g(f(\tilde \Vx)) - \Vx \|^2$的平均值时,\gls{reconstruction} $g(f(\tilde \Vx))$估计$\SetE_{\RVx, \tilde{\RVx} \sim p_{\text{data}}(\RVx) C(\tilde{\RVx} \mid \RVx)}[\RVx \mid \tilde{\Vx}]$。
$g(f(\tilde \Vx))$对可能产生$\tilde \Vx$的原始点$\Vx$的质心进行估计,所以向量$ g(f(\tilde \Vx)) - \tilde \Vx $近似指向\gls{manifold}上最近的点。
因此\gls{AE}可以学习由绿色箭头表示的向量场$ g(f( \Vx)) -  \Vx $。
该向量场将得分$\nabla_{\Vx} \log p_{\text{data}}(\Vx)$估计为一个乘性因子,即\gls{reconstruction_error}均方根的平均。
}
\label{fig:chap14_denoising_task}
\end{figure}

对一类采用高斯噪声和\gls{mean_squared_error}作为\gls{reconstruction_error}的特定\gls{DAE}(具有sigmoid~\gls{hidden_unit}和线性\gls{reconstruction}单元)的\gls{denoising}训练过程,与训练一类特定的被称为~\glssymbol{RBM}~的无向概率模型是等价的\citep{Vincent-NC-2011-small}。
这类模型将在\secref{sec:gaussian_bernoulli_rbms}给出更详细的介绍;对于现在的讨论,我们只需知道这个模型能显式的给出$p_{\text{model}}(\Vx; \Vtheta)$。
当~\glssymbol{RBM}~使用\firstgls{denoising_score_matching}算法~\citep{Kingma+LeCun-2010-small}训练时,它的学习算法与训练对应的\gls{DAE}是等价的。
在一个确定的噪声水平下,\gls{regularization}的\gls{score_matching}不是一致估计量;相反它会恢复分布的一个模糊版本。
然而,当噪声水平趋向于0且训练样本数趋向与无穷时,一致性就会恢复。
我们将会在\secref{sec:denoising_score_matching}更详细地讨论\gls{denoising_score_matching}。


\gls{AE}和~\glssymbol{RBM}~还存在其他联系。
在~\glssymbol{RBM}~上应用\gls{score_matching}后,其\gls{cost_function}将等价于\gls{reconstruction_error}结合类似~\glssymbol{CAE}~惩罚的正则项 \citep{Swersky-ICML2011}。
\citet{Bengio+Delalleau-2009}指出\gls{AE}的\gls{gradient}是对~\glssymbol{RBM}~\gls{contrastive_divergence}训练的近似。


对于连续的$\Vx$,高斯损坏和\gls{reconstruction}分布的\gls{denoising}\gls{criterion}得到的\gls{score}估计适用于一般\gls{encoder}和\gls{decoder}的参数化\citep{Alain+Bengio-ICLR2013-small}。
这意味着一个使用平方误差\gls{criterion}
\begin{align}
 \| g(f(\tilde \Vx)) - \Vx \|^2
\end{align}
和噪声方差为$\sigma^2 $的损坏
\begin{align}
 C(\tilde x = \tilde \Vx \mid \Vx) = N(\tilde \Vx; \mu=\Vx, \Sigma = \sigma^2 I)
\end{align}
的通用\gls{encoder}-\gls{decoder}架构可以用来训练估计\gls{score}。
\figref{fig:chap14_vector_field_color}展示其中的工作原理。

\begin{figure}[!htb]
\ifOpenSource
\centerline{\includegraphics{figure.pdf}}
\else
\centerline{\includegraphics[width=0.8\textwidth]{Chapter14/figures/vector_field_color}}
\fi
\caption{由\gls{DAE}围绕$1$维弯曲\gls{manifold}学习的向量场,其中数据集中在$2$维空间中。
每个箭头与\gls{reconstruction}向量减去\gls{AE}的输入向量后的向量成比例,并且根据隐式估计的概率分布指向较高的概率。
向量场在估计的密度函数的最大值处(在数据\gls{manifold}上)和密度函数的最小值处都为零。
例如,螺旋臂形成局部最大值彼此连接的$1$维\gls{manifold}。
局部最小值出现在两个臂间隙的中间附近。
当\gls{reconstruction}误差的范数(由箭头的长度示出)很大时,在箭头的方向上移动可以显著增加概率,并且在低概率的地方大多也是如此。
\gls{AE}将这些低概率点映射到较高的概率\gls{reconstruction}。
在概率最大的情况下,\gls{reconstruction}变得更准确,因此箭头会收缩。
经~\citet{Alain+Bengio-ICLR2013-small}许可转载此图。
}
\label{fig:chap14_vector_field_color}
\end{figure}

一般情况下,不能保证\gls{reconstruction}函数$g(f(\Vx))$减去输入$\Vx$后对应于某个函数的\gls{gradient},更不用说\gls{score} 。
这是早期工作~\citep{Vincent-NC-2011-small}专用于特定参数化的原因(其中$g(f(\Vx)) - \Vx$能通过另一个函数的导数获得)。
\citet{Kamyshanska+Memisevic-2015}通过标识一类特殊的浅层\gls{AE}家族,使$g(f(\Vx)) - \Vx$对应于这个家族所有成员的一个\gls{score},以此推广~\citet{Vincent-NC-2011-small}的结果。

% -- 504 --

目前为止我们所讨论的仅限于\gls{DAE}如何学习表示一个概率分布。
更一般的,我们可能希望使用\gls{AE}作为\gls{generative_model},并从其分布中进行采样。
这将在\secref{sec:drawing_samples_from_autoencoders}中讨论。

% -- 505 --

\subsection{历史展望}
\label{sec:historical_perspective_chap14}
采用~\glssymbol{MLP}~\gls{denoising}的想法可以追溯到~\cite{Lecun-these87}和~\citet{Gallinari87}的工作。
\citet{Behnke-2001}也曾使用\gls{recurrent_network}对图像去噪。
在某种意义上,\gls{DAE}仅仅是被训练\gls{denoising}的~\glssymbol{MLP}。
然而,``\gls{DAE}''的命名指的不仅仅是学习\gls{denoising},而且可以学到一个好的内部\gls{representation}(作为学习\gls{denoising}的副效用)。
这个想法提出较晚\citep{VincentPLarochelleH2008-small,Vincent-JMLR-2010-small}。
学习到的\gls{representation}可以被用来预训练更深的\gls{unsupervised}网络或\gls{supervised}网络。
与\gls{sparse}\gls{AE}、\gls{sparse_coding}、\gls{CAE}等\gls{regularization}的\gls{AE}类似, \glssymbol{DAE}~的动机是允许学习\gls{capacity}很高的\gls{encoder},同时防止在\gls{encoder}和\gls{decoder}学习一个无用的恒等函数 。


在引入现代~\glssymbol{DAE}~之前,\citet{Inayoshi-and-Kurita-2005}探索了其中一些相同的方法和目标。
他们除了在\gls{supervised}目标的情况下最小化\gls{reconstruction_error}之外,还在监督~\glssymbol{MLP}~的\gls{hidden_layer}注入噪声,通过引入\gls{reconstruction_error}和注入噪声提升泛化能力。
然而,他们的方法基于线性\gls{encoder},因此无法学习到现代~\glssymbol{DAE}~能学习的强大函数族。



\section{使用\glsentrytext{AE}学习\glsentrytext{manifold}}
\label{sec:learning_manifolds_with_autoencoders}

如\secref{sec:manifold_learning}描述,\gls{AE}跟其他很多\gls{ML}算法一样,也利用了数据集中在一个低维\gls{manifold}或者一小组这样的\gls{manifold}的思想。
其中一些\gls{ML}算法仅能学习到在\gls{manifold}上表现良好但给定不在\gls{manifold}上的输入会导致异常的函数。
\gls{AE}进一步借此想法,旨在学习\gls{manifold}的结构。


要了解\gls{AE}如何做到这一点,我们必须介绍\gls{manifold}的一些重要特性。


\gls{manifold}的一个重要特征是\firstgls{tangent_plane}的集合。
$d$维\gls{manifold}上的一点$\Vx$,\gls{tangent_plane}由能张成\gls{manifold}上允许变动的局部方向的$d$维基向量给出。
如\figref{fig:chap14_tangent_plane_color}所示,这些局部方向决定了我们能如何微小地变动$\Vx$而保持于\gls{manifold}上。

\begin{figure}[!htb]
\ifOpenSource
\centerline{\includegraphics{figure.pdf}}
\else
\centerline{\includegraphics{Chapter14/figures/tangent_plane_color}}
\fi
\caption{正切超平面概念的图示。
我们在$784$维空间中创建了$1$维\gls{manifold}。
我们使用一张784像素的MNIST图像,并通过垂直平移来转换它。
垂直平移的量定义沿着1维\gls{manifold}的坐标,轨迹为通过图像空间的弯曲路径。
该图显示了沿着该\gls{manifold}的几个点。
为了可视化,我们使用~\glssymbol{PCA}~将\gls{manifold}投影到$2$维空间中。
$n$维\gls{manifold}在每个点处都具有$n$维切平面。
该切平面恰好在该点接触\gls{manifold},并且在该点处平行于\gls{manifold}表面。
它定义了为保持在\gls{manifold}上可以移动的方向空间。
该$1$维\gls{manifold}具有单个切线。
我们在图中示出了一个点处的示例切线,其中图像表示该切线方向在图像空间中是怎样的。
灰色像素表示沿着切线移动时不改变的像素,白色像素表示变亮的像素,黑色像素表示变暗的像素。
}
\label{fig:chap14_tangent_plane_color}
\end{figure}

% -- 506 --

所有\gls{AE}的训练过程涉及两种推动力的折衷:
\begin{enumerate}
 \item 学习训练样本$\Vx$的\gls{representation} $\Vh$使得$\Vx$能通过\gls{decoder}近似地从$\Vh$中恢复。
$\Vx$是从训练数据挑出的这一事实很关键,因为这意味着在\gls{AE}不需要成功\gls{reconstruction}不属于数据生成分布下的输入。
 \item 满足约束或正则惩罚。
这可以是限制\gls{AE}\gls{capacity}的架构约束,也可以是加入到\gls{reconstruction}代价的一个正则项。
这些技术一般倾向那些对输入较不敏感的解。
\end{enumerate}

% -- 507 --

显然,单一的推动力是无用的——从它本身将输入复制到输出是无用的,同样忽略输入也是没用的。
相反,两种推动力结合是有用的,因为它们驱使隐藏的表示能捕获有关数据分布结构的信息。
重要的原则是,\gls{AE}必须有能力表示\emph{\gls{reconstruction}训练实例所需的变化}。
如果该数据生成分布集中靠近一个低维\gls{manifold},\gls{AE}能隐式产生捕捉这个\gls{manifold}局部坐标系的表示:仅在$\Vx$周围关于\gls{manifold}的相切变化需要对应于$\Vh=f(\Vx)$中的变化。
因此,\gls{encoder}学习从输入空间$\Vx$到表示空间的映射,映射仅对沿着\gls{manifold}方向的变化敏感,并且对\gls{manifold}正交方向的变化不敏感。


\figref{fig:chap14_1d_autoencoder_color}中一维的例子说明,我们可以通过构建对数据点周围的输入扰动不敏感的\gls{reconstruction}函数,使得\gls{AE}恢复\gls{manifold}结构。


\begin{figure}[!htb]
\ifOpenSource
\centerline{\includegraphics{figure.pdf}}
\else
\centerline{\includegraphics{Chapter14/figures/1d_autoencoder_color}}
\fi
\caption{如果\gls{AE}学习到对数据点附近的小扰动不变的\gls{reconstruction}函数,它就能捕获数据的\gls{manifold}结构。
这里,\gls{manifold}结构是0维\gls{manifold}的集合。
虚线对角线表示\gls{reconstruction}的恒等函数目标。 
最佳\gls{reconstruction}函数会在存在数据点的任意处穿过恒等函数。
图底部的水平箭头表示在输入空间中基于箭头的$r(\Vx)-\Vx$重建方向向量,总是指向最近的``\gls{manifold}''(1维情况下的单个数据点 )。
在数据点周围,\gls{DAE}明确地尝试将\gls{reconstruction}函数$r(\Vx)$的导数限制为很小。
\gls{CAE}的\gls{encoder}执行相同操作。
虽然在数据点周围,$r(\Vx)$的导数被要求很小,但在数据点之间它可能会很大。
数据点之间的空间对应于\gls{manifold}之间的区域,为将损坏点映射回\gls{manifold},\gls{reconstruction}函数必须具有大的导数。
} \label{fig:chap14_1d_autoencoder_color}
\end{figure}

% -- 508 --

为了理解\gls{AE}可用于\gls{manifold_learning}的原因,我们可以将\gls{AE}和其他方法进行对比。
学习表征\gls{manifold}最常见的是\gls{manifold}上(或附近)数据点的\firstgls{representation}。
对于特定的实例,这样的表示也被称为\gls{embedding}。
它通常由一个低维向量给出,具有比这个\gls{manifold}的``外围''空间更少的维数。
有些算法(下面讨论的\gls{nonparametric}\gls{manifold_learning}算法)直接学习每个训练样例的\gls{embedding},而其他算法学习更一般的映射(有时被称为\gls{encoder}或表示函数),将周围空间(输入空间)的任意点映射到它的\gls{embedding}。


\gls{manifold_learning}大多专注于试图捕捉到这些\gls{manifold}的\gls{unsupervised_learning}过程。
最初始的学习非线性\gls{manifold}的\gls{ML}研究专注基于\firstgls{nearest_neighbor_graph}的\firstgls{nonparametric}方法。
该图中每个训练样例对应一个节点,它的边连接近邻点对。
如\figref{fig:chap14_faces_graph_manifold}所示,这些方法\citep{Scholkopf98,Roweis2000-lle-small,Tenenbaum2000-isomap,Brand2003-small,Belkin+Niyogi-2003,Donoho+Carrie-03,Weinberger04a-small,SNE-nips15-small,VanDerMaaten08-small}将每个节点与张成实例和近邻之间的差向量变化方向的\gls{tangent_plane}相关联。

\begin{figure}[!htb]
\ifOpenSource
\centerline{\includegraphics{figure.pdf}}
\else
\centerline{\includegraphics{Chapter14/figures/faces_graph_manifold}}
\fi
\caption{非参数\gls{manifold_learning}过程构建的最近邻图,其中节点表示训练样本,有向边指示最近邻关系。
因此,各种过程可以获得与图的邻域相关联的切平面以及将每个训练样本与实值向量位置或\firstgls{embedding}相关联的坐标系。
我们可以通过插值将这种表示概括为新的样本。
只要样本的数量大到足以覆盖\gls{manifold}的弯曲和扭转,这些方法工作良好。 
图片来自QMUL 多角度人脸数据集\citep{Gong-et-al-2000}。
}
\label{fig:chap14_faces_graph_manifold}
\end{figure}

全局坐标系则可以通过优化或求解线性系统获得。
\figref{fig:chap14_tiling-a-manifold}展示了如何通过大量局部线性的类高斯样平铺(或``薄煎饼'',因为高斯块在\gls{tangent_plane}方向是扁平的)得到一个\gls{manifold}。

\begin{figure}[!htb]
\ifOpenSource
\centerline{\includegraphics{figure.pdf}}
\else
\centerline{\includegraphics[width=0.8\textwidth]{Chapter14/figures/tiling-a-manifold}}
\fi
\caption{如果每个位置处的切平面(见\figref{fig:chap14_tangent_plane_color})是已知的,则它们可以平铺后形成全局坐标系或密度函数。
每个局部块可以被认为是局部欧几里德坐标系或者是局部平面高斯或``薄饼'',在与薄饼正交的方向上具有非常小的方差而在定义坐标系的方向上具有非常大的方差。
这些高斯的混合提供了估计的密度函数,如\gls{manifold}中的~\ENNAME{Parzen}~窗口算法\citep{Vincent-Bengio-2003-short} 或其非局部的基于\gls{NN}的变体\citep{Bengio-Larochelle-NLMP-NIPS-2006-short}。
}
\label{fig:chap14_tiling-a-manifold}
\end{figure}

然而,\citet{Bengio+Monperrus-2005}指出了这些局部\gls{nonparametric}方法应用于\gls{manifold_learning}的根本困难:如果\gls{manifold}不是很光滑(它们有许多波峰、波谷和曲折),为覆盖其中的每一个变化,我们可能需要非常多的训练样本,导致没有能力泛化到没见过的变化。
实际上,这些方法只能通过内插,概括相邻实例之间\gls{manifold}的形状。
不幸的是,\glssymbol{AI}~问题中涉及的\gls{manifold}可能具有非常复杂的结构,难以仅从局部插值捕获特征。
考虑\figref{fig:chap14_tangent_plane_color}转换所得的\gls{manifold}样例。
如果我们只观察输入向量内的一个坐标$x_i$,当平移图像,我们可以观察到当这个坐标遇到波峰或波谷时,图像的亮度也会经历一个波峰或波谷。
换句话说,底层图像模板亮度的模式复杂性决定执行简单的图像变换所产生的\gls{manifold}的复杂性。
这是采用\gls{distributed_representation}和\gls{DL}捕获\gls{manifold}结构的动机。

% -- 509 --

\section{\glsentrytext{CAE}}
\label{sec:contractive_autoencoders}
\gls{CAE}~\citep{Rifai+al-2011-small,Salah+al-2011-small}在\gls{code} $\Vh = f(\Vx)$的基础上添加了显式的正则项,鼓励$f$的导数尽可能小:
\begin{align}
 \Omega(\Vh) = \lambda \Bigg\| \frac{\partial f(\Vx)}{\partial \Vx} \Bigg\|_F^2 .
\end{align}
惩罚项$\Omega(\Vh)$为平方~\ENNAME{Frobenius}~范数(元素平方之和),作用于与\gls{encoder}的函数相关\gls{partial_derivatives}的~\gls{jacobian}~矩阵。


\gls{DAE}和\gls{CAE}之间存在一定联系:\citet{Alain+Bengio-ICLR2013-small}指出在小高斯噪声的限制下,当\gls{reconstruction}函数将$\Vx$映射到$\Vr = g(f(\Vx))$时,\gls{denoising}\gls{reconstruction_error}与\gls{contractive}惩罚项是等价的。
换句话说,\gls{DAE}能抵抗小且有限的输入扰动,而\gls{CAE}使特征提取函数能抵抗极小的输入扰动。

分类任务中,基于~\gls{jacobian}~的\gls{contractive}惩罚预训练特征函数$f(\Vx)$,将收缩惩罚应用在$f(\Vx)$而不是$g(f(\Vx))$可以产生最好的分类精度。
如\secref{sec:estimating_the_score}所讨论,应用于$f(\Vx)$的\gls{contractive}惩罚与\gls{score_matching}也有紧密的联系。

\firstgls{contractive}源于~\glssymbol{CAE}~弯曲空间的方式。
具体来说,由于~\glssymbol{CAE}~训练为抵抗输入扰动,鼓励将输入点邻域映射到输出点处更小的邻域。
我们能认为这是将输入的邻域\gls{contractive}到更小的输出邻域。


说得更清楚一点,\glssymbol{CAE}~只在局部\gls{contractive}——一个训练样本$\Vx$的所有扰动都映射到$f(\Vx)$的附近。
全局来看,两个不同的点$\Vx$和$\Vx'$会分别被映射到远离原点的两个点$f(\Vx)$和$f(\Vx')$。
$f$扩展到数据\gls{manifold}的中间或远处是合理的(见\figref{fig:chap14_1d_autoencoder_color}中小例子的情况)。
当$\Omega(\Vh)$惩罚应用于~\ENNAME{sigmoid}~单元时,\gls{contractive}~\gls{jacobian}~的简单方式是令~\ENNAME{sigmoid}~趋向饱和的0或1。
这鼓励~\glssymbol{CAE}~使用~\ENNAME{sigmoid}~的极值编码输入点,或许可以解释为二进制\gls{code}。
它也保证了~\glssymbol{CAE}~可以穿过大部分~\ENNAME{sigmoid}~\gls{hidden_unit}能张成的超立方体,进而扩散其\gls{code}值。

我们可以认为点$\Vx$处的~\gls{jacobian}~矩阵$\MJ$能将非线性\gls{encoder}近似为线性算子。
这允许我们更形式地使用``\gls{contractive}''这个词。
在线性理论中,当$\MJ\Vx$的范数对于所有单位$\Vx$都小于等于1时,$\MJ$被称为\gls{contractive}的。
换句话说,如果$\MJ$收缩了单位球,他就是\gls{contractive}的。
我们可以认为~\glssymbol{CAE}~为鼓励每个局部线性算子具有收缩性,而在每个训练数据点处将~\ENNAME{Frobenius}~范数作为$f(\Vx)$的局部线性近似的惩罚。


如\secref{sec:learning_manifolds_with_autoencoders}中描述,正则\gls{AE}基于两种相反的推动力学习\gls{manifold}。
在~\glssymbol{CAE}~的情况下,这两种推动力是\gls{reconstruction_error}和\gls{contractive}惩罚$\Omega(\Vh)$。
单独的\gls{reconstruction_error}鼓励~\glssymbol{CAE}~学习一个恒等函数。
单独的\gls{contractive}惩罚将鼓励~\glssymbol{CAE}~学习关于$\Vx$是恒定的特征。
这两种推动力的折衷产生导数$\frac{\partial f(\Vx)}{\partial \Vx}$大多是微小的\gls{AE}。
只有少数\gls{hidden_unit},对应于一小部分输入数据的方向,可能有显著的导数。


\glssymbol{CAE}~的目标是学习数据的\gls{manifold}结构。
使$\MJ\Vx$很大的方向$\Vx$,会快速改变$\Vh$,因此很可能是近似\gls{manifold}\gls{tangent_plane}的方向。
\citet{Rifai+al-2011-small,Salah+al-2011-small}的实验显示训练~\glssymbol{CAE}~会导致$\MJ$中大部分奇异值(幅值)比1小,因此是收缩的。
然而,有些奇异值仍然比1大,因为\gls{reconstruction_error}的惩罚鼓励~\glssymbol{CAE}~对最大局部变化的方向进行编码。
对应于最大奇异值的方向被解释为\gls{CAE}学到的切方向。
理想情况下,这些切方向应对应于数据的真实变化。
比如,一个应用于图像的~\glssymbol{CAE}~应该能学到显示图像改变的切向量,如\figref{fig:chap14_tangent_plane_color}图中物体渐渐改变状态。
如\figref{fig:chap14_cifar_cae}所示,实验获得的奇异向量的可视化似乎真的对应于输入图象有意义的变换。

% -- 512 --

\begin{figure}[ht]
\centering
\ifOpenSource
\centerline{\includegraphics{figure.pdf}}
\else
% ?? I do NOT which is correct ...
\begin{tabular}{p{.1\figwidth}|p{.86\figwidth}}
Input point & Tangent vectors\\
\hline 
\includegraphics[width=\linewidth]{Chapter14/figures/cifar_tangent_input.png} & 
\includegraphics[width=\linewidth]{Chapter14/figures/cifar_local_pca.png}\\
 & Local PCA (no sharing across regions)\\
 \hline
\includegraphics[width=\linewidth]{Chapter14/figures/cifar_tangent_input.png} & 
\includegraphics[width=\linewidth]{Chapter14/figures/cifar_cae.png}\\
 & Contractive autoencoder\\
\end{tabular}
\fi
\caption{通过局部~\glssymbol{PCA}~和\gls{CAE}估计的\gls{manifold}切向量的图示。
\gls{manifold}的位置由来自CIFAR-10数据集中狗的输入图像定义。
切向量通过输入到代码映射的~\gls{jacobian}~矩阵$\frac{\partial \Vh}{\partial \Vx}$ 的前导奇异向量估计。
虽然局部~\glssymbol{PCA}~和~\glssymbol{CAE}~都可以捕获局部切方向,但~\glssymbol{CAE}~能够从有限训练数据形成更准确的估计,因为它利用了不同位置的参数共享(共享激活的\gls{hidden_unit}子集)。
\glssymbol{CAE}切方向通常对应于物体的移动或改变部分(例如头或腿)。
经~\citet{Dauphin-et-al-NIPS2011-small}许可转载此图。
}
\label{fig:chap14_cifar_cae}
\end{figure}


\gls{CAE}\gls{regularization}\gls{criterion}的一个实际问题是,尽管它在单一\gls{hidden_layer}的\gls{AE}情况下是容易计算的,但在更深的\gls{AE}情况下会变的难以计算。
根据~\citet{Rifai+al-2011-small}的策略,分别训练一系列单层的\gls{AE},并且每个被训练为\gls{reconstruction}前一个\gls{AE}的\gls{hidden_layer}。
这些\gls{AE}的组合就组成了一个深度\gls{AE}。
因为每个层分别训练成局部\gls{contractive},深度\gls{AE}自然也是\gls{contractive}的。
这个结果与联合训练深度模型完整架构(带有关于\gls{jacobian}的惩罚项)获得的结果是不同的,但它抓住了许多理想的定性特征。


另一个实际问题是,如果我们不对\gls{decoder}强加一些约束,\gls{contractive}惩罚可能导致无用的结果。
例如,\gls{encoder}将输入乘一个小常数$\epsilon$,\gls{decoder}将\gls{code}除以一个小常数$\epsilon$。
随着$\epsilon$趋向于0,\gls{encoder}会使\gls{contractive}惩罚项$\Omega(\Vh)$趋向于0而学不到任何关于分布的信息。
同时,\gls{decoder}保持完美的\gls{reconstruction}。
\citet{Rifai+al-2011-small}通过绑定$f$和$g$的权重来防止这种情况。
$f$和$g$都是由线性仿射变换后进行逐元素非线性变换的标准\gls{NN}层组成,因此将$g$的权重矩阵设成$f$权重矩阵的转置是很直观的。

% -- 513 --

\section{\glsentrytext{PSD}}
\label{sec:predictive_sparse_decomposition}

\firstall{PSD}是\gls{sparse_coding}和参数化\gls{AE}~\citep{koray-psd-08}的混合模型。
参数化\gls{encoder}被训练为能预测迭代推断的输出。
\glssymbol{PSD}~被应用于图片和视频中对象识别的\gls{unsupervised}特征学习\citep{Koray-08-small,koray-nips-10-small,Jarrett-ICCV2009-small,farabet-suml-11},在音频中也有所应用\citep{henaff-ismir-11-small}。
这个模型由一个\gls{encoder} $f(\Vx)$和一个\gls{decoder} $g(\Vh)$组成,并且都是参数化的。
在训练过程中,$\Vh$由优化算法控制。
优化过程是最小化
\begin{align}
 \| \Vx - g(\Vh) \| ^2 + \lambda | \Vh |_1 + \gamma \| \Vh - f(\Vx) \|^2.
\end{align}
就像\gls{sparse_coding},训练算法交替地相对$\Vh$和模型的参数最小化上述目标。
相对$\Vh$最小化较快,因为$f(\Vx)$提供$\Vh$的良好初始值以及\gls{loss_function}将$\Vh$约束在$f(\Vx)$附近。
简单的\gls{GD}算法只需10步左右就能获得理想的$\Vh$。


\glssymbol{PSD}~所使用的训练程序不是先训练\gls{sparse_coding}模型,然后训练$f(\Vx)$来预测\gls{sparse_coding}的特征。
\glssymbol{PSD}~训练过程正则化\gls{decoder},使用$f(\Vx)$可以推断出良好\gls{code}的参数。


\gls{PSD}是\firstgls{learned_approximate_inference}的一个例子。
在\secref{sec:learned_approximate_inference}中,这个话题将会进一步展开。
\chapref{chap:approximate_inference}中展示的工具能让我们了解到,\glssymbol{PSD}~能够被解释为通过最大化模型的对数似然下界训练有向\gls{sparse_coding}的概率模型。


在~\glssymbol{PSD}~的实际应用中,迭代优化仅在训练过程中使用。
模型被部署后,参数\gls{encoder} $f$用于计算已经习得的特征。
相比通过\gls{GD}推断$\Vh$,计算$f$是很容易的。
因为$f$是一个可微带参函数,\glssymbol{PSD}~模型可堆叠,并用于初始化其他训练\gls{criterion}的深度网络。

% -- 514 --

\section{\glsentrytext{AE}的应用}
\label{sec:applications_of_autoencoders}

\gls{AE}已成功应用于\gls{dimensionality_reduction}和\gls{information_retrieval}任务。
\gls{dimensionality_reduction}是\gls{representation_learning}和\gls{DL}的第一批应用之一。
它是研究\gls{AE}早期驱动力之一。
例如, \citet{Hinton-Science2006}训练了一个栈式~\glssymbol{RBM},然后利用它们的权重初始化一个\gls{hidden_layer}逐渐减小的深度\gls{AE},终结于30个单元的瓶颈。
生成的\gls{code}比30维的~\glssymbol{PCA}~产生更少的\gls{reconstruction_error},所学到的表示更容易定性解释,并能联系基础类别,这些类别表现为分离良好的集群。


低维\gls{representation}可以提高许多任务的性能,例如分类。
小空间的模型消耗更少的内存和运行时间。
据~\citet{Salakhutdinov+Hinton2007-small}和~\citet{Torralba+Fergus+Weiss-2008}观察,许多\gls{dimensionality_reduction}的形式会将语义上相关的样本置于彼此邻近的位置。
映射到低维空间所提供的线索有助于泛化。


相比普通任务,\firstgls{information_retrieval}从\gls{dimensionality_reduction}中获益更多,此任务需要找到数据库中类似查询的条目。
此任务不仅和其他任务一样从\gls{dimensionality_reduction}中获得一般益处,还使某些低维空间中的搜索变得极为高效。
特别的,如果我们训练\gls{dimensionality_reduction}算法生成一个低维且\emph{二值}的\gls{code},那么我们就可以将所有数据库条目在哈希表映射为二值编码向量。
这个哈希表允许我们返回具有相同二值编码的数据库条目作为查询结果进行\gls{information_retrieval}。
我们也可以非常高效地搜索稍有不同条目,只需反转查询编码的各个位。
这种通过\gls{dimensionality_reduction}和二值化的\gls{information_retrieval}方法被称为\firstgls{semantic_hashing}\citep{Salakhutdinov+Hinton2007-small,Salakhutdinov+Geoff-2009},已经被用于文本输入\citep{Salakhutdinov+Hinton2007-small,Salakhutdinov+Geoff-2009}和图像\citep{Torralba+Fergus+Weiss-2008,WeissTF08,KrizhevskyH11}。


通常在最终层上使用~\ENNAME{sigmoid}~编码函数产生\gls{semantic_hashing}的二值\gls{code}。
\ENNAME{sigmoid}~单元必须被训练为到达饱和,对所有输入值都接近0或接近1。
能做到这一点的窍门就是训练时在~\ENNAME{sigmoid}~非线性单元前简单地注入加性噪声。
噪声的大小应该随时间增加。
要对抗这种噪音并且保存尽可能多的信息,网络必须加大输入到~\ENNAME{sigmoid}~函数的幅度,直到饱和。

% -- 515 --

学习哈希函数的思想已在其他多个方向进一步探讨,包括改变损失训练\gls{representation}的想法,其中所需优化的损失与哈希表中查找附近样本的任务有更直接的联系\citep{Norouzi+Fleet-ICML2011}。

% !Mode:: "TeX:UTF-8"
% Translator: Yujun Li 
\chapter{\glsentrytext{representation_learning}}
\label{chap:representation_learning}

% 517 head

在本章中,首先我们会讨论学习表示是什么意思,以及表示的概念如何有助于深度框架的设计。
我们探讨学习算法如何在不同任务中共享统计信息,包括使用\gls{unsupervised}任务中的信息来完成\gls{supervised}任务。
共享表示有助于处理多模式或多领域,或是将已学到的知识迁移到样本很少或没有、但任务表示依然存在的任务上。
最后,我们回过头探讨\gls{representation_learning}成功的原因,从\gls{distributed_representation}~\citep{Hinton-et-al-PDP1986}和深度表示的理论优势,最后会讲到\gls{DGP}潜在假设的更一般概念,特别是观测数据的基本成因。


很多信息处理任务可能非常容易,也可能非常困难,这取决于信息是如何表示的。
这是一个广泛适用于日常生活、计算机科学及\gls{ML}的基本原则。
例如,对于人而言,可以直接使用长除法计算$210$除以$6$。
但如果使用罗马数字表示,这个问题就没那么直接了。
大部分现代人在使用罗马数字计算~\ENNAME{CCX}~除以~\ENNAME{VI}~时,都会将其转化成阿拉伯数字,从而使用位值系统的长除法。
更具体地,我们可以使用合适或不合适的表示来量化不同操作的渐近运行时间。
例如,插入一个数字到有序表中的正确位置,如果该数列表示为链表,那么所需时间是$O(n)$;如果该列表表示为红黑树,那么只需要$O(\log n)$的时间。

% -- 517 --

在\gls{ML}中,到底是什么因素决定了一种表示比另一种表示更好呢?
一般而言,一个好的表示可以使后续的学习任务更容易。
选择什么表示通常取决于后续的学习任务。

% 518 head 


我们可以将\gls{supervised_learning}训练的\gls{feedforward_network}视为\gls{representation_learning}的一种形式。
具体地,网络的最后一层通常是\gls{linear_classifier},如~\gls{softmax_chap15}~回归分类器。
网络的其余部分学习出该分类器的表示。
监督学习训练模型,一般会使得模型的各个\gls{hidden_layer}(特别是接近顶层的\gls{hidden_layer})的表示能够更加容易地完成训练任务。
例如,输入特征线性不可分的类别可能在最后一个\gls{hidden_layer}变成线性\gls{separable}。
原则上,最后一层可以是另一种模型,如\gls{nearest_neighbor}分类器~\citep{SalakhutdinovR2007-small}。
倒数第二层的特征应该根据最后一层的类型学习不同的性质。
% 518 mid


\gls{feedforward_network}的\gls{supervised}训练并没有给\gls{learned}的中间特征明确强加任何条件。
其他的\gls{representation_learning}算法往往会以某种特定的方式明确设计表示。
例如,我们想要学习一种使得密度估计更容易的表示。
具有更多独立性的分布会更容易建模,因此,我们可以设计鼓励表示向量$\Vh$中元素之间相互独立的\gls{objective_function}。
就像\gls{supervised}网络,\gls{unsupervised}深度学习算法有一个主要的训练目标,但也额外地学习出了表示。
不论该表示是如何得到的,它都可以用于其他任务。
或者,多个任务(有些是\gls{supervised}的,有些是\gls{unsupervised}的)可以通过共享的内部表示一起学习。


% 518 mid 

大多数\gls{representation_learning}算法都会在尽可能多地保留与输入相关的信息和追求良好的性质(如独立性)之间作出权衡。


% 518 mid 
\gls{representation_learning}特别有趣,因为它提供了进行\gls{unsupervised_learning}和\gls{semi_supervised_learning}的一种方法。
我们通常会有巨量的\gls{unlabeled}训练数据和相对较少的\gls{labeled}训练数据。
在非常有限的\gls{labeled}数据集上\gls{supervised_learning}通常会导致严重的\gls{overfitting}。
\gls{semi_supervised_learning}通过进一步学习\gls{unlabeled}数据,来解决\gls{overfitting}的问题。
具体地,我们可以从\gls{unlabeled}数据上学习出很好的表示,然后用这些表示来解决\gls{supervised_learning}问题。
% 518 end


人类和动物能够从非常少的\gls{labeled}样本中学习。
我们至今仍不知道这是如何做到的。
有许多假说解释人类的卓越学习能力——例如,大脑可能使用了大量的分类器或者\gls{bayesian_inference}技术的\gls{ensemble}。
一种流行的假说是,大脑能够利用\gls{unsupervised_learning}和\gls{semi_supervised_learning}。
利用\gls{unlabeled}数据有多种方式。
在本章中,我们主要使用的假说是\gls{unlabeled}数据可以学习出良好的表示。

% -- 519 head


\section{\glsentrytext{greedy_layer_wise_unsupervised_pretraining}}
\label{sec:greedy_layer_wise_unsupervised_pretraining}
\gls{unsupervised_learning}在\gls{DNN}的复兴上起到了关键的、历史性的作用,它使研究者首次可以训练不含诸如\gls{convolution}或者\gls{recurrence}这类特殊结构的深度\gls{supervised}网络。
我们将这一过程称为\firstgls{unsupervised_pretraining},或者更精确地,\firstgls{greedy_layer_wise_unsupervised_pretraining}。
此过程是一个任务(\gls{unsupervised_learning},尝试获取输入分布的形状)的表示如何有助于另一个任务(具有相同输入域的\gls{supervised_learning})的典型示例。
% 519 mid


\gls{greedy_layer_wise_unsupervised_pretraining}依赖于单层\gls{representation_learning}算法,例如~\glssymbol{RBM}、单层\gls{AE}、\gls{sparse_coding}模型或其他学习\gls{latent}表示的模型。
每一层使用\gls{unsupervised_learning}\gls{pretraining},将前一层的输出作为输入,输出数据的新的表示。
这个新的表示的分布(或者是和其他变量比如要预测类别的关系)有可能是更简单的。
如\algref{alg:pretraining}所示的正式表述。

\begin{algorithm}
\caption{ {\em \gls{greedy_layer_wise_unsupervised_pretraining}的协定}\\
给定如下:无监督特征学习算法$\CalL$,$\CalL$使用训练集样本并返回\gls{encoder}或特征函数$f$。
原始输入数据是$\MX$,每行一个样本,并且$f^{(1)}(\MX)$是第一阶段\gls{encoder}关于$\MX$的输出。
在执行\gls{fine_tune}的情况下,我们使用学习者$\CalT$,并使用初始函数$f$,输入样本$\MX$(以及在监督\gls{fine_tune}情况下关联的目标$\MY$),并返回细调好函数。 阶段数为$m$。
}
\label{alg:pretraining}
\begin{algorithmic}
\STATE $f \leftarrow$ 恒等函数
\STATE $\tilde{\MX} = \MX$
\FOR {$k=1, \ldots, m$}
  \STATE $f^{(k)} = {\CalL}(\tilde{\MX})$
  \STATE $f \leftarrow f^{(k)} \circ f$
  \STATE $\tilde{\MX} \leftarrow f^{(k)}(\tilde{\MX})$
\ENDFOR
\IF {\em fine-tuning}
  \STATE $f \leftarrow {\CalT}(f,\MX,\MY)$
\ENDIF
\STATE {\bf Return} $f$
\end{algorithmic}
\end{algorithm}
% 521 head


基于\gls{unsupervised}标准的贪心逐层训练过程,早已被用来规避\gls{supervised}问题中\gls{DNN}难以联合训练多层的问题。
这种方法至少可以追溯神经认知机~\citep{Fukushima75}。
深度学习的复兴始于2006年,源于发现这种贪心学习过程能够为多层联合训练过程找到一个好的初始值,甚至可以成功训练全连接的结构~\citep{Hinton06-small,Hinton-Science2006,HintonG2006,Bengio-nips-2006,ranzato-07-small}。
在此发现之前,只有深度卷积网络或深度循环网络这类特殊结构的深度网络被认为是有可能训练的。
现在我们知道训练具有全连接的深度结构时,不再需要使用\gls{greedy_layer_wise_unsupervised_pretraining},但\gls{unsupervised}\gls{pretraining}是第一个成功的方法。

% 519 end

\gls{greedy_layer_wise_unsupervised_pretraining}被称为\firstgls{greedy}的,是因为它是一个\firstgls{greedy_algorithm},
这意味着它独立地优化解决方案的每一个部分,每一步解决一个部分,而不是联合优化所有部分。
它被称为\firstgls{layer_wise},是因为这些独立的解决方案是网络层。
具体地,\gls{greedy_layer_wise_unsupervised_pretraining}每次处理一层网络,训练第$k$层时保持前面的网络层不变。
特别地,低层网络(最先训练的)不会在引入高层网络后进行调整。
它被称为\firstgls{unsupervised}的,是因为每一层用\gls{unsupervised}\gls{representation_learning}算法训练。
然而,它也被称为\firstgls{pretraining},是因为它只是在联合训练算法\firstgls{fine_tune}所有层之前的第一步。
在\gls{supervised_learning}任务中,它可以被看作是\gls{regularizer}(在一些实验中,\gls{pretraining}不能降低训练误差,但能降低测试误差)和参数初始化的一种形式。

% 520 head

通常而言,``\gls{pretraining}''不仅单指\gls{pretraining}阶段,也指结合\gls{pretraining}和\gls{supervised_learning}的两阶段学习过程。
\gls{supervised_learning}阶段可能会使用\gls{pretraining}阶段得到的顶层特征训练一个简单分类器,或者可能会对\gls{pretraining}阶段得到的整个网络进行\gls{supervised}\gls{fine_tune}。
不管采用什么类型的\gls{supervised_learning}算法和模型,在大多数情况下,整个训练过程几乎是相同的。
虽然\gls{unsupervised_learning}算法的选择将明显影响到细节,但是大多数\gls{unsupervised_pretraining}应用都遵循这一基本方法。
% 520 mid


\gls{greedy_layer_wise_unsupervised_pretraining}也能用作其他\gls{unsupervised_learning}算法的初始化,比如深度\gls{AE}~\citep{Hinton-Science2006}和具有很多\gls{latent_variable}层的概率模型。
这些\gls{model}包括\gls{DBN}~\citep{Hinton06-small}和\gls{DBM}~\citep{SalHinton09}。
这些\gls{DGM}会在\chapref{chap:deep_generative_models}中讨论。
% 520 mid


正如\secref{sec:supervised_pretraining}所探讨的,我们也可以进行\gls{greedy}逐层\emph{\gls{supervised}}\gls{pretraining}。
这是建立在训练浅层模型比\gls{deep_model}更容易的前提下,而该前提似乎在一些情况下已被证实~\citep{Erhan+al-2010-small}。

% 520 end

\subsection{何时以及为何\glsentrytext{unsupervised_pretraining}有效?}
\label{sec:when_and_why_does_unsupervised_pretraining_work}

% 520 end

在很多分类任务中,\gls{greedy_layer_wise_unsupervised_pretraining}能够在\gls{test_error}上获得重大提升。
这一观察结果始于2006年对\gls{DNN}的重新关注~\citep{Hinton06-small,Bengio-nips-2006,ranzato-07-small}。
然而,在很多其他问题上,\gls{unsupervised}\gls{pretraining}不能带来改善,甚至还会带来明显的负面影响。
\cite{Ma-et-al-2015}研究了\gls{pretraining}对\gls{ML_model}在化学活性预测上的影响。
结果发现,平均而言\gls{pretraining}是有轻微负面影响的,但在有些问题上会有显著帮助。
由于\gls{unsupervised_pretraining}有时有效,但经常也会带来负面效果,因此很有必要了解它何时有效以及有效的原因,以确定它是否适合用于特定的任务。

% 521 mid


首先,要注意的是这个讨论大部分都是针对\gls{greedy_unsupervised_pretraining}而言。
还有很多其他完全不同的方法使用\gls{semi_supervised_learning}来训练\gls{NN},比如\secref{sec:adversarial_training}介绍的\gls{virtual_adversarial_training}。
我们还可以在训练\gls{supervised_model}的同时训练\gls{AE}或\gls{generative_model}。
这种单阶段方法的例子包括\gls{discriminative_RBM}~\citep{Larochelle+Bengio-2008}和梯形网络~\citep{Rasmus-et-al-arxiv2015},其中整体目标是两项之和(一个使用\gls{label},另一个仅仅使用输入)。

% 521 end

\gls{unsupervised_pretraining}结合了两种不同的想法。
第一,它利用了\gls{DNN}对初始参数的选择,可以对模型有着显著的\gls{regularization}效果(在较小程度上,可以改进优化)的想法。
第二,它利用了更一般的想法——学习\gls{input_distribution}有助于学习从\gls{input}到\gls{output}的映射。

% 522 head

这两个想法都涉及到\gls{ML}算法中多个未能完全理解的部分之间复杂的相互作用。

% 522 mid
第一个想法,即\gls{DNN}初始参数的选择对其性能具有很强的\gls{regularization}效果,很少有关于这个想法的理解。
在\gls{pretraining}变得流行时,在一个位置初始化模型被认为会使其接近某一个\gls{local_minimum},而不是另一个\gls{local_minimum}。%?? 难以理解
如今,\gls{local_minima}不再被认为是\gls{NN}优化中的严重问题。
现在我们知道标准的\gls{NN}训练过程通常不会到达任何形式的\gls{critical_points}。
仍然可能的是,\gls{pretraining}会初始化模型到一个可能不会到达的位置——例如,某种区域,其中\gls{cost_function}从一个样本点到另一个样本点变化很大,而\gls{minibatch}只能提供\gls{noise}严重的梯度估计,或是某种区域中的~\gls{hessian}~矩阵条件数是病态的,\gls{GD}必须使用非常小的步长。
然而,我们很难准确判断\gls{supervised_learning}期间\gls{pretraining}参数的哪些部分应该保留。
这是现代方法通常同时使用\gls{unsupervised_learning}和\gls{supervised_learning},而不是依序使用两个学习阶段的原因之一。
除了这些复杂的方法可以让\gls{supervised_learning}阶段保持\gls{unsupervised_learning}阶段提取的信息之外,还有一种简单的方法,固定\gls{feature_extractor}的参数,仅仅将\gls{supervised_learning}作为顶层学成特征的分类器。

% 522 mid

另一个想法有更好的理解,即学习算法可以使用\gls{unsupervised}阶段学习的信息,在\gls{supervised_learning}的阶段表现得更好。
其基本想法是对于\gls{unsupervised}任务有用的一些特征对于\gls{supervised_learning}任务也可能是有用的。
例如,如果我们训练汽车和摩托车图像的\gls{generative_model},它需要知道轮子的概念,以及一张图中应该有多少个轮子。
如果我们幸运的话,\gls{unsupervised}阶段学习的轮子表示会适合于\gls{supervised_learning}。
然而我们还未能从数学、理论层面上证明,因此并不总是能够预测哪种任务能以这种形式从\gls{unsupervised_learning}中受益。
这种方法的许多方面高度依赖于具体使用的模型。
例如,如果我们希望在\gls{pretraining}特征的顶层添加\gls{linear_classifier},那么(学习到的)特征必须使潜在的类别是线性\gls{separable}。%??    学习到的, 译者加, 便于理解
这些性质通常会在\gls{unsupervised_learning}阶段自然发生,但也并非总是如此。
这是另一个\gls{supervised}和\gls{unsupervised_learning}同时训练更可取的原因——输出层施加的约束很自然地从一开始就包括在内。

% 523 head

从\gls{unsupervised_pretraining}作为学习一个表示的角度来看,我们可以期望\gls{unsupervised_pretraining}在初始表示较差的情况下更有效。
一个重要的例子是\gls{word_embeddings}。
使用~\gls{one_hot}~向量表示的词并不具有很多信息,因为任意两个不同的~\gls{one_hot}~向量之间的距离(平方$L^2$距离都是$2$ )都是相同的。
\gls{learned}的\gls{word_embeddings}自然会用它们彼此之间的距离来编码词之间的相似性。
因此,\gls{unsupervised_pretraining}在处理单词时特别有用。
然而在处理图像时是不太有用的,可能是因为图像已经在一个很丰富的向量空间中,其中的距离只能提供低质量的相似性度量。

% 523 mid

从\gls{unsupervised_pretraining}作为\gls{regularizer}的角度来看,我们可以期望\gls{unsupervised_pretraining}在\gls{labeled}样本数量非常小时很有帮助。
因为\gls{unsupervised_pretraining}添加的信息来源于\gls{unlabeled}数据,所以当\gls{unlabeled}样本的数量非常大时,我们也可以期望\gls{unsupervised_pretraining}的效果最好。
\gls{unsupervised_pretraining}的大量\gls{unlabeled}样本和少量\gls{labeled}样本构成的\gls{semi_supervised_learning}的优势特别明显。
在2011年,\gls{unsupervised_pretraining}赢得了两个国际\gls{transfer_learning}比赛~\citep{UTLC+LISA-2011-small,goodfellow+all-NIPS2011}。
在该情景中,目标任务中\gls{labeled}样本的数目很少(每类几个到几十个)。
这些效果也出现在被~\citet{paine2014analysis}严格控制的实验中。

% 523 mid

还可能涉及到一些其他的因素。
例如,当我们要学习的函数非常复杂时,\gls{unsupervised_pretraining}可能会非常有用。
\gls{unsupervised_learning}不同于\gls{weight_decay}这样的\gls{regularizer},它不偏向于学习一个简单的函数,而是学习对\gls{unsupervised_learning}任务有用的特征函数。
如果真实的潜在函数是复杂的,并且由\gls{input_distribution}的规律塑造,那么\gls{unsupervised_learning}更适合作为\gls{regularizer}。

% 523 end

除了这些注意事项外,我们现在分析一些\gls{unsupervised_pretraining}改善性能的成功示例,并解释这种改进发生的已知原因。
\gls{unsupervised_pretraining}通常用来改进分类器,并且从减少\gls{test_set}误差的观点来看是很有意思的。
然而,\gls{unsupervised_pretraining}还有助于分类以外的任务,并且可以用于改进优化,而不仅仅只是作为\gls{regularizer}。
例如,它可以提高\gls{DAE}的训练和测试\gls{reconstruction_error}~\citep{Hinton-Science2006}。

% 524 head

\cite{Erhan+al-2010-small}进行了许多实验来解释\gls{unsupervised_pretraining}的几个成功原因。
对\gls{training_error}和\gls{test_error}的改进都可以解释为,\gls{unsupervised_pretraining}将参数引入到了其他方法可能探索不到的区域。
\gls{NN}训练是非确定性的,并且每次运行都会收敛到不同的函数。
训练可以停止在梯度很小的点;也可以\gls{early_stopping}结束训练,以防\gls{overfitting};
还可以停止在梯度很大,但由于诸如随机性或~\gls{hessian}~矩阵\gls{poor_conditioning}等问题难以找到合适下降方向的点。
经过\gls{unsupervised_pretraining}的神经网络会一致地停止在一片相同的函数空间区域,但未经过\gls{pretraining}的\gls{NN}会一致地停在另一个区域。
\figref{fig:chap15_isomap}可视化了这种现象。
经过\gls{pretraining}的网络到达的区域是较小的,这表明\gls{pretraining}减少了估计过程的方差,这进而又可以降低严重\gls{overfitting}的风险。
换言之,\gls{unsupervised_pretraining}将\gls{NN}参数初始化到它们不易逃逸的区域,并且遵循这种初始化的结果更加一致,和没有这种初始化相比,结果很差的可能性更低。

% 524 mid

\cite{Erhan+al-2010-small}也回答了\emph{何时}\gls{pretraining}效果最好——预训练的网络越深,\gls{test_error}的均值和方差下降得越多。
值得注意的是,这些实验是在训练非常深层网络的现代方法发明和流行(\gls{ReLU},\gls{dropout}~和\gls{batch_normalization})之前进行的,因此对于\gls{unsupervised_pretraining}与当前方法的结合,我们所知甚少。

% 524 mid


一个重要的问题是\gls{unsupervised_pretraining}是如何起到\gls{regularizer}作用的。
一个假设是,\gls{pretraining}鼓励学习算法发现那些与生成观察数据的潜在原因相关的特征。
这也是启发除\gls{unsupervised_pretraining}之外许多其他算法的重要思想,将会在\secref{sec:semi_supervised_disentangling_of_causal_factors}中进一步讨论。

% 524 mid  


与\gls{unsupervised_learning}的其他形式相比,\gls{unsupervised_pretraining}的缺点是其使用了两个单独的训练阶段。
很多\gls{regularization}技术都具有一个优点,允许用户通过调整单一\gls{hyperparameter}的值来控制\gls{regularization}的强度。
\gls{unsupervised_pretraining}没有一种明确的方法来调整\gls{unsupervised}阶段\gls{regularization}的强度。
相反,\gls{unsupervised_pretraining}有许多\gls{hyperparameter},但其效果只能之后度量,通常难以提前预测。
当我们同时执行\gls{unsupervised}和\gls{supervised_learning}而不使用\gls{pretraining}策略时,会有单个\gls{hyperparameter}(通常是附加到\gls{unsupervised}\gls{cost}的系数)控制\gls{unsupervised}\gls{objective}\gls{regularize}\gls{supervised}模型的强度。
减少该系数,总是能够可预测地获得较少\gls{regularization}强度。
在\gls{unsupervised_pretraining}的情况下,没有一种灵活调整\gls{regularization}强度的方式——要么\gls{supervised}模型初始化为\gls{pretraining}的参数,要么不是。

% 525 end

% 525 head

\begin{figure}[!htb]
\ifOpenSource
\centerline{\includegraphics{figure.pdf}}
\else
\centerline{\includegraphics{Chapter15/figures/isomap_color}}
\fi
\caption{在\emph{函数空间}(并非参数空间,避免从参数向量到函数的多对一映射)不同神经网络的学习轨迹的非线性映射的可视化。
不同网络采用不同的随机初始化,并且有的使用了\gls{unsupervised_pretraining},有的没有。
每个点对应着训练过程中一个特定时间的神经网络。
经\citet{Erhan+al-2010-small}许可改编此图。
函数空间中的坐标是关于每组输入$\Vx$和它的一个输出$\Vy$的\gls{infinite}维向量。
\cite{Erhan+al-2010-small}将很多特定$\Vx$的$\Vy$连接起来,线性投影到高维空间中。
然后他们使用~\gls{isomap}~\citep{Tenenbaum2000-isomap}进行进一步的非线性投影并投到二维空间。
颜色表示时间。
所有的网络初始化在上图的中心点附近(对应的函数区域在不多数输入上具有近似\gls{uniform_distribution}的类别$y$)。
随着时间推移,学习将函数向外移动到预测得更好的点。
当使用\gls{pretraining}时,训练会一致地收敛到同一个区域;
而不使用\gls{pretraining}时,训练会收敛到另一个不重叠的区域。
\gls{isomap}~试图维持全局相对距离(体积因此也保持不变),因此使用\gls{pretraining}的模型对应的较小区域意味着,基于\gls{pretraining}的\gls{estimator}具有较小的方差。
}
\label{fig:chap15_isomap}
\end{figure}

% 525 head

% 525 end 

具有两个单独的训练阶段的另一个缺点是每个阶段都具有各自的\gls{hyperparameter}。
第二阶段的性能通常不能在第一阶段期间预测,因此在第一阶段提出\gls{hyperparameter}和第二阶段根据反馈来更新之间存在较长的延迟。
最通用的方法是在\gls{supervised}阶段使用\gls{validation_set}上的误差来挑选\gls{pretraining}阶段的\gls{hyperparameter},如~\cite{Larochelle-jmlr-2009}中讨论的。
在实际中,有些\gls{hyperparameter},如\gls{pretraining}迭代的次数,很方便在\gls{pretraining}阶段设定,通过\gls{unsupervised}目标上使用\gls{early_stopping}策略完成。
这个策略并不理想,但是在计算上比使用\gls{supervised}目标代价小得多。

% 526 head

如今,大部分算法已经不使用\gls{unsupervised_pretraining}了,除了在\gls{NLP}领域中单词作为~\gls{one_hot}~向量的自然表示不能传达相似性信息,并且有非常多的\gls{unlabeled}数据集可用。
在这种情况下,\gls{pretraining}的优点是可以对一个巨大的\gls{unlabeled}集合(例如用包含数十亿单词的语料库)进行\gls{pretraining},学习良好的表示(通常是单词,但也可以是句子),然后使用该表示或\gls{fine_tune}它,使其适合于训练集样本大幅减少的\gls{supervised}任务。
这种方法由~\cite{CollobertR2008-small}、\cite{Turian+Ratinov+Bengio-2010-small}和~\cite{collobert2011natural}开创,至今仍在使用。

% 526 mid


基于\gls{supervised_learning}的\gls{DL}技术,通过~\gls{dropout}~或\gls{batch_normalization}来\gls{regularize},能够在很多任务上达到人类级别的性能,但仅仅是在极大的\gls{labeled}数据集上。
在中等大小的数据集(例如~\ENNAME{CIFAR}-10和~\ENNAME{MNIST},每个类大约有5,000个\gls{labeled}样本)上,这些技术的效果比\gls{unsupervised_pretraining}更好。
在极小的数据集,例如\gls{alternative_splicing_dataset},贝叶斯方法要优于基于\gls{unsupervised_pretraining}的方法~\citep{Srivastava-master-small}。
由于这些原因,\gls{unsupervised_pretraining}已经不如以前流行。
然而,\gls{unsupervised_pretraining}仍然是\gls{DL}研究历史上的一个重要里程碑,并将继续影响当代方法。
\gls{pretraining}的想法已经推广到\firstgls{supervised_pretraining},这将在\secref{sec:supervised_pretraining}中讨论,在\gls{transfer_learning}中这是非常常用的方法。
\gls{transfer_learning}中的\gls{supervised_pretraining}流行~\citep{Oquab-et-al-CVPR2014,yosinski-nips2014}于在~\ENNAME{ImageNet}~数据集上使用\gls{convolutional_network}\gls{pretraining}。
由于这个原因,实践者们公布了这些网络训练出的参数,就像自然语言任务公布\gls{pretraining}的单词向量一样~\citep{collobert2011natural,Mikolov-et-al-ICLR2013}。

% 526 end

\section{\glsentrytext{transfer_learning}和\glsentrytext{domain_adaption}}
\label{sec:transfer_learning_and_domain_adaptation}
\gls{transfer_learning}和\gls{domain_adaption}指的是利用一个情景(例如,分布$P_1$)中已经学到的内容去改善另一个情景(比如分布$P_2$)中的\gls{generalization}情况。
这点概括了上一节提出的想法,即在\gls{unsupervised_learning}任务和\gls{supervised_learning}任务之间转移表示。

%527 head

在\firstgls{transfer_learning}中,\gls{learner}必须执行两个或更多个不同的任务,但是我们假设能够解释$P_1$变化的许多因素和学习$P_2$需要抓住的变化相关。
这通常能够在\gls{supervised_learning}中解释,输入是相同的,但是输出不同的性质。
例如,我们可能在第一种情景中学习了一组视觉类别,比如猫和狗,然后在第二种情景中学习一组不同的视觉类别,比如蚂蚁和黄蜂。
如果第一种情景(从$P_1$采样)中具有非常多的数据,那么这有助于学习到能够使得从$P_2$抽取的非常少样本中快速\gls{generalize}的表示。%??
许多视觉类别\emph{共享}一些低级概念,比如边缘、视觉形状、几何变化、光照变化的影响等等。
一般而言,当存在对不同情景或任务有用特征时,并且这些特征对应多个情景出现的潜在因素,\gls{transfer_learning}、\gls{multitask_learning}(\secref{sec:multitask_learning})和\gls{domain_adaption}可以使用\gls{representation_learning}来实现。
如\figref{fig:chap7_multi_factor_output}所示,这是具有共享底层和任务相关上层的学习框架。

% 527 mid

然而,有时不同任务之间共享的不是输入的语义,而是输出的语义。
例如,\gls{SR}系统需要在输出层产生有效的句子,但是输入附近的较低层可能需要识别相同\gls{phoneme}或子\gls{phoneme}发音的非常不同的版本(这取决于说话人)。
在这样的情况下,共享神经网络的上层(输出附近)和进行任务特定的预处理是有意义的,如\figref{fig:chap15_multi_task_input}所示。

% 527 mid

% 528 head

\begin{figure}[!htb]
\ifOpenSource
\centerline{\includegraphics{figure.pdf}}
\else
\centerline{\includegraphics{Chapter15/figures/multi_task_input}}
\fi
\caption{\gls{multitask_learning}或者\gls{transfer_learning}的架构示例。
输出变量$\RVy$在所有的任务上具有相同的语义;输入变量$\RVx$在每个任务(或者,比如每个用户)上具有不同的意义(甚至可能具有不同的维度),图上三个任务为$\RVx^{(1)}$,$\RVx^{(2)}$,$\RVx^{(3)}$。
底层结构(决定了选择方向)是面向任务的,上层结构是共享的。
底层结构学习将面向特定任务的输入转化为通用特征。
}
\label{fig:chap15_multi_task_input}
\end{figure}

% 528 head

在\firstgls{domain_adaption}的相关情况下,在每个情景之间任务(和最优的输入到输出的映射)都是相同的,但是\gls{input_distribution}稍有不同。 
例如,考虑情感分析的任务,如判断一条评论是表达积极的还是消极的情绪。 
网上的评论有许多类别。
在书、视频和音乐等媒体内容上训练的顾客评论情感预测器,被用于分析诸如电视机或智能电话的消费电子产品的评论时,\gls{domain_adaption}情景可能会出现。%?? 
可以想象,存在一个潜在的函数可以判断任何语句是正面的、中性的还是负面的,但是词汇和风格可能会因领域而有差异,使得跨域的\gls{generalize}训练变得更加困难。
简单的\gls{unsupervised_pretraining}(\gls{DAE})已经能够非常成功地用于\gls{domain_adaption}的情感分析\citep{Glorot+al-ICML-2011}。
% 527 end 

% 528 head

一个相关的问题是\firstgls{concept_drift},我们可以将其视为一种\gls{transfer_learning},因为数据分布随时间而逐渐变化。
\gls{concept_drift}和\gls{transfer_learning}都可以被视为\gls{multitask_learning}的特定形式。
``\gls{multitask_learning}''这个术语通常指\gls{supervised_learning}任务,而更广义的\gls{transfer_learning}的概念也适用于\gls{unsupervised_learning}和\gls{RL}。

% 528 mid

在所有这些情况下,我们的目标是利用第一个\gls{setting}下的数据,提取那些在第二种\gls{setting}中学习时或直接进行预测时可能有用的信息。
\gls{representation_learning}的核心思想是相同的表示可能在两种\gls{setting}中都是有用的。
两个\gls{setting}使用相同的表示,使得表示可以受益于两个任务的训练数据。

% 528 end

如前所述,\gls{transfer_learning}中\gls{unsupervised}\gls{DL}已经在一些\gls{ML}比赛中取得了成功~\citep{UTLC+LISA-2011-small,goodfellow+all-NIPS2011}。
这些比赛中的某一个实验配置如下。
首先每个参与者获得一个第一种\gls{setting}(来自分布$P_1$)的数据集,其中含有一些类别的样本。
参与者必须使用这个来学习一个良好的特征空间(将原始输入映射到某种表示),使得当我们将这个\gls{learned}变换用于来自迁移\gls{setting}(分布$P_2$)的输入时,\gls{linear_classifier}可以在很少\gls{labeled}样本上训练、并\gls{generalize}得很好。
这个比赛中最引人注目的结果之一是,学习表示的网络架构越深(在第一个\gls{setting}$P_1$中的数据使用纯\gls{unsupervised}方式学习),在第二个\gls{setting}(迁移)$P_2$的新类别上学习到的曲线就越好。
对于深度表示而言,迁移任务只需要少量\gls{labeled}样本就能显著地提升\gls{generalization}性能。

% 529 head

\gls{transfer_learning}的两种极端形式是\firstgls{one_shot_learning}和\firstgls{zero_shot_learning},有时也被称为\firstgls{zero_data_learning}。
只有一个\gls{labeled}样本的迁移任务被称为\gls{one_shot_learning};没有\gls{labeled}样本的迁移任务被称为\gls{zero_shot_learning}。

% 529 mid

因为第一阶段学习出的表示就可以清楚地分离出潜在的类别,所以\gls{one_shot_learning}~\citep{Fei-Fei+al-2006}是可能的。
在\gls{transfer_learning}阶段,仅需要一个\gls{labeled}样本来推断表示空间中聚集在相同点周围许多可能测试样本的\gls{label}。
这使得在\gls{learned}的表示空间中,对应于不变性的变化因子已经与其他因子完全分离,在区分某些类别的对象时,我们可以学习到哪些因素具有决定意义。

% 529 mid

考虑一个\gls{zero_shot_learning}\gls{setting}的例子,\gls{learner}已经读取了大量文本,然后要解决\gls{object_recognition}的问题。
如果文本足够好地描述了对象,那么即使没有看到某对象的图像,也能识别出该对象的类别。
例如,已知猫有四条腿和尖尖的耳朵,那么\gls{learner}可以在没有见过猫的情况下猜测该图像中是猫。

% 529 end


只有在训练时使用了额外信息,\gls{zero_data_learning}~\citep{Larochelle2008}和\gls{zero_shot_learning}~\citep{Palatucci2009,Socher-2013}才是有可能的。
我们可以认为\gls{zero_data_learning}场景包含三个随机变量:传统输入$\Vx$,传统输出或目标$\Vy$,以及描述任务的附加随机变量$T$。
该模型被训练来估计条件分布$p(\Vy \mid \Vx, T)$,其中$T$是我们希望执行的任务的描述。
在我们的例子中,读取猫的文本信息然后识别猫,输出是二元变量$y$,$y=1$表示``是'',$y=0$表示``不是''。
任务变量$T$表示要回答的问题,例如``这个图像中是否有猫?''
如果训练集包含和$T$在相同空间的\gls{unsupervised}对象样本,我们也许能够推断未知的$T$实例的含义。
在我们的例子中,没有提前看到猫的图像而去识别猫,所以拥有一些\gls{unlabeled}文本数据包含句子诸如``猫有四条腿''或``猫有尖耳朵'',对于学习非常有帮助。

% 530  head

\gls{zero_shot_learning}要求$T$被表示为某种形式的\gls{generalization}。
例如,$T$不能仅是指示对象类别的\gls{one_hot}编码。
通过使用每个类别词的\gls{word_embeddings}表示,\cite{Socher-2013}提出了对象类别的\gls{distributed_representation}。

% 530 mid

我们还可以在\gls{machine_translation}中发现一种类似的现象~\citep{Klementiev-et-al-COLING2012,Mikolov-et-al-arxiv2013,Gouws-et-al-arxiv2014}:我们已经知道一种语言中的单词,还可以学到单一语言语料库中词与词之间的关系;另一方面,我们已经翻译了一种语言中的单词与另一种语言中的单词相关的句子。
即使我们可能没有将语言$X$中的单词$A$翻译成语言$Y$中的单词$B$的\gls{labeled}样本,我们也可以\gls{generalize}并猜出单词$A$的翻译,这是由于我们已经学习了语言$X$和$Y$单词的\gls{distributed_representation},并且通过两种语言句子的匹配对组成的训练样本,产生了关联于两个空间的链接(可能是双向的)。
如果联合学习三种成分(两种表示形式和它们之间的关系),那么这种迁移将会非常成功。

% 530 mid

\gls{zero_shot_learning}是\gls{transfer_learning}的一种特殊形式。
同样的原理可以解释如何能执行\firstgls{multimodal_learning},学习两种\gls{modality}的表示,和一种\gls{modality}中的观察结果$\Vx$与另一种\gls{modality}中的观察结果$\Vy$组成的对$(\Vx, \Vy)$之间的关系(通常是一个联合分布)\citep{Srivastava+Salakhutdinov-NIPS2012-small}。
通过学习所有的三组参数(从$\Vx$到它的表示、从$\Vy$到它的表示,以及两个表示之间的关系),一个表示中的概念被锚定在另一个表示中,反之亦然,从而可以有效地推广到新的对组。 %?? allowing one 
这个过程如\figref{fig:chap15_maps_between_representations}所示。

% 530 end

% 531

\begin{figure}[!htb]
\ifOpenSource
\centerline{\includegraphics{figure.pdf}}
\else
\centerline{\includegraphics{Chapter15/figures/maps_between_representations}}
\fi
\caption{两个域$\Vx$和$\Vy$之间的\gls{transfer_learning}能够进行\gls{zero_shot_learning}。
\gls{labeled}或\gls{unlabeled}样本$\Vx$可以学习表示函数$f_{\Vx}$。
同样地,样本$\Vy$也可以学习表示函数$f_{\Vy}$。
上图中$f_{\Vx}$和$f_{\Vy}$旁都有一个向上的箭头,不同的箭头表示不同的作用函数。
并且箭头的类型表示使用了哪一种函数。
$\Vh_{\Vx}$空间中的相似性度量表示$\Vx$空间中任意点对之间的距离,这种度量方式比直接度量$\Vx$空间的距离更好。
同样地,$\Vh_{\Vy}$空间中的相似性度量表示$\Vy$空间中任意点对之间的距离。
这两种相似函数都使用带点的双向箭头表示。
\gls{labeled}样本(水平虚线)$(\Vx, \Vy)$能够学习表示$f_{\Vx}(\Vx)$和表示$f_{\Vy}(\Vy)$之间的单向或双向映射(实双向箭头),以及这些表示之间如何锚定。
\gls{zero_data_learning}可以通过以下方法实现。
像$\Vx_{\text{test}}$可以和单词$\Vy_{\text{test}}$关联起来,即使该单词没有像,仅仅是因为单词表示$f_{\Vy}(\Vy_{\text{test}})$和像表示$f_{\Vx}(\Vx_{\text{test}})$可以通过表示空间的映射彼此关联。
这种方法有效的原因是,尽管像和单词没有匹配成队,但是它们各自的特征向量$f_{\Vx}(\Vx_{\text{test}})$和$f_{\Vy}(\Vy_{\text{test}})$互相关联。
上图受~\ENNAME{Hrant Khachatrian}~的建议启发。
}
\label{fig:chap15_maps_between_representations}
\end{figure}

% 531

\section{\glsentrytext{semi_supervised}解释因果关系}
\label{sec:semi_supervised_disentangling_of_causal_factors}

% 532 head

\gls{representation_learning}的一个重要问题是``什么原因能够使一个表示比另一个表示更好?''
一种假设是,理想表示中的特征对应到观测数据的\gls{underlying_cause},特征空间中不同的特征或方向对应着不同的原因,从而表示能够区分这些原因。
这个假设促使我们去寻找表示$p(\Vx)$的更好方法。
如果$\Vy$是$\Vx$的重要成因之一,那么这种表示也可能是计算$p(\Vy \mid \Vx)$的一种良好表示。
从20世纪90年代以来,这个想法已经指导了大量的\gls{DL}研究工作~\citep{Becker92,hinton1999unsupervised}。
关于\gls{semi_supervised_learning}可以超过纯\gls{supervised_learning}的其他论点,请读者参考\cite{Chapelle-2006}的第1.2节。

% 532 head

在\gls{representation_learning}的其他方法中,我们大多关注易于建模的表示——例如,数据稀疏或是各项之间相互独立的情况。
能够清楚地分离出潜在因素的表示可能并不一定易于建模。
然而,该假设促使\gls{semi_supervised_learning}使用\gls{unsupervised}\gls{representation_learning}的一个更深层原因是,对于很多\gls{AI}任务而言,有两个相随的特点:一旦我们能够获得观察结果基本成因的解释,那么将会很容易分离出个体属性。
具体来说,如果表示向量$\Vh$表示观察值$\Vx$的很多潜在因素,并且输出向量$\Vy$是最为重要的原因之一,那么从$\Vh$预测$\Vy$会很容易。

% 532 mid 

首先,让我们看看$p(\RVx)$的\gls{unsupervised_learning}无助于学习$p(\RVy\mid\RVx)$时,\gls{semi_supervised_learning}为何失败。
例如,考虑一种情况,$p(\RVx)$是均匀分布的,我们希望学习$f(\Vx) = \SetE[\RVy \mid \Vx]$。
显然,仅仅观察\gls{training_set}的值$\Vx$不能给我们关于$p(\RVy \mid \RVx)$的任何信息。

% 532 mid 

接下来,让我们看看\gls{semi_supervised_learning}成功的一个简单例子。
考虑这样的情况,$\RVx$来自一个混合分布,每个$\RVy$值具有一个混合分量,如\figref{fig:chap15_mixture_model}所示。
如果混合分量很好地分出来了,那么建模$p(\RVx)$可以精确地指出每个分量的位置,每个类一个\gls{labeled}样本的\gls{training_set}足以精确学习$p(\RVy \mid \RVx)$。
但是更一般地,什么能将$p(\RVy \mid \RVx)$和$p(\RVx)$关联在一起呢?

% 532 mid-end

% 533 head

\begin{figure}[!htb]
\ifOpenSource
\centerline{\includegraphics{figure.pdf}}
\else
\centerline{\includegraphics{Chapter15/figures/mixture_model_color}}
\fi
\caption{混合模型。具有三个混合分量的$x$上混合密度示例。
混合分量的内在本质是潜在\gls{explanatory_factor}$y$。
因为混合分量(例如,图像数据中的自然对象类别)在统计学上是显著的,所以仅仅使用\gls{unlabeled}样本\gls{unsupervised}建模$p(x)$也能揭示\gls{explanatory_factor}$y$。
}
\label{fig:chap15_mixture_model}
\end{figure}

% 533 head


如果$\RVy$与$\RVx$的成因之一非常相关,那么$p(\RVx)$和$p(\RVy \mid \RVx)$也会紧密关联,试图找到变化\gls{underlying}因素的\gls{unsupervised}\gls{representation_learning}可能像\gls{semi_supervised_learning}一样有用。

% -- 532 end

% 533 mid

假设$\RVy$是$\RVx$的成因之一,让$\RVh$代表所有这些成因。
真实的生成过程可以被认为是根据这个\gls{directed_graphical_model}结构化出来的,其中$\RVh$是$\RVx$的父节点:
\begin{equation}
	p(\RVh, \RVx) = p(\RVx \mid \RVh) p(\RVh).
\end{equation} % 533 mid
因此,数据的边缘概率是
\begin{equation}
	p(\Vx) = \SetE_{\RVh} p(\Vx \mid \Vh).
\end{equation}
从这个直观的观察中,我们得出结论,$\RVx$最好可能的模型(从广义的观点)是会表示上述``真实''结构的,其中$\Vh$作为\gls{latent_variable}解释$\Vx$中可观察的变化。
上文讨论的``理想''的\gls{representation_learning}应该能够反映出这些\gls{latent}因子。
如果$\RVy$是其中之一(或是紧密关联于其中之一),那么将很容易从这种表示中预测$\RVy$。
我们会看到给定$\RVx$下$\RVy$的条件分布通过\gls{bayes_rule}关联到上式中的分量:
\begin{equation}
	p(\RVy \mid \RVx) = \frac{ p(\RVx \mid \RVy) p(\RVy) }{p(\RVx)}.
\end{equation}
因此边缘概率$p(\RVx)$和条件概率$p(\RVy \mid \RVx)$密切相关,前者的结构信息应该有助于学习后者。
因此,在这些假设情况下,\gls{semi_supervised_learning}应该能提高性能。

% 533 end

关于这个事实的一个重要的研究问题是,大多数观察是由极其大量的\gls{underlying_cause}形成的。
假设$\RVy = \RSh_i$,但是\gls{unsupervised}\gls{learner}并不知道是哪一个$\RSh_i$。
对于一个\gls{unsupervised}\gls{learner}暴力求解就是学习一种表示,这种表示能够捕获\emph{所有}合理的重要生成因子$\RSh_j$,并将它们彼此区分开来,因此不管$\RSh_i$是否关联于$\RVy$,从$\RVh$预测$\RVy$都是容易的。

% -- 534 head

在实践中,暴力求解是不可行的,因为不可能捕获影响观察的所有或大多数变化因素。
例如,在视觉场景中,表示是否应该对背景中的所有最小对象进行编码?
根据一个有据可查的心理学现象,人们不会察觉到环境中和他们所在进行的任务并不立刻相关的变化,具体例子可以参考~\cite{simons1998failure}。
\gls{semi_supervised_learning}的一个重要研究前沿是确定每种情况下要编码\emph{什么}。
目前,处理大量\gls{underlying}原因的两个主要策略是,同时使用\gls{unsupervised_learning}和\gls{supervised_learning}信号,从而使得模型捕获最相关的变动因素,或是使用纯\gls{unsupervised_learning}学习更大规模的表示。

% 534 mid

\gls{unsupervised_learning}的另一个思路是选择一个更好的确定哪些\gls{underlying}因素最为关键的定义。
之前,\gls{AE}和\gls{generative_model}被训练来优化一个类似于\gls{mean_squared_error}的固定标准。
这些固定标准确定了哪些因素是重要的。
例如,图像像素的\gls{mean_squared_error}隐式地指定,一个\gls{underlying}因素只有在其显著地改变大量像素的亮度时,才是重要影响因素。
如果我们希望解决的问题涉及到小对象之间的相互作用,那么这将有可能遇到问题。
如\figref{fig:chap15_pingpong}所示,在机器人任务中,\gls{AE}未能学习到编码小乒乓球。
同样是这个机器人,它可以成功地与更大的对象进行交互(例如棒球,\gls{mean_squared_error}在这种情况下很显著)。

% 534 mid

% 535 head

\begin{figure}[!htb]
\ifOpenSource
\centerline{\includegraphics{figure.pdf}}
\else
\begin{tabular}{cc}
输入 & 重构 \\
\includegraphics[width=0.4\textwidth]{Chapter15/figures/ping_pong_input} &
\includegraphics[width=0.4\textwidth]{Chapter15/figures/ping_pong_reconstruction}
\end{tabular}
\fi
\caption{机器人任务上,基于\gls{mean_squared_error}训练的\gls{AE}不能重构乒乓球。
乒乓球的存在及其所有空间坐标,是生成图像且与机器人任务相关的重要潜在因素。
不幸的是,\gls{AE}具有有限的容量,基于\gls{mean_squared_error}的训练没能将乒乓球作为显著物体识别出来编码。
以上图像由Chelsea Finn提供。
}
\label{fig:chap15_pingpong}
\end{figure}

% 535 head

% 534 end

还有一些其他的显著性的定义。
例如,如果一组像素具有高度可识别的模式,那么即使该模式不涉及到极端的亮度或暗度,该模式还是会被认为非常显著。
实现这样一种定义显著的方法是使用最近提出的\firstgls{generative_adversarial_networks}\citep{Goodfellow-et-al-NIPS2014-small}。
在这种方法中,\gls{generative_model}被训练来愚弄\gls{feedforward_classifier}。
\gls{feedforward_classifier}尝试将来自\gls{generative_model}的所有样本识别为假的,并将来自\gls{training_set}的所有样本识别为真的。
在这个框架中,\gls{feedforward_network}能够识别出的任何结构化模式都是非常显著的。
\gls{generative_adversarial_networks}会在\secref{sec:generative_adversarial_networks}中更详细地介绍。
为了叙述方便,知道它能\emph{学习}出如何决定什么是显著的就可以了。
\cite{lotter2015unsupervised}表明,生成人类头部头像的模型在使用\gls{mean_squared_error}训练时往往会忽视耳朵,但是\gls{adversarial}式框架学习能够成功地生成耳朵。
因为耳朵与周围的皮肤相比不是非常明亮或黑暗,所以根据\gls{mean_squared_error}损失它们不是特别突出,但是它们高度可识别的形状和一致的位置意味着\gls{feedforward_network}能够轻易地学习出如何检测它们,从而使得它们在\gls{generative_adversarial_framework}下是高度突出的。
\figref{fig:chap15_manface}给了一些样例图片。
\gls{generative_adversarial_networks}只是确定应该表示哪些因素的一小步。
我们期望未来的研究能够发现更好的方式来确定表示哪些因素,并且根据任务来开发表示不同因素的机制。

% -- 535 mid


% 536 head
\begin{figure}[!htb]
\ifOpenSource
\centerline{\includegraphics{figure.pdf}}
\else
\begin{tabular}{ccc}
真实图 & \glssymbol{mean_squared_error} & 对抗学习 \\
\includegraphics[width=0.3\textwidth]{Chapter15/figures/PGN_face1_GT} &
\includegraphics[width=0.3\textwidth]{Chapter15/figures/PGN_face1_MSE} &
\includegraphics[width=0.3\textwidth]{Chapter15/figures/PGN_face1_AL}
\end{tabular}
\fi
\caption{
预测生成网络是一个学习哪些特征显著的例子。
在这个例子中,预测生成网络已被训练成在特定视角预测人头的3D模型。
\emph{(左)}真实情况。
这是一张网络应该生成的正确图片。
\emph{(中)}由具有\gls{mean_squared_error}的预测生成网络生成的图片。
因为与相邻皮肤相比,耳朵不会引起亮度的极大差异,所以它们的显著性不足以让模型学习表示它们。
\emph{(右)}由具有\gls{mean_squared_error}和对抗损失的模型生成的图片。
使用这个\gls{learned}的\gls{cost_function},由于耳朵遵循可预测的模式,因此耳朵是显著重要的。
学习哪些原因对于模型而言是足够重要和相关的,是一个重要的活跃研究领域。
以上图片由~\cite{lotter2015unsupervised}提供。
}
\label{fig:chap15_manface}
\end{figure}
% 536 head


% 535 head

正如~\cite{Janzing-et-al-ICML2012}指出,学习\gls{underlying}因素的好处是,如果真实的生成过程中$\RVx$是结果,$\RVy$是原因,那么建模$p(\RVx \mid \RVy)$对于$p(\RVy)$的变化是鲁棒的。
如果因果关系被逆转,这是不对的,因为根据\gls{bayes_rule},$p(\RVx \mid \RVy)$将会对$p(\RVy)$的变化十分敏感。
很多时候,我们考虑分布的变化(由于不同领域、时间不稳定性或任务性质的变化)时,\emph{因果机制是保持不变的}(``宇宙定律不变''),而\gls{underlying}因素的边缘分布是会变化的。
因此,通过学习试图恢复成因向量$\RVh$和$p(\RVx \mid \RVh)$的\gls{generative_model},我们可以期望最后的模型对所有种类的变化有更好的\gls{generalization}和鲁棒性。

% 536 mid

\section{\glsentrytext{distributed_representation}}
\label{sec:distributed_representation}
% 536 end

\gls{distributed_representation}的概念(由很多元素组合的表示,这些元素之间可以设置成可分离的)是\gls{representation_learning}最重要的工具之一。
\gls{distributed_representation}非常强大,因为他们能用具有$k$个值的$n$个特征去描述$k^n$个不同的概念。
正如我们在本书中看到的,具有多个\gls{hidden_unit}的\gls{NN}和具有多个\gls{latent_variable}的概率模型都利用了\gls{distributed_representation}的策略。
我们现在再介绍一个观察结果。
许多\gls{DL}算法基于的假设是,\gls{hidden_unit}能够学习表示出解释数据的\gls{underlying}\gls{causal_factor},就像\secref{sec:semi_supervised_disentangling_of_causal_factors}中讨论的一样。
这种方法在\gls{distributed_representation}上是自然的,因为表示空间中的每个方向都对应着一个不同的\gls{underlying}配置变量的值。

%  537 head

$n$维二元向量是一个\gls{distributed_representation}的示例,有$2^n$种配置,每一种都对应输入空间中的一个不同区域,如\figref{fig:chap15_distributed}所示。
这可以与\emph{\gls{symbolic_representation}}相比较,其中输入关联到单一符号或类别。
如果字典中有$n$个符号,那么可以想象有$n$个特征监测器,每个特征探测器监测相关类别的存在。
在这种情况下,只有表示空间中$n$个不同配置才有可能在输入空间中刻画$n$个不同的区域,如\figref{fig:chap15_nondistributed}所示。
这样的\gls{symbolic_representation}也被称为~\gls{one_hot}~表示,因为它可以表示成相互排斥的$n$维二元向量(其中只有一位是激活的)。
\gls{symbolic_representation}是更广泛的\gls{nondistributed_representation}类中的一个具体示例,它可以包含很多条目,但是每个条目没有显著意义的单独控制作用。

% 537 mid

% 538 1 page

\begin{figure}[!htb]
\ifOpenSource
\centerline{\includegraphics{figure.pdf}}
\else
\centerline{\includegraphics{Chapter15/figures/distributed}}
\fi
\caption{基于\gls{distributed_representation}的学习算法如何将输入空间分割成多个区域的图示。
这个例子具有二元变量$h_1$,$h_2$,$h_3$。
每个特征通过为\gls{learned}的线性变换设定输出阀值而定义。
每个特征将$\SetR^2$分成两个半平面。
令$h_i^+$表示输入点$h_i=1$的集合;$h_i^-$表示输入点$h_i=0$的集合。
在这个图示中,每条线代表着一个$h_i$的决策边界,对应的箭头指向边界的$h_i^+$区域。
整个表示在这些半平面的每个相交区域都指定一个唯一值。
例如,表示值为$[1,1,1]^\top$对应着区域$h_1^+ \cap h_2^+ \cap h_3^+$。
可以将以上表示和\figref{fig:chap15_nondistributed}中的\gls{nondistributed_representation}进行比较。
在输入维度是$d$的一般情况下,\gls{distributed_representation}通过半空间(而不是半平面)的交叉分割$\SetR^d$。
具有$n$个特征的\gls{distributed_representation}给$O(n^d)$个不同区域分配唯一的编码,而具有$n$个样本的\gls{nearest_neighbor}算法只能给$n$个不同区域分配唯一的编码。
因此,\gls{distributed_representation}能够比\gls{nondistributed_representation}多分配指数级的区域。
注意并非所有的$\Vh$值都是可取的(这个例子中没有$\Vh=\mathbf{0}$),在\gls{distributed_representation}上的\gls{linear_classifier}不能向每个相邻区域分配不同的类别标识;
甚至深度线性阀值网络的~\glssymbol{VC}~维只有$O(w\log w)$(其中$w$是权重数目)\citep{sontag1998vc}。
强表示层和弱分类器层的组合是一个强\gls{regularizer}。
试图学习``人''和``非人''概念的分类器不需要给表示为``戴眼镜的女人''和``没有戴眼镜的男人''的输入分配不同的类别。
容量限制鼓励每个分类器关注少数几个$h_i$,鼓励$\Vh$以线性可分的方式学习表示这些类别。
}
\label{fig:chap15_distributed}
\end{figure}

% 538 1 page

% 539 head

\begin{figure}[!htb]
\ifOpenSource
\centerline{\includegraphics{figure.pdf}}
\else
\centerline{\includegraphics{Chapter15/figures/non_distributed}}
\fi
\caption{\gls{nearest_neighbor}算法如何将输入空间分成不同区域的图示。
\gls{nearest_neighbor}算法是一个基于\gls{nondistributed_representation}的学习算法的示例。
不同的\gls{nondistributed}算法可以具有不同的几何形状,但是它们通常将输入空间分成区域,\emph{每个区域具有不同的参数}。
\gls{nondistributed}方法的优点是,给定足够的参数,它能够拟合一个\gls{training_set},而不需要复杂的优化算法。
因为它直接为每个区域\emph{独立地}设置不同的参数。
缺点是,\gls{nondistributed_representation}的模型只能通过平滑先验来局部地\gls{generalize},因此学习波峰波谷多于样本的复杂函数时,该方法是不可行的。
和\gls{distributed_representation}的对比,可以参照\figref{fig:chap15_distributed}。
}
\label{fig:chap15_nondistributed}
\end{figure}

% 539 head


% 537 mid

以下是基于\gls{nondistributed_representation}的学习算法的示例:
\begin{itemize}
	\item 聚类算法,包含$k$-means算法:每个输入点恰好分配到一个类别。

	\item $k$-\gls{nearest_neighbor}算法:给定一个输入,一个或几个模板或原型样本与之关联。
	在$k >  1$ 的情况下,每个输入都使用多个值来描述,但是它们不能彼此分开控制,因此这不能算真正的\gls{distributed_representation}。

	\item \gls{decision_tree}:给定输入时,只有一个叶节点(和从根到该叶节点路径上的点)是被激活的。

	\item \gls{gauss_mixture}和\gls{mixture_of_experts}:模板(聚类中心)或专家关联一个激活的\emph{程度}。
	和$k$-\gls{nearest_neighbor}算法一样,每个输入用多个值表示,但是这些值不能轻易地彼此分开控制。

	\item 具有\gls{gaussian_kernel} (或其他类似的\gls{local_kernel})的\gls{kernel_machines}: 尽管每个``\gls{support_vectors}''或模板样本的激活程度是连续值,但仍然会出现和\gls{gauss_mixture}相同的问题。

% 537 end

	\item 基于~\gls{n_gram}~的语言或翻译模型:根据后缀的树结构划分上下文集合(符号序列)。
	例如,一个叶节点可能对应于最后两个单词$w_1$和$w_2$。
	树上的每个叶节点分别估计单独的参数(有些共享也是可能的)。
\end{itemize}

% -- 539 mid  



对于部分\gls{nondistributed}算法而言,有些输出并非是恒定的,而是在相邻区域之间内插。
参数(或样本)的数量和它们能够定义区域的数量之间仍保持线性关系。

% -- 539 mid  

将\gls{distributed_representation}和\gls{symbolic_representation}区分开来的一个重要概念是,由不同概念之间的\emph{共享属性而产生的\gls{generalization}}。
作为纯符号,``猫''和``狗''之间的距离和任意其他两种符号的距离一样。
然而,如果将它们与有意义的\gls{distributed_representation}相关联,那么关于猫的很多特点可以推广到狗,反之亦然。
例如,我们的\gls{distributed_representation}可能会包含诸如``具有皮毛''或``腿的数目''这类在``猫''和``狗''的嵌入上具有相同值的项。
正如\secref{sec:neural_language_models}所讨论的,作用于单词\gls{distributed_representation}的神经语言模型比其他直接对单词~\gls{one_hot}~表示进行操作的模型\gls{generalize}得更好。
\gls{distributed_representation}具有丰富的\emph{相似性空间},语义上相近的概念(或输入)在距离上接近,这是纯粹的\gls{symbolic_representation}所缺少的特点。

% -- 540 head

在学习算法中使用\gls{distributed_representation}何时以及为什么具有统计优势?
当一个明显复杂的结构可以用较少参数紧致地表示时,\gls{distributed_representation}具有统计上的优点。
一些传统的\gls{nondistributed}学习算法仅仅在平滑假设的情况下能够\gls{generalize},也就是说如果$u\approx v$,那么学习到的目标函数$f$通常具有$f(u) \approx f(v)$的性质。
有许多方法来形式化这样一个假设,但其结果是如果我们有一个样本$(x,y)$,并且我们知道$f(x) \approx y$,那么我们可以选取一个\gls{estimator}$\hat{f}$近似地满足这些限制,并且当我们移动到附近的输入$x + \epsilon$时,$\hat{f}$尽可能少地发生改变。
显然这个假设是非常有用的,但是它会遭受\gls{curse_of_dimensionality}:
学习出一个能够在很多不同区域上增加或减少很多次的目标函数
\footnote{一般来说,我们可能会想要学习一个函数,这个函数在指数级数量区域的表现都是不同的:在$d$-维空间中,为了区分每一维,至少有两个不同的值。
我们想要函数$f$区分这$2^d$个不同的区域,需要$O(2^d)$量级的训练样本},
我们可能需要至少和可区分区域数量一样多的样本。
我们可以将每一个区域视为一个类别或符号:通过让每个符号(或区域)具有单独的自由度,我们可以学习出从符号映射到值的任意\gls{decoder}。
然而,这不能推广到新区域的新符号上。

% -- 540 mid

如果我们幸运的话,除了平滑之外,目标函数可能还有一些其他规律。
例如,具有\gls{max_pooling}的\gls{convolutional_network}可以在不考虑对象在图像中位置(即使对象的空间变换不对应输入空间的平滑变换)的情况下识别出对象。

% -- 540 mid

让我们检查\gls{distributed_representation}学习算法的一个特殊情况,它通过对输入的线性函数进行阀值处理来提取二元特征。
该表示中的每个二元特征将$\SetR^d$分成一对半空间,如\figref{fig:chap15_distributed}所示。
$n$个相应半空间的指数级数量的交集确定了该\gls{distributed_representation}\gls{learner}能够区分多少区域。
空间$\SetR^d$中的$n$个超平面的排列组合能够生成多少区间?
通过应用关于超平面交集的一般结果~\citep{Zaslavsky-1975},我们发现~\citep{Pascanu+et+al-ICLR2014b}这个二元特征表示能够区分的空间数量是
\begin{equation}
	\sum_{j=0}^d \binom{n}{j} = O(n^d).
\end{equation}
因此,我们会发现关于输入大小呈指数级增长,关于\gls{hidden_unit}的数量呈多项式级增长。

% -- 541 head

这提供了\gls{distributed_representation}\gls{generalization}能力的一种几何解释:$O(nd)$个参数(空间$\SetR^d$中的$n$个线性阀值特征)能够明确表示输入空间中$O(n^d)$个不同区域。
如果我们没有对数据做任何假设,并且每个区域使用唯一的符号来表示,每个符号使用单独的参数去识别$\SetR^d$中的对应区域,那么指定$O(n^d)$个区域需要$O(n^d)$个样本。
更一般地,\gls{distributed_representation}的优势还可以体现在我们对\gls{distributed_representation}中的每个特征使用非线性的、可能连续的\gls{feature_extractor},而不是\gls{linear_threshold_units}的情况。
在这种情况下,如果具有$k$个参数的参数变换可以学习输入空间中的$r$个区域($k\ll r$),并且如果学习这样的表示有助于关注的任务,那么这种方式会比\gls{nondistributed}\gls{setting}(我们需要$O(r)$个样本来获得相同的特征,将输入空间相关联地划分成$r$个区域。)\gls{generalize}得更好。
使用较少的参数来表示模型意味着我们只需拟合较少的参数,因此只需要更少的训练样本去获得良好的\gls{generalization}。

% -- 541 mid

另一个解释基于\gls{distributed_representation}的模型\gls{generalization}能力更好的说法是,尽管能够明确地编码这么多不同的区域,但它们的\gls{capacity}仍然是很有限的。
例如,\gls{linear_threshold_units}\gls{NN}的~\glssymbol{VC}~维仅为$O(w\log w)$, 其中$w$是权重的数目~\citep{sontag1998vc}。
这种限制出现的原因是,虽然我们可以为表示空间分配非常多的唯一码,但是我们不能完全使用所有的码空间,也不能使用\gls{linear_classifier}学习出从表示空间$\Vh$到输出$\Vy$的任意函数映射。
因此使用与\gls{linear_classifier}相结合的\gls{distributed_representation}传达了一种先验信念,待识别的类在$\Vh$代表的\gls{underlying}\gls{causal_factor}的函数下是线性可分的。
我们通常想要学习类别,例如所有绿色对象的图像集合,或是所有汽车图像集合,但不会是需要非线性~\ENNAME{XOR}~逻辑的类别。
例如,我们通常不会将数据划分成所有红色汽车和绿色卡车作为一个集合,所有绿色汽车和红色卡车作为另一个集合。

% -- 542 head

到目前为止讨论的想法都是抽象的,但是它们可以通过实验验证。
\cite{Zhou-et-al-ICLR2015}发现,在~\ENNAME{ImageNet}~和~\ENNAME{Places}~基准数据集上训练的深度\gls{convolutional_network}中的\gls{hidden_unit}\gls{learned}的特征通常是可以解释的,对应人类自然分配的\gls{label}。
在实践中,\gls{hidden_unit}并不能总是学习出具有简单语言学名称的事物,但有趣的是,这些事物会在那些最好的\gls{CV}深度网络的顶层附近出现。
这些特征的共同之处在于,我们可以设想\emph{学习其中的每个特征不需要知道所有其他特征的所有配置}。
\cite{radford2015unsupervised}发现\gls{generative_model}可以学习人脸图像的表示,在表示空间中的不同方向捕获不同的\gls{underlying}\gls{factors_of_variation}。
\figref{fig:chap15_generative_glasses}展示表示空间中的一个方向对应着该人是男性还是女性,而另一个方向对应着该人是否戴着眼镜。
这些特征都是自动发现的,而非先验固定的。
我们没有必要为\gls{hidden_unit}分类器提供\gls{label}:只要该任务需要这样的特征,\gls{GD}就能在感兴趣的\gls{objective_function}上自然地学习出语义上有趣的特征。
我们可以学习出男性和女性之间的区别,或者是眼镜的存在与否,而不必通过涵盖所有这些值组合的样本来表征其他$n-1$个特征的所有配置。
这种形式的统计可分离性质能够\gls{generalize}到训练期间从未见过的新特征上。

% 543 head

% 542 end

\begin{figure}[!htb]
\ifOpenSource
\centerline{\includegraphics{figure.pdf}}
\else
\begin{tabular}{ccccccc}
\includegraphics[width=0.1\textwidth]{Chapter15/figures/man_with_glasses} &
- &
\includegraphics[width=0.1\textwidth]{Chapter15/figures/man_without_glasses} &
+ &
\includegraphics[width=0.1\textwidth]{Chapter15/figures/woman_without_glasses} &
= &
\includegraphics[width=0.3\textwidth]{Chapter15/figures/woman_with_glasses}
\end{tabular}
\fi
\caption{\gls{generative_model}学到了\gls{distributed_representation},能够从戴眼镜的概念中区分性别的概念。
如果我们从一个戴眼镜的男人的概念表示向量开始,然后减去一个没戴眼镜的男人的概念表示向量,最后加上一个没戴眼镜的女人的概念表示向量,那么我们会得到一个戴眼镜的女人的概念表示向量。
\gls{generative_model}将所有这些表示向量正确地解码为可被识别为正确类别的图像。
图片转载许可自~\cite{radford2015unsupervised}。
}
\label{fig:chap15_generative_glasses}
\end{figure}

% 542 end

\section{得益于深度的指数增益}
\label{sec:exponential_gains_from_depth}
我们已经在\secref{sec:universal_approximation_properties_and_depth}中看到,\gls{MLP}是\gls{universal_approximator},相比于浅层网络,一些函数能够用指数级小的\gls{deep_network}表示。
缩小模型规模能够提高统计效率。
在本节中,我们描述如何将类似结果更一般地应用于其他具有分布式隐藏表示的模型。

% 543 mid

在\secref{sec:distributed_representation}中,我们看到了一个\gls{generative_model}的示例,能够学习人脸图像的\gls{underlying}\gls{explanatory_factor},包括性别以及是否佩戴眼镜。
完成这个任务的\gls{generative_model}是基于一个\gls{DNN}的。
浅层网络例如线性网络不能学习出这些抽象\gls{explanatory_factor}和图像像素之间的复杂关系。
在这个任务和其他~\glssymbol{AI}~任务中,这些因子几乎彼此独立地被抽取,但仍然对应到有意义输入的因素,很有可能是高度抽象的,并且和输入呈高度非线性的关系。
我们认为这需要\emph{深度}\gls{distributed_representation},需要许多非线性组合来获得较高级的特征(被视为输入的函数)或因子(被视为生成原因)。

% 543 mid

在许多不同\gls{setting}中已经证明,
非线性和重用特征层次结构的组合来组织计算,可以使\gls{distributed_representation}获得指数级加速之外,还可以获得统计效率的指数级提升。
许多种类的只有一个\gls{hidden_layer}的网络(例如,具有饱和非线性,布尔门,和/积,或~\glssymbol{RBF}~单元的网络)都可以被视为\gls{universal_approximator}。
在给定足够多\gls{hidden_unit}的情况下,这个模型族是一个\gls{universal_approximator},可以在任意非零允错级别近似一大类函数(包括所有连续函数)。
然而,\gls{hidden_unit}所需的数量可能会非常大。
关于深层架构表达能力的理论结果表明,有些函数族可以高效地通过深度$k$层的网络架构表示,但是深度不够(深度为2或$k-1$)时会需要指数级(相对于输入大小而言)的\gls{hidden_unit}。

% -- 543 end

% 544 head

在\secref{sec:universal_approximation_properties_and_depth}中,我们看到确定性\gls{feedforward_network}是函数的\gls{universal_approximator}。
许多具有单个\gls{hidden_layer}(\gls{latent_variable})的\gls{structured_probabilistic_models}(包括\gls{RBM},\gls{DBN})是概率分布的\gls{universal_approximator}~\citep{LeRoux-Bengio-2008,LeRoux+Bengio-2010,Montufar-2011,Montufar-arxiv2013,Krause-et-al-ICML2013}。

% 544 mid

在\secref{sec:universal_approximation_properties_and_depth}中,我们看到足够深的\gls{feedforward_network}会比深度不够的网络具有指数级优势。
这样的结果也能从诸如概率模型的其他模型中获得。
\firstall{sum_product_network}%,\glssymbol{sum_product_network}
~\citep{Poon+Domingos-2011}是这样的一种概率模型。
这些模型使用多项式回路来计算一组\gls{RV}的\gls{PD}。
\cite{Delalleau+Bengio-2011-small}表明存在一种\gls{PD},对~\glssymbol{sum_product_network}~的最小深度有要求,以避免模型规模呈指数级增长。
后来,\cite{Martens+Medabalimi-arxiv2014}表明,任意两个有限深度的~\glssymbol{sum_product_network}~之间都会存在显著差异,并且一些使~\glssymbol{sum_product_network}~易于处理的约束可能会限制其表示能力。

% 544 mid

另一个有趣的进展是,一系列和\gls{convolutional_net}相关的\gls{deep_circuit}族表达能力的理论结果,即使让\gls{shadow_circuit}只去近似\gls{deep_circuit}计算的函数,也能突出反映\gls{deep_circuit}的指数级优势~\citep{Cohen-et-al-arXiv2015}。
相比之下,以前的理论工作只研究了\gls{shadow_circuit}必须精确复制特定函数的情况。

% 544 mid

\section{提供发现潜在原因的线索}
\label{sec:providing_clues_to_discover_underlying_causes}

% 544 end

我们回到最初的问题之一来结束本章:什么原因能够使一个表示比另一个表示更好?
首先在\secref{sec:semi_supervised_disentangling_of_causal_factors}中介绍的一个答案是,一个理想的表示能够区分生成数据变化的\gls{underlying}\gls{causal_factor},特别是那些与我们的应用相关的因素。
\gls{representation_learning}的大多数策略都会引入一些有助于学习\gls{underlying}\gls{factors_of_variation}的线索。
这些线索可以帮助\gls{learner}将这些观察到的因素与其他因素分开。
\gls{supervised_learning}提供了非常强的线索:每个观察向量$\Vx$的\gls{label}$\Vy$,它通常直接指定了至少一个\gls{factors_of_variation}。
更一般地,为了利用丰富的\gls{unlabeled}数据,\gls{representation_learning}会使用关于\gls{underlying}因素的其他不太直接的提示。
这些提示包含一些我们(学习算法的设计者)为了引导\gls{learner}而强加的隐式先验信息。
诸如\gls{no_free_lunch_theorem}的这些结果表明,\gls{regularization}策略对于获得良好\gls{generalization}是很有必要的。
当不可能找到一个普遍良好的\gls{regularization}策略时,\gls{DL}的一个目标是找到一套相当通用的\gls{regularization}策略,使其能够适用于各种各样的~\glssymbol{AI}~任务(类似于人和动物能够解决的任务)。

% 545 head

在此,我们提供了一些通用\gls{regularization}策略的列表。
该列表显然是不详尽的,但是给出了一些学习算法是如何发现对应\gls{underlying}因素的特征的具体示例。
该列表在~\cite{Bengio-Courville-Vincent-TPAMI-2012}的第3.1节中提出,这里进行了部分拓展。
\begin{itemize}
	\item \emph{平滑}:假设对于单位$\Vd$和小量$\epsilon$有$f(\Vx + \epsilon \Vd) \approx f(\Vx)$。
	这个假设允许\gls{learner}从训练样本\gls{generalize}到输入空间中附近的点。
	许多\gls{ML}算法都利用了这个想法,但它不能克服\gls{curse_of_dimensionality}难题。


	\item \emph{线性}:很多学习算法假定一些变量之间的关系是线性的。
	这使得算法能够预测远离观测数据的点,但有时可能会导致一些极端的预测。
	大多数简单的学习算法不会做平滑假设,而会做线性假设。
	这些假设实际上是不同的,具有很大权重的线性函数在高维空间中可能不是非常平滑的。
	参看~\cite{Goodfellow-2015-adversarial}了解关于线性假设局限性的进一步讨论。

% 545 mid

	\item \emph{多个\gls{explanatory_factor}}:许多\gls{representation_learning}算法受以下假设的启发,数据是由多个\gls{underlying}\gls{explanatory_factor}生成的,并且给定每一个因子的状态,大多数任务都能轻易解决。
	\secref{sec:semi_supervised_disentangling_of_causal_factors}描述了这种观点如何通过\gls{representation_learning}来启发\gls{semi_supervised_learning}的。
	学习$p(\Vx)$的结构要求学习出一些对建模$p(\Vy\mid\Vx)$同样有用的特征,因为它们都涉及到相同的\gls{underlying}\gls{explanatory_factor}。
	\secref{sec:distributed_representation}介绍了这种观点如何启发\gls{distributed_representation}的使用,表示空间中\gls{separate}方向对应着\gls{separate}\gls{factors_of_variation}。


	\item \emph{\gls{causal_factor}}:该模型认为\gls{learned}表示所描述的\gls{factors_of_variation}是观察数据$\Vx$的成因,而并非反过来。
	正如\secref{sec:semi_supervised_disentangling_of_causal_factors}中讨论的,这对于\gls{semi_supervised_learning}是有利的,当\gls{underlying_cause}上的分布发生改变,或者我们应用模型到一个新的任务上时,\gls{learned}的模型都会更加鲁棒。

% -- 546 head

	\item \emph{深度,或者\gls{explanatory_factor}的层次组织}:高级抽象概念能够通过将简单概念层次化来定义。
	从另一个角度来看,深度架构表达了我们认为任务应该由多个程序步骤完成的观念,其中每一个步骤回溯到先前步骤处理之后的输出。


	\item \emph{任务间共享因素}:
	当多个对应到不同变量$\RSy_i$的任务共享相同的输入$\RVx$时,或者当每个任务关联到全局输入$\RVx$的子集或者函数$f^{(i)}(\RVx)$时,我们会假设每个变量$\RSy_i$关联到来自相关因素$\RVh$公共池的不同子集。
	因为这些子集有重叠,所以通过共享的中间表示$ P(\RVh \mid \RVx)$来学习所有的$P(\RSy_i \mid \RVx)$能够使任务间共享统计强度。

% -- 546 mid

	\item \emph{\gls{manifold}}:概率质量集中,并且集中区域是局部连通的,且占据很小的体积。
	在连续情况下,这些区域可以用比数据所在原始空间低很多维的低维\gls{manifold}来近似。
	很多\gls{ML}算法只在这些\gls{manifold}上有效\citep{Goodfellow-2015-adversarial}。
	一些\gls{ML}算法,特别是\gls{AE},会试图显式地学习\gls{manifold}的结构。


	\item \emph{自然聚类}:很多\gls{ML}算法假设输入空间中每个连通\gls{manifold}可以被分配一个单独的类。
	数据分布在许多个不连通的\gls{manifold}上,但相同\gls{manifold}上数据的类别是相同的。
	这个假设激励了各种学习算法,包括\gls{tangent_propagation}、\gls{double_backprop}、\gls{manifold_tangent_classifier}和\gls{adversarial_training}。


	\item \emph{时间和空间相干性}:\gls{SFA}和相关的算法假设,最重要的\gls{explanatory_factor}随时间变化很缓慢,或者至少假设预测真实的\gls{underlying}\gls{explanatory_factor}比预测诸如像素值这类原始观察会更容易些。
	读者可以参考\secref{sec:slow_feature_analysis},进一步了解这个方法。


	\item \emph{稀疏性}:假设大部分特征和大部分输入不相关,如在表示猫的图像时,没有必要使用象鼻的特征。
	因此,我们可以强加一个先验,任何可以解释为``存在''或``不存在''的特征在大多数时间都是不存在的。

% -- 547 head --

	\item \emph{简化因子\gls{dependency}}:在良好的高级表示中,因子会通过简单的\gls{dependency}相互关联。
	最简单的可能是边缘独立,即$P(\RVh) = \prod_i P(\RVh_i)$。
	但是线性\gls{dependency}或浅层\gls{AE}所能表示的\gls{dependency}关系也是合理的假设。
	这可以从许多物理定律中看出来,并且假设在\gls{learned}表示的顶层插入线性预测器或\gls{factorized}先验。
\end{itemize}


\gls{representation_learning}的概念将许多\gls{DL}形式联系在了一起。
\gls{feedforward_network}和\gls{recurrent_network},\gls{AE}和深度概率模型都在学习和使用表示。
学习最佳表示仍然是一个令人兴奋的研究方向。

% -- 547 --

% !Mode:: "TeX:UTF-8"
% Translator: Tianfan Fu
\chapter{\glsentrytext{DL}中的\glsentrytext{structured_probabilistic_models}}
\label{chap:structured_probabilistic_models_for_deep_learning}
% 549


%\gls{DL}为研究者们提供了许多指导性的建模和设计算法的思路。%??
\gls{DL}为研究者们提供了许多建模方式,用以设计以及描述算法。
其中一种形式是\firstgls{structured_probabilistic_models}的思想。
我们曾经在\secref{sec:structured_probabilistic_models_chap3}中简要讨论过\gls{structured_probabilistic_models}。
此前简要的介绍已经足够使我们充分了解如何使用\gls{structured_probabilistic_models}作为描述第二部分中某些算法的语言。
现在在第三部分,我们可以看到\gls{structured_probabilistic_models}是许多\gls{DL}重要研究方向的关键组成部分。
作为讨论这些研究方向的预备知识,本章将更加详细地描述\gls{structured_probabilistic_models}。
本章内容是自洽的,所以在阅读本章之前读者不需要回顾之前的介绍。
% 549


\gls{structured_probabilistic_models}使用图来描述概率分布中随机变量之间的直接相互作用,从而描述一个概率分布。
在这里我们使用了图论(一系列结点通过一系列边来连接)中``图''的概念,由于模型结构是由图定义的,所以这些模型也通常被称为\firstgls{graphical_models}。
% 549


\gls{graphical_models}的研究社群是巨大的,并提出过大量的模型、训练算法和推断算法。
在本章中,我们将介绍\gls{graphical_models}中几个核心方法的基本背景,并且重点描述已被证明对\gls{DL}社群最有用的观点。
如果你已经熟知\gls{graphical_models},那么你可以跳过本章的绝大部分。
然而,我们相信即使是资深的\gls{graphical_models}方向的研究者也会从本章的最后一节中获益匪浅,详见\secref{sec:the_deep_learning_approach_to_structured_probabilistic_models},其中我们强调了在\gls{DL}算法中使用图模型的独特方式。
相比于其他\gls{graphical_models}研究领域的是,\gls{DL}的研究者们通常会使用完全不同的模型结构、学习算法和推断过程。
在本章中,我们将指明这种区别并解释其中的原因。
% 550 head


我们首先介绍了构建大规模概率模型时面临的挑战。
之后,我们介绍如何使用一个图来描述概率分布的结构。
尽管这个方法能够帮助我们解决许多挑战和问题,它本身仍有很多缺陷。
\gls{graphical_models}中的一个主要难点就是判断哪些变量之间存在直接的相互作用关系,也就是对于给定的问题哪一种图结构是最适合的。
在\secref{sec:learning_about_dependencies}中,我们通过了解\firstgls{dependency},简要概括了解决这个难点的两种方法。
%最后,在\secref{sec:the_deep_learning_approach_to_structured_probabilistic_models}中,我们讨论并强调了\gls{graphical_models}在\gls{DL}中的一些独特之处和一些特有的方法,作为本章的收尾。
最后,作为本章的收尾,我们在\secref{sec:the_deep_learning_approach_to_structured_probabilistic_models}中讨论\gls{DL}研究者使用\gls{graphical_models}特定方式的独特之处。
% 550  ok



\section{非结构化建模的挑战}
\label{sec:the_challenge_of_unstructured_modelling}
% 16.1   p 550 


\gls{DL}的目标是使得\gls{ML}能够解决许多\gls{AI}中亟需解决的挑战。
这也意味着它们能够理解具有丰富结构的高维数据。
举个例子,我们希望\glssymbol{AI}的算法能够理解自然图片\footnote{自然图片指的是能够在正常的环境下被照相机拍摄的图片,不同于合成的图片,或者一个网页的截图等等。},表示语音的声音信号和包含许多词和标点的文档。
% 550  ok


分类问题可以把这样一个来自高维分布的数据作为输入,然后使用一个类别的标签来概括它——这个标签可以是照片中是什么物品,一段语音中说的是哪个单词,也可以是一段文档描述的是哪个话题。
这个分类过程丢弃了输入数据中的大部分信息,然后产生单个值的输出(或者是关于单个输出值的概率分布)。
这个分类器通常可以忽略输入数据的很多部分。
例如,当我们识别一张照片中的一个物体时,我们通常可以忽略图片的背景。
% 550  ok


我们也可以使用概率模型完成许多其他的任务。
这些任务通常相比于分类成本更高。
其中的一些任务需要产生多个输出。
大部分任务需要对输入数据整个结构的完整理解,所以并不能舍弃数据的一部分。
这些任务包括以下几个:
\begin{itemize}
\item \textbf{估计密度函数}:给定一个输入$\Vx$,\gls{ML}系统返回一个对数据生成分布的真实密度函数$p(\Vx)$的估计。
这只需要一个输出,但它需要完全理解整个输入。
即使向量中只有一个元素不太正常,系统也会给它赋予很低的概率。
% 551  ok
	
	
\item
\textbf{\gls{denoising}}:给定一个受损的或者观察有误的输入数据$\tilde{\Vx}$,\gls{ML}系统返回一个对原始的真实$\Vx$的估计。
举个例子,有时候\gls{ML}系统需要从一张老相片中去除灰尘或者抓痕。
这个系统会产生多个输出值(对应着估计的干净样本$\Vx$的每一个元素),并且需要我们有一个对输入的整体理解(因为即使只有一个损坏的区域,仍然会显示最终估计被损坏)。%??
	% 551 ok
	
\item
\textbf{缺失值的填补}:给定$\Vx$的某些元素作为观察值,模型被要求返回一个$\Vx$
一些或者全部未观察值的估计或者概率分布。
这个模型返回的也是多个输出。
由于这个模型需要恢复$\Vx$的每一个元素,所以它必须理解整个输入。
% 551 ok
	
	
\item \textbf{采样}: 模型从分布$p(\Vx)$中抽取新的样本。
其应用包括语音合成,即产生一个听起来很像人说话的声音。
这个模型也需要多个输出以及对输入整体的良好建模。
即使样本只有一个从错误分布中产生的元素,那么采样的过程也是错误的。 
% 551
\end{itemize}

\figref{fig:chap16_fig-ssrbm}中描述了一个使用较小的自然图片的采样任务。
% 551 ok 

\begin{figure}[!htb]
\ifOpenSource
\centerline{\includegraphics{figure.pdf}}
\else
	\centerline{\includegraphics[width=0.9\textwidth]{Chapter16/figures/fig-ssrbm_nearest_train}}\ \\
     \centerline{\includegraphics[width=0.9\textwidth]{Chapter16/figures/fig-ssrbm_samples}}
\fi
	\caption{自然图片的概率建模。
\emph{(上)}~CIFAR-10数据集\citep{KrizhevskyHinton2009}中的$32\times 32$像素的样例图片。
\emph{(下)}从这个数据集上训练的\gls{structured_probabilistic_models}中抽出的样本。
每一个样本都出现在与其欧式距离最近的训练样本的格点中。
这种比较使得我们发现这个模型确实能够生成新的图片,而不是记住训练样本。
为了方便展示,两个集合的图片都经过了微调。
图片经~\citet{Courville+al-2011-small}许可转载。}
	\label{fig:chap16_fig-ssrbm}
\end{figure}

对上千甚至是上百万随机变量的分布建模,无论从计算上还是从统计意义上说,都是一个极具挑战性的任务。
假设我们只想对二值的随机变量建模。
这是一个最简单的例子,但是我们仍然无能为力。
对一个只有$32\times 32$像素的彩色(RGB)图片来说,存在$2^{3072}$种可能的二值图片。
这个数量已经超过了$10^{800}$,比宇宙中的原子总数还要多。
% 551 

通常意义上讲,如果我们希望对一个包含$n$个离散变量并且每个变量都能取$k$个值的$\Vx$的分布建模,那么最简单的表示$P(\Vx)$的方法需要存储一个可以查询的表格。
这个表格记录了每一种可能值的概率,则需要$k^n$个参数。
% 551 

基于下述几个原因,这种方式是不可行的:
\begin{itemize}
\item \emph{内存: 存储参数的开销。}
除了极小的$n$和$k$的值,用表格的形式来表示这样一个分布需要太多的存储空间。
% 551 
	
\item  \emph{统计的高效性}: 
当模型中的参数个数增加时,使用统计估计器估计这些参数所需要的训练数据数量也需要相应地增加。
因为基于查表的模型拥有天文数字级别的参数,为了准确地拟合,相应的训练集的大小也是相同级别的。
任何这样的模型都会导致严重的\gls{overfitting},除非我们添加一些额外的假设来联系表格中的不同元素(正如\secref{sec:n_grams}中所举的\gls{backoff}或者平滑~\gls{n_gram}~模型)。
% 552  end
	
\item \emph{运行时间:推断的开销。}
假设我们需要完成这样一个推断的任务,其中我们需要使用联合分布$P(\RVx)$来计算某些其他的分布,比如说边缘分布$P(\RSx_1)$或者是条件分布$P(\RSx_2\mid \RSx_1)$。
计算这样的分布需要对整个表格的某些项进行求和操作,因此这样的操作的运行时间和上述高昂的内存开销是一个级别的。
% 553 head
	
	
\item \emph{运行时间: 采样的开销。}
类似的,假设我们想要从这样的模型中采样。
最简单的方法就是从均匀分布中采样,$u\sim \text{U}(0,1)$,然后把表格中的元素累加起来,直到和大于$u$,然后返回最后一个加上的元素。
最差情况下,这个操作需要读取整个表格,所以和其他操作一样,它也需要指数级别的时间。
\end{itemize}
% 553 



基于表格操作的方法的主要问题是我们显式地对每一种可能的变量子集所产生的每一种可能类型的相互作用建模。
在实际问题中我们遇到的概率分布远比这个简单。
通常,许多变量只是间接地相互作用。
% 553


例如,我们想要对接力跑步比赛中一个队伍完成比赛的时间进行建模。
假设这个队伍有三名成员:Alice, Bob和Carol。
在比赛开始时,Alice拿着接力棒,开始跑第一段距离。
在跑完她的路程以后,她把棒递给了Bob。
然后Bob开始跑,再把棒给Carol,Carol跑最后一棒。
我们可以用连续变量来建模他们每个人完成的时间。
因为Alice第一个跑,所以她的完成时间并不依赖于其他的人。
Bob的完成时间依赖于Alice的完成时间,因为Bob只能在Alice跑完以后才能开始跑。
如果Alice跑得更快,那么Bob也会完成得更快。
所有其他关系都可以被类似地推出。
最后,Carol的完成时间依赖于她的两个队友。
如果Alice跑得很慢,那么Bob也会完成得更慢。
结果,Carol将会更晚开始跑步,因此她的完成时间也更有可能要晚。
然而,在给定Bob完成时间的情况下,Carol的完成时间只是\emph{间接地}依赖于Alice的完成时间。
如果我们已经知道了Bob的完成时间,知道Alice的完成时间对估计Carol的完成时间并无任何帮助。
这意味着我们可以通过仅仅两个相互作用来建模这个接力赛。
这两个相互作用分别是Alice的完成时间对Bob的完成时间的影响和Bob的完成时间对Carol的完成时间的影响。
在这个模型中,我们可以忽略第三种间接的相互作用,即Alice的完成时间对Carol的完成时间的影响。
% 553


\gls{structured_probabilistic_models}为随机变量之间的直接作用提供了一个正式的建模框架。
这种方式大大减少了模型的参数个数以致于模型只需要更少的数据来进行有效的估计。
%这些更轻便的模型在模型存储,模型推断以及从模型中采样时有着更小的计算开销。
这些更小的模型大大减小了在模型存储、模型推断以及从模型中采样时的计算开销。
% 554 head  


\section{使用图描述模型结构}
\label{sec:using_graphs_to_describe_model_structure}
% 554


\gls{structured_probabilistic_models}使用图(在图论中``结点''是通过``边''来连接的)来表示随机变量之间的相互作用。
每一个结点代表一个随机变量。
每一条边代表一个直接相互作用。
这些直接相互作用隐含着其他的间接相互作用,但是只有直接的相互作用会被显式地建模。
% 554


使用图来描述概率分布中相互作用的方法不止一种。
在下文中我们会介绍几种最为流行和有用的方法。
\gls{graphical_models}可以被大致分为两类:基于有向无环图的模型和基于无向图的模型。
% 554


\subsection{\glsentrytext{directed_model}}
\label{sec:directed_models}
% 554


\firstgls{directed_graphical_model}是一种\gls{structured_probabilistic_models},也被称为\firstgls{BN}或者\firstgls{bayesian_network}\footnote{当我们希望``强调''从网络中计算出的值的``推断''本质,即强调这些值代表的是置信程度大小而不是事件的频率时,Judea Pearl建议使用``\gls{bayesian_network}''这个术语。}%??
 \citep{pearl85bayesian}。
% 554


之所以命名为\gls{directed_graphical_model}是因为所有的边都是有方向的,即从一个结点指向另一个结点。
这个方向可以通过画一个箭头来表示。
箭头所指的方向表示了这个随机变量的概率分布是由其他变量的概率分布所定义的。
画一个从结点$\RSa$到结点$\RSb$的箭头表示了我们用一个条件分布来定义$\RSb$,而$\RSa$是作为这个条件分布符号右边的一个变量。
换句话说,$\RSb$的概率分布依赖于$\RSa$的取值。
% 554


我们继续\secref{sec:the_challenge_of_unstructured_modelling}所讲的接力赛的例子,我们假设Alice的完成时间为$\RSt_0$,Bob的完成时间为$\RSt_1$,Carol的完成时间为$\RSt_2$。
就像我们之前看到的一样,$\RSt_1$的估计是依赖于$\RSt_0$的,$\RSt_2$的估计是直接依赖于$\RSt_1$的,但是仅仅间接地依赖于$\RSt_0$。
我们用一个\gls{directed_graphical_model}来建模这种关系,如\figref{fig:relay_race_graph}所示。
% 554 end


\begin{figure}[!htb]
\ifOpenSource
\centerline{\includegraphics{figure.pdf}}
\else
\centerline{\includegraphics{Chapter16/figures/relay_race_graph}}	
\fi
\caption{描述接力赛例子的\gls{directed_graphical_model}。
Alice的完成时间$\RSt_0$影响了Bob的完成时间$\RSt_1$,因为Bob只能在Alice完成比赛后才开始。
类似的,Carol也只会在Bob完成之后才开始,
所以Bob的完成时间$\RSt_1$直接影响了Carol的完成时间$\RSt_2$。}
\label{fig:relay_race_graph}
\end{figure}


% 555 head 
正式地说,变量$\RVx$的有向概率模型是通过有向无环图$\CalG$ (每个结点都是模型中的随机变量)
和一系列\firstgls{local_conditional_probability_distribution} $p(\RSx_i\mid Pa_{\CalG}(\RSx_i))$来定义的,其中$Pa_{\CalG}(\RSx_i)$表示结点$\RSx_i$的所有父结点。
$\RVx$的概率分布可以表示为
\begin{align}
\label{eqn:161}
p(\RVx) = \prod_{i} p(\RSx_i\mid Pa_{\CalG}(\RSx_i)).
\end{align}
% 555 


在之前所述的接力赛的例子中,参考\figref{fig:relay_race_graph},这意味着概率分布可以被表示为
\begin{align}
\label{eqn:162}
p(\RSt_0,\RSt_1,\RSt_2) = p(\RSt_0)p(\RSt_1\mid \RSt_0)p(\RSt_2\mid \RSt_1).
\end{align}
% 555 


这是我们看到的第一个\gls{structured_probabilistic_models}的实际例子。
我们能够检查这样建模的计算开销,为了验证相比于非结构化建模,结构化建模为什么有那么多的优势。
% 555 


假设我们采用从第$0$分钟到第$10$分钟每$6$秒一块的方式离散化地表示时间。
这使得$\RSt_0$,$\RSt_1$和$\RSt_2$都是一个有$100$个取值可能的离散变量。
如果我们尝试着用一个表来表示$p(\RSt_0, \RSt_1, \RSt_2)$,那么我们需要存储$999,999$个值
(100个$\RSt_0$的可能取值 $\times$ $\RSt_1$的可能取值 $\times$ 100个$\RSt_2$的可能取值 减去1,由于存在所有的概率之和为$1$的限制,所以其中有$1$个值的存储是多余的)。
反之,如果我们用一个表来记录每一种条件概率分布,那么表中记录$\RSt_0$的分布需要存储$99$个值,给定$\RSt_0$情况下$\RSt_1$的分布需要存储9900个值,给定$\RSt_1$情况下$\RSt_2$的分布也需要存储$9900$个值。
加起来总共需要存储$19, 899$个值。
这意味着使用\gls{directed_graphical_model}将参数的个数减少了超过$50$倍!
% 555 


通常意义上说,对每个变量都能取$k$个值的$n$个变量建模,基于建表的方法需要的复杂度是$O(k^n)$,就像我们之前观察到的一样。
现在假设我们用一个\gls{directed_graphical_model}来对这些变量建模。
如果$m$代表\gls{graphical_models}的单个条件概率分布中最大的变量数目(在条件符号的左右皆可),那么对这个\gls{directed_model}建表的复杂度大致为$O(k^m)$。
只要我们在设计模型时使其满足$m\ll n$,那么复杂度就会被大大地减小。
% 555 


换一句话说,只要图中的每个变量都只有少量的父结点,那么这个分布就可以用较少的参数来表示。
图结构上的一些限制条件,比如说要求这个图为一棵树,也可以保证一些操作(例如求一小部分变量的边缘或者条件分布)更加地高效。
% 556 head   16.2 to here 



决定哪些信息需要被包含在图中而哪些不需要是很重要的。
如果变量之间可以被假设为是\gls{conditional_independent}的,那么这个图可以包含这种简化假设。
当然也存在其他类型的简化\gls{graphical_models}的假设。
例如,我们可以假设无论Alice的表现如何,Bob总是跑得一样快
(实际上,Alice的表现很大概率会影响Bob的表现,这取决于Bob的性格,
如果在之前的比赛中Alice跑得特别快,这有可能鼓励Bob更加努力并取得更好的成绩,当然这也有可能使得Bob过分自信或者变得懒惰)。
那么Alice对Bob的唯一影响就是在计算Bob的完成时间时需要加上Alice的时间。
这个假设使得我们所需要的参数量从$O(k^2)$降到了$O(k)$。
然而,值得注意的是在这个假设下$\RSt_0$和$\RSt_1$仍然是直接相关的,因为$\RSt_1$表示的是Bob完成时的时间,并不是他跑的总时间。
这也意味着图中会有一个从$\RSt_0$指向$\RSt_1$的箭头。
``Bob的个人跑步时间相对于其他因素是独立的''这个假设无法在$\RSt_0$,$\RSt_1$,$\RSt_2$的图中被表示出来。
反之,我们只能将这个关系表示在条件分布的定义中。
这个条件分布不再是一个大小为$k\times k-1$的分别对应着$\RSt_0$,  $\RSt_1$的表格,而是一个包含了$k-1$个参数的略微复杂的公式。
\gls{directed_graphical_model}的语法并不能对我们如何定义条件分布作出任何限制。
它只定义了哪些变量可以作为其中的参数。
% 556  16.2 to 



\subsection{\glsentrytext{undirected_model}}
\label{sec:undirected_models}
% 556


\gls{directed_graphical_model}为我们提供了一种描述\gls{structured_probabilistic_models}的语言。
而另一种常见的语言则是\firstgls{undirected_model},也被称为\firstall{MRF}或者是\firstgls{markov_network} \citep{kindermann-book-1980}。
就像它们的名字所说的那样,\gls{undirected_model}中所有的边都是没有方向的。
% 556 end


%\gls{directed_model}显然适用于当存在一个很明显的理由来描述每一个箭头时。
当存在很明显的理由画出每一个指向特定方向的箭头时,\gls{directed_model}显然最适用。
\gls{directed_model}中,经常存在我们理解的具有因果关系以及因果关系有明确方向的情况。
接力赛的例子就是一个这样的情况。
之前运动员的表现会影响后面运动员的完成时间,而后面运动员却不会影响前面运动员的完成时间。
% 557 head 


然而并不是所有情况的相互作用都有一个明确的方向关系。
当相互的作用并没有本质性的指向,或者是明确的双向相互作用时,使用\gls{undirected_model}更加合适。
% 557 


作为一个这种情况的例子,假设我们希望对三个二值随机变量建模:你是否生病,你的同事是否生病以及你的室友是否生病。
就像在接力赛的例子中所作的简化假设一样,我们可以在这里做一些关于相互作用的简化假设。
假设你的室友和同事并不认识,所以他们不太可能直接相互传染一些疾病,比如说感冒。
这个事件太过罕见,所以我们不对此事件建模。
然而,很有可能其中之一将感冒传染给你,然后通过你再传染给了另一个人。
我们通过对你的同事传染给你以及你传染给你的室友建模来对这种间接的从你的同事到你的室友的感冒传染建模。
% 557 


在这种情况下,你传染给你的室友和你的室友传染给你都是非常容易的,所以模型不存在一个明确的单向箭头。
这启发我们使用\gls{undirected_model}。
其中随机变量对应着图中的相互作用的结点。
与\gls{directed_model}相同的是,如果在\gls{undirected_model}中的两个结点通过一条边相连接,那么对应这些结点的随机变量相互之间是直接作用的。
不同于\gls{directed_model},在\gls{undirected_model}中的边是没有方向的,并不与一个条件分布相关联。
% 557 


我们把对应你健康状况的随机变量记作$\RSh_y$,对应你的室友健康状况的随机变量记作$\RSh_r$,你的同事健康的变量记作$\RSh_c$。
\figref{fig:cold_undirected_graph}表示这种关系。
% 557 end

\begin{figure}[!htb]
\ifOpenSource
\centerline{\includegraphics{figure.pdf}}
\else
	\centerline{\includegraphics{Chapter16/figures/cold_undirected_graph}}	
\fi
\caption{表示你室友健康状况的$\RSh_r$、你健康状况的$\RSh_y$和你同事健康状况的$\RSh_c$之间如何相互影响的一个无向图。
你和你的室友可能会相互传染感冒,你和你的同事之间也是如此,
但是假设你室友和同事之间相互不认识,他们只能通过你来间接传染。}
	\label{fig:cold_undirected_graph}
\end{figure}

正式地说,一个\gls{undirected_model}是一个定义在\gls{undirected_model} $\CalG$上的\gls{structured_probabilistic_models}。
对于图中的每一个\gls{clique}\footnote{图的一个\gls{clique}是图中结点的一个子集,并且其中的点是全连接的}$~\CalC$,
一个\firstgls{factor} $\phi(\CalC)$(也称为\firstgls{clique_potential}),衡量了\gls{clique}中变量每一种可能的联合状态所对应的密切程度。
这些\gls{factor}都被限制为是非负的。
它们一起定义了\firstgls{unnormalized_probability_function}:
\begin{align}
\label{eqn:163}
\tilde{p}(\RVx) = \prod_{\CalC\in\CalG} \phi(\CalC).
\end{align}
% 558 head


只要所有\gls{clique}中的结点数都不大,那么我们就能够高效地处理这些\gls{unnormalized_probability_function}。
它包含了这样的思想,密切度越高的状态有越大的概率。
然而,不像\gls{bayesian_network},几乎不存在\gls{clique}定义的结构,所以不能保证把它们乘在一起能够得到一个有效的概率分布。
\figref{fig:example_undirected}展示了一个从\gls{undirected_model}中读取分解信息的例子。
% 558

\begin{figure}[!htb]
\ifOpenSource
\centerline{\includegraphics{figure.pdf}}
\else
	\centerline{\includegraphics{Chapter16/figures/example_undirected}}
\fi
	\caption{这个图说明通过选择适当的$\phi$,函数
		$p(\RSa,\RSb,\RSc,\RSd,\RSe,\RSf)$可以写作
		$\frac{1}{Z}\phi_{\RSa,\RSb}(\RSa,\RSb)\phi_{\RSb,\RSc}(\RSb,\RSc)\phi_{\RSa,\RSd}(\RSa,\RSd)\phi_{\RSb,\RSe}(\RSb,\RSe)\phi_{\RSe,\RSf}(\RSe,\RSf)$。}
	\label{fig:example_undirected}
\end{figure}



在你、你的室友和同事之间感冒传染的例子中包含了两个\gls{clique}。
一个\gls{clique}包含了$\RSh_y$和$\RSh_c$。
这个\gls{clique}的\gls{factor}可以通过一个表来定义,可能取到下面的值:
% \begin{table*}[!hbp]
	% \centering
\begin{table}
	\centering
\begin{tabular}{c|cc}
		& $\RSh_y = 0$ & $\RSh_y = 1$ \\ \hline
		$\RSh_c = 0$ & 2 & 1 \\
		$\RSh_c = 1$  & 1 & 10 \\
\end{tabular}
\end{table}
% \end{table*}
% 558


状态为$1$代表了健康的状态,相对的状态为$0$则表示不好的健康状态(即感染了感冒)。
你们两个通常都是健康的,所以对应的状态拥有最高的密切程度。
两个人中只有一个人是生病的密切程度是最低的,因为这是一个很罕见的状态。
两个人都生病的状态(通过一个人来传染给了另一个人)有一个稍高的密切程度,尽管仍然不及两个人都健康的密切程度。
% 559 head  


为了完整地定义这个模型,我们需要对包含$\RSh_y$和$\RSh_r$的\gls{clique}定义类似的\gls{factor}。
% 559 head


\subsection{\glsentrytext{partition_function}}
\label{sec:the_partition_function}
%  16.2.3   p 559



尽管这个\gls{unnormalized_probability_function}处处不为零,我们仍然无法保证它的概率之和或者积分为$1$。
为了得到一个有效的概率分布,我们需要使用对应的归一化的概率分布
\footnote{一个通过归一化\gls{clique_potential}乘积定义的分布也被称作是\firstgls{gibbs_distribution}}:
\begin{align}
p(\RVx) = \frac{1}{Z}\tilde{p}(\RVx),
\end{align}
其中,$Z$是使得所有的概率之和或者积分为$1$的常数,并且满足:
\begin{align}
Z = \int \tilde{p}(\RVx)d\RVx.
\end{align}
当函数$\phi$固定时,我们可以把$Z$当成是一个常数。
值得注意的是如果函数$\phi$带有参数时,那么$Z$是这些参数的一个函数。
在相关文献中为了节省空间忽略控制$Z$的变量而直接写$Z$是一个常用的方式。
归一化常数$Z$被称作是\gls{partition_function},这是一个从统计物理学中借鉴的术语。
% 559 


由于$Z$通常是由对所有可能的$\RVx$状态的联合分布空间求和或者求积分得到的,它通常是很难计算的。
为了获得一个\gls{undirected_model}的归一化概率分布,模型的结构和函数$\phi$的定义通常需要设计为有助于高效地计算$Z$。
在\gls{DL}中,$Z$通常是难以处理的。
由于$Z$难以精确地计算出,我们只能使用一些近似的方法。
这样的近似方法是\chapref{chap:confronting_the_partition_function}的主要内容。
% 559 


% 559 end
%在设计\gls{undirected_model}时我们必须牢记在心的一个要点是设置一些\gls{factor}使得$Z$不存在这样的方法也是有可能的。
在设计\gls{undirected_model}时,我们必须牢记在心的一个要点是设定一些使得$Z$不存在的\gls{factor}也是有可能的。
当模型中的一些变量是连续的,且$\tilde{p}$在其定义域上的积分发散时这种情况就会发生。
例如, 当我们需要对一个单独的标量变量$\RSx\in\SetR$建模,并且单个\gls{clique_potential}定义为$\phi(x) = x^2$时。
在这种情况下,
\begin{align}
Z = \int x^2 dx.
\end{align}
由于这个积分是发散的,所以不存在一个对应着这个势能函数$\phi(x)$的概率分布。
有时候$\phi$函数某些参数的选择可以决定相应的概率分布是否能够被定义。
例如,对$\phi$函数$\phi(x;\beta) = \text{exp}(-\beta x^2)$来说,参数$\beta$决定了归一化常数$Z$是否存在。
正的$\beta$使得$\phi$函数是一个关于$\RSx$的高斯分布,但是非正的参数$\beta$则使得$\phi$不可能被归一化。


% P560   
有向建模和无向建模之间一个重要的区别就是\gls{directed_model}是通过从起始点的概率分布直接定义的,
反之\gls{undirected_model}的定义显得更加宽松,通过$\phi$函数转化为概率分布而定义。
这改变了我们处理这些建模问题的直觉。
当我们处理\gls{undirected_model}时需要牢记一点,每一个变量的定义域对于一系列给定的$\phi$函数所对应的概率分布有着重要的影响。
举个例子,我们考虑一个$n$维向量的随机变量$\RVx$以及一个由偏置向量$\Vb$参数化的\gls{undirected_model}。
假设$\RVx$的每一个元素对应着一个\gls{clique},并且满足$\phi^{(i)}(\RSx_i) = \exp(b_i\RSx_i)$。
在这种情况下概率分布是怎样的呢?
答案是我们无法确定,因为我们并没有指定$\RVx$的定义域。
如果$\RVx$满足$\RVx \in \SetR^n$,那么有关归一化常数$Z$的积分是发散的,这导致了对应的概率分布是不存在的。
如果$\RVx\in\{0,1\}^n$,那么$p(\RVx)$可以被分解成$n$个独立的分布,并且满足$p(\RSx_i=1) = \text{sigmoid}(b_i)$。
如果$\RVx$的定义域是\gls{ebv} $(\{[1,0,\ldots,0],[0,1,\ldots,0],\ldots,[0,0,\ldots,1]\})$的集合,
那么$p(\RSx) = \text{softmax}(\Vb)$,因此对于$j\neq i$,一个较大的$b_i$的值会降低所有$p(\RSx_j = 1)$的概率。
通常情况下,通过仔细选择变量的定义域,能够从一个相对简单的$\phi$函数的集合可以获得一个相对复杂的表达。
我们会在\secref{sec:convolutional_boltzmann_machines}中讨论这个想法的实际应用。
% P560   


\subsection{\glsentrytext{energy_based_model}}
\label{sec:energy_based_models}
% 560


\gls{undirected_model}中许多有趣的理论结果都依赖于$\forall \Vx,\ \tilde{p}(\Vx)>0$这个假设。
使这个条件满足的一种简单方式是使用\firstall{energy_based_model},其中
\begin{align}
\label{eqn:167}
\tilde{p}(\RVx) = \exp(-E(\RVx)),
\end{align}
$E(\RVx)$被称作是\firstgls{energy_function}。
对所有的$\RSz$, $\exp(\RSz)$都是正的,这保证了没有一个\gls{energy_function}会使得某一个状态$\RVx$的概率为$0$。
我们可以完全自由地选择那些能够简化学习过程的\gls{energy_function}。
如果我们直接学习各个\gls{clique_potential},我们需要利用\gls{constrained_optimization}方法来任意地指定一些特定的最小概率值。
学习\gls{energy_function}的过程中,我们可以采用无约束的优化方法\footnote{对于某些模型,我们可以仍然使用\gls{constrained_optimization}方法来确保$Z$存在。}。
\gls{energy_based_model}中的概率可以无限趋近于$0$但是永远达不到$0$。
% p 561  head


服从\eqnref{eqn:167}形式的任意分布都是\firstgls{boltzmann_distribution}的一个实例。
正是基于这个原因,我们把许多\gls{energy_based_model}称为\firstgls{BM}~\citep{Fahlman83,Ackley85,Hinton84,Hinton86a}。
关于什么时候称之为\gls{energy_based_model},什么时候称之为\gls{BM}不存在一个公认的判别标准。
一开始\gls{BM}这个术语是用来描述一个只有二值变量的模型,但是如今许多模型,比如\gls{mcrbm2},也涉及到了实值变量。
虽然\gls{BM}最初的定义既可以包含\gls{latent_variable}也可以不包含\gls{latent_variable},但是时至今日\gls{BM}这个术语通常用于指拥有\gls{latent_variable}的模型,而没有\gls{latent_variable}的\gls{BM}则经常被称为\gls{MRF}或\gls{log_linear_model}。
% 561



\gls{undirected_model}中的\gls{clique}对应于\gls{unnormalized_probability_function}中的\gls{factor}。 
通过$\exp(a+b) = \exp(a) \exp(b)$,我们发现\gls{undirected_model}中的不同\gls{clique}对应于\gls{energy_function}的不同项。
换句话说,\gls{energy_based_model}只是一种特殊的\gls{markov_network}:求幂使\gls{energy_function}中的每个项对应于不同\gls{clique}的一个\gls{factor}。
关于如何从\gls{undirected_model}结构中获得\gls{energy_function}形式的示例可以参考\figref{fig:example_undirected_2}。
人们可以将\gls{energy_function}中带有多个项的\gls{energy_based_model}视作是\firstgls{product_of_expert}~\citep{Hinton99}。
\gls{energy_function}中的每一项对应的是概率分布中的一个\gls{factor}。
\gls{energy_function}中的每一项都可以看作决定一个特定的软约束是否能够满足的``专家''。
%每个专家只执行仅涉及一个随机变量低维投影的约束,但是当其结合概率的乘法时,专家们合理构造了复杂的高维约束。
每个专家只执行一个约束,而这个约束仅仅涉及随机变量的一个低维投影,但是当其结合概率的乘法时,专家们一同构造了复杂的高维约束。
% 561



\begin{figure}[!htb]
\ifOpenSource
\centerline{\includegraphics{figure.pdf}}
\else
	\centerline{\includegraphics{Chapter16/figures/example_undirected}}
\fi
\caption{这个图说明通过为每个\gls{clique}选择适当的\gls{energy_function} $E(\RSa,\RSb,\RSc,\RSd,\RSe,\RSf)$可以写作$E_{\RSa,\RSb}(\RSa,\RSb) + E_{\RSb,\RSc}(\RSb,\RSc) + E_{\RSa,\RSd}(\RSa,\RSd)+  E_{\RSb,\RSe}(\RSb,\RSe) + E_{\RSe,\RSf}(\RSe,\RSf)$。
值得注意的是,我们令$\phi$等于对应负能量的指数,可以
%我们可以通过令$\phi$等于对应负能量的指数来
获得图\ref{fig:example_undirected}中的$\phi$函数,比如,$\phi_{\RSa,\RSb}(\RSa,\RSb) = \exp(-E(\RSa,\RSb))$。}
	\label{fig:example_undirected_2}
\end{figure}
%  562 head


\gls{energy_based_model}定义的一部分无法用\gls{ML}观点来解释:即\eqnref{eqn:167}中的``-''符号。
这个``-''符号可以被包含在$E$的定义之中。
对于很多$E$函数的选择来说,学习算法可以自由地决定能量的符号。
这个负号的存在主要是为了保持\gls{ML}文献和物理学文献之间的兼容性。
概率建模的许多研究最初都是由统计物理学家做出的,其中$E$是指实际的、物理概念的能量,没有任何符号。
诸如``能量''和``\gls{partition_function}''这类术语仍然与这些技术相关联,尽管它们的数学适用性比在物理中更宽。
一些\gls{ML}研究者(例如,\citet{Smolensky86}将负能量称为~\textbf{harmony})发出了不同的声音,但这些都不是标准惯例。
% 562  head


许多对概率模型进行操作的算法不需要计算$p_{\text{model}}(\Vx)$,而只需要计算$\log \tilde{p}_{\text{model}}(\Vx)$。
对于具有\gls{latent_variable} $\Vh$的\gls{energy_based_model}, 这些算法有时会将该量的负数称为\firstgls{free_energy}:
\begin{align}
\label{eqn:168}
\CalF (\Vx) = -\log \sum_{\Vh} \exp(-E(\Vx,\Vh)).
\end{align}
在本书中,我们更倾向于更为通用的基于$\log \tilde{p}_{\text{model}}(\Vx)$的定义。
% 562


\subsection{\glsentrytext{separation}和\glsentrytext{dseparation}}
\label{sec:separation_and_d_separation}

\gls{graphical_models}中的边告诉我们哪些变量直接相互作用。
我们经常需要知道哪些变量\emph{间接}相互作用。 
某些间接相互作用可以通过观察其他变量来启用或禁用。
更正式地,我们想知道在给定其他变量子集的值时,哪些变量子集彼此\gls{conditional_independent}。
% 562 end 


在\gls{undirected_model}中,识别图中的\gls{conditional_independent}性是非常简单的。 
在这种情况下,图中隐含的\gls{conditional_independent}性称为\firstgls{separation}。
如果图结构显示给定变量集$\SetS$的情况下变量集$\SetA$与变量集$\SetB$无关,
那么我们声称给定变量集$\SetS$时,变量集$\SetA$与另一组变量集$\SetB$是\gls{separation}的。
%如果两个变量$\RSa$和$\RSb$通过涉及未观察变量的路径连接,那么这些变量不是\gls{separation}的。
如果连接两个变量$\RSa$和$\RSb$的连接路径仅涉及未观察变量,那么这些变量不是\gls{separation}的。
如果它们之间没有路径,或者所有路径都包含可观测的变量,那么它们是\gls{separation}的。
我们认为仅涉及未观察到的变量的路径是``活跃''的,而包括可观察变量的路径称为``非活跃''的。
% 563  head


当我们画图时,我们可以通过加阴影来表示观察到的变量。
\figref{fig:undirected_paths_active}用于描述当以这种方式绘图时\gls{undirected_model}中的活跃和非活跃路径的样子。
\figref{fig:example_sep}描述了一个从\gls{undirected_model}中读取\gls{separation}信息的例子。
% 563


\begin{figure}[!htb]
\ifOpenSource
\centerline{\includegraphics{figure.pdf}}
\else
\centering
\begin{tabular}{cc}
\includegraphics{Chapter16/figures/undirected_paths_active.pdf} &%
\includegraphics{Chapter16/figures/undirected_paths_inactive.pdf} \\
(a) & (b)
\end{tabular}
\fi
\caption{\emph{(a)}随机变量$\RSa$和随机变量$\RSb$之间穿过$\RSs$的路径是活跃的,因为$\RSs$是观察不到的。
这意味着$\RSa$,$\RSb$之间不是\gls{separation}的。
\emph{(b)}图中$\RSs$用阴影填充,表示它是可观察的。
因为$\RSa$和$\RSb$之间的唯一路径通过$\RSs$,并且这条路径是不活跃的,
我们可以得出结论,在给定$\RSs$的条件下$\RSa$和$\RSb$是\gls{separation}的。}
\label{fig:undirected_paths_active}
\end{figure}
% 563 mid

\begin{figure}[!htb]
\ifOpenSource
\centerline{\includegraphics{figure.pdf}}
\else
	\centerline{\includegraphics{Chapter16/figures/example_sep}}
\fi
\caption{从一个无向图中读取\gls{separation}性质的一个例子。
这里$\RSb$用阴影填充,表示它是可观察的。
由于$\RSb$挡住了从$\RSa$到$\RSc$的唯一路径,我们说在给定$\RSb$的情况下$\RSa$和$\RSc$是相互\gls{separation}的。
观察值$\RSb$同样挡住了从$\RSa$到$\RSd$的一条路径,但是它们之间有另一条活跃路径。
因此给定$\RSb$的情况下$\RSa$和$\RSd$不是\gls{separation}的。}
	\label{fig:example_sep}
\end{figure}
% 563 end



类似的概念适用于\gls{directed_model},只是在\gls{directed_model}中,这些概念被称为\firstgls{dseparation}。
``d''代表``\gls{dependency}''的意思。
有向图中\,\gls{dseparation}的定义与\gls{undirected_model}中\gls{separation}的定义相同:
如果图结构显示给定变量集$\SetS$时 ,变量集$\SetA$与变量集$\SetB$无关, 那么我们认为给定变量集$\SetS$时,变量集$\SetA$~\gls{dseparation}于变量集$\SetB$。
% 563


与\gls{undirected_model}一样,我们可以通过查看图中存在的活跃路径来检查图中隐含的独立性。
如前所述,如果两个变量之间存在活跃路径,则两个变量是依赖的,如果没有活跃路径,则为\gls{dseparation}。
在有向网络中,确定路径是否活跃有点复杂。
关于在\gls{directed_model}中识别活跃路径的方法可以参考\figref{fig:168}。 
\figref{fig:169}是从一个图中读取一些属性的例子。
% 564  head 





尤其重要的是要记住\gls{separation}和\gls{dseparation}只能告诉我们\emph{图中隐含}的\gls{conditional_independent}性。
图并不需要表示所有存在的独立性。 
进一步的,使用完全图(具有所有可能的边的图)来表示任何分布总是合法的。
事实上,一些分布包含不可能用现有图形符号表示的独立性。
\firstgls{context_specific_independence}指的是取决于网络中一些变量值的独立性。
例如,考虑三个二值变量的模型:$\RSa$,$\RSb $和$\RSc$。
假设当$\RSa$是0时,$\RSb$和$\RSc$是独立的, 但是当$\RSa$是1时,$\RSb$确定地等于$\RSc$。
当$\RSa = 1$时\gls{graphical_models}需要连接$\RSb$和$\RSc$的边。
但是图不能说明当$\RSa = 0$时$\RSb$和$\RSc$不是独立的。
% 564

一般来说,当独立性不存在时,图不会显示独立性。 
然而,图可能无法编码独立性。
% 564



% 565  1 page
\begin{figure}[!htb]
	\ifOpenSource
	\centerline{\includegraphics{figure.pdf}}
	\else
	\centerline{\includegraphics{Chapter16/figures/directed_paths.pdf}}
	\fi
\caption{两个随机变量$\RSa$,$\RSb$之间存在的长度为$2$的所有种类的活跃路径。
%存在的所有种类的长度为$2$的活跃路径。
\emph{(a)}箭头方向从$\RSa$指向$\RSb$的任何路径,反过来也一样。
如果$\RSs$可以被观察到,这种路径就是阻塞的。
在接力赛的例子中,我们已经看到过这种类型的路径。
\emph{(b)}变量$\RSa$和$\RSb$通过\emph{\gls{common_cause}} $\RSs$相连。
举个例子,假设$\RSs$是一个表示是否存在飓风的变量,$\RSa$和$\RSb$表示两个相邻气象监控区域的风速。
如果我们在$\RSa$处观察到很高的风速,我们可以期望在$b$处也观察到高速的风。
如果观察到$\RSs$,那么这条路径就被阻塞了。
如果我们已经知道存在飓风,那么无论$\RSa$处观察到什么,我们都能期望$\RSb$处有较高的风速。
在$\RSa$处观察到一个低于预期的风速(对飓风而言)并不会改变我们对$\RSb$处风速的期望(已知有飓风的情况下)。
然而,如果$\RSs$不被观测到,那么$\RSa$和$\RSb$是依赖的,即路径是活跃的。
\emph{(c)}变量$\RSa$和$\RSb$都是$\RSs$的父节点。
这称为\firstgls{vstructure}或者\firstgls{collider}。
根据\firstgls{explaining_away_effect},\gls{vstructure}导致$\RSa$和$\RSb$是相关的。
在这种情况下,当$\RSs$被观测到时路径是活跃的。
举个例子,假设$\RSs$是一个表示你的同事不在工作的变量。
变量$\RSa$表示她生病了,而变量$\RSb$表示她在休假。
如果你观察到了她不在工作,你可以假设她很有可能是生病了或者是在度假,但是这两件事同时发生是不太可能的。
如果你发现她在休假,那么这个事实足够\emph{解释}她的缺席了。
你可以推断她很可能没有生病。
\emph{(d)}即使$\RSs$的任意后代都被观察到,\gls{explaining_away_effect}也会起作用。
举个例子,假设$\RSc$是一个表示你是否收到你同事的报告的一个变量。
如果你注意到你还没有收到这个报告,这会增加你估计的她今天不在工作的概率,这反过来又会增加她今天生病或者度假的概率。
阻塞\,\gls{vstructure}中路径的唯一方法就是共享子节点的后代一个都观察不到。}
\label{fig:168}
\end{figure}
% 565  1 page


% 566 head
\begin{figure}[!htb]
	\centering
	\ifOpenSource
	\includegraphics{figure.pdf}
	\else
	\includegraphics{Chapter16/figures/example_dsep}
	\fi
	\captionsetup{singlelinecheck=off}
\caption[d-separation example]{从这张图中,我们可以发现一些\,\gls{dseparation}的性质。这包括了:
\begin{itemize}
\item 给定空集的情况下,$\RSa$和$\RSb$是\,\gls{dseparation}的。
\item 给定$\RSc$的情况下,$\RSa$和$\RSe$是\,\gls{dseparation}的。
\item 给定$\RSc$的情况下,$\RSd$和$\RSe$是\,\gls{dseparation}的。
\end{itemize}
我们还可以发现当我们观察到一些变量时,一些变量不再是\,\gls{dseparation}的:
\begin{itemize}
\item 给定$\RSc$的情况下,$\RSa$和$\RSb$不是\,\gls{dseparation}的。
\item 给定$\RSd$的情况下,$\RSa$和$\RSb$不是\,\gls{dseparation}的。
\end{itemize}
}
	\label{fig:169}
\end{figure}
% 566 head

\subsection{在\glsentrytext{directed_model}和\glsentrytext{undirected_model}中转换}
\label{sec:converting_between_undirected_and_directed_graphs}

我们经常将特定的\gls{ML}模型称为\gls{undirected_model}或\gls{directed_model}。
例如,我们通常将\gls{RBM}称为\gls{undirected_model}, 而\gls{sparse_coding}则被称为\gls{directed_model}。
这种措辞的选择可能有点误导,因为没有概率模型本质上是有向或无向的。
但是,一些模型很适合使用有向图\emph{描述},而另一些模型很适合使用\gls{undirected_model}描述。
% 564 end

\gls{directed_model}和\gls{undirected_model}都有其优点和缺点。
这两种方法都不是明显优越和普遍优选的。
相反,我们根据具体的每个任务来决定使用哪一种模型。 
这个选择部分取决于我们希望描述的概率分布。
根据哪种方法可以最大程度地捕捉到概率分布中的独立性,或者哪种方法使用最少的边来描述分布,我们可以决定使用有向建模还是无向建模。
还有其他因素可以影响我们决定使用哪种建模方式。 
即使在使用单个概率分布时,我们有时也可以在不同的建模方式之间切换。
有时,如果我们观察到变量的某个子集,或者如果我们希望执行不同的计算任务,换一种建模方式可能更合适。
例如,\gls{directed_model}通常提供了一种高效地从模型中抽取样本(在\secref{sec:sampling_from_graphical_models}中描述)的直接方法。
而\gls{undirected_model}形式通常对于推导\gls{approximate_inference}过程(我们将在\chapref{chap:approximate_inference}中看到,\eqnref{eqn:1956}强调了\gls{undirected_model}的作用)是很有用的。
% 566


每个概率分布可以由\gls{directed_model}或由\gls{undirected_model}表示。
在最坏的情况下,我们可以使用``\gls{complete_graph}''来表示任何分布。
在\gls{directed_model}的情况下,\gls{complete_graph}是任意有向无环图,其中我们对随机变量排序,并且每个变量在排序中位于其之前的所有其他变量作为其图中的祖先。
对于\gls{undirected_model},\gls{complete_graph}只是包含所有变量的单个\gls{clique}。 
\figref{fig:complete}给出了一个实例。
% 566

\begin{figure}[!htb]
\ifOpenSource
\centerline{\includegraphics{figure.pdf}}
\else
	\centerline{\includegraphics{Chapter16/figures/complete}}	
\fi
\caption{\gls{complete_graph}的例子,\gls{complete_graph}能够描述任何的概率分布。
这里我们展示了一个带有四个随机变量的例子。
\emph{(左)}完全无向图。
在无向图中,\gls{complete_graph}是唯一的。
\emph{(右)}一个完全有向图。
在有向图中,并不存在唯一的\gls{complete_graph}。
我们选择一种变量的排序,然后对每一个变量,从它本身开始,向每一个指向顺序在其后面的变量画一条弧。
因此存在着关于变量数阶乘数量级的不同种\gls{complete_graph}。
在这个例子中,我们从左到右从上到下地排序变量。}
	\label{fig:complete}
\end{figure}


当然,\gls{graphical_models}的优势在于图能够包含一些变量不直接相互作用的信息。 
\gls{complete_graph}并不是很有用,因为它并不隐含任何独立性。
% 566


当我们用图表示概率分布时,我们想要选择一个包含尽可能多独立性的图,但是并不会假设任何实际上不存在的独立性。
% 566 end


从这个角度来看,一些分布可以使用\gls{directed_model}更高效地表示,而其他分布可以使用\gls{undirected_model}更高效地表示。
换句话说,\gls{directed_model}可以编码一些\gls{undirected_model}所不能编码的独立性,反之亦然。
% 567 head


\gls{directed_model}能够使用一种\gls{undirected_model}无法完美表示的特定类型的子结构。
这个子结构被称为\firstgls{immorality}。
这种结构出现在当两个随机变量$\RSa$和$\RSb$都是第三个随机变量$\RSc$的父结点,并且不存在任一方向上直接连接$\RSa$和$\RSb$的边时。
(``\gls{immorality}''的名字可能看起来很奇怪; 它在\gls{graphical_models}文献中使用源于一个关于未婚父母的笑话。)
为了将\gls{directed_model}图$\CalD$转换为\gls{undirected_model},我们需要创建一个新图$\CalU$。
对于每对变量$\RSx$和$\RSy$,如果存在连接$\CalD$中的$\RSx$和$\RSy$的有向边(在任一方向上),或者如果$\RSx$和$\RSy$都是图$\CalD$中另一个变量$\RSz$的父节点,则在$\CalU$中添加连接$\RSx$和$\RSy$的无向边。 
得到的图$\CalU$被称为是\firstgls{moralized_graph}。
关于一个通过\gls{moralization}将\gls{directed_graphical_model}转化为\gls{undirected_model}的例子可以参考\figref{fig:dm_to_um}。
% 567


% 568 1 page
\begin{figure}[!htb]
\ifOpenSource
\centerline{\includegraphics{figure.pdf}}
\else
	\centerline{\includegraphics{Chapter16/figures/dm_to_um}}	
\fi
\caption{通过构造\gls{moralized_graph}将有向模型(上一行)转化为无向模型(下一行)的例子。
\emph{(左)}只需要把有向边替换成无向边就可以把这个简单的链转化为一个\gls{moralized_graph}。
得到的无向模型包含了完全相同的独立关系和\gls{conditional_independent}关系。
\emph{(中)}这个图是在不丢失独立性的情况下是无法
%不丢失独立性的情况下无法
转化为无向模型的最简单的有向模型。
这个图包含了单个完整的\gls{immorality}结构。
因为$\RSa$和$\RSb$都是$\RSc$的父节点,当$\RSc$被观察到时,它们之间通过活跃路径相连。
为了捕捉这个\gls{dependency},无向模型必须包含一个含有所有三个变量的\gls{clique}。
这个\gls{clique}无法编码$\RSa \perp \RSb$这个信息。
\emph{(右)}一般来说,\gls{moralization}的过程会给图添加许多边,因此丢失了一些隐含的独立性。
举个例子,这个\gls{sparse_coding}图需要在每一对\gls{hidden_unit}之间添加\gls{moralization}的边,因此也引入了二次数量级的新的直接\gls{dependency}。}
	\label{fig:dm_to_um}
\end{figure}
% 568 1 page



同样的,\gls{undirected_model}可以包括\gls{directed_model}不能完美表示的子结构。
具体来说,如果$\CalU$包含长度大于$3$的\firstgls{loop},则有向图$\CalD$不能捕获\gls{undirected_model} $\CalU$所包含的所有\gls{conditional_independent}性,除非该\gls{loop}还包含\firstgls{chord}。
\gls{loop}指的是由无向边连接的变量序列,并且满足序列中的最后一个变量连接回序列中的第一个变量。
\gls{chord}是定义\gls{loop}序列中任意两个非连续变量之间的连接。
如果$\CalU$具有长度为$4$或更大的环,并且这些环没有\gls{chord},我们必须在将它们转换为\gls{directed_model}之前添加\gls{chord}。
添加这些\gls{chord}会丢弃在$\CalU$中编码的一些独立信息。
通过将\gls{chord}添加到$\CalU$形成的图被称为\firstgls{chordal_graph}或者\firstgls{triangulated_graph},
因为我们现在可以用更小的、三角的环来描述所有的环。
要从\gls{chordal_graph}构建有向图$\CalD$,我们还需要为边指定方向。
当这样做时,我们不能在$\CalD$中创建有向循环,否则将无法定义有效的有向概率模型。
为$\CalD$中的边分配方向的一种方法是对随机变量排序,然后将每个边从排序较早的节点指向排序稍后的节点。
一个简单的实例可以参考\figref{fig:um_to_dm}。
% 569 mid



% 569 head
\begin{figure}[!htb]
\ifOpenSource
\centerline{\includegraphics{figure.pdf}}
\else
	\centerline{\includegraphics{Chapter16/figures/um_to_dm}}	
\fi
\caption{将一个无向模型转化为一个有向模型。
\emph{(左)}这个无向模型无法转化为有向模型,因为它有一个长度为$4$且不带有\gls{chord}的\gls{loop}。
具体说来,这个无向模型包含了两种不同的独立性,并且不存在一个有向模型可以同时描述这两种性质:$\RSa\perp \RSc \mid \{\RSb,\RSd\}$和$\RSb \perp \RSd \mid \{\RSa,\RSc\}$。
\emph{(中)}为了将无向图转化为有向图,我们必须通过保证所有长度大于$3$的\gls{loop}都有\gls{chord}来\gls{triangulate}图。
为了实现这个目标,我们可以加一条连接$\RSa$和$\RSc$或者连接$\RSb$和$\RSd$的边。
在这个例子中,我们选择添加一条连接$\RSa$和$\RSc$的边。
\emph{(右)}为了完成转化的过程,我们必须给每条边分配一个方向。
执行这个任务时,我们必须保证不产生任何有向环。
避免出现有向环的一种方法是赋予节点一定的顺序,然后将每个边从排序较早的节点指向排序稍后的节点。
在这个例子中,我们根据变量名的字母进行排序。}
	\label{fig:um_to_dm}
\end{figure}
% 569 head



\subsection{\glsentrytext{factor_graph}}
\label{sec:factor_graphs}
% 569      16.2.7



\firstgls{factor_graph}是从\gls{undirected_model}中抽样的另一种方法,它可以解决标准\gls{undirected_model}语法中图表达的模糊性。
在\gls{undirected_model}中,每个$\phi$函数的范围必须是图中某个\gls{clique}的\emph{子集}。
我们无法确定每一个\gls{clique}是否含有一个作用域包含整个\gls{clique}的\gls{factor}——比如说一个包含三个结点的\gls{clique}可能对应的是一个有三个结点的\gls{factor},也可能对应的是三个\gls{factor}并且每个\gls{factor}包含了一对结点,这通常会导致模糊性。
通过显式地表示每一个$\phi$函数的作用域,\gls{factor_graph}解决了这种模糊性。
%然而,$\phi$没有必要包含每个\gls{clique}的全部。
%\gls{factor_graph}明确表示每个$\phi$函数的范围。
具体来说,\gls{factor_graph}是一个包含无向二分图的\gls{undirected_model}的图形化表示。
一些节点被绘制为圆形。 
就像在标准\gls{undirected_model}中一样,这些节点对应于随机变量。
其余节点绘制为方块。
这些节点对应于\gls{unnormalized_probability_function}的\gls{factor} $\phi$。
变量和\gls{factor}可以通过无向边连接。
当且仅当变量包含在\gls{unnormalized_probability_function}的\gls{factor}中时,变量和\gls{factor}在图中存在连接。
没有\gls{factor}可以连接到图中的另一个\gls{factor},也不能将变量连接到变量。
\figref{sec:factor_graphs}给出了一个例子来说明\gls{factor_graph}如何解决无向网络中的模糊性。
% 570  


% 570 head
\begin{figure}[!htb]
\ifOpenSource
\centerline{\includegraphics{figure.pdf}}
\else
	\centerline{\includegraphics{Chapter16/figures/factor_graph}}	
\fi
\caption{\gls{factor_graph}如何解决无向网络中的模糊性的一个例子。
\emph{(左)}一个包含三个变量($\RSa$、$\RSb$和$\RSc$)的\gls{clique}组成的无向网络。
\emph{(中)}对应这个无向模型的\gls{factor_graph}。
这个\gls{factor_graph}有一个包含三个变量的因子。
\emph{(右)}对应这个无向模型的另一种有效的\gls{factor_graph}。
这个\gls{factor_graph}包含了三个因子,每个因子只对应两个变量。
即使它们表示的是同一个无向模型,这个\gls{factor_graph}上进行的表示、推断和学习相比于中图描述的\gls{factor_graph}都要渐近地廉价。}
	\label{fig:factor_graph}
\end{figure}
% 570 head


\section{从\glsentrytext{graphical_models}中采样}
\label{sec:sampling_from_graphical_models}

% 570  
\gls{graphical_models}同样简化了从模型中采样的过程。


\gls{directed_graphical_model}的一个优点是,可以通过一个简单高效的过程从模型所表示的联合分布中产生样本,这个过程被称为\firstgls{ancestral_sampling}。
%可以通过一个简单高效的被称作是\firstgls{ancestral_sampling}的过程从由模型表示的联合分布中抽取样本。
% 570  


\gls{ancestral_sampling}的基本思想是将图中的变量$\RSx_i$使用拓扑排序,使得对于所有$i$和$j$,
如果$\RSx_i$是$\RSx_j$的一个父亲结点,则$j$大于$i$。
然后可以按此顺序对变量进行采样。
换句话说,我们可以首先采$\RSx_1\sim P(\RSx_1)$,然后采$\RSx_2\sim P(\RSx_2\mid Pa_{\CalG}(\RSx_2))$,以此类推,直到最后我们从$ P(\RSx_n\mid Pa_{\CalG}(\RSx_n))$中采样。
只要不难从每个条件分布$\RSx_i\sim P(\RSx_i\mid Pa_{\CalG}(\RSx_i))$中采样,那么从整个模型中采样也是容易的。
%那么很容易从整个模型中抽样。
拓扑排序操作保证我们可以按照\eqnref{eqn:161}中条件分布的顺序依次采样。
如果没有拓扑排序,我们可能会在其父节点可用之前试图对该变量进行抽样。
% 571  


有些图可能存在多个拓扑排序。 
\gls{ancestral_sampling}可以使用这些拓扑排序中的任何一个。
% 571

\gls{ancestral_sampling}通常非常快(假设从每个条件分布中采样都是很容易的)并且非常简便。
% 571  


\gls{ancestral_sampling}的一个缺点是其仅适用于\gls{directed_graphical_model}。 
另一个缺点是它并不是每次采样都是条件采样操作。%??
当我们希望从\gls{directed_graphical_model}中变量的子集中采样时,给定一些其他变量,我们经常要求所有给定的条件变量在顺序图中比要采样的变量的顺序要早。
在这种情况下,我们可以从模型分布指定的局部条件概率分布中采样。 
否则,我们需要采样的条件分布是给定观测变量的后验分布。
这些后验分布在模型中通常没有明确指定和参数化。 
推断这些后验分布的代价可能是很高的。 
在这种情况下的模型中,\gls{ancestral_sampling}不再有效。
% 571   



不幸的是,\gls{ancestral_sampling}仅适用于\gls{directed_model}。 
我们可以通过将\gls{undirected_model}转换为\gls{directed_model}来实现从\gls{undirected_model}中抽样,但是这通常需要解决棘手的推断问题(要确定新有向图的根节点上的边缘分布),或者需要引入许多边从而会使得到的\gls{directed_model}变得难以处理。
从\gls{undirected_model}采样,而不首先将其转换为\gls{directed_model}的做法似乎需要解决循环依赖的问题。 
每个变量与每个其他变量相互作用,因此对于采样过程没有明确的起点。
不幸的是,从\gls{undirected_model}中抽取样本是一个成本很高的多次迭代的过程。
理论上最简单的方法是\firstgls{gibbs_sampling}。
假设我们在一个$n$维向量的随机变量$\RVx$上有一个\gls{graphical_models}。 
我们迭代地访问每个变量$x_i$,在给定其他变量的条件下从$p(\RSx_i \mid \RSx_{-i})$中抽样。
由于\gls{graphical_models}的\gls{separation}性质,抽取$x_i$时我们可以等价地仅对$\RSx_i$的邻居条件化。
不幸的是,在我们遍历\gls{graphical_models}一次并采样所有$n$个变量之后,我们仍然无法得到一个来自$p(\RVx)$的客观样本。
相反,我们必须重复该过程并使用它们邻居的更新值对所有$n$个变量重新取样。
在多次重复之后,该过程渐近地收敛到正确的目标分布。
我们很难确定样本何时达到所期望分布的足够精确的近似。
\gls{undirected_model}的采样技术是一个高级的研究方向,\chapref{chap:monte_carlo_methods}将对此进行更详细的讨论。
% 572 head




\section{结构化建模的优势}
\label{sec:advantages_of_structured_modelling}
% 572


使用\gls{structured_probabilistic_models}的主要优点是它们能够显著降低表示概率分布、学习和推断的成本。
\gls{directed_model}中采样还可以被加速,但是对于\gls{undirected_model}情况则较为复杂。
选择不对某些变量的相互作用进行建模是允许所有这些操作使用较少的运行时间和内存的主要机制。
%允许所有这些操作使用较少的运行时间和内存的主要机制是选择不对某些变量的相互作用进行建模。
\gls{graphical_models}通过省略某些边来传达信息。
在没有边的情况下,模型假设不对变量间直接的相互作用建模。
% 572


%使用\gls{structured_probabilistic_models}的一个更加不容易量化的益处是它们允许我们明确地将给定的现有的知识与知识的学习或者推断分开。
\gls{structured_probabilistic_models}允许我们明确地将给定的现有知识与知识的学习或者推断分开,这是一个不容易量化的益处。
这使我们的模型更容易开发和调试。 
我们可以设计、分析和评估适用于更广范围的图的学习算法和推断算法。
同时,我们可以设计能够捕捉到我们认为数据中存在的重要关系的
%认为重要的关系的
模型。
然后,我们可以组合这些不同的算法和结构,并获得不同可能性的笛卡尔乘积。
然而,为每种可能的情况设计\gls{end_to_end}算法会更加困难。
% 572



\section{学习\glsentrytext{dependency}关系}
\label{sec:learning_about_dependencies}
% 572

良好的生成模型需要准确地捕获所观察到的或``可见''变量$\RVv$上的分布。
通常$\RVv$的不同元素彼此高度依赖。
在\gls{DL}中,最常用于建模这些\gls{dependency}关系的方法是引入几个\gls{latent}或``隐藏''变量$\RVh$。
然后,该模型可以捕获任何对(变量$\RSv_i$和$\RSv_j$间接\gls{dependency}可以通过$\RSv_i$和$\RVh$之间直接\gls{dependency}和$\RVh$和$\RSv_j$直接\gls{dependency}捕获)之间的\gls{dependency}关系。%??
% 572  end 


如果一个良好的关于$\RVv$的模型不包含任何\gls{latent_variable},那么它在\gls{bayesian_network}中的每个节点需要具有大量父节点或在\gls{markov_network}中具有非常大的\gls{clique}。
仅仅表示这些高阶相互作用的成本就很高了,首先从计算角度上考虑,存储在存储器中的参数数量是\gls{clique}中成员数量的指数级别,接着在统计学意义上,因为这些指数数量的参数需要大量的数据来准确估计。
% 573  head  


当模型旨在描述直接连接的可见变量之间的\gls{dependency}关系时,通常不可能连接所有变量,因此设计\gls{graphical_models}时需要连接那些紧密相关的变量,并忽略其他变量之间的作用。
\gls{ML}中有一个称为\firstgls{structure_learning}的领域专门讨论这个问题。
\citet{koller-book2009}是一个不错的\gls{structure_learning}参考资料。
大多数\gls{structure_learning}技术基于一种贪婪搜索的形式。
它们提出了一种结构,对具有该结构的模型进行训练,然后给出分数。 
该分数奖励训练集上的高精度并对模型的复杂度进行惩罚。
然后提出添加或移除少量边的候选结构作为搜索的下一步。
搜索向一个预计会增加分数的新结构发展。 %??
% 573


使用\gls{latent_variable}而不是自适应结构避免了离散搜索和多轮训练的需要。 
可见变量和\gls{latent_variable}之间的固定结构可以使用可见单元和\gls{hidden_unit}之间的直接作用,从而建模可见单元之间的间接作用。
%从而使得可见单元之间间接作用。
使用简单的参数学习技术,我们可以学习到一个具有固定结构的模型,这个模型在边缘分布$p(\Vv)$上拥有正确的结构。
% 573 


\gls{latent_variable}除了发挥本来的作用,即能够高效地描述$p(\RVv)$以外,还具有另外的优势。%??
新变量$\RVh$还提供了$\RVv$的替代表示。
例如,如\secref{sec:mixtures_of_distributions}所示,\gls{GMM}学习了一个\gls{latent_variable},这个\gls{latent_variable}对应于输入样本是从哪一个混合体中抽出。
这意味着\gls{GMM}中的\gls{latent_variable}可以用于做分类。
我们可以看到\chapref{chap:autoencoders}中简单的概率模型如\gls{sparse_coding},是如何学习可以用作分类器输入特征或者作为\gls{manifold}上坐标的\gls{latent_variable}的。
其他模型也可以使用相同的方式,但是更深的模型和具有多种相互作用方式的模型可以获得更丰富的输入描述。
许多方法通过学习\gls{latent_variable}来完成特征学习。
通常,给定$\RVv$和$\RVh$,实验观察显示$\SetE[\RVh\mid\RVv]$或${\arg\max}_{\Vh}\ p(\Vh,\Vv)$都是$\Vv$的良好特征映射。
% 573

\section{推断和\glsentrytext{approximate_inference}}
\label{sec:inference_and_approximate_inference}
% 573 end


%我们可以使用概率模型的主要方法之一是提出关于变量如何相互关联的问题。 
解决变量之间如何相互关联的问题是我们使用概率模型的一个主要方式。 
给定一组医学测试,我们可以询问患者可能患有什么疾病。
在一个\gls{latent_variable}模型中,我们可能需要提取能够描述可观察变量$\RVv$的特征$\SetE[\RVh \mid \RVv]$。
有时我们需要解决这些问题来执行其他任务。 
我们经常使用最大似然的准则来训练我们的模型。
由于
\begin{align}
\label{eqn:169}
\log p(\Vv) = \SetE_{\RVh \sim p(\RVh\mid \Vv)} [\log p(\Vh,\Vv) -  \log p(\Vh\mid\Vv)],
\end{align}
学习过程中,%为了执行学习规则,
我们经常需要计算$p(\RVh\mid\Vv)$。
所有这些都是\firstgls{inference}问题的例子,其中我们必须预测给定其他变量的情况下一些变量的值,或者在给定其他变量值的情况下预测一些变量的概率分布。
% 574

%不幸的是,对于大多数有趣的深度模型来说,这些推断问题都是难以处理的,即使我们使用结构化\gls{graphical_models}来简化它们。
不幸的是,对于大多数有趣的深度模型来说,即使我们使用结构化\gls{graphical_models}来简化这些推断问题,它们仍然是难以处理的。
图结构允许我们用合理数量的参数来表示复杂的高维分布,但是用于\gls{DL}的图并不满足这样的条件,从而难以实现高效地推断。
% 574


我们可以直接看出,计算一般\gls{graphical_models}的边缘概率是\#P-hard的。
复杂性类别\#P是复杂性类别NP的泛化。
NP中的问题只需确定其中一个问题是否有解决方案,并找到一个解决方案(如果存在)就可以解决。
\#P中的问题需要计算解决方案的数量。
为了构建最坏情况的\gls{graphical_models},我们可以设想一下我们在3-SAT问题中定义二值变量的\gls{graphical_models}。
我们可以对这些变量施加均匀分布。
然后我们可以为每个子句添加一个二值\gls{latent_variable},来表示每个子句是否成立。
然后,我们可以添加另一个\gls{latent_variable},来表示所有子句是否成立。
这可以通过构造一个\gls{latent_variable}的缩减树来完成,树中的每个结点表示其他两个变量是否成立,从而不需要构造一个大的\gls{clique}。
该树的叶是每个子句的变量。
树的根表示整个问题是否成立。
由于子句的均匀分布,缩减树根结点的边缘分布表示子句有多少比例是成立的。
虽然这是一个设计的最坏情况的例子,NP-hard图确实会频繁地出现在现实世界的场景中。
% 574


这促使我们使用\gls{approximate_inference}。
在\gls{DL}中,这通常涉及\gls{variational_inference},其中通过寻求尽可能接近真实分布的近似分布$q(\RVh\mid\RVv)$来逼近真实分布$p(\RVh\mid\Vv)$。
这个技术将在\chapref{chap:approximate_inference}中深入讨论。
% 574 end



\section{结构化概率模型的\glsentrytext{DL}方法}
\label{sec:the_deep_learning_approach_to_structured_probabilistic_models}
% 575 head    16.7  

%\gls{DL}实践者通常使用与从事\gls{structured_probabilistic_models}研究的其它\gls{ML}研究者相同的基本计算工具。
\gls{DL}从业者通常与其他从事\gls{structured_probabilistic_models}研究的\gls{ML}研究者使用相同的基本计算工具。
然而,在\gls{DL}中,我们通常对如何组合这些工具作出不同的设计决定,导致总体算法、模型与更传统的\gls{graphical_models}具有非常不同的风格。
% 575 head


\gls{DL}并不总是涉及特别深的\gls{graphical_models}。
在\gls{graphical_models}中,我们可以根据\gls{graphical_models}的图而不是\gls{computational_graph}来定义模型的深度。
如果从\gls{latent_variable} $h_i$到可观察变量的最短路径是$j$步,我们可以认为\gls{latent_variable} $h_j$处于深度$j$。
我们通常将模型的深度描述为任何这样的$h_j$的最大深度。 
这种深度不同于由\gls{computational_graph}定义的深度。
用于\gls{DL}的许多\gls{generative_model}没有\gls{latent_variable}或只有一层\gls{latent_variable},但使用深度\gls{computational_graph}来定义模型中的条件分布。
% 575


\gls{DL}基本上总是利用\gls{distributed_representation}的思想。
即使是用于\gls{DL}目的的浅层模型(例如\gls{pretraining}浅层模型,稍后将形成深层模型),也几乎总是具有单个大的\gls{latent_variable}层。
\gls{DL}模型通常具有比可观察变量更多的\gls{latent_variable}。
变量之间复杂的非线性相互作用通过多个\gls{latent_variable}的间接连接来实现。
% 575  ok


相比之下,传统的\gls{graphical_models}通常包含至少是偶尔观察到的变量,即使一些训练样本中的许多变量随机地丢失。
传统模型大多使用高阶项和\gls{structure_learning}来捕获变量之间复杂的非线性相互作用。
如果有\gls{latent_variable},它们的数量通常很少。
% 575



\gls{latent_variable}的设计方式在\gls{DL}中也有所不同。
\gls{DL}从业者通常不希望\gls{latent_variable}提前包含了任何特定的含义——训练算法可以自由地开发对特定数据集建模所需要的概念。
在事后解释\gls{latent_variable}通常是很困难的,但是可视化技术可以得到它们表示的一些粗略表征。
当\gls{latent_variable}在传统\gls{graphical_models}中使用时,它们通常被赋予一些特定含义——比如文档的主题、学生的智力、导致患者症状的疾病等。
这些模型通常由研究者解释,并且通常具有更多的理论保证,但是不能扩展到复杂的问题,并且不能像深度模型一样在许多不同背景中重复使用。
% 576  head


另一个明显的区别是\gls{DL}方法中经常使用的连接类型。
深度图模型通常具有大的与其他单元组全连接的单元组,使得两个组之间的相互作用可以由单个矩阵描述。
传统的\gls{graphical_models}具有非常少的连接,并且每个变量的连接选择可以单独设计。
模型结构的设计与推断算法的选择紧密相关。
\gls{graphical_models}的传统方法通常旨在保持精确推断的可解性。
当这个约束太强时,我们可以采用一种流行的被称为\firstgls{loopy_belief_propagation}的\gls{approximate_inference}算法。
这两种方法通常在稀疏连接图上都有很好的效果。
相比之下,在\gls{DL}中使用的模型倾向于将每个可见单元$\RSv_i$连接到非常多的\gls{hidden_unit} $\RSh_j$上,从而使得$\RVh$可以获得一个$\RSv_i$的\gls{distributed_representation}(也可能是其他几个可观察变量)。
\gls{distributed_representation}具有许多优点,但是从\gls{graphical_models}和计算复杂性的观点来看,
\gls{distributed_representation}有一个缺点就是很难产生对于精确推断和\gls{loopy_belief_propagation}等传统技术来说足够稀疏的图。
结果,大规模\gls{graphical_models}和深度图模型最大的区别之一就是\gls{DL}中几乎从来不会使用\gls{loopy_belief_propagation}。
相反的,许多\gls{DL}模型可以设计来加速~\gls{gibbs_sampling}或者\gls{variational_inference}。
此外,\gls{DL}模型包含了大量的\gls{latent_variable},使得高效的数值计算代码显得格外重要。
除了选择高级推断算法之外,这提供了另外的动机,用于将结点分组成层,相邻两层之间用一个矩阵来描述相互作用。
这要求实现算法的单个步骤可以实现高效的矩阵乘积运算,或者专门适用于稀疏连接的操作,例如块对角矩阵乘积或\gls{convolution}。
% 576



最后,\gls{graphical_models}的\gls{DL}方法的一个主要特征在于对未知量的较高容忍度。
与简化模型直到它的每一个量都可以被精确计算不同的是,我们仅仅直接使用数据运行或者是训练,以增强模型的能力。%让模型保持了较高的自由度
我们一般使用边缘分布不能计算的模型,但可以从中简单地采近似样本。%但是可以简单从中采样的模型。
我们经常训练具有难以处理的目标函数的模型,我们甚至不能在合理的时间内近似,但是如果我们能够高效地获得这样一个函数的梯度估计,我们仍然能够近似训练模型。
深度学习方法通常是找出我们绝对需要的最小量信息,然后找出如何尽快得到该信息的合理近似。
% 577 head



\subsection{实例:\glsentrytext{RBM}}
\label{sec:example_the_restricted_boltzmann_machine}
\firstall{RBM}\citep{Smolensky86}或者\firstgls{harmonium}是\gls{graphical_models}如何用于深度学习的典型例子。 
\glssymbol{RBM}~本身不是一个深层模型。 
相反,它有一层\gls{latent_variable},可用于学习输入的表示。 
在\chapref{chap:deep_generative_models}中,我们将看到~\glssymbol{RBM}~如何被用来构建许多的深层模型。
在这里,我们举例展示了~\glssymbol{RBM}~在许多深度图模型中使用的实践:
它的单元被分成很大的组,这种组称作层,层之间的连接由矩阵描述,连通性相对密集。
该模型被设计为能够进行高效的~\gls{gibbs_sampling},并且模型设计的重点在于以很高的自由度来学习\gls{latent_variable},而\gls{latent_variable}的含义并不是设计者指定的。
之后在\secref{sec:restricted_boltzmann_machines},我们将更详细地再次讨论~\glssymbol{RBM}。
% 577


标准的~\glssymbol{RBM}~是具有二值的可见和隐藏单元的\gls{energy_based_model}。 其\gls{energy_function}为
\begin{align}
\label{eqn:1610}
E(\Vv,\Vh) = -\Vb^{\top}\Vv - \Vc^{\top}\Vh - \Vv^{\top}\MW\Vh,
\end{align}
其中$\Vb,\Vc$和$\MW$都是无约束、实值的可学习参数。
我们可以看到,模型被分成两组单元:$\Vv$和$\Vh$,
它们之间的相互作用由矩阵$\MW$来描述。
该模型在\figref{fig:rbm}中以图的形式描绘。
%可以看到。
该图能够使我们更清楚地发现,该模型的一个重要方面是在任何两个可见单元之间或任何两个\gls{hidden_unit}之间没有直接的相互作用(因此称为``受限'',一般的\gls{BM}可以具有任意连接)。
% 577



\begin{figure}[!htb]
\ifOpenSource
\centerline{\includegraphics{figure.pdf}}
\else
	\centerline{\includegraphics{Chapter16/figures/rbm}}	
\fi
	\caption{一个画成\gls{markov_network}形式的\glssymbol{RBM}。}
	\label{fig:rbm}
\end{figure}



对~\glssymbol{RBM}~结构的限制产生了良好的属性
\begin{align}
\label{eqn:1611}
p(\RVh\mid\RVv) = \prod_i p(\RSh_i\mid \RVv)
\end{align}
以及
\begin{align}
\label{eqn:1612}
p(\RVv\mid\RVh) = \prod_i p(\RSv_i\mid \RVh).
\end{align}
% 578 head
独立的条件分布很容易计算。
对于二元的\gls{RBM},我们可以得到:
\begin{align}
\label{eqn:1613}
p(\RSh_i = 1\mid\RVv) &= \sigma\big(\RVv^{\top}\MW_{:,i} + b_i\big),\\
p(\RSh_i = 0\mid\RVv) &= 1 - \sigma\big(\RVv^{\top}\MW_{:,i} + b_i\big).
\end{align}
结合这些属性可以得到高效的\firstgls{block_gibbs_sampling},它在同时采样所有$\Vh$和同时采样所有$\Vv$之间交替。
\glssymbol{RBM}~模型通过~\gls{gibbs_sampling}产生的样本展示在\figref{fig:rbm_sample}中。
% 578 

\begin{figure}[!htb]
\ifOpenSource
\centerline{\includegraphics{figure.pdf}}
\else
	\centerline{\includegraphics[width=0.9\textwidth]{Chapter16/figures/rbm_samples}}	
\fi
\caption{训练好的~\glssymbol{RBM}~的样本及其权重。
\emph{(左)}用MNIST训练模型,然后用~\gls{gibbs_sampling}进行采样。
每一列是一个单独的~\gls{gibbs_sampling}过程。
每一行表示另一个$1000$步后~\gls{gibbs_sampling}的输出。
连续的样本之间彼此高度相关。
\emph{(右)}对应的权重向量。
将本图结果与图\ref{fig:s3c_samples}中描述的\gls{linear_factor}的样本和权重相比。
由于~\glssymbol{RBM}~的先验$p(\Vh)$没有限制为\gls{factorial},这里的样本表现得好很多。
采样时~\glssymbol{RBM}~能够学习到哪些特征需要一起出现。
另一方面说,\glssymbol{RBM}~后验$p(\Vh \mid \Vv)$是\gls{factorial}的,而\gls{sparse_coding}的后验并不是,所以在特征提取上\gls{sparse_coding}模型表现得更好。
其他的模型可以使用非\gls{factorial}的$p(\Vh)$和非\gls{factorial}的$p(\Vh \mid \Vh)$。
图片经~\citet{lisa_tutorial_rbm}允许转载。}
\label{fig:rbm_sample}
\end{figure}


由于\gls{energy_function}本身只是参数的线性函数,很容易获取\gls{energy_function}的导数。 例如,
\begin{align}
\label{eqn:1615}
\frac{\partial}{\partial W_{i,j}} E(\RVv,\RVh) = - \RSv_i \RSh_j.
\end{align}


% 579 head
这两个属性,高效的~\gls{gibbs_sampling}和导数计算,使训练过程变得非常方便。
在\chapref{chap:confronting_the_partition_function}中,我们将看到,可以通过计算应用于这种来自模型样本的导数来训练\gls{undirected_model}。
% 579

训练模型可以得到数据$\Vv$的表示$\Vh$。
我们经常使用$\SetE_{\RVh\sim p(\RVh\mid\Vv)}[\Vh]$ 作为一组描述$\Vv$的特征。
% 579



总的来说,\glssymbol{RBM}~展示了典型的\gls{graphical_models}\gls{DL}方法:
%结合由矩阵参数化的层之间的高效相互作用通过多层\gls{latent_variable}完成\gls{representation_learning}。
使用多层\gls{latent_variable},并由矩阵参数化层之间的高效相互作用来完成\gls{representation_learning}。
% 579


\gls{graphical_models}为描述概率模型提供了一种优雅、灵活、清晰的语言。 
在未来的章节中,我们将使用这种语言,以其他视角来描述各种各样的深度概率模型。
% 579





% !Mode:: "TeX:UTF-8"
% Translator: Tianfan Fu
\chapter{\glsentrytext{monte_carlo}方法}
\label{chap:monte_carlo_methods}
% 581

随机算法可以粗略地分为两类:\ENNAME{Las Vegas}\,算法和\gls{monte_carlo}算法。
\ENNAME{Las Vegas}\,算法总是精确地返回一个正确答案 (或者返回算法失败了)。   
这类方法通常需要占用随机量的计算资源(一般指内存或运行时间)。    
与此相对的,\gls{monte_carlo}方法返回的答案具有随机大小的错误。 
花费更多的计算资源(通常包括内存和运行时间)可以减少这种错误。
在任意固定的计算资源下, \gls{monte_carlo}算法可以得到一个近似解。

对于\gls{ML}中的许多问题来说,我们很难得到精确的答案。
这类问题很难用精确的确定性算法如\,\ENNAME{Las Vegas}\,算法解决。
取而代之的是确定性的近似算法或\gls{monte_carlo}近似方法。
这两种方法在\gls{ML}中都非常普遍。
本章主要关注\gls{monte_carlo}方法。
% 581

\section{采样和\glsentrytext{monte_carlo}方法}
\label{sec:sampling_and_monte_carlo_methods}

\gls{ML}中的许多重要工具都基于从某种分布中采样以及用这些样本对目标量做一个\gls{monte_carlo}估计。

\subsection{为什么需要采样?}
\label{sec:why_sampling}

有许多原因使我们希望从某个分布中采样。  
当我们需要以较小的代价近似许多项的和或某个积分时,采样是一种很灵活的选择。  
有时候,我们使用它加速一些很费时却易于处理的求和估计,就像我们使用\gls{minibatch}对整个训练代价进行\gls{subsample}一样。 
在其他情况下,我们需要近似一个难以处理的求和或积分,例如估计一个\gls{undirected_model}中\gls{partition_function}对数的梯度时。 
在许多其他情况下,抽样实际上是我们的目标,例如我们想训练一个可以从训练分布采样的模型。  
% 582 

\subsection{\glsentrytext{monte_carlo}采样的基础}
\label{sec:basics_of_monte_carlo_sampling}

当无法精确计算和或积分(例如,和具有指数数量个项,且无法被精确简化)时,通常可以使用\gls{monte_carlo}采样来近似它。
这种想法把和或者积分视作某分布下的期望,然后\emph{通过估计对应的平均值来近似这个期望}。
令
\begin{align}
s = \sum_{\Vx} p(\Vx)f(\Vx) = E_p[f(\RVx)]
\end{align}
或者
\begin{align}
\label{eqn:integra}
s = \int p(\Vx)f(\Vx)d\Vx = E_p[f(\RVx)]
\end{align}
为我们所需要估计的和或者积分,写成期望的形式,$p$是一个关于随机变量$\RVx$的概率分布(求和时)或者\gls{PDF}(求积分时)。
% 582 

我们可以通过从$p$中抽取$n$个样本$\Vx^{(1)},\ldots,\Vx^{(n)}$来近似$s$并得到一个经验平均值   
\begin{align}	
\hat{s}_n = \frac{1}{n}\sum_{i=1}^{n}f(\Vx^{(i)}).
\end{align}
下面几个性质表明了这种近似的合理性。
首先很容易观察到$\hat{s}$这个估计是\gls{unbiased}的,由于
\begin{align}
\SetE[\hat{s}_n] = \frac{1}{n}\sum_{i=1}^{n}\SetE[f(\Vx^{(i)})] = \frac{1}{n}\sum_{i=1}^{n}s = s.
\end{align}
% 582
此外,根据\firstgls{law_of_large_numbers},如果样本$\Vx^{(i)}$是\gls{iid_chap18}的,那么其平均值几乎必然收敛到期望值,即
\begin{align}
\lim_{n\xrightarrow{}\infty} \hat{s}_n = s,
\end{align}
只需要满足各个单项的方差$\text{Var}[f(\Vx^{(i)})]$有界。 
详细地说,我们考虑当$n$增大时$\hat{s}_n$的方差。
只要满足$\text{Var}[f(\RVx^{(i)})]<\infty$,方差$\text{Var}[\hat{s}_n]$就会减小并收敛到$0$:
\begin{align}
\text{Var}[\hat{s}_n] & = \frac{1}{n^2}\sum_{i=1}^{n}\text{Var}[f(\RVx)]\\
&  = \frac{\text{Var}[f(\RVx)]}{n}.
\end{align}
这个简单有用的结果启迪我们如何估计\gls{monte_carlo}均值中的不确定性,或者等价地说是\gls{monte_carlo}估计的期望误差。 
我们计算了$f(\Vx^{(i)})$的经验均值和方差\footnote{通常我们会倾向于计算方差的\gls{unbiased}估计,它由偏差的平方和除以$n-1$而非$n$得到。}, 
然后将估计的方差除以样本数$n$来得到$\text{Var}[\hat{s}_n]$的估计。
\firstgls{central_limit_theorem}告诉我们$\hat{s}_n$的分布收敛到以$s$为均值以$\frac{\text{Var}[f(\RVx)]}{n}$为方差的\gls{normal_distribution}。
这使得我们可以利用\gls{normal_distribution}的\gls{cumulative_function}来估计$\hat{s}_n$的置信区间。
% 583


以上的所有结论都依赖于我们可以从基准分布$p(\RVx)$中轻易地采样,但是这个假设并不是一直成立的。
当我们无法从$p$中采样时,一个备选方案是用\secref{sec:importance_sampling_chap17}讲到的\gls{importance_sampling}。 
一种更加通用的方式是构建一个收敛到目标分布的估计序列。 
这就是\gls{mcmc}方法(见\secref{sec:markov_chain_monte_carlo_methods})。
% 583

\section{\glsentrytext{importance_sampling}}
\label{sec:importance_sampling_chap17}

如方程~\eqref{eqn:integra}所示,在\gls{monte_carlo}方法中,对积分(或者和)分解,确定积分中哪一部分作为概率分布$p(\Vx)$以及哪一部分作为被积的函数$f(\Vx)$(我们感兴趣的是估计$f(\Vx)$在概率分布$p(\Vx)$下的期望)是很关键的一步。  
$p(\Vx)f(\Vx)$不存在唯一的分解,因为它总是可以被写成   
\begin{align}
\label{eqn:decomposition}
p(\Vx)f(\Vx) = q(\Vx) \frac{p(\Vx)f(\Vx)}{q(\Vx)},
\end{align}
在这里,我们从$q$分布中采样,然后估计$\frac{pf}{q}$在此分布下的均值。
许多情况中,我们希望在给定$p$和$f$的情况下计算某个期望,这个问题既然是求期望,那么很自然地$p$和$f$是一种分解选择。 
然而,如果考虑达到某给定精度所需要的样本数量,这个问题最初的分解选择不是最优的选择。
幸运的是,最优的选择$q^*$可以被简单地推导出来。   
这种最优的采样函数$q^*$对应所谓的最优\gls{importance_sampling}。 
% 584


从\eqnref{eqn:decomposition}所示的关系中可以发现,任意\gls{monte_carlo}估计   
\begin{align}
\hat{s}_p = \frac{1}{n}\sum_{i=1,\Vx^{(i)}\sim p}^{n}f(\Vx^{(i)})
\end{align}
可以被转化为一个\gls{importance_sampling}的估计
\begin{align}
\hat{s}_q = \frac{1}{n}\sum_{i=1,\Vx^{(i)}\sim q}^{n}\frac{p(\Vx^{(i)})f(\Vx^{(i)})}{q(\Vx^{(i)})}.
\end{align}
我们可以容易地发现估计的期望与$q$分布无关:   
\begin{align}
\SetE_q [\hat{s}_q] = \SetE_p [\hat{s}_p] = s.      
\end{align}
然而,\gls{importance_sampling}的方差可能对$q$的选择非常敏感。 
这个方差可以表示为
\begin{align}
\label{eqn:variance_of_is}
\Var [\hat{s}_q] = \Var \left[\frac{p(\RVx)f(\RVx)}{q(\RVx)}\right]/n.
\end{align}
方差想要取到最小值,$q$需要满足
\begin{align}
q^*(\Vx) = \frac{p(\Vx)\vert f(\Vx)\vert}{Z},
\end{align}
% 584
在这里$Z$表示归一化常数,选择适当的$Z$使得$q^*(\Vx)$之和或者积分为$1$。
一个更好的\gls{importance_sampling}分布会把更多的权重放在被积函数较大的地方。 
事实上,当$f(\Vx)$的正负符号不变时,$\Var[\hat{s}_{q^*}]=0$, 这意味着当使用最优的$q$分布时,\emph{只需要一个样本就足够了}。
当然,这仅仅是因为计算$q^*$时已经解决了原问题。   
所以在实践中这种只需要采样一个样本的方法往往是无法实现的。 
% 584


对于\gls{importance_sampling}来说任意$q$分布都是可行的(从得到一个期望上正确的值的角度来说),$q^*$指的是最优的$q$分布(从得到最小方差的角度上考虑)。
从$q^*$中采样往往是不可行的,但是其他仍然能降低方差的$q$的选择还是可行的。 
% 584  end


另一种方法是采用\firstgls{biased_importance_sampling},这种方法有一个优势,即不需要归一化的$p$或$q$分布。
在处理离散变量时,\gls{biased_importance_sampling}估计可以表示为
\begin{align}
\label{eqn:bis}
\hat{s}_{\text{BIS}} & = \frac{\sum_{i=1}^{n} \frac{p(\Vx^{(i)})}{q(\Vx^{(i)})} f(\Vx^{(i)})}{\sum_{i=1}^{n}\frac{p(\Vx^{(i)})}{q(\Vx^{(i)})}} \\
& = \frac{\sum_{i=1}^{n} \frac{p(\Vx^{(i)})}{\tilde{q}(\Vx^{(i)})} f(\Vx^{(i)})}{\sum_{i=1}^{n}\frac{p(\Vx^{(i)})}{\tilde{q}(\Vx^{(i)})}} \\
& = \frac{\sum_{i=1}^{n} \frac{\tilde{p}(\Vx^{(i)})}{\tilde{q}(\Vx^{(i)})} f(\Vx^{(i)})}{\sum_{i=1}^{n}\frac{\tilde{p}(\Vx^{(i)})}{\tilde{q}(\Vx^{(i)})}},
\end{align}
其中$\tilde{p}$和$\tilde{q}$分别是分布${p}$和${q}$的未经归一化的形式,$\Vx^{(i)}$是从分布${q}$中抽取的样本。
这种估计是有偏的,因为$\SetE[\hat{s}_{\text{BIS}}]\neq s$,只有当$n \to \infty$且方程~\eqnref{eqn:bis}的分母收敛到$1$时,等式才渐近地成立。
所以这一估计也被称为渐近\gls{unbiased}的。   
% 585


一个好的$q$分布的选择可以显著地提高\gls{monte_carlo}估计的效率,而一个糟糕的$q$分布选择则会使效率更糟糕。
我们回过头来看看方程~\eqnref{eqn:variance_of_is}会发现,如果存在一个$q$使得$\frac{p(\Vx)f(\Vx)}{q(\Vx)}$很大,那么这个估计的方差也会很大。
当$q(\Vx)$很小,而$f(\Vx)$和$p(\Vx)$都较大并且无法抵消$q$时,这种情况会非常明显。
$q$分布经常会取一些简单常用的分布使得我们能够从$q$分布中容易地采样。
当$\Vx$是高维数据时,$q$分布的简单性使得它很难与$p$或者$p\vert f\vert$相匹配。   
当$q(\Vx^{(i)})\gg p(\Vx^{(i)}) \vert f(\Vx^{(i)})\vert $时,\gls{importance_sampling}采到了很多无用的样本(很小的数或零相加)。
另一种相对少见的情况是$q(\Vx^{(i)})\ll p(\Vx^{(i)}) \vert f(\Vx^{(i)})\vert $,相应的比值会非常大。
正因为后一个事件是很少发生的,这种样本很难被采到,通常使得对$s$的估计出现了典型的\gls{underestimation},很难被整体的\gls{overestimation}抵消。
这样的不均匀情况在高维数据屡见不鲜,因为在高维度分布中联合分布的动态域可能非常大。
% 585 end


% 585 end
尽管存在上述的风险,但是\gls{importance_sampling}及其变种在\gls{ML}的应用中仍然扮演着重要的角色,包括\gls{DL}算法。
例如,\gls{importance_sampling}被应用于加速训练具有大规模词表的神经网络\gls{language_model}的过程中(见\secref{sec:importance_sampling_chap12})或者其他有着大量输出结点的\gls{NN}中。   
此外,还可以看到\gls{importance_sampling}应用于估计\gls{partition_function}(一个概率分布的归一化常数),详见\secref{sec:estimating_the_partition_function},以及在深度有向图模型比如\gls{VAE}中估计对数似然(详见\secref{sec:variational_autoencoders})。
采用\gls{SGD}训练模型参数时\gls{importance_sampling}可以用来改进对\gls{cost_function}梯度的估计,尤其是分类器这样的模型,其中\gls{cost_function}的大部分\gls{cost}来自于少量错误分类的样本。    
在这种情况下,更加频繁地抽取这些困难的样本可以减小梯度估计的方差~\citep{Hinton06}。 
% 586 


\section{\glsentrytext{mcmc}方法}
\label{sec:markov_chain_monte_carlo_methods}

在许多实例中,我们希望采用\gls{monte_carlo}方法,然而往往又不存在一种简单的方法可以直接从目标分布$p_{\text{model}}(\RVx)$中精确采样或者一个好的(方差较小的)\gls{importance_sampling}分布$q(\Vx)$。 
在深度学习中,当分布$p_{\text{model}}(\RVx)$表示成无向模型时,这种情况往往会发生。
在这种情况下,为了从分布$p_{\text{model}}(\RVx)$中近似采样,我们引入了一种称为\firstgls{markov_chain}的数学工具。
利用\gls{markov_chain}来进行\gls{monte_carlo}估计的这一类算法被称为\firstall{mcmc}方法。
\citet{koller-book2009}花了大量篇幅来描述\gls{mcmc}算法在\gls{ML}中的应用。
\glssymbol{mcmc}~技术最标准、最一般的理论保证只适用于那些各状态概率均不为零的模型。
因此,这些技术最方便的使用方法是用于从\firstgls{energy_based_model}即$p(\Vx)\propto \exp(-E(\Vx))$中采样,见\secref{sec:energy_based_models}。
在~\glssymbol{energy_based_model}~的公式表述中,每一个状态所对应的概率都不为零。
事实上,\glssymbol{mcmc}~方法可以被广泛地应用在包含$0$概率状态的许多概率分布中。
然而,在这种情况下,关于~\glssymbol{mcmc}~方法性能的理论保证只能依据具体不同类型的分布具体分析证明。
在\gls{DL}中,我们通常依赖于那些一般的理论保证,其在所有\gls{energy_based_model}都能自然成立。

% 586 end


为了解释从\gls{energy_based_model}中采样困难的原因,我们考虑一个包含两个变量的~\glssymbol{energy_based_model}~的例子,记$p(\RSa,\RSb)$为其分布。  
为了采$\RSa$,我们必须先从$p(\RSa\mid \RSb)$中采样;为了采$\RSb$,我们又必须从$p(\RSb\mid \RSa)$中采样。   
这似乎成了棘手的先有鸡还是先有蛋的问题。
\gls{directed_model}避免了这一问题因为它的图是有向无环的。
为了完成\firstgls{ancestral_sampling},在给定每个变量的所有父结点的条件下,我们根据拓扑顺序采样每一个变量,这个变量是确定能够被采样的(详见\secref{sec:sampling_from_graphical_models})。   
\gls{ancestral_sampling}定义了一种高效的、单遍的方法来抽取一个样本。   
% 587 begin


在~\glssymbol{energy_based_model}~中,我们通过使用\gls{markov_chain}来采样,从而避免了先有鸡还是先有蛋的问题。  
\gls{markov_chain}的核心思想是从某个可取任意值的状态$\Vx$出发。
随着时间的推移,我们随机地反复更新状态$\Vx$。  
最终$\Vx$成为了一个从$p(\Vx)$中抽出的(非常接近)比较一般的样本。  %?? 建议 比较公正 换成 标准
在正式的定义中,\gls{markov_chain}由一个随机状态$x$和一个转移分布$T(\Vx'\mid \Vx)$定义而成,$T(\Vx'\mid \Vx)$是一个概率分布,说明了给定状态$\Vx$的情况下随机地转移到$\Vx'$的概率。
运行一个\gls{markov_chain}意味着根据转移分布$T(\RVx' \mid \Vx)$采出的值$\Vx'$来更新状态$\Vx$。  
% 587  mid


为了给出~\glssymbol{mcmc}~方法为何有效的一些理论解释,重参数化这个问题是很有用的。 
首先我们关注一些简单的情况,其中随机变量$\RVx$有可数个状态。
我们将这种状态简单地记作正整数$x$。
不同的整数$x$的大小对应着原始问题中$\Vx$的不同状态。
% 587   mid


接下来我们考虑如果并行地运行无穷多个\gls{markov_chain}的情况。  
不同\gls{markov_chain}的所有状态都采样自某一个分布$q^{(t)}(x)$,在这里$t$表示消耗的时间数。
开始时,对每个\gls{markov_chain},我们采用一个分布$q^{{0}}$来任意地初始化$x$。  
之后,$q^{(t)}$与所有之前运行的\gls{markov_chain}有关。
我们的目标是$q^{(t)}(x)$收敛到$p(x)$。 
% 587 mid 


因为我们已经用正整数$x$重参数化了这个问题,我们可以用一个向量$\Vv$来描述这个概率分布$q$,  
\begin{align}
q(\RSx = i) = v_i.
\end{align}
% 587


然后我们考虑更新单一的\gls{markov_chain},从状态$x$到新状态$x'$。
单一状态转移到$x'$的概率可以表示为
\begin{align}
\label{eqn:transition1}
q^{(t+1)}(x') = \sum_{x} q^{(t)}(x) T(x'\mid x).
\end{align}
% 587 end


根据状态为整数的参数化设定,我们可以将转移算子$T$表示成一个矩阵$\MA$。
矩阵$\MA$的定义如下:
\begin{align}
\MA_{i,j} = T(\RVx' = i\mid \RVx = j).
\end{align}
使用这一定义,我们可以改写\eqnref{eqn:transition1}。
不同于之前使用$q$和$T$来理解单个状态的更新,我们现在可以使用$\Vv$和$\MA$来描述当我们更新时(并行运行的)不同\gls{markov_chain}上整个分布是如何变化的: 
\begin{align}
\Vv^{(t)} = \MA \Vv^{(t-1)}.
\end{align}
% 588   head
重复地使用\gls{markov_chain}更新相当于重复地与矩阵$\MA$相乘。  
换言之,我们可以认为这一过程就是关于$\MA$的幂乘:  
\begin{align}
\Vv^{(t)} = \MA^{t} \Vv^{(0)}.
\end{align}
% 588


矩阵$\MA$有一种特殊的结构,因为它的每一列都代表一个概率分布。
这样的矩阵被称为\firstgls{stochastic_matrix}。
如果对于任意状态$x$到任意其他状态$x'$存在一个$t$使得转移概率不为$0$,那么Perron-Frobenius定理~\citep{perron1907theorie,frobenius1908matrizen}可以保证这个矩阵的最大特征值是实数且大小为$1$。
我们可以看到所有的特征值随着时间呈现指数变化: 
\begin{align}
\Vv^{(t)} = (\MV \text{diag}(\Vlambda)\MV^{-1})^{t} \Vv^{(0)} = \MV \text{diag}(\Vlambda)^t \MV^{-1} \Vv^{(0)}.
\end{align}
这个过程导致了所有不等于$1$的特征值都衰减到$0$。  
在一些额外的较为宽松的假设下,我们可以保证矩阵$\MA$只有一个对应特征值为$1$的特征向量。  
所以这个过程收敛到\firstgls{stationary_distribution},有时也被称为\firstgls{equilibrium_distribution}。 
收敛时,我们得到 
\begin{align}
\label{eqn:1723}
\Vv ' = \MA \Vv = \Vv,
\end{align}
这个条件也适用于收敛之后的每一步。
这就是特征向量方程。
作为收敛的稳定点,$\Vv$一定是特征值为$1$所对应的特征向量。 
这个条件保证收敛到了\gls{stationary_distribution}以后,再重复转移采样过程不会改变所有不同\gls{markov_chain}上状态的\emph{分布}(尽管转移算子自然而然地会改变每个单独的状态)。  
% 588


如果我们正确地选择了转移算子$T$,那么最终的\gls{stationary_distribution}$q$将会等于我们所希望采样的分布$p$。  
我们会将\secref{sec:gibbs_sampling}介绍如何选择$T$。 
% 588


可数状态\gls{markov_chain}的大多数性质可以被推广到连续状态的\gls{markov_chain}中。 
在这种情况下,一些研究者把这种\gls{markov_chain}称为\firstgls{harris_chain},但是我们将这两种情况都称为\gls{markov_chain}。
通常在一些宽松的条件下,一个带有转移算子$T$的\gls{markov_chain}都会收敛到一个不动点,这个不动点可以写成如下形式:
\begin{align}
q' (\RVx') = \SetE_{\RVx\sim q}T(\RVx'\mid \RVx),
\end{align}
这个方程的离散版本就相当于重新改写方程~\eqnref{eqn:1723}。
当$\RVx$是离散值时,这个期望对应着求和,而当$\RVx$是连续值时,这个期望对应的是积分。 
% 589 head


无论状态是连续的还是离散的,所有的\gls{markov_chain}方法都包括了重复、随机地更新直到最后状态开始从\gls{equilibrium_distribution}中采样。 
运行\gls{markov_chain}直到它达到\gls{equilibrium_distribution}的过程通常被称为\gls{markov_chain}的\firstgls{burn_in}过程。
在\gls{markov_chain}达到\gls{equilibrium_distribution}之后,我们可以从\gls{equilibrium_distribution}中抽取一个无限多数量的样本序列。 
这些样本服从同一分布,但是两个连续的样本之间会高度相关。 
所以一个有限的序列无法完全表达\gls{equilibrium_distribution}。
一种解决这个问题的方法是每隔$n$个样本返回一个样本,从而使得我们对于\gls{equilibrium_distribution}的统计量的估计不会被\,\glssymbol{mcmc}\,方法的样本之间的相关性所干扰。
所以\gls{markov_chain}的计算代价很高,主要源于达到\gls{equilibrium_distribution}前需要\gls{burn_in}的时间以及在达到\gls{equilibrium_distribution}之后从一个样本转移到另一个足够无关的样本所需要的时间。
如果我们想要得到完全独立的样本,那么我们可以同时并行地运行多个\gls{markov_chain}。  
这种方法使用了额外的并行计算来减少时延。 
使用一条\gls{markov_chain}来生成所有样本的策略和(使用多条\gls{markov_chain})每条\gls{markov_chain}只产生一个样本的策略是两种极端。
深度学习的从业者们通常选取的\gls{markov_chain}的数目和\gls{minibatch}中的样本数相近,然后从这些固定的\gls{markov_chain}集合中抽取所需要的样本。 
\gls{markov_chain}的数目通常选为$100$。
% 589

另一个难点是我们无法预先知道\gls{markov_chain}需要运行多少步才能到达\gls{equilibrium_distribution}。 
这段时间通常被称为\firstgls{mixing_time}。
检测一个\gls{markov_chain}是否达到平衡是很困难的。
我们并没有足够完善的理论来解决这个问题。
理论只能保证\gls{markov_chain}会最终收敛,但是无法保证其他。
如果我们从矩阵$\MA$作用在概率向量$\Vv$上的角度来分析\gls{markov_chain},那么我们可以发现当$\MA^t$除了单个$1$以外的特征值都趋于$0$时,\gls{markov_chain}混合成功(收敛到了\gls{equilibrium_distribution})。
这也意味着矩阵$\MA$的第二大特征值决定了\gls{markov_chain}的\gls{mixing_time}。
然而,在实践中,我们通常不能真的将\gls{markov_chain}表示成矩阵的形式。 
我们的概率模型所能够达到的状态是变量数的指数级别,所以表达$\Vv$,$\MA$或者$\MA$的特征值是不现实的。 
由于以上在内的诸多阻碍,我们通常无法知道\gls{markov_chain}是否已经混合成功。
作为替代,我们只能运行一定量时间\gls{markov_chain}直到我们粗略估计这段时间是足够的,然后使用启发式的方法来判断\gls{markov_chain}是否混合成功。
这些启发性的算法包括了手动检查样本或者衡量前后样本之间的相关性。
% 590 head



\section{\glsentrytext{gibbs_sampling}}
\label{sec:gibbs_sampling}

目前为止我们已经了解了如何通过反复更新$\Vx \xleftarrow{} \Vx'\sim T(\Vx'\mid\Vx)$从一个分布$q(\Vx)$中采样。 
然而我们还没有介绍过如何确定$q(\Vx)$是否是一个有效的分布。
本书中将会描述两种基本的方法。  
第一种方法是从已经学习到的分布$p_{\text{model}}$中推导出$T$,下文描述了如何从\gls{energy_based_model}中采样。
第二种方法是直接用参数描述$T$,然后学习这些参数,其\gls{stationary_distribution}隐式地定义了我们所感兴趣的模型$p_{\text{model}}$。
我们将在\secref{sec:generative_stochastic_networks}和\secref{sec:other_generation_schemes}中讨论第二种方法的例子。  
% 590  mid


在\gls{DL}中,我们通常使用\gls{markov_chain}从定义为\gls{energy_based_model}的分布$p_{\text{model}}(\Vx)$中采样。
在这种情况下,我们希望\gls{markov_chain}的$q(\Vx)$分布就是$p_{\text{model}}(\Vx)$。
为了得到所期望的$q(\Vx)$分布,我们必须选取合适的$T(\Vx'\mid \Vx)$。
% 590


\firstgls{gibbs_sampling}是一种概念简单而又有效的方法。
它构造一个从$p_{\text{model}}(\Vx)$中采样的\gls{markov_chain},其中在\gls{energy_based_model}中从$T(\RVx'\mid \RVx)$采样是通过选择一个变量$\RSx_i$,然后从$p_{\text{model}}$中该点关于在无向图$\CalG$(定义了\gls{energy_based_model}结构)中邻接点的条件分布中采样。
只要一些变量在给定相邻变量时是条件独立的,那么这些变量就可以被同时采样。
正如在\secref{sec:example_the_restricted_boltzmann_machine}中看到的~\glssymbol{RBM}~示例一样,\glssymbol{RBM}~中所有的\gls{hidden_unit}可以被同时采样,因为在给定所有可见单元的条件下它们相互条件独立。
同样地,所有的可见单元也可以被同时采样,因为在给定所有\gls{hidden_unit}的情况下它们相互条件独立。 
以这种方式同时更新许多变量的\,\gls{gibbs_sampling}通常被称为\firstgls{block_gibbs_sampling}。   
% 590 mid

设计从$p_{\text{model}}$中采样的\gls{markov_chain}还存在其他备选方法。  
比如说,Metropolis-Hastings算法在其他领域中广泛使用。
不过在\gls{DL}的\gls{undirected_model}中,我们主要使用\,\gls{gibbs_sampling},很少使用其他方法。 
改进采样技巧也是一个潜在的研究热点。
% 590 end


\section{不同的\glsentrytext{mode}之间的\glsentrytext{mixing}挑战}
\label{sec:the_challenge_of_mixing_between_separated_modes}
% 591  head 

使用\,\glssymbol{mcmc}\,方法的主要难点在于\gls{markov_chain}的\firstgls{mixing}通常不理想。
在理想情况下,从设计好的\gls{markov_chain}中采出的连续样本之间是完全独立的,而且在$\Vx$空间中,\gls{markov_chain}会按概率大小访问许多不同区域。

然而,\,\glssymbol{mcmc}\,方法采出的样本可能会具有很强的相关性,尤其是在高维的情况下。
我们把这种现象称为慢\gls{mixing}甚至\gls{mixing}失败。
具有缓慢\gls{mixing}的\,\glssymbol{mcmc}\,方法可以被视为对\gls{energy_function}无意地执行类似于带噪声的\gls{GD}的操作,或者说等价于相对于链的状态(被采样的随机变量)依据概率进行噪声爬坡。
(在\gls{markov_chain}的状态空间中)从$\Vx^{(t-1)}$到$\Vx^{(t)}$该链倾向于选取很小的步长,其中能量$E(\Vx^{(t)})$通常低于或者近似等于能量$E(\Vx^{(t-1)})$,
倾向于向较低能量的区域移动。
当从可能性较小的状态(比来自$p(\Vx)$的典型样本拥有更高的能量)开始时,链趋向于逐渐减少状态的能量,并且仅仅偶尔移动到另一个\gls{mode}。 
一旦该链已经找到低能量的区域(例如,如果变量是图像中的像素,则低能量的区域可以是同一对象所对应图像的一个连通的\gls{manifold}),我们称之为\gls{mode},链将倾向于围绕着这个\gls{mode}游走(按某一种形式随机游走)。
它时不时会走出该\gls{mode},但是结果通常会返回该\gls{mode}或者(如果找到一条离开的路线)移向另一个\gls{mode}。
问题是对于很多有趣的分布来说成功的离开路线很少,所以\gls{markov_chain}将在一个\gls{mode}附近抽取远超过需求的样本。
% 591 


当我们考虑~\gls{gibbs_sampling}算法(见\secref{sec:gibbs_sampling})时,这种现象格外明显。
在这种情况下,我们考虑在一定步数内从一个\gls{mode}移动到一个临近\gls{mode}的概率。
决定这个概率的是两个\gls{mode}之间的``能量障碍''的形状。
隔着一个巨大``能量障碍'' (低概率的区域)的两个\gls{mode}之间的转移概率是(随着能量障碍的高度)指数下降的,如\figref{fig:chap17_good_bad_really_bad_mixing_color}所示。
当目标分布有多个高概率\gls{mode}并且被低概率区域所分割,尤其当\,\gls{gibbs_sampling}的每一步都只是更新变量的一小部分而这一小部分变量又严重依赖其他的变量时,就会产生问题。 
% 591


% 592 head 
\begin{figure}[!htb]
\ifOpenSource
\centerline{\includegraphics{figure.pdf}}
\else
	\centerline{\includegraphics{Chapter17/figures/good_bad_really_bad_mixing_color}}
\fi
\caption{对于三种分布使用~\gls{gibbs_sampling}所产生的路径,所有的分布\gls{markov_chain}初始值都设为\gls{mode}。
\emph{(左)}一个带有两个独立变量的\gls{multivariate_normal_distribution}。
由于变量之间是相互独立的,\gls{gibbs_sampling}\gls{mixing}得很好。
\emph{(中)}变量之间存在高度相关性的一个\gls{multivariate_normal_distribution}。
变量之间的相关性使得\gls{markov_chain}很难\gls{mixing}。
因为每一个变量的更新需要相对其他变量求条件分布,相关性减慢了\gls{markov_chain}远离初始点的速度。
\emph{(右)}\,\gls{mode}之间间距很大且不在轴上对齐的混合高斯分布。
\gls{gibbs_sampling}\gls{mixing}得很慢,因为每次更新仅仅一个变量很难跨越不同的\gls{mode}。}
\label{fig:chap17_good_bad_really_bad_mixing_color}
\end{figure}


%  592
举一个简单的例子,考虑两个变量$\RSa$,$\RSb$的\gls{energy_based_model},这两个变量都是二值的,取值$+1$或者$-1$。   %??  
如果对某个较大的正数$w$,$E(\RSa,\RSb) = - w \RSa \RSb$,那么这个模型传达了一个强烈的信息,$\RSa$和$\RSb$有相同的符号。
当$\RSa=1$时用~\gls{gibbs_sampling}更新$\RSb$。
给定$\RSb$时的条件分布满足$p(\RSb=1\mid \RSa=1) = \sigma(w)$。
如果$w$的值很大,\gls{sigmoid}~函数趋近于饱和,那么$b$也取到$1$的概率趋近于$1$。
同理,如果$\RSa=-1$,那么$\RSb$取到$-1$的概率也趋于$1$。 
根据模型$p_{\text{model}}(\RSa,\RSb)$,两个变量取一样的符号的概率几乎相等。
根据$p_{\text{model}}(\RSa\mid \RSb)$,两个变量应该有相同的符号。
这也意味着~\gls{gibbs_sampling}很难会改变这些变量的符号。
% 592

在更实际的问题中,这种挑战更加艰巨因为在实际问题中我们不能仅仅关注在两个\gls{mode}之间的转移,更要关注在多个\gls{mode}之间的转移。 
如果由于\gls{mode}之间\gls{mixing}困难,而导致某几个这样的转移难以完成,那么得到一些可靠的覆盖大部分\gls{mode}的样本集合的计算代价是很高的,同时\gls{markov_chain}收敛到它的\gls{stationary_distribution}的过程也会非常缓慢。
% 592

通过寻找一些高度依赖变量的组以及分块同时更新块(组)中的变量,这个问题有时候是可以被解决的。
然而不幸的是,当依赖关系很复杂时,从这些组中采样的过程从计算角度上说是难以处理的。
归根结底,\gls{markov_chain}最初就是被提出来解决这个问题,即从大量变量中采样的问题。
% 592


% 593 head
在定义了一个联合分布$p_{\text{model}}(\Vx,\Vh)$的\gls{latent_variable}模型中,我们经常通过交替地从$p_{\text{model}}(\Vx\mid \Vh)$和$p_{\text{model}}(\Vh\mid \Vx)$中采样来达到抽$\Vx$的目的。  
从快速\gls{mixing}的角度上说,我们更希望$p_{\text{model}}(\Vh\mid \Vx)$有很大的熵。
然而,从学习一个$\Vh$的有用表示的角度上考虑,我们还是希望$\Vh$能够包含$\Vx$的足够信息从而能够较完整地重构它,这意味$\Vh$和$\Vx$要有非常高的互信息。
这两个目标是相互矛盾的。
我们经常学习到能够将$\Vx$精确地编码为$\Vh$的\gls{generative_model},但是无法很好\gls{mixing}。
这种情况在\gls{BM}中经常出现,一个\gls{BM}学到的分布越尖锐,该分布的\gls{markov_chain}采样越难\gls{mixing}得好。
这个问题在\figref{fig:chap17_fig-dbm-bad-mixing}中有所描述。
% 593


% 593 end
\begin{figure}[!htb]
\ifOpenSource
\centerline{\includegraphics{figure.pdf}}
\else
    \centering
    \begin{tabular}{cc}
    \includegraphics[width=0.45\figwidth]{Chapter17/figures/fig-dbm-bad-mixing}
    \includegraphics[width=0.45\figwidth]{Chapter17/figures/fig-adversarial}    
    \end{tabular}
\fi
\caption{深度概率模型中一个\gls{mixing}缓慢问题的例证。
每张图都是按照从左到右从上到下的顺序的。
\emph{(左)}~\gls{gibbs_sampling}从MNIST数据集训练成的\gls{DBM}中采出的连续样本。
这些连续的样本之间非常相似。
由于\,\gls{gibbs_sampling}作用于一个深度图模型,相似度更多地是基于语义而非原始视觉特征。
但是对于吉布斯链来说从分布的一个\gls{mode}转移到另一个仍然是很困难的,比如说改变数字。
\emph{(右)}从\gls{generative_adversarial_networks}中抽出的连续原始样本。
因为\gls{ancestral_sampling}生成的样本之间互相独立,所以不存在\gls{mixing}问题。}
\label{fig:chap17_fig-dbm-bad-mixing}
\end{figure}
% 593 end


% 593 mid
当感兴趣的分布对于每个类具有单独的\gls{manifold}结构时,所有这些问题都使\,\glssymbol{mcmc}\,方法变得不那么有用:分布集中在许多\gls{mode}周围,并且这些\gls{mode}由大量高能量区域分割。
我们在许多分类问题中遇到的是这种类型的分布,由于\gls{mode}之间\gls{mixing}缓慢,它将使得MCMC方法非常缓慢地收敛。
% 593



\subsection{不同\gls{mode}之间通过\glsentrytext{tempering}来\glsentrytext{mixing}}
\label{sec:tempering_to_mix_between_modes}
% 594

当一个分布有一些陡峭的峰并且被低概率区域包围时,很难在分布的不同\gls{mode}之间\gls{mixing}。   %?? 峰值
一些加速\gls{mixing}的方法是基于构造一个概率分布替代目标分布,这个概率分布的\gls{mode}没有那么高,\gls{mode}周围的低谷也没有那么低。
\gls{energy_based_model}为这个想法提供一种简单的做法。
目前为止,我们一直将\gls{energy_based_model}描述为定义一个概率分布:   
\begin{align}
\label{eqn:1725}
p(\Vx) \propto \exp(-E(\Vx)).
\end{align}
\gls{energy_based_model}可以通过添加一个额外的控制\gls{mode}尖锐程度的参数$\beta$来加强:
\begin{align}
\label{eqn:1726}
p_{\beta}(\Vx) \propto \exp(-\beta E(\Vx)).
\end{align}
$\beta$参数可以被理解为\firstgls{temperature}的倒数,反映了\gls{energy_based_model}的统计物理学起源。
当\gls{temperature}趋近于0时,$\beta$趋近于无穷大,此时的\gls{energy_based_model}是确定性的。
当\gls{temperature}趋近于无穷大时,$\beta$趋近于零,\gls{energy_based_model}(对离散的$\Vx$)成了均匀分布。
% 594

通常情况下,在$\beta = 1$时训练一个模型。
但我们也可以利用其他\gls{temperature},尤其是$\beta < 1$的情况。
\firstgls{tempering}作为一种通用的策略,它通过从$\beta<1$模型中采样来实现在$p_1$的不同\gls{mode}之间快速\gls{mixing}。
% 594

基于\firstgls{tempering_transition}~\citep{Neal94b}的\gls{markov_chain}临时从高\gls{temperature}的分布中采样使其在不同\gls{mode}之间\gls{mixing},然后继续从单位\gls{temperature}的分布中采样。
这些技巧被应用在一些模型比如\,\glssymbol{RBM}中~\citep{Salakhutdinov-2010}。
另一种方法是利用\firstgls{parallel_tempering}~\citep{Iba-2001}。
其中\gls{markov_chain}并行地模拟许多不同\gls{temperature}的不同状态。
最高\gls{temperature}的状态\gls{mixing}较慢,相比之下最低\gls{temperature}的状态,即\gls{temperature}为$1$时,采出了精确的样本。
转移算子包括了两个\gls{temperature}之间的随机跳转,所以一个高\gls{temperature}状态分布槽中的样本有足够大的概率跳转到低\gls{temperature}分布的槽中。
这个方法也被应用到了\,\glssymbol{RBM}中~\citep{desjardins2010tempered,Cho10IJCNN}。
尽管\gls{tempering}这种方法前景可期,现今它仍然无法让我们在采样复杂的\gls{energy_based_model}中更进一步。
一个可能的原因是在\firstgls{critical_temperatures}时\gls{temperature}转移算子必须设置得非常慢(因为\gls{temperature}需要逐渐下降)来确保\gls{tempering}的有效性。
% 594


% 595
\subsection{深度也许会有助于\glsentrytext{mixing}}
\label{sec:depth_may_help_mixing}

当我们从\gls{latent_variable}模型$p(\Vh,\Vx)$中采样时,我们可以发现如果$p(\Vh\mid \Vx)$将$\Vx$编码得非常好,那么从$p(\Vx \mid \Vh)$中采样时,并不会太大地改变$\Vx$,那么\gls{mixing}结果会很糟糕。
解决这个问题的一种方法是使得$\Vh$成为一种将$\Vx$编码为$\Vh$的深度表示,从而使得\gls{markov_chain}在$\Vh$空间中更容易\gls{mixing}。
在许多\gls{representation_learning}算法如\gls{AE}和\,\glssymbol{RBM}\,中,$\Vh$的边缘分布相比于$\Vx$上的原始数据分布,通常表现为更加均匀、更趋近于\gls{unimodal}。
或许可以说,这是因为利用了所有可用的表示空间并尽量减小\gls{reconstruction_error}。
因为当训练集上的不同样本之间在$\Vh$空间能够被非常容易地区分时,我们也会很容易地最小化\gls{reconstruction_error}。
\citet{Bengio-et-al-ICML2013-small}观察到这样的现象,堆叠越深的\gls{regularize}\gls{AE}或者\,\glssymbol{RBM},顶端$\Vh$空间的边缘分布越趋向于均匀和发散,而且不同\gls{mode}(比如说实验中的类别)所对应区域之间的间距也会越小。
在高层空间中训练\,\glssymbol{RBM}\,会使得~\gls{gibbs_sampling}在\gls{mode}间\gls{mixing}得更快。
然而,如何利用这种观察到的现象来辅助训练深度\gls{generative_model}或者从中采样仍然有待探索。
% 595

尽管存在\gls{mixing}的难点,\gls{monte_carlo}技术仍然是一个有用的工具,通常也是最好的可用工具。  
事实上,在遇到难以处理的\gls{undirected_model}中的\gls{partition_function}时,\gls{monte_carlo}方法仍然是最主要的工具,这将在下一章详细阐述。
% 595

% !Mode:: "TeX:UTF-8"
% Translator: Yujun Li 
\chapter{直面\glsentrytext{partition_function}}
\label{chap:confronting_the_partition_function}
% 597
在\secref{sec:undirected_models}中,我们看到许多概率模型(通常是\gls{undirected_graphical_model})由一个未归一化的\gls{PD} $\tilde{p}(\RVx, \theta)$定义。
我们必须通过除以\gls{partition_function} $Z(\Vtheta)$来归一化$\tilde{p}$,以获得一个有效的\gls{PD}:
\begin{equation}
	p(\RVx; \Vtheta) = \frac{1}{Z(\Vtheta)} \tilde{p}(\RVx; \Vtheta).
\end{equation}
\gls{partition_function}是未归一化概率所有状态的积分(对于连续变量)或求和(对于离散变量):
\begin{equation}
	\int \tilde{p}(\Vx) d \Vx
\end{equation}
或者
\begin{equation}
	\sum_{\Vx} \tilde{p} (\Vx).
\end{equation}


对于很多有趣的模型而言,以上积分或求和难以计算。


正如我们将在\chapref{chap:deep_generative_models}看到的,有些\gls{DL}模型被设计成具有一个易于处理的归一化常数,或被设计成能够在不涉及计算$p(\RVx)$的情况下使用。
然而,其他一些模型会直接面对难以计算的\gls{partition_function}的挑战。
在本章中,我们会介绍用于训练和评估那些具有难以处理的\gls{partition_function}的模型的技术。
% -- 597 end


\section{对数似然梯度}
\label{sec:the_log_likelihood_gradient}
通过\gls{maximum_likelihood}学习\gls{undirected_model}特别困难的原因在于\gls{partition_function}依赖于参数。
对数似然相对于参数的梯度具有一项对应于\gls{partition_function}的梯度:
\begin{equation}
	\nabla_{\Vtheta} \log p(\RVx; \Vtheta) = \nabla_{\Vtheta} \log \tilde{p}(\RVx; \Vtheta) -
\nabla_{\Vtheta} \log Z(\Vtheta).
\end{equation}
% 598 head

这是机器学习中非常著名的\firstgls{positive_phase}和\firstgls{negative_phase}的分解。


对于大多数感兴趣的\gls{undirected_model}而言,\gls{negative_phase}是困难的。
没有\gls{latent_variable}或\gls{latent_variable}之间很少相互作用的模型通常会有一个易于计算的\gls{positive_phase}。
\glssymbol{RBM}\,的\gls{hidden_unit}在给定可见单元的情况下彼此条件独立,是一个典型的具有简单\gls{positive_phase}和困难\gls{negative_phase}的模型。
\gls{positive_phase}计算困难,\gls{latent_variable}之间具有复杂相互作用的情况将主要在\chapref{chap:approximate_inference}中讨论。本章主要探讨\gls{negative_phase}计算中的难点。


让我们进一步分析$\log Z$的梯度:
\begin{equation}
	\nabla_{\Vtheta} \log Z
\end{equation}
\begin{equation}
	= \frac{ \nabla_{\Vtheta} Z }{Z}
\end{equation}
\begin{equation}
	= \frac{ \nabla_{\Vtheta} \sum_{\RVx} \tilde{p}(\RVx) }{Z}
\end{equation}
\begin{equation}
	= \frac{ \sum_{\RVx} \nabla_{\Vtheta} \tilde{p}(\RVx) }{Z}.
\end{equation}


对于保证所有的$\RVx$都有$p(\RVx) > 0$的模型,我们可以用$\exp(\log \tilde{p}(\RVx))$代替$\tilde{p}(\RVx)$:
\begin{equation}
	\frac{ \sum_{\RVx} \nabla_{\Vtheta} \exp (\log \tilde{p} (\RVx)) }{ Z }
\end{equation}
\begin{equation}
	= \frac{  \sum_{\RVx}  \exp (\log \tilde{p} (\RVx)) \nabla_{\Vtheta} \log \tilde{p}(\RVx)  }{Z}
\end{equation}
\begin{equation}
	=\frac{  \sum_{\RVx} \tilde{p} (\RVx)  \nabla_{\Vtheta} \log \tilde{p}(\RVx)  }{ Z}
\end{equation}
\begin{equation}
	=\sum_{\RVx} p(\RVx) \nabla_{\Vtheta} \log \tilde{p} ( \RVx )
 \end{equation}
\begin{equation}
	=\SetE_{\RVx \sim p(\RVx)} \nabla_{\Vtheta} \log \tilde{p} (\RVx).
\end{equation}
% -- 598 end


上述推导对离散的$\Vx$进行求和,对连续的$\Vx$进行积分也可以得到类似结果。
在连续版本的推导中,使用在积分符号内取微分的\gls{leibniz_rule}可以得到等式
\begin{equation}
	\nabla_{\Vtheta} \int \tilde{p} (\RVx) d \Vx  = \int \nabla_{\Vtheta} 
\tilde{p} (\RVx) d \Vx.
\end{equation}
该等式只适用于$\tilde{p}$和$\nabla_{\Vtheta} \tilde{p} (\RVx)$上的一些特定规范条件。
在测度论术语中,这些条件是:
(1) 对每一个$\Vtheta$而言,未归一化分布$\tilde{p}$必须是$\Vx$的\gls{lebesgue_integrable}函数。
(2) 对于所有的$\Vtheta$和几乎所有$\Vx$,梯度$\nabla_{\Vtheta} \tilde{p}(\RVx)$必须存在。
(3) 对于所有的$\Vtheta$和几乎所有的$\Vx$,必须存在一个可积函数$R(\Vx)$使得$\max_i | \frac{\partial}{\partial \theta_i } \tilde{p} (\RVx) | \leq R(\Vx)$。
幸运的是,大多数感兴趣的\gls{ML}模型都具有这些性质。
% 599 mid

等式
\begin{equation}
\label{eq:18.15}
	\nabla_{\Vtheta} \log Z = \SetE_{\RVx \sim p(\RVx)} \nabla_{\Vtheta} \log \tilde{p}(\RVx)
\end{equation}
是使用各种\gls{monte_carlo}方法近似最大化(具有难计算\gls{partition_function}模型的)似然的基础。
% 599 mid

\gls{monte_carlo}方法为学习\gls{undirected_model}提供了直观的框架,我们能够在其中考虑\gls{positive_phase}和\gls{negative_phase}。
在\gls{positive_phase}中,我们增大从数据中采样得到的$\log \tilde{p}(\RVx)$。
在\gls{negative_phase}中,我们通过降低从模型分布中采样的$\log \tilde{p}(\RVx)$来降低\gls{partition_function}。
% 599 mid

在\gls{DL}文献中,经常会看到用\gls{energy_function}(\eqnref{eqn:167})来参数化$\log \tilde{p}$。
在这种情况下,\gls{positive_phase}可以解释为压低训练样本的能量,\gls{negative_phase}可以解释为提高模型抽出的样本的能量,如\figref{fig:chap18_pos_and_neg_phase}所示。
% 599 mid

\section{随机\glsentrytext{maximum_likelihood}和\glsentrytext{contrastive_divergence}}
\label{sec:stochastic_maximum_likelihood_and_contrastive_divergence}
% 599 end
实现\eqnref{eq:18.15}的一个朴素方法是,每次需要计算梯度时,\gls{burn_in}随机初始化的一组\gls{markov_chain}。
当使用\gls{SGD}进行学习时,这意味着\gls{markov_chain}必须在每次梯度步骤中\gls{burn_in}。
这种方法引导下的训练过程如\algref{alg:naive_cd}所示。
内循环中\gls{burn_in}\gls{markov_chain}的计算代价过高,导致这个过程在实际中是不可行的,
但是这个过程是其他更加实际的近似算法的基础。
%不过该过程启发了其他计算代价较低的近似算法。

% -- 599 --
\begin{algorithm}[ht]
\caption{一种朴素的\,\glssymbol{mcmc}\,算法,使用梯度上升最大化具有难以计算\gls{partition_function}的对数似然。}
\label{alg:naive_cd}
\begin{algorithmic}
\STATE 设步长 $\epsilon$ 为一个小正数。
\STATE 设\gls{gibbs_steps} $k$ 大到足以允许\gls{burn_in}。在小图像集上训练一个~\glssymbol{RBM}~大致设为100。
\WHILE{不收敛}
\STATE 从训练集中采包含 $m$ 个样本$\{ \RVx^{(1)}, \dots, \RVx^{(m)}\}$的\gls{minibatch}。
\STATE $\RVg \leftarrow \frac{1}{m} \sum_{i=1}^m \nabla_{\Vtheta} \log \tilde{p}(\RVx^{(i)}; \Vtheta)$.
\STATE 初始化$m$ 个样本 $\{ \tilde{\RVx}^{(1)}, \dots, \tilde{\RVx}^{(m)} \}$ 为随机值(例如,从均匀或正态分布中采,或大致与模型边缘分布匹配的分布)。
\FOR{$i=1$ to $k$}
    \FOR{$j=1$ to $m$}
        \STATE $\tilde{\RVx}^{(j)} \leftarrow \text{gibbs\_update}(\tilde{\RVx}^{(j)}).$
    \ENDFOR
\ENDFOR
\STATE $\RVg \leftarrow \RVg - \frac{1}{m} \sum_{i=1}^m \nabla_{\Vtheta} \log \tilde{p}( \tilde{\RVx}^{(i)} ; \Vtheta ).$
\STATE $\Vtheta \leftarrow \Vtheta + \epsilon \RVg.$
\ENDWHILE
\end{algorithmic}
\end{algorithm}


我们可以将最大化似然的\,\glssymbol{mcmc}\,方法视为在两种力之间平衡,一种力拉高数据出现时的模型分布,一种拉低模型采样出现时的模型分布。%??   从而最大化似然     还是  将最大似然提前?
\figref{fig:chap18_pos_and_neg_phase}展示了这个过程。
这两种力分别对应最大化$\log \tilde{p}$和最小化$\log Z$。
对于\gls{negative_phase}会有一些近似方法。
这些近似都可以被理解为使\gls{negative_phase}更容易计算,但是也可能将其推向错误的位置。%??  push down  翻成 拉低/压低  ?推向 不对吧?  push 翻成 推 可以啊

\begin{figure}[!htb]
\ifOpenSource
\centerline{\includegraphics{figure.pdf}}
\else
\centerline{\includegraphics{Chapter18/figures/pos_and_neg_phase_color}}
\fi
\caption{\algref{alg:naive_cd}角度的``\gls{positive_phase}''和``\gls{negative_phase}''。
\emph{(左)}在\gls{positive_phase}中,我们从数据分布中采样,然后推高它们未归一化的概率。这意味着概率越高的数据点未归一化的概率被推高得越多。
\emph{(右)}在\gls{negative_phase}中,我们从模型分布中采样,然后压低它们未归一化的概率。
这与\gls{positive_phase}的倾向相反,给未归一化的概率处处添加了一个大常数。
当数据分布和模型分布相等时,\gls{positive_phase}推高数据点和\gls{negative_phase}压低数据点的机会相等。
此时,不再有任何的梯度(期望上说),训练也必须停止。}
\label{fig:chap18_pos_and_neg_phase}
\end{figure}


% -- 600 end
因为\gls{negative_phase}涉及到从模型分布中抽样,所以我们可以认为它在找模型信任度很高的点。
因为\gls{negative_phase}减少了这些点的概率,它们一般被认为代表了模型不正确的信念。
在文献中,它们经常被称为``幻觉''或``幻想粒子''。
事实上,\gls{negative_phase}已经被作为人类和其他动物做梦的一种可能解释~\citep{CrickMitchison83}。
这个想法是说,大脑维持着世界的概率模型,并且在醒着经历真实事件时会遵循$\log \tilde{p}$的梯度,在睡觉时会遵循$\log \tilde{p}$的负梯度最小化$\log Z$,其经历的样本采样自当前的模型。
这个视角解释了具有\gls{positive_phase}和\gls{negative_phase}的大多数算法,但是它还没有被神经科学实验证明是正确的。
在\gls{ML}模型中,通常有必要同时使用\gls{positive_phase}和\gls{negative_phase},而不是按不同时间阶段分为清醒和REM睡眠时期。
正如我们将在\secref{sec:learned_approximate_inference}中看到的,一些其他\gls{ML}算法出于其他原因从模型分布中采样,这些算法也能提供睡觉做梦的解释。
% 601 mid

这样理解学习\gls{positive_phase}和\gls{negative_phase}的作用之后,我们设计了一个比\algref{alg:naive_cd}计算代价更低的替代算法。
简单的\,\glssymbol{mcmc}\,算法的计算成本主要来自每一步的随机初始化\gls{burn_in}\gls{markov_chain}。
一个自然的解决方法是初始化\gls{markov_chain}为一个非常接近模型分布的分布,从而大大减少\gls{burn_in}步骤。

% -- 601 --
\begin{algorithm}[ht]
\caption{\gls{contrastive_divergence}算法,使用梯度上升作为优化过程。}
\label{alg:cd}
\begin{algorithmic}
\STATE 设步长 $\epsilon$ 为一个小正数。
\STATE 设\gls{gibbs_steps} $k$ 大到足以让从 $p_\text{data}$初始化并从 $p(\RVx; \Vtheta)$采样的\gls{markov_chain}\gls{mixing} 。在小图像集上训练一个~\glssymbol{RBM}~大致设为1-20。
\WHILE{不收敛}
\STATE 从训练集中采包含 $m$ 个样本 $\{ \RVx^{(1)}, \dots, \RVx^{(m)}\}$ 的\gls{minibatch}。
\STATE $\RVg \leftarrow \frac{1}{m} \sum_{i=1}^m \nabla_{\Vtheta} \log \tilde{p}(\RVx^{(i)}; \Vtheta).$
    \FOR{$i=1$ to $m$}
        \STATE $\tilde{\RVx}^{(i)} \leftarrow \RVx^{(i)}.$
    \ENDFOR
\FOR{$i=1$ to $k$}
    \FOR{$j=1$ to $m$}
        \STATE $\tilde{\RVx}^{(j)} \leftarrow \text{gibbs\_update}(\tilde{\RVx}^{(j)}).$
    \ENDFOR
\ENDFOR
\STATE $\RVg \leftarrow \RVg - \frac{1}{m} \sum_{i=1}^m \nabla_{\Vtheta} \log \tilde{p}( \tilde{\RVx}^{(i)} ; \Vtheta ) .$
\STATE $\Vtheta \leftarrow \Vtheta + \epsilon \RVg.$
\ENDWHILE
\end{algorithmic}
\end{algorithm}

% 601 end
\textbf{\gls{contrastive_divergence}}(\glssymbol{contrastive_divergence},或者是具有$k$个Gibbs步骤的~\glssymbol{contrastive_divergence}-$k$)算法在每个步骤中初始化\gls{markov_chain}为采样自数据分布中的样本~\citep{Hinton-PoE-2000,Hinton-RBMguide-small},如\algref{alg:cd}所示。
从数据分布中获取样本是计算代价最小的,因为它们已经在数据集中了。
初始时,数据分布并不接近模型分布,因此\gls{negative_phase}不是非常准确。
幸运的是,\gls{positive_phase}仍然可以准确地增加数据的模型概率。
进行\gls{positive_phase}阶段一段时间之后,模型分布会更接近于数据分布,并且\gls{negative_phase}开始变得准确。
% 602 mid

当然,\glssymbol{contrastive_divergence}\,仍然是真实\gls{negative_phase}的一个近似。
\glssymbol{contrastive_divergence}\,未能定性地实现真实\gls{negative_phase}的主要原因是,它不能抑制远离真实训练样本的高概率区域。
这些区域在模型上具有高概率,但是在数据生成区域上具有低概率,被称为\firstgls{spurious_modes}。
\figref{fig:chap18_spurious_mode}解释了这种现象发生的原因。
基本上,除非$k$非常大,模型分布中远离数据分布的\gls{mode}不会被使用训练数据初始化的\gls{markov_chain}访问到。
% -- 602 end



\begin{figure}[!htb]
\ifOpenSource
\centerline{\includegraphics{figure.pdf}}
\else
\centerline{\includegraphics{Chapter18/figures/spurious_mode_color}}
\fi
\caption{一个\gls{spurious_modes}。
说明\gls{contrastive_divergence}(\algref{alg:cd})的\gls{negative_phase}为何无法抑制\gls{spurious_modes}的例子。
一个\gls{spurious_modes}指的是一个在模型分布中出现数据分布中却不存在的模式。
由于\gls{contrastive_divergence}从数据点中初始化它的\gls{markov_chain}然后仅仅运行了几步\gls{markov_chain},不太可能到达模型中离数据点较远的模式。
这意味着从模型中采样时,我们有时候会得到一些与数据并不相似的样本。
这也意味着由于在这些模式上浪费了一些概率质量,模型很难把较高的概率质量集中于正确的模式上。
出于可视化的目的,这个图使用了某种程度上说更加简单的距离的概念——在$\SetR$的数轴上\gls{spurious_modes}与正确的模式有很大的距离。
这对应着基于局部移动$\SetR$上的单个变量$x$的\gls{markov_chain}。
对于大部分深度概率模型来说,\gls{markov_chain}是基于\,\gls{gibbs_sampling}的,并且对于单个变量产生非局部的移动但是无法同时移动所有的变量。
对于这些问题来说,考虑编辑距离比欧式距离通常更好。
然而,高维空间的编辑距离很难在二维空间作图展示。}
\label{fig:chap18_spurious_mode}
\end{figure}


\cite{Perpinan+Hinton-2005-small}实验上证明\,\glssymbol{contrastive_divergence}\,\gls{estimator}偏向于~\glssymbol{RBM}~和完全可见的\gls{BM},因为它会收敛到与\gls{MLE}不同的点。
他们认为,由于偏差较小,\glssymbol{contrastive_divergence}\,可以作为一种计算代价低的方式来初始化模型,之后可以通过计算代价高的\,\glssymbol{mcmc}\,方法进行\gls{fine_tune}。
\cite{Bengio+Delalleau-2009}表明,\glssymbol{contrastive_divergence}\,可以被理解为去掉了正确\,\glssymbol{mcmc}\,梯度更新中的最小项,这解释了偏差的由来。


在训练诸如\,\glssymbol{RBM}\,的浅层网络时\,\glssymbol{contrastive_divergence}\,是很有用的。
反过来,这些可以堆叠起来初始化更深的模型,如\,\glssymbol{DBN}\,或\,\glssymbol{DBM}。
但是\,\glssymbol{contrastive_divergence}\,并不直接有助于训练更深的模型。
这是因为在给定可见单元样本的情况下,很难获得\gls{hidden_unit}的样本。
由于\gls{hidden_unit}不包括在数据中,所以使用训练点初始化无法解决这个问题。
即使我们使用数据初始化可见单元,我们仍然需要\gls{burn_in}在给定这些可见单元的\gls{hidden_unit}条件分布上采样的\gls{markov_chain}。

% -- 603 end

\glssymbol{contrastive_divergence}\,算法可以被理解为惩罚某类模型,这类模型的\gls{markov_chain}会快速改变来自数据的输入。 %??   这句话总是读不通
这意味着使用\,\glssymbol{contrastive_divergence}\,训练从某种程度上说类似于训练\gls{AE}。
即使\,\glssymbol{contrastive_divergence}\,估计比一些其他训练方法具有更大偏差,但是它有助于\gls{pretraining}之后会堆叠起来的浅层模型。
这是因为堆栈中最早的模型会受激励复制更多的信息到其\gls{latent_variable},使其可用于随后的模型。
这应该更多地被认为是\,\glssymbol{contrastive_divergence}\,训练中经常可利用的副产品,而不是主要的设计优势。


\cite{sutskever2010convergence-small}表明,\glssymbol{contrastive_divergence}\,的更新方向不是任何函数的梯度。
这使得\,\glssymbol{contrastive_divergence}\,可能存在永久循环的情况,但在实践中这并不是一个严重的问题。


另一个解决\,\glssymbol{contrastive_divergence}\,中许多问题的不同策略是,在每个梯度步骤中初始化\gls{markov_chain}为先前梯度步骤的状态值。
这个方法首先被应用数学和统计学社群发现,命名为\textbf{\gls{SML}}(\glssymbol{SML})\citep{Younes98onthe},后来又在\gls{DL}社群中以名称\textbf{\gls{persistent_contrastive_divergence}}(\glssymbol{persistent_contrastive_divergence},或者每个更新中具有$k$个Gibbs步骤的\,\glssymbol{persistent_contrastive_divergence}-k)独立地被重新发现~\citep{Tieleman08-small}。
具体可以参考\algref{alg:sml}。
这种方法的基本思想是,只要随机梯度算法得到的步长很小,那么前一步骤的模型将类似于当前步骤的模型。
因此,来自先前模型分布的样本将非常接近来自当前模型分布的客观样本,用这些样本初始化的\gls{markov_chain}将不需要花费很多时间来完成\gls{mixing}。


因为每个\gls{markov_chain}在整个学习过程中不断更新,而不是在每个梯度步骤中重新开始,\gls{markov_chain}可以自由探索很远,以找到模型的所有\gls{mode}。 
因此,\glssymbol{SML}\,比\,\glssymbol{contrastive_divergence}\,更不容易形成具有\gls{spurious_modes}的模型。
此外,因为可以存储所有采样变量的状态,无论是可见的还是\gls{latent}的,\glssymbol{SML}\,为\gls{hidden_unit}和可见单元都提供了初始值。
\glssymbol{contrastive_divergence}\,只能为可见单元提供初始化,因此\gls{deep_model}需要进行\gls{burn_in}步骤。
\glssymbol{SML}\,能够高效地训练\gls{deep_model}。
\cite{Marlin10Inductive-small}将\,\glssymbol{SML}\,与本章中提出的许多其他标准方法进行比较。
他们发现,\glssymbol{SML}\,在\,\glssymbol{RBM}\,上得到了最佳的测试集对数似然,并且如果\,\glssymbol{RBM}\,的\gls{hidden_unit}被用作\,\glssymbol{SVM}\,分类器的特征,那么\,\glssymbol{SML}\,会得到最好的分类精度。
% -- 604 end

\begin{algorithm}[ht]
\caption{\gls{SML}/\gls{persistent_contrastive_divergence}算法,使用梯度上升作为优化过程。}
\label{alg:sml}
\begin{algorithmic}
\STATE 设步长 $\epsilon$ 为一个小正数。
\STATE 设\gls{gibbs_steps} $k$ 大到足以让从 $p(\RVx; \Vtheta + \epsilon \RVg)$ 采样的\gls{markov_chain}\gls{burn_in}(从采自$p(\RVx; \Vtheta)$的样本开始)。
在小图像集上训练一个\glssymbol{RBM}大致设为1,对于更复杂的模型如\gls{DBM}可能要设为5到50。
\STATE  初始化$m$ 个样本 $\{ \tilde{\RVx}^{(1)}, \dots, \tilde{\RVx}^{(m)} \}$ 为随机值(例如,从均匀或正态分布中采,或大致与模型边缘分布匹配的分布)。
\WHILE{不收敛}
\STATE 从训练集中采包含 $m$ 个样本 $\{ \RVx^{(1)}, \dots, \RVx^{(m)}\}$ 的\gls{minibatch}。
\STATE $\RVg \leftarrow \frac{1}{m} \sum_{i=1}^m \nabla_{\Vtheta} \log \tilde{p}(\RVx^{(i)}; \Vtheta).$
\FOR{$i=1$ to $k$}
    \FOR{$j=1$ to $m$}
        \STATE $\tilde{\RVx}^{(j)} \leftarrow \text{gibbs\_update}(\tilde{\RVx}^{(j)}).$
    \ENDFOR
\ENDFOR
\STATE $\RVg \leftarrow \RVg - \frac{1}{m} \sum_{i=1}^m \nabla_{\Vtheta} \log \tilde{p}( \tilde{\RVx}^{(i)} ; \Vtheta ).$
\STATE $\Vtheta \leftarrow \Vtheta + \epsilon \RVg.$
\ENDWHILE
\end{algorithmic}
\end{algorithm}

% 605 mid
在$k$太小或$\epsilon$太大时,随机梯度算法移动模型的速率比\gls{markov_chain}在迭代步中\gls{mixing}更快,此时\,\glssymbol{SML}\,容易变得不准确。
不幸的是,这些值的容许范围高度依赖于具体问题。
现在还没有方法能够正式地测试\gls{markov_chain}是否能够在迭代步骤之间成功\gls{mixing}。
主观地,如果对于Gibbs步骤数目而言\gls{learning_rate}太大的话,那么
梯度步骤中\gls{negative_phase}采样的方差会比不同\gls{markov_chain}中\gls{negative_phase}采样的方差更大。
例如,一个MNIST模型在一个步骤中只采样得到了$7$。
然后学习过程将会极大降低$7$对应的\gls{mode},在下一个步骤中,模型可能会只采样得到$9$。

% 605 end
从使用\,\glssymbol{SML}\,训练的模型中评估采样必须非常小心。
在模型训练完之后,有必要从一个随机起点初始化的新\gls{markov_chain}抽取样本。
用于训练的连续负相链中的样本受到了模型最近几个版本的影响,会使模型看起来具有比其实际更大的\gls{capacity}。

% -- 605 --


\cite{BerglundR13}进行了实验来检验由\,\glssymbol{contrastive_divergence}\,和\,\glssymbol{SML}\,进行梯度估计带来的偏差和方差。
结果证明\,\glssymbol{contrastive_divergence}\,比基于精确采样的\gls{estimator}具有更低的方差。
而\,\glssymbol{SML}\,有更高的方差。
\glssymbol{contrastive_divergence}\,方差低的原因是,其在\gls{positive_phase}和\gls{negative_phase}中使用了相同的训练点。
如果从不同的训练点来初始化\gls{negative_phase},那么方差会比基于精确采样的\gls{estimator}的方差更大。


所有基于\,\glssymbol{mcmc}\,从模型中抽取样本的方法在原则上几乎可以与\,\glssymbol{mcmc}\,的任何变体一起使用。
这意味着诸如\,\glssymbol{SML}\,这样的技术可以使用\chapref{chap:monte_carlo_methods}中描述的任何增强\,\glssymbol{mcmc}\,的技术(例如\gls{parallel_tempering})来加以改进~\citep{desjardins2010tempered,Cho10IJCNN-small}。
% 606 mid

一种在学习期间加速\gls{mixing}的方法是,不改变\gls{monte_carlo}采样技术,而是改变模型的参数化和\gls{cost_function}。
\firstgls{FPCD},或者\,\glssymbol{FPCD}~\citep{TielemanT2009-small}使用如下表达式去替换传统模型的参数$\Vtheta$

\begin{equation}
	\Vtheta = \Vtheta^{\text{(slow)}} + \Vtheta^{\text{(fast)}}.
\end{equation}
现在的参数是以前的两倍多,将其逐个相加以定义原始模型的参数。
快速复制参数可以使用更大的\gls{learning_rate}来训练,从而使其快速响应学习的\gls{negative_phase},并促使\gls{markov_chain}探索新的区域。
这能够使\gls{markov_chain}快速\gls{mixing},尽管这种效应只会发生在学习期间快速权重可以自由改变的时候。
通常,在短时间地将快速权重设为大值并保持足够长时间,使\gls{markov_chain}改变\gls{mode}之后,我们会对快速权重使用显著的\gls{weight_decay},促使它们收敛到较小的值。


本节介绍的基于\,\glssymbol{mcmc}\,的方法的一个关键优点是它们提供了$\log Z$梯度的估计,因此我们可以从本质上将问题分解为$\log \tilde{p}$和$\log Z$两块。
然后我们可以使用任何其他的方法来处理$\log \tilde{p}(\RVx)$,只需将我们的\gls{negative_phase}梯度加到其他方法的梯度中。
特别地,这意味着\gls{positive_phase}可以使用那些仅提供$\tilde{p}$下限的方法。
然而,本章介绍处理$\log Z$的大多数其他方法都和基于边界的\gls{positive_phase}方法是不兼容的。

% -- 606 end

\section{\glsentrytext{pseudolikelihood}}
\label{sec:pseudolikelihood}
\gls{monte_carlo}近似\gls{partition_function}及其梯度需要直接处理\gls{partition_function}。
有些其他方法通过训练不需要计算\gls{partition_function}的模型来绕开这个问题。
这些方法大多数都基于以下观察:无向概率模型中很容易计算概率的比率。
这是因为\gls{partition_function}同时出现在比率的分子和分母中,互相抵消:
\begin{equation}
	\frac{p(\RVx)}{ p(\RVy) } = \frac{ \frac{1}{Z} \tilde{p}(\RVx) }{ \frac{1}{Z} \tilde{p}(\RVy) } =
\frac{ \tilde{p}(\RVx) }{ \tilde{p}(\RVy) }.
\end{equation}


\gls{pseudolikelihood}正是基于\gls{conditional_probability}可以采用这种基于比率的形式,因此可以在没有\gls{partition_function}的情况下进行计算。
假设我们将$\RVx$分为$\RVa$,$\RVb$和$\RVc$,其中$\RVa$包含我们想要的条件分布的变量,$\RVb$包含我们想要条件化的变量,$\RVc$包含除此之外的变量:
\begin{equation}
	p(\RVa \mid \RVb) = \frac{ p(\RVa, \RVb) }{ p(\RVb) } = \frac{p(\RVa, \RVb)}{ \sum_{\RVa, \RVc} p(\RVa, \RVb, \RVc) } = \frac{ \tilde{p}(\RVa, \RVb) }{ \sum_{\RVa, \RVc} \tilde{p}(\RVa, \RVb, \RVc) }.
\end{equation}
以上计算需要边缘化$\RVa$,假设$\RVa$和$\RVc$包含的变量并不多,那么这将是非常高效的操作。
在极端情况下,$\RVa$可以是单个变量,$\RVc$可以为空,那么该计算仅需要估计与单个\gls{RV}值一样多的$\tilde{p}$。


不幸的是,为了计算对数似然,我们需要边缘化很多变量。
如果总共有$n$个变量,那么我们必须边缘化$n-1$个变量。
根据概率的\gls{chain_rule},我们有
\begin{equation}
	\log p(\RVx) = \log p(x_1) + \log p(x_2 \mid x_1) + \dots + \log p(x_n \mid \RVx_{1:n−1}).
\end{equation}
在这种情况下,我们已经使$\RVa$尽可能小,但是$\RVc$可以大到$\RVx_{2:n}$。
如果我们简单地将$\RVc$移到$\RVb$中以减少计算代价,那么会发生什么呢?
这便产生了\firstgls{pseudolikelihood}~\citep{Besag75pseudolikelihood}\gls{objective_function},给定所有其他特征$\Vx_{-i}$,预测特征$x_i$的值:
\begin{equation}
	\sum_{i=1}^n \log p(x_i \mid \Vx_{-i}).
\end{equation}


如果每个\gls{RV}有$k$个不同的值,那么计算$\tilde{p}$需要$k\times n$次估计,而计算\gls{partition_function}需要$k^n$次估计。

% -- 607 --

这看起来似乎是一个没有道理的策略,但可以证明最大化\gls{pseudolikelihood}的估计是渐近一致的~\citep{Mase1995}。
当然,在数据集不趋近于大采样极限的情况下,\gls{pseudolikelihood}可能表现出与\gls{MLE}不同的结果。
% 608 head

我们可以使用\firstgls{generalized_pseudolikelihood_estimator}来权衡计算复杂度和最大似然表现的偏差~\citep{Huang02}。
\gls{generalized_pseudolikelihood_estimator}使用$m$个不同的集合$\SetS^{(i)}$,$i=1, \dots, m$作为变量的指标出现在条件棒的左侧。
在$m = 1$和$\SetS^{(1)}= 1, \dots, n$的极端情况下,\gls{generalized_pseudolikelihood_estimator}会变为对数似然。
在$m = n$和$\SetS^{(i)} = \{i\}$的极端情况下,\gls{generalized_pseudolikelihood}会恢复为\gls{pseudolikelihood}。%??   recover the pseudolikeli
\gls{generalized_pseudolikelihood_estimator}\gls{objective_function}如下所示
\begin{equation}
	\sum_{i=1}^m \log p(\RVx_{\SetS^{(i)}} \mid \RVx_{- \SetS^{(i)} }).
\end{equation}


基于\gls{pseudolikelihood}的方法的性能在很大程度上取决于模型是如何使用的。
对于完全联合分布$p(\RVx)$模型的任务(例如密度估计和采样),\gls{pseudolikelihood}通常效果不好。
对于在训练期间只需要使用条件分布的任务而言,它的效果比最大似然更好,例如填充少量的缺失值。
如果数据具有规则结构,使得$\SetS$索引集可以被设计为表现最重要的相关性质,同时略去相关性可忽略的变量,那么\gls{generalized_pseudolikelihood}策略将会非常有效。
例如,在自然图像中,空间中相隔很远的像素也具有弱相关性,因此\gls{generalized_pseudolikelihood}可以应用于每个$\SetS$集是小的局部空间窗口的情况。
% 608 mid

\gls{pseudolikelihood}估计的一个弱点是它不能与仅在$\tilde{p}(\RVx)$上提供下界的其他近似一起使用,例如\chapref{chap:approximate_inference}中介绍的\gls{variational_inference}。这是因为$\tilde{p}$出现在了分母中。
分母的下界仅提供了整个表达式的上界,然而最大化上界没有什么意义。
这使得我们难以将\gls{pseudolikelihood}方法应用于诸如\gls{DBM}的\gls{deep_model},因为变分方法是近似边缘化互相作用的多层\gls{hidden_variable}的主要方法之一。
尽管如此,\gls{pseudolikelihood}仍然可以用在\gls{DL}中,它可以用于单层模型,或使用不基于下界的近似推断方法的\gls{deep_model}中。

% -- 608 end

\gls{pseudolikelihood}比\,\glssymbol{SML}\,在每个梯度步骤中的计算代价要大得多,这是由于其对所有条件进行显式计算。
但是,如果每个样本只计算一个随机选择的条件,那么\gls{generalized_pseudolikelihood}和类似标准仍然可以很好地运行,从而使计算代价降低到和\,\glssymbol{SML}\,差不多的程度~\citep{Goodfellow-et-al-NIPS2013}。


虽然\gls{pseudolikelihood}估计没有显式地最小化$\log Z$,但是我们仍然认为它具有类似\gls{negative_phase}的效果。
每个条件分布的分母会使得学习算法降低所有仅具有一个变量不同于训练样本的状态的概率。


读者可以参考~\cite{Marlin11-small}了解\gls{pseudolikelihood}渐近效率的理论分析,。

% 609 mid
\section{\glsentrytext{score_matching}和\glsentrytext{ratio_matching}}
\label{sec:score_matching_and_ratio_matching}
\gls{score_matching}~\citep{Hyvarinen-2005-small}提供了另一种训练模型而不需要估计$Z$或其导数的一致性方法。
对数密度关于参数的导数$\nabla_{\Vx} \log p(\Vx)$,被称为其\firstgls{score},\emph{\gls{score_matching}}这个名称正是来自这样的术语。
\gls{score_matching}采用的策略是,最小化模型对数密度和数据对数密度关于输入的导数之间的平方差期望:
\begin{align}
	L(\Vx, \Vtheta) & = \frac{1}{2} \norm{  \nabla_{\Vx} \log p_{\text{model}}(\Vx; \Vtheta) - \nabla_{\Vx} \log p_{\text{data}} (\Vx)  }_2^2, \\
%\end{equation}
%\begin{equation}
	J(\Vtheta) & = \frac{1}{2} \SetE_{p_{\text{data}}(\Vx)}  L(\Vx, \Vtheta), \\
%\end{equation}
%\begin{equation}
	\Vtheta^* & = \min_{\Vtheta} J(\Vtheta) .
\end{align}


该\gls{objective_function}避免了微分\gls{partition_function} $Z$带来的难题,因为$Z$不是$x$的函数,所以$\nabla_{\RVx} Z = 0$。
最初,\gls{score_matching}似乎有一个新的困难:计算数据分布的\gls{score}需要知道生成训练数据的真实分布$p_{\text{data}}$。
幸运的是,最小化$L(\Vx, \Vtheta)$的期望等价于最小化下式的期望
\begin{equation}
	\tilde{L} (\Vx, \Vtheta) = \sum_{j=1}^n \left( \frac{\partial^2}{ \partial x_j^2 } 
	\log p_{\text{model}} (\Vx; \Vtheta) + \frac{1}{2} \left( \frac{\partial}{ \partial x_j }
	\log p_{\text{model}} (\Vx; \Vtheta)
  \right)^2
\right),
\end{equation}
其中$n$是$\Vx$的维度。

% -- 609 end

因为\gls{score_matching}需要关于$\RVx$的导数,所以它不适用于具有离散数据的模型,但是模型中的\gls{latent_variable}可以是离散的。


类似于\gls{pseudolikelihood},\gls{score_matching}只有在我们能够直接估计$\log \tilde{p}(\RVx)$及其导数的时候才有效。
它与对$\log \tilde{p}(\RVx)$仅提供下界的方法不兼容,因为\gls{score_matching}需要$\log \tilde{p}(\RVx)$的导数和二阶导数,而下限不能传达关于导数的任何信息。
这意味着\gls{score_matching}不能应用于\gls{hidden_unit}之间具有复杂相互作用的模型估计,例如\gls{sparse_coding}模型或\gls{DBM}。
虽然\gls{score_matching}可以用于\gls{pretraining}较大模型的第一个\gls{hidden_layer},但是它没有被用于预训练较大模型的较深层网络。
这可能是因为这些模型的\gls{hidden_layer}通常包含一些离散变量。


虽然\gls{score_matching}没有明确显示具有\gls{negative_phase}信息,但是它可以被视为使用特定类型\gls{markov_chain}的\gls{contrastive_divergence}的变种~\citep{Hyvarinen-2007b}。
在这种情况下,\gls{markov_chain}并没有采用\,\gls{gibbs_sampling},而是采用一种由梯度引导局部更新的不同方法。
当局部更新的大小接近于零时,\gls{score_matching}等价于具有这种\gls{markov_chain}的\gls{contrastive_divergence}。
% 610 mid

\cite{Lyu09}将\gls{score_matching}推广到离散的情况(但是推导有误,后由~\cite{Marlin10Inductive-small}修正)。
\cite{Marlin10Inductive-small}发现,\textbf{\gls{GSM}}(\textbf{generalized score matching}, GSM)在许多样本观测概率为$0$的高维离散空间中不起作用。


一种更成功地将\gls{score_matching}的基本想法扩展到离散数据的方法是\firstgls{ratio_matching}~\citep{Hyvarinen-2007}。
\gls{ratio_matching}特别适用于二值数据。
\gls{ratio_matching}最小化以下\gls{objective_function}在样本上的均值:
\begin{equation}
	L^{(\text{RM})} (\Vx, \Vtheta) = \sum_{j=1}^n \left( 
	\frac{1}{ 1 + \frac{ p_{\text{model}}(\Vx; \Vtheta) }{ p_{\text{model}}(f(\Vx), j; \Vtheta) } } 
\right)^2.
\end{equation}
其中$f(\Vx, j)$返回$j$处位值取反的$\RVx$。
\gls{ratio_matching}使用了与\gls{pseudolikelihood}估计相同的策略来绕开\gls{partition_function}:\gls{partition_function}会在两个概率的比率中抵消掉。
\cite{Marlin10Inductive-small}发现,
训练模型给测试集图像\gls{denoising}时,\gls{ratio_matching}的效果要优于\,\glssymbol{SML}、\gls{pseudolikelihood}和\,\glssymbol{GSM}。


% -- 610 end

类似于\gls{pseudolikelihood}估计,\gls{ratio_matching}对每个数据点都需要$n$个$\tilde{p}$的估计,因此每次更新的计算代价大约比~\glssymbol{SML}~的计算代价高出$n$倍。


与\gls{pseudolikelihood}估计一样,我们可以认为\gls{ratio_matching}减小了所有只有一个变量不同于训练样本的状态的概率。
由于\gls{ratio_matching}特别适用于二值数据,这意味着在与数据的\gls{hamming_distance}为$1$内的所有状态上,\gls{ratio_matching}都是有效的。


\gls{ratio_matching}还可以作为处理高维稀疏数据(例如词计数向量)的基础。
这类稀疏数据对基于\,\glssymbol{mcmc}\,的方法提出了挑战,因为以密集格式表示数据是非常消耗计算资源的,而只有在模型学会表示数据分布的稀疏性之后,\glssymbol{mcmc}\,采样才会产生稀疏值。
\cite{Dauphin+Bengio-NIPS2013}设计了\gls{ratio_matching}的\gls{unbiased}随机近似来解决这个问题。
该近似只估计随机选择的目标子集,不需要模型生成完整的样本。


读者可以参考~\cite{Marlin11-small}了解\gls{ratio_matching}渐近效率的理论分析,。
% 611 mid

\section{\glsentrytext{denoising_score_matching}}
\label{sec:denoising_score_matching}
某些情况下,我们希望拟合以下分布来\gls{regularize}\gls{score_matching}
\begin{equation}
	p_{\text{smoothed}}(\Vx) = \int p_{\text{data}} (\Vy) q( \Vx \mid \Vy) d \Vy
\end{equation}
而不是拟合真实分布$p_{\text{data}}$。
分布$q(\Vx \mid \Vy)$是一个损坏过程,通常在形成$\Vx$的过程中会向$\Vy$中添加少量\gls{noise}。


\gls{denoising_score_matching}非常有用,因为在实践中,通常我们不能获取真实的$p_{\text{data}}$,而只能得到其样本确定的\gls{empirical_distribution}。
给定足够\gls{capacity},任何一致估计都会使$p_{\text{model}}$成为一组以训练点为中心的\,\gls{dirac_distribution}。
考虑在\secref{sec:consistency}介绍的渐近一致性上的\gls{loss},通过$q$来平滑有助于缓解这个问题。
\cite{Kingma+LeCun-2010}介绍了平滑分布$q$为正态分布\gls{noise}的\gls{regularize}\gls{score_matching}。


回顾\secref{sec:estimating_the_score},有一些\gls{AE}训练算法等价于\gls{score_matching}或\gls{denoising_score_matching}。
因此,这些\gls{AE}训练算法也是解决\gls{partition_function}问题的一种方式。

% -- 611 --

\section{\glsentrytext{NCE}}
\label{sec:noise_contrastive_estimation}
具有难求解的\gls{partition_function}的大多数模型估计都没有估计\gls{partition_function}。
\glssymbol{SML}\,和\,\glssymbol{contrastive_divergence}\,只估计对数\gls{partition_function}的梯度,而不是估计\gls{partition_function}本身。
\gls{score_matching}和\gls{pseudolikelihood}避免了和\gls{partition_function}相关的计算。 


\textbf{\gls{NCE}}\,(\textbf{noise-contrastive estimation},\glssymbol{NCE})\citep{Gutmann+Hyvarinen-2010}采取了一种不同的策略。
 在这种方法中,模型估计的\gls{PD}被明确表示为
\begin{equation}
	\log p_{\text{model}} (\RVx) = \log \tilde{p}_{\text{model}} (\RVx; \Vtheta) + c,
\end{equation}
其中$c$是$-\log Z(\Vtheta)$的近似。
\gls{NCE}过程将$c$视为另一参数,使用相同的算法同时估计$\Vtheta$和$c$,而不是仅仅估计$\Vtheta$,。
因此,所得到的$\log p_{\text{model}}(\RVx)$可能并不完全对应有效的\gls{PD},但随着$c$估计的改进,它将变得越来越接近有效值\footnote{\glssymbol{NCE}~也适用于具有易于处理的,不需要引入额外参数$c$的\gls{partition_function}的问题。它已经是最令人感兴趣的,估计具有复杂\gls{partition_function}模型的方法。}。


这种方法不可能使用最大似然作为估计的标准。
最大似然标准可以设置$c$为任意大的值,而不是设置$c$以创建一个有效的\gls{PD}。


\glssymbol{NCE}~将估计$p(\RVx)$的\gls{unsupervised_learning}问题转化为学习一个概率二元分类器,其中一个类别对应模型生成的数据。
该\gls{supervised_learning}问题中的\gls{MLE}定义了原始问题的渐近一致估计。


具体地说,我们引入第二个分布,\firstgls{noise_distribution} $p_{\text{noise}}(\RVx)$。
\gls{noise_distribution}应该易于估计和从中采样。
我们现在可以构造一个联合$\RVx$和新二值变量$y$的模型。
在新的联合模型中,我们指定
\begin{equation}
	p_{\text{joint}} (y = 1) = \frac{1}{2},
\end{equation}
\begin{equation}
	p_{\text{joint}} (\RVx \mid y = 1) = p_{\text{model}} (\RVx),
\end{equation}
和
\begin{equation}
	p_{\text{joint}} (\RVx \mid y = 0) = p_{\text{noise}}(\RVx).
\end{equation}
换言之,$y$是一个决定我们从模型还是从\gls{noise_distribution}中生成$\RVx$的开关变量。

% -- 612 --

我们可以在训练数据上构造一个类似的联合模型。
在这种情况下,开关变量决定是从\,\textbf{数据}\,还是从\gls{noise}分布中抽取$\RVx$。
正式地,$p_{\text{train}}(y = 1) = \frac{1}{2}$,$p_{\text{train}}( \RVx \mid y = 1) = p_{\text{data}}( \RVx)$,和
$p_{\text{train}}(\RVx \mid y = 0) = p_{\text{noise}}(\RVx)$。


现在我们可以应用标准的最大似然学习拟合$p_{\text{joint}}$到$p_{\text{train}}$的\,\textbf{监督}\,学习问题:
\begin{equation}
	\Vtheta, c = \underset{ \Vtheta, c}{\arg\max} \SetE_{\RVx, \RSy \sim p_{\text{train}}} \log 
	p_{\text{joint}} (y \mid \RVx).
\end{equation}


分布$p_{\text{joint}}$本质上是将\gls{logistic_regression}模型应用于模型和\gls{noise_distribution}之间的对数概率之差:
\begin{equation}
	p_{\text{joint}} (y = 1 \mid \RVx) = \frac{ p_{\text{model}}(\RVx) }{ p_{\text{model}}(\RVx) + p_{\text{noise}}(\RVx) }
\end{equation}
\begin{equation}
 = \frac{1}{1 + \frac{ p_{\text{noise}}(\RVx) }{ p_{\text{model}}(\RVx) }}
\end{equation}
\begin{equation}
= \frac{1}{1 + \exp \left( \log \frac{ p_{\text{noise}}(\RVx) }{ p_{\text{model}}(\RVx) } \right)}
\end{equation}
\begin{equation}
	= \sigma \left( - \log \frac{ p_{\text{noise}} (\RVx) }{ p_{\text{model}} (\RVx) } \right)
\end{equation}
\begin{equation}
	= \sigma ( \log p_{\text{model}} (\RVx) - \log p_{\text{noise}} (\RVx)  ) .
\end{equation}


因此,只要$\log \tilde{p}_{\text{model}}$易于\gls{back_propagation},
并且如上所述,$p_{\text{noise}}$应易于估计(以便评估$p_{\text{joint}}$)和采样(以生成训练数据),那么\,\glssymbol{NCE}\,就易于使用。


\glssymbol{NCE}\,能够非常成功地应用于\gls{RV}较少的问题,但即使\gls{RV}有很多可以取的值时,它也很有效。
例如,它已经成功地应用于给定单词上下文建模单词的条件分布~\citep{Mnih2013}。
虽然单词可以采样自一个很大的词汇表,但是只能采样一个单词。

% -- 613 --

当\,\glssymbol{NCE}\,应用于具有许多\gls{RV}的问题时,其效率会变得较低。
当\gls{logistic_regression}分类器发现某个变量的取值不大可能时,它会拒绝这个\gls{noise}样本。
这意味着在$p_{\text{model}}$学习了基本的边缘统计之后,学习进程会大大减慢。
想象一个使用非结构化高斯\gls{noise}作为$p_{\text{noise}}$来学习面部图像的模型。
如果$p_{\text{model}}$学会了眼睛,就算没有学习任何其他面部特征,比如嘴, 它也会拒绝几乎所有的非结构化\gls{noise}样本。
% 614 head

\gls{noise_distribution}~$p_{\text{noise}}$必须是易于估计和采样的约束可能是过于严格的限制。
当$p_{\text{noise}}$比较简单时,大多数采样可能与数据有着明显不同,而不会迫使$p_{\text{model}}$进行显著改进。


类似于\gls{score_matching}和\gls{pseudolikelihood},如果$\tilde{p}$只有下界,那么\,\glssymbol{NCE}\,不会有效。
这样的下界能够用于构建$p_{\text{joint}}( y = 1 \mid \RVx)$的下界,但是它只能用于构建$p_{\text{joint}}(y = 0 \mid \RVx)$(出现在一半的\,\glssymbol{NCE}\,对象中)的上界。
同样地,$p_{\text{noise}}$的下界也没有用,因为它只提供了$p_{\text{joint}}( y = 1 \mid \RVx)$的上界。


在每个梯度步骤之前,模型分布被复制来定义新的\gls{noise_distribution}时,\glssymbol{NCE}\,定义了一个被称为\firstgls{self_contrastive_estimation}的过程,其梯度期望等价于最大似然的梯度期望~\citep{Goodfellow-ICLR2015}。
特殊情况的\,\glssymbol{NCE}(\gls{noise}采样由模型生成)表明最大似然可以被解释为使模型不断学习以将现实与自身发展的信念区分的过程,而\gls{NCE}通过让模型区分现实和固定的基准(\gls{noise}模型),我们降低了计算成本。%??  以  是不是多余的?


在训练样本和生成样本(使用模型能量函数定义分类器)之间进行分类以得到模型的梯度的方法,已经在更早的时候以各种形式提出来~\citep{Welling2003b,Bengio-2009-book}。


\gls{NCE}是基于良好\gls{generative_model}应该能够区分数据和\gls{noise}的想法。
一个密切相关的想法是,良好的\gls{generative_model}能够生成分类器无法将其与数据区分的样本。
这个想法诞生了\gls{generative_adversarial_networks}(\secref{sec:generative_adversarial_networks})。


\section{估计\glsentrytext{partition_function}}
\label{sec:estimating_the_partition_function}
尽管本章中的大部分内容都在避免计算与\gls{undirected_graphical_model}相关的难以计算的\gls{partition_function} $Z(\Vtheta)$,但在本节中我们将会讨论几种直接估计\gls{partition_function}的方法。

% -- 614 --

估计\gls{partition_function}可能会很重要,当我们希望计算数据的归一化似然时,我们会需要它。
在\emph{评估}模型,监控训练性能,和比较模型时,这通常是很重要的。


例如,假设我们有两个模型:\gls{PD}为$p_A(\RVx; \Vtheta_A)= \frac{1}{Z_A} \tilde{p}_A(\RVx; \Vtheta_A)$的模型$\CalM_A$和\gls{PD}为$p_B(\RVx; \Vtheta_B)= \frac{1}{Z_B} \tilde{p}_B(\RVx; \Vtheta_B)$的模型$\CalM_B$。
比较模型的常用方法是评估和比较两个模型分配给\gls{iid_chap18}测试数据集的似然。
假设测试集含$m$个样本$\{ \Vx^{(1)}, \dots, \Vx^{(m)} \}$。
如果 $\prod_i p_A ( \RSx^{(i)}; \Vtheta_A) > \prod_i p_B( \RSx^{(i)}; \Vtheta_B)$,或等价地,如果
\begin{equation}
\label{eq:18.38}
	\sum_i \log p_A (\RSx^{(i)}; \Vtheta_A) - \sum_i \log p_B(\RSx^{(i)}; \Vtheta_B) > 0,
\end{equation}
那么我们说$\CalM_A$是一个比$\CalM_B$更好的模型(或者,至少可以说,它在测试集上是一个更好的模型),这是指它有一个更好的测试对数似然。
不幸的是,测试这个条件是否成立需要知道\gls{partition_function}。
\eqnref{eq:18.38}看起来需要估计模型分配给每个点的对数概率,因而需要估计\gls{partition_function}。
我们可以通过将\eqnref{eq:18.38}重新转化为另一种形式来简化情况,在该形式中我们只需要知道两个模型的\gls{partition_function}的\textbf{比率}:
\begin{equation}
\label{eq:18.39}
	\sum_i \log p_A(\RVx^{(i)}; \Vtheta_A) - \sum_i \log p_B(\RVx^{(i)}; \Vtheta_B) =
	\sum_i \left(  \log \frac{ \tilde{p}_A(\RVx^{(i)}; \Vtheta_A) }{ \tilde{p}_B(\RVx^{(i)}; \Vtheta_B) } \right)  - m\log \frac{Z(\Vtheta_A)}{ Z(\Vtheta_B) }.
\end{equation}
因此,我们可以在不知道任一模型的\gls{partition_function},而只知道它们比率的情况下,判断模型$\CalM_A$是否比模型$\CalM_B$更优。
正如我们将很快看到的,在两个模型相似的情况下,我们可以使用\gls{importance_sampling}来估计比率。


然而,如果我们想要计算测试数据在$\CalM_A$或$\CalM_B$上的真实概率,我们需要计算\gls{partition_function}的真实值。
如果我们知道两个\gls{partition_function}的比率,$r = \frac{Z(\Vtheta_B)}{ Z(\Vtheta_A) }$,并且我们知道两者中一个的实际值,比如说$Z(\Vtheta_A)$,那么我们可以计算另一个的值:
\begin{equation}
	Z(\Vtheta_B) = r Z(\Vtheta_A) = \frac{ Z(\Vtheta_B) }{ Z(\Vtheta_A) } Z(\Vtheta_A).
\end{equation}


一种估计\gls{partition_function}的简单方法是使用\gls{monte_carlo}方法,例如简单\gls{importance_sampling}。
以下用连续变量积分来表示该方法,也可以替换积分为求和,很容易将其应用到离散变量的情况。
我们使用\gls{proposal_distribution} $p_0(\RVx) = \frac{1}{Z_0} \tilde{p}_0( \RVx)$,其在\gls{partition_function} $Z_0$和未归一化分布$\tilde{p}_0(\RVx)$上易于采样和估计。

% -- 615 --

\begin{align}
	Z_1 &= \int \tilde{p}_1 (\RVx) d \RVx \\
	&= \int  \frac{ p_0(\RVx) }{ p_0(\RVx) }   \tilde{p}_1 (\RVx) d \RVx \\
	&= Z_0 \int  p_0(\RVx)   \frac{ \tilde{p}_1 (\RVx) }{ \tilde{p}_0 (\RVx) } d \RVx \\
	\hat{Z}_1 &= \frac{Z_0}{K} \sum_{k=1}^K \frac{ \tilde{p}_1(\RVx^{(k)})  }{ \tilde{p}_0(\RVx^{(k)}) }  \quad\quad \text{s.t.}: \RVx^{(k)} \sim p_0 \label{eq:18.44}
\end{align}

在最后一行,我们使用\gls{monte_carlo}估计,使用从$p_0(\RVx)$中抽取的采样计算积分$\hat{Z}_1$,然后用未归一化的$\tilde{p}_1$和\gls{proposal_distribution} $p_0$的比率对每个采样加权。


这种方法使得我们可以估计\gls{partition_function}之间的比率:
\begin{equation}
	\frac{1}{K} \sum_{k=1}^K \frac{ \tilde{p}_1 (\RVx^{(k)}) }{ \tilde{p}_0 (\RVx^{(k)}) }
	\quad\quad \text{s.t.}: \RVx^{(k)} \sim p_0.
\end{equation}
然后该值可以直接比较\eqnref{eq:18.39}中的两个模型。


如果分布$p_0$接近$p_1$,那么\eqnref{eq:18.44}能够有效地估计\gls{partition_function}~\citep{Minka_2005}。
不幸的是,大多数时候$p_1$都很复杂(通常是\gls{multimodal}的),并且定义在高维空间中。
很难找到一个易求解的$p_0$,既能易于评估,又能充分接近$p_1$以保持高质量的近似。
如果$p_0$和$p_1$不接近,那么$p_0$的大多数采样将在$p_1$中具有较低的概率,从而在\eqnref{eq:18.44}的求和中产生(相对的)可忽略的贡献。


如果求和中只有少数几个具有显著权重的样本,那么将会由于高方差而导致估计的效果很差。
这可以通过估计$\hat{Z}_1$的方差来定量地理解:
\begin{equation}
	\hat{\text{Var}} \left( \hat{Z}_1 \right)  = \frac{Z_0 }{K^2} \sum_{k=1}^K
\left(  \frac{ \tilde{p}_1(\RVx^{(k)}) }{  \tilde{p}_0(\RVx^{(k)}) } - \hat{Z}_1  \right)^2.
\end{equation}
当重要性权重$\frac{ \tilde{p}_1(\RVx^{(k)}) }{ \tilde{p}_0(\RVx^{(k)}) } $存在显著偏差时,上式的值是最大的。

% -- 616 --

我们现在关注两个解决高维空间复杂分布上估计\gls{partition_function}的方法:
\gls{AIS}和\gls{bridge_sampling}。
两者都始于上面介绍的简单\gls{importance_sampling}方法,并且都试图通过引入\emph{缩小}$p_0$和$p_1$之间\emph{差距}的中间分布,来解决$p_0$远离$p_1$的问题。


\subsection{\glsentrytext{AIS}}
\label{sec:annealed_importance_sampling}
在$D_{KL}(p_0 \| p_1)$很大的情况下(即$p_0$和$p_1$之间几乎没有重叠),一种称为\textbf{\gls{AIS}}(\textbf{annealed importance sampling},AIS)的方法试图通过引入中间分布来缩小这种差距~\citep{Jarzynski1997,Neal-2001}。
考虑分布序列$p_{\eta_0},\dots,p_{\eta_n}$,其中$0=\eta_0 < \eta_1 < \dots < \eta_{n-1} < \eta_n = 1$,分布序列中的第一个和最后一个分别是$p_0$和$p_1$。


这种方法使我们能够估计定义在高维空间多峰分布(例如训练~\glssymbol{RBM}~时定义的分布)上的\gls{partition_function}。
我们从一个已知\gls{partition_function}的简单模型(例如,权重为零的\,\glssymbol{RBM})开始,估计两个模型\gls{partition_function}之间的比率。
该比率的估计基于许多个相似分布的比率估计,例如在零和学习到的权重之间插值一组权重不同的\,\glssymbol{RBM}。


现在我们可以将比率$\frac{Z_1}{Z_0}$写作
\begin{align}
\frac{Z_1}{Z_0} &= \frac{Z_1}{Z_0} \frac{Z_{\eta_1}}{Z_{\eta_1}} \dots \frac{Z_{\eta_{n-1}}}{Z_{\eta_{n-1}}} \\
&= \frac{Z_{\eta_1}}{Z_{0}}  \frac{Z_{\eta_2}}{Z_{\eta_1}}  \dots \frac{Z_{\eta_{n-1}}}{Z_{\eta_{n-2}}} \frac{Z_{1}}{Z_{\eta_{n-1}}} \\
&= \prod_{j=0}^{n-1} \frac{ Z_{\eta_{j+1}} }{Z_{\eta_j}}. \label{eq:18.49}
\end{align}
如果对于所有的$0 \leq j \leq n-1$,分布$p_{\eta_j}$和$p_{\eta_{j+1}}$足够接近,那么我们能够使用简单的\gls{importance_sampling}来估计每个因子$\frac{Z_{\eta_{j+1}}}{ Z_{\eta_j}}$,然后使用这些得到$\frac{Z_1}{Z_0}$的估计。

% -- 617 --

这些中间分布是从哪里来的呢?
正如最先的\gls{proposal_distribution} $p_0$是一种设计选择,分布序列$p_{\eta_1} \dots p_{\eta_{n-1}}$也是如此。
也就是说,它们可以被特别设计为特定的问题领域。
中间分布的一个通用和流行选择是使用目标分布$p_1$的加权几何平均,起始分布(其\gls{partition_function}是已知的)为$p_0$:
\begin{equation}
	p_{\eta_j} \propto p_1^{\eta_j} p_0^{1-\eta_j}.
\end{equation}


为了从这些中间分布中采样,我们定义了一组\gls{markov_chain}转移函数$T_{\eta_j}(\Vx' \mid \Vx)$,定义了给定$\Vx$转移到$\Vx'$的\gls{conditional_probability_distribution}。
转移算子$T_{\eta_j}(\Vx' \mid \Vx)$定义如下,保持$p_{\eta_j}(\Vx)$不变:
\begin{equation}
	p_{\eta_j}(\Vx) = \int p_{\eta_j} (\Vx') T_{\eta_j} (\Vx \mid \Vx') d\Vx'.
\end{equation}
这些转移可以被构造为任何\gls{mcmc}方法(例如,Metropolis-Hastings,Gibbs),包括涉及多次遍历所有\gls{RV}或其他迭代的方法。


然后,\glssymbol{AIS}\,采样方法从$p_0$开始生成样本,并使用转移算子从中间分布顺序地生成采样,直到我们得到目标分布$p_1$的采样:
\begin{itemize}
	\item 对于 $k=1 \dots K$ 
		\newline
		\quad\quad -- 采样 $ \Vx_{\eta_1}^{(k)} \sim p_0(\RVx) $
		\newline
		\quad\quad -- 采样 $ \Vx_{\eta_2}^{(k)} \sim T_{\eta_1}(\RVx_{\eta_2}^{(k)} \mid \Vx_{\eta_1}^{(k)} ) $
		\newline
		\quad\quad -- $\dots$
		\newline
		\quad\quad -- 采样 $ \Vx_{\eta_{n-1}}^{(k)} \sim T_{\eta_{n-2}}(\RVx_{\eta_{n-1}}^{(k)} \mid \Vx_{\eta_{n-2}}^{(k)} ) $
		\newline
		\quad\quad -- 采样 $ \Vx_{\eta_n}^{(k)} \sim T_{\eta_{n-1}}(\RVx_{\eta_n}^{(k)} \mid \Vx_{\eta_{n-1}}^{(k)} ) $
	\item 结束
\end{itemize}


对于采样$k$,通过连接\eqnref{eq:18.49}给出的中间分布之间的重要性权重,我们可以导出目标重要性权重:
\begin{equation}
\label{eq:18.52}
	w^{(k)} = \frac{ \tilde{p}_{\eta_1} ( \Vx_{\eta_1}^{(k)} )  }{  \tilde{p}_{0} ( \Vx_{\eta_1}^{(k)} )  }
\frac{ \tilde{p}_{\eta_2} ( \Vx_{\eta_2}^{(k)} )  }{  \tilde{p}_{\eta_1} ( \Vx_{\eta_2}^{(k)} )  }
\dots
\frac{ \tilde{p}_{1} ( \Vx_{1}^{(k)} )  }{  \tilde{p}_{\eta_{n-1}} ( \Vx_{\eta_n}^{(k)} )  } .
\end{equation}
为了避免诸如\gls{overflow}的数值问题,最佳方法可能是通过加法或减法计算$\log w^{(k)}$,而不是通过概率乘法和除法计算$w^{(k)}$。

% -- 618 --

利用由此定义的采样过程和\eqnref{eq:18.52}中给出的重要性权重,\gls{partition_function}的比率估计如下所示:
\begin{equation}
	\frac{Z_1}{Z_0} \approx \frac{1}{K} \sum_{k=1}^K w^{(k)}
\end{equation}


为了验证该过程定义的\gls{importance_sampling}方案是否有效,我们可以展示\citep{Neal-2001}~\glssymbol{AIS}~过程对应着扩展状态空间上的简单\gls{importance_sampling},其中数据点采样自乘积空间$[\Vx_{\eta_1},\dots,\Vx_{\eta_{n-1}},\Vx_1]$。
为此,我们将扩展空间上的分布定义为
\begin{align}
&\tilde{p} (\Vx_{\eta_1}, \dots, \Vx_{\eta_{n-1}}, \Vx_1) \\
= &\tilde{p}_1 (\Vx_1) \tilde{T}_{\eta_{n-1}} (\Vx_{\eta_{n-1}} \mid \Vx_1)
 	\tilde{T}_{\eta_{n-2}}  (\Vx_{\eta_{n-2}} \mid \Vx_{\eta_{n-1}}) \dots
 	\tilde{T}_{\eta_1} (\Vx_{\eta_1} \mid \Vx_{\eta_2}) \label{eq:18.55} ,
\end{align}
其中$\tilde{T}_a$是由$T_a$定义的转移算子的逆(应用\gls{bayes_rule}):
\begin{equation}
	\tilde{T}_a (\Vx' \mid \Vx) = \frac{p_a(\Vx')}{ p_a(\Vx) } T_a(\Vx \mid \Vx') = 
\frac{  \tilde{p}_a(\Vx')}{ \tilde{p}_a(\Vx) } T_a(\Vx \mid \Vx') .
\end{equation}
将以上代入到\eqnref{eq:18.55}给出的扩展状态空间上的联合分布中,我们得到:
\begin{align}
	&\tilde{p} (\Vx_{\eta_1}, \dots, \Vx_{\eta_{n-1}}, \Vx_1) \\
	= &\tilde{p}_1 (\Vx_1) \frac{ \tilde{p}_{\eta_{n-1}} (\Vx_{\eta_{n-1}})  }{ \tilde{p}_{\eta_{n-1}}(\Vx_1)} T_{\eta_{n-1}} (\Vx_1 \mid \Vx_{\eta_{n-1}})
\prod_{i=1}^{n-2} \frac{ \tilde{p}_{\eta_i}(\Vx_{\eta_i}) }{ \tilde{p}_{\eta_i}(\Vx_{\eta_{i+1}})} T_{\eta_i} (\Vx_{\eta_{i+1}} \mid \Vx_{\eta_i}) \\
	= &\frac{ \tilde{p}_1(\Vx_1) }{ \tilde{p}_{\eta_{n-1}}(\Vx_1) } T_{\eta_{n-1}} (\Vx_1 \mid \Vx_{\eta_{n-1}})
\tilde{p}_{\eta_1} (\Vx_{\eta_1}) \prod_{i=1}^{n-2} \frac{ \tilde{p}_{\eta_{i+1}}(\Vx_{\eta_{i+1}}) }{ \tilde{p}_{\eta_i}(\Vx_{\eta_{i+1}}) } T_{\eta_i} (\Vx_{\eta_{i+1}} \mid \Vx_{\eta_i}) . \label{eq:18.59}
\end{align}
通过上面给定的采样方案,现在我们可以从扩展样本上的联合\gls{proposal_distribution} $q$上生成采样,联合分布如下
\begin{equation}
	q(\Vx_{\eta_1}, \dots, \Vx_{\eta_{n-1}}, \Vx_1)  = p_0(\Vx_{\eta_1}) T_{\eta_1}( \Vx_{\eta_2} \mid \Vx_{\eta_1} ) \dots T_{\eta_{n-1}} (\Vx_1 \mid \Vx_{\eta_{n-1}}) .
\end{equation}
\eqnref{eq:18.59}给出了扩展空间上的联合分布。
将$q(\Vx_{\eta_1}, \dots, \Vx_{\eta_{n-1}}, \Vx_1)$作为扩展状态空间上的\gls{proposal_distribution}(我们会从中抽样),重要性权重如下
\begin{equation}
	w^{(k)} = \frac{ \tilde{p}(\Vx_{\eta_1}, \dots, \Vx_{\eta_{n-1}}, \Vx_1) }{ q( \Vx_{\eta_1}, \dots, \Vx_{\eta_{n-1}}, \Vx_1 ) } =
\frac{ \tilde{p}_1(\Vx_1^{(k)}) }{ \tilde{p}_{\eta_{n-1}}(\Vx_{\eta_{n-1}}^{(k)}) } \dots
\frac{ \tilde{p}_{\eta_{2}}(\Vx_{\eta_{2}}^{(k)}) }{ \tilde{p}_{\eta_1}(\Vx_{\eta_{1}}^{(k)}) } 
\frac{ \tilde{p}_{\eta_{1}}(\Vx_{\eta_{1}}^{(k)}) }{ \tilde{p}_{0}(\Vx_{0}^{(k)}) } .
\end{equation}
这些权重和\,\glssymbol{AIS}\,上的权重相同。
因此,我们可以将\,\glssymbol{AIS}\,解释为应用于扩展状态上的简单\gls{importance_sampling},其有效性直接来源于\gls{importance_sampling}的有效性。

% -- 619 --

\gls{AIS}首先由~\cite{Jarzynski1997}发现,然后由~\cite{Neal-2001}再次独立发现。
目前它是估计无向概率模型的\gls{partition_function}的最常用方法。
其原因可能与一篇有影响力的论文~\citep{Salakhutdinov+Murray-2008}有关,该论文并没有讨论该方法相对于其他方法的优点,而是介绍了将其应用于估计\gls{RBM}和\gls{DBN}的\gls{partition_function}。


关于\,\glssymbol{AIS}\,估计性质(例如,方差和效率)的讨论,请参看~\cite{Neal-2001}。


\subsection{\glsentrytext{bridge_sampling}}
\label{sec:bridge_sampling}
类似于\,\glssymbol{AIS},\gls{bridge_sampling}~\citep{Bennet76}是另一种处理\gls{importance_sampling}缺点的方法。
并非将一系列中间分布连接在一起,\gls{bridge_sampling}依赖于单个分布$p_*$(被称为桥),在已知\gls{partition_function}的分布$p_0$和分布$p_1$(我们试图估计其\gls{partition_function} $Z_1$)之间插值。


\gls{bridge_sampling}估计比率$Z_1 / Z_0$:$\tilde{p}_0$和$\tilde{p}_*$之间重要性权重期望与$\tilde{p}_1$和$\tilde{p}_*$之间重要性权重的比率,
\begin{equation}
	\frac{Z_1}{ Z_0} \approx \sum_{k=1}^K \frac{ \tilde{p}_*(\Vx_0^{(k)}) }{ \tilde{p}_0(\Vx_0^{(k)}) } \bigg/ \sum_{k=1}^K \frac{ \tilde{p}_*(\Vx_1^{(k)}) }{ \tilde{p}_1(\Vx_1^{(k)}) } .
\end{equation}
如果仔细选择\gls{bridge_sampling} $p_*$,使其与$p_0$和$p_1$都有很大重合的话,那么\gls{bridge_sampling}能够允许两个分布(或更正式地,$D_{\text{KL}}(p_0 \| p_1)$)之间有较大差距(相对标准\gls{importance_sampling}而言)。

% -- 620 --

可以表明,最优的\gls{bridge_sampling}是$p_*^{(opt)} (\RVx) \propto \frac{ \tilde{p}_0(\Vx) \tilde{p}_1(\Vx) }{ r\tilde{p}_0(\Vx) + \tilde{p}_1(\Vx) }$,其中$r = Z_1 / Z_0$。
这似乎是一个不可行的解决方案,因为它似乎需要我们估计数值$Z_1 / Z_0$。
然而,可以从粗糙的$r$开始估计,然后使用得到的\gls{bridge_sampling}逐步迭代以改进估计~\citep{Neal05estimatingratios}。
也就是说,我们会迭代地重新估计比率,并使用每次迭代更新$r$的值。


\paragraph{\gls{linked_importance_sampling}}
\glssymbol{AIS}\,和\gls{bridge_sampling}各有优点。
如果$D_{\text{KL}}(p_0 \| p_1)$不太大(由于$p_0$和$p_1$足够接近)的话,那么\gls{bridge_sampling}能比\,\glssymbol{AIS}\,更高效地估计\gls{partition_function}比率。
然而,如果对于单个分布$p_*$而言,两个分布相距太远难以桥接差距,那么\,\glssymbol{AIS}\,至少可以使用许多潜在中间分布来跨越$p_0$和$p_1$之间的差距。
\cite{Neal05estimatingratios}展示\gls{linked_importance_sampling}方法如何利用\gls{bridge_sampling}的优点,桥接\,\glssymbol{AIS}\,中使用的中间分布,并且显著改进了整个\gls{partition_function}的估计。


\paragraph{在训练期间估计\gls{partition_function}}
虽然\,\glssymbol{AIS}\,已经被认为是用于估计许多无向模型\gls{partition_function}的标准方法,但是它在计算上代价很高,以致其在训练期间仍然不很实用。
研究者探索了一些在训练过程中估计\gls{partition_function}的替代方法。


使用\gls{bridge_sampling}、短链\,\glssymbol{AIS}\,和\gls{parallel_tempering}的组合,\cite{Desjardins+al-NIPS2011}设计了一种在训练过程中追踪\,\glssymbol{RBM}\,\gls{partition_function}的方法。
该策略的基础是,在\gls{parallel_tempering}方法操作的每个温度下,\glssymbol{RBM}\,\gls{partition_function}的独立估计会一直保持。
作者将相邻链(来自\gls{parallel_tempering})的\gls{partition_function}比率的\gls{bridge_sampling}估计和跨越时间的\,\glssymbol{AIS}\,估计组合起来,提出一个在每次迭代学习时估计\gls{partition_function}的(且方差较小的)方法。


本章中描述的工具提供了许多不同的方法,以解决难处理的\gls{partition_function}问题,但是在训练和使用\gls{generative_model}时,可能会存在一些其他问题。
其中最重要的是我们接下来会遇到的难以推断的问题。

% -- 621 --

% !Mode:: "TeX:UTF-8"
% Translator: Tianfan Fu 
\chapter{\glsentrytext{approximate_inference}}
\label{chap:approximate_inference}
% 623


许多概率模型很难训练的原因是很难进行推断。
在\gls{DL}中,通常我们有一系列可见变量$\Vv$和一系列\gls{latent_variable} $\Vh$。
推断困难通常是指难以计算$p(\Vh\mid\Vv)$或其期望。
而这样的操作在一些诸如最大似然学习的任务中往往是必需的。

% 623

许多仅含一个\gls{hidden_layer}的简单\gls{graphical_models}会定义成易于计算$p(\Vh\mid\Vv)$或其期望的形式,例如\gls{RBM}和\gls{PPCA}。
不幸的是,大多数具有多层\gls{hidden_variable}的\gls{graphical_models}的后验分布都很难处理。
对于这些模型而言,精确推断算法需要指数量级的运行时间。
即使一些只有单层的模型,如\gls{sparse_coding},也存在着这样的问题。
% 623


在本章中,我们将会介绍几个用来解决这些难以处理的推断问题的技巧。
稍后,在\chapref{chap:deep_generative_models}中,我们还将描述如何将这些技巧应用到训练其他方法难以奏效的概率模型中,如\gls{DBN}、\gls{DBM}。
% 623


在\gls{DL}中难以处理的推断问题通常源于结构化图模型中\gls{latent_variable}之间的相互作用。
读者可以参考\figref{fig:intractable_graphs}的几个例子。
这些相互作用可能是\gls{undirected_model}的直接相互作用,也可能是\gls{directed_model}中同一个可见变量的共同祖先之间的``\gls{explaining_away}''作用。
% 623 end



% 624 head
\begin{figure}[!htb]
\ifOpenSource
\centerline{\includegraphics{figure.pdf}}
\else
	\centerline{\includegraphics[width=0.8\textwidth]{Chapter19/figures/intractable_graphs}}
\fi
\caption{\gls{DL}中难以处理的推断问题通常是由于结构化图模型中\gls{latent_variable}的相互作用。
这些相互作用产生于一个\gls{latent_variable}与另一个\gls{latent_variable}或者当\gls{vstructure}的子节点可观察时与更长的激活路径相连。
\emph{(左)}一个\gls{hidden_unit}存在连接的\firstgls{srbm}~\citep{Osindero+Hinton-2008}。
由于存在大量\gls{latent_variable}的\gls{clique},\gls{latent_variable}的直接连接使得后验分布难以处理。
\emph{(中)}一个\gls{DBM},被分层从而使得不存在层内连接,由于层之间的连接其后验分布仍然难以处理。
\emph{(右)}当可见变量可观察时这个有向模型的\gls{latent_variable}之间存在相互作用,因为每两个\gls{latent_variable}都是\gls{coparent}。
即使拥有上图中的某一种结构,一些概率模型依然能够获得易于处理的关于\gls{latent_variable}的后验分布。
如果我们选择条件概率分布来引入相对于图结构描述的额外的独立性这种情况也是可能出现的。
举个例子,\gls{PPCA}的图结构如右图所示,然而由于其条件分布的特殊性质(带有相互正交基向量的线性高斯条件分布)依然能够进行简单的推断。}
\label{fig:intractable_graphs}
\end{figure}
% 624 head



\section{把推断视作优化问题}
\label{sec:inference_as_optimization}
% 624

% 许多难以利用观察值进行精确推断的问题往往可以描述为一个优化问题。
精确推断问题可以描述为一个优化问题,有许多方法正是由此解决了推断的困难。
通过近似这样一个潜在的优化问题,我们往往可以推导出\gls{approximate_inference}算法。
% 624


为了构造这样一个优化问题,假设我们有一个包含可见变量$\Vv$和\gls{latent_variable} $\Vh$的概率模型。
我们希望计算观察数据的对数概率$\log p(\Vv;\Vtheta)$。
有时候如果边缘化消去$\Vh$的操作很费时,我们会难以计算$\log p(\Vv;\Vtheta)$。
作为替代,我们可以计算一个$\log p(\Vv;\Vtheta)$的下界$\CalL(\Vv,{\Vtheta},q)$。
这个下界被称为\firstall{ELBO}。
这个下界的另一个常用名称是负\firstgls{variational_free_energy}。
具体地,这个\gls{ELBO}是这样定义的:
% 625 head 
\begin{align}
\CalL(\Vv,{\Vtheta},q) = \log p(\Vv;{\Vtheta}) - D_{\text{KL}}(q(\Vh\mid\Vv) \Vert p(\Vh\mid\Vv;{\Vtheta})),
\end{align}
其中$q$是关于$\Vh$的一个任意概率分布。
% 625


因为$\log p(\Vv)$和$\CalL(\Vv,{\Vtheta},q)$之间的距离是由~\gls{KL}来衡量的,且~\gls{KL}总是非负的,我们可以发现$\CalL$总是小于等于所求的对数概率。
当且仅当分布$q$完全相等于$p(\Vh\mid\Vv)$时取到等号。
% 625


令人吃惊的是,对于某些分布$q$,计算$\CalL$可以变得相当简单。
通过简单的代数运算我们可以把$\CalL$重写成一个更加简单的形式:

\begin{align}
\CalL(\Vv,{\Vtheta},q) = & \log p(\Vv;{\Vtheta})- D_{\text{KL}}(q(\Vh\mid\Vv)\Vert p(\Vh\mid\Vv;{\Vtheta})) \\
= & \log p(\Vv;{\Vtheta}) - \SetE_{\RVh\sim q}\log \frac{q(\Vh\mid\Vv)}{p(\Vh\mid\Vv)} \\
= & \log p(\Vv;{\Vtheta}) -  \SetE_{\RVh\sim q} \log \frac{q(\Vh\mid\Vv) }{ \frac{p(\Vh,\Vv;{\Vtheta})}{p(\Vv; {\Vtheta})} } \\
= & \log p(\Vv; {\Vtheta}) -  \SetE_{\RVh\sim q} [\log q(\Vh\mid\Vv) - \log {p(\Vh,\Vv; {\Vtheta})} + \log {p(\Vv; {\Vtheta})} ]\\
= & - \SetE_{\RVh\sim q}[\log q(\Vh\mid\Vv) - \log {p(\Vh,\Vv;{\Vtheta})}].
\end{align}
% 625
这也给出了\gls{ELBO}的标准定义:
\begin{align}
\CalL(\Vv,{\Vtheta},q) = \SetE_{\RVh\sim q}[\log p(\Vh , \Vv)] + H(q).
\end{align}


对于一个选择的合适分布$q$来说,$\CalL$是容易计算的。
对任意分布$q$的选择来说,$\CalL$提供了似然函数的一个下界。
越好地近似$p(\Vh\mid\Vv)$的分布$q(\Vh\mid\Vv)$,得到的下界就越紧,换言之,就是与$\log p(\Vv)$更加接近。
当$q(\Vh\mid\Vv) = p(\Vh\mid\Vv)$时,这个近似是完美的,也意味着$\CalL(\Vv,{\Vtheta},q) = \log {p(\Vv;{\Vtheta})} $。
% 625


因此我们可以将推断问题看作是找一个分布$q$使得$\CalL$最大的过程。
精确推断能够在包含分布$p(\Vh\mid\Vv)$的函数族中搜索一个函数,完美地最大化$\CalL$。
在本章中,我们将会讲到如何通过近似优化寻找分布$q$的方法来推导出不同形式的\gls{approximate_inference}。
我们可以通过限定分布$q$的形式或者使用并不彻底的优化方法来使得优化的过程更加高效(却更粗略),但是优化的结果是不完美的,
不求彻底地最大化$\CalL$,而只要显著地提升$\CalL$。
%因为只能显著地提升$\CalL$而无法彻底地最大化$\CalL$。
% 625  end


无论我们选择什么样的分布$q$,$\CalL$始终是一个下界。
我们可以通过选择一个更简单或更复杂的计算过程来得到对应的更松或更紧的下界。
通过一个不彻底的优化过程或者将分布$q$做很强的限定(并且使用一个彻底的优化过程)我们可以获得一个很差的分布$q$,但是降低了计算开销。
% 626 head


\section{\glsentrytext{EM}}
\label{sec:expectation_maximization}
% 626

我们介绍的第一个最大化下界$\CalL$的算法是\firstall{EM}算法。
在\gls{latent_variable}模型中,这是一个非常常见的训练算法。
在这里我们描述 \citet{emview} 所提出的~\glssymbol{EM}~算法。
与大多数我们在本章中介绍的其他算法不同的是,\glssymbol{EM}~并不是一个\gls{approximate_inference}算法,而是一种能够学到近似后验的算法。
% 626


\glssymbol{EM}算法由交替迭代,直到收敛的两步运算组成:
\begin{itemize}
\item \firstgls{e_step}: 令${\Vtheta^{(0)}}$表示在这一步开始时的参数值。
对任何我们想要训练的(对所有的或者\gls{minibatch}数据均成立)索引为$i$的训练样本$\Vv^{(i)}$,令$q(\Vh^{(i)}\mid \Vv) = p(\Vh^{(i)}\mid\Vv^{(i)};\Vtheta^{(0)})$。
通过这个定义,我们认为$q$在\emph{当前}参数$\Vtheta^{(0)}$下定义。
如果我们改变$\Vtheta$,那么$p(\Vh\mid\Vv;\Vtheta)$将会相应地变化,但是$q(\Vh\mid\Vv)$还是不变并且等于$p(\Vh\mid\Vv;\Vtheta^{(0)})$。
\item \firstgls{m_step}:使用选择的优化算法完全地或者部分地关于$\Vtheta$最大化
\begin{align}
\sum_i \CalL(\Vv^{(i)},\Vtheta,q).
\end{align}
\end{itemize}
% 626  


这可以被看作通过\gls{coordinate_ascent}算法来最大化$\CalL$。
在第一步中,我们更新分布$q$来最大化$\CalL$,而在另一步中,我们更新$\Vtheta$来最大化$\CalL$。
% 626


基于\gls{latent_variable}模型的\gls{SGA}可以被看作是一个~\glssymbol{EM}~算法的特例,其中~\gls{m_step}包括了单次梯度操作。
\glssymbol{EM}~算法的其他变种可以实现多次梯度操作。
对一些模型族来说,\gls{m_step}甚至可以直接推出解析解,不同于其他方法,在给定当前$q$的情况下直接求出最优解。
% 626 end


尽管\gls{e_step}采用的是精确推断,我们仍然可以将~\glssymbol{EM}~算法视作是某种程度上的\gls{approximate_inference}。
具体地说,\gls{m_step}假设一个分布$q$可以被所有的$\Vtheta$值分享。
当~\gls{m_step}越来越远离~\gls{e_step}中的$\Vtheta^{(0)}$时,这将会导致$\CalL$和真实的$\log p(\Vv)$之间出现差距。
幸运的是,在进入下一个循环时,\gls{e_step}把这种差距又降到了$0$。
% 627 head



\glssymbol{EM}~算法还包含一些不同的见解。
首先,它包含了学习过程的一个基本框架,就是我们通过更新模型参数来提高整个数据集的似然,其中缺失变量的值是通过后验分布来估计的。
这种特定的性质并非~\glssymbol{EM}~算法独有的。
例如,使用\gls{GD}来最大化对数似然函数的方法也有相同的性质。
计算对数似然函数的梯度需要对\gls{hidden_unit}的后验分布求期望。 
\glssymbol{EM}~算法另一个关键的性质是当我们移动到另一个$\Vtheta$时候,我们仍然可以使用旧的分布$q$。
在传统\gls{ML}中,这种特有的性质在推导大~\gls{m_step}更新时候得到了广泛的应用。
在\gls{DL}中,大多数模型太过于复杂以致于在最优大~\gls{m_step}更新中很难得到一个简单的解。
所以~\glssymbol{EM}~算法的第二个特质,更多为其所独有,较少被使用。
% 627


\section{\glsentrytext{MAP}推断和\glsentrytext{sparse_coding}}
\label{sec:map_inference_and_sparse_coding}
% 627


我们通常使用\firstgls{inference}这个术语来指代给定一些其他变量的情况下计算某些变量概率分布的过程。
当训练带有\gls{latent_variable}的概率模型时,我们通常关注于计算$p(\Vh\mid\Vv)$。
另一种可选的推断形式是计算一个缺失变量的最可能值来代替在所有可能值的完整分布上的推断。
在\gls{latent_variable}模型中,这意味着计算
\begin{align}
\Vh^* = \underset{\Vh}{\arg\max} \ \  p(\Vh\mid\Vv).
\end{align}
这被称作\firstgls{MAP}推断,简称~\glssymbol{MAP}~推断。
% 627  



\glssymbol{MAP}~推断并不被视作是一种\gls{approximate_inference},它只是精确地计算了最有可能的一个$\Vh^*$。
然而,如果我们希望设计一个最大化$\CalL(\Vv,\Vh,q)$的学习过程,那么把~\glssymbol{MAP}~推断视作是输出一个$q$值的学习过程是很有帮助的。
在这种情况下,我们可以将~\glssymbol{MAP}~推断视作是\gls{approximate_inference},因为它并不能提供一个最优的$q$。
% 627 end



我们回过头来看看\secref{sec:inference_as_optimization}中所描述的精确推断,它指的是关于一个在无限制的概率分布族中的分布$q$使用精确的优化算法来最大化
\begin{align}
\CalL(\Vv,{\Vtheta},q)
 = \SetE_{\RVh\sim q}[\log p(\Vh , \Vv)] + H(q).
\end{align}
我们通过限定分布$q$属于某个分布族,能够使得~\glssymbol{MAP}~推断成为一种形式的\gls{approximate_inference}。
具体地说,我们令分布$q$满足一个\,\gls{dirac_distribution}:
\begin{align}
q(\Vh\mid\Vv) = \delta(\Vh - {\Vmu}).
\end{align}
这也意味着现在我们可以通过$\Vmu$来完全控制分布$q$。
将$\CalL$中不随$\Vmu$变化的项丢弃,我们只需解决一个优化问题:
\begin{align}
\Vmu^*  =  \underset{\Vmu}{\arg\max}\ \log p(\Vh = \Vmu,\Vv),
\end{align}
这等价于~\glssymbol{MAP}~推断问题
\begin{align}
\Vh^* = \underset{\Vh}{\arg\max}\  p(\Vh\mid\Vv).
\end{align}
% 628




因此我们能够证明一种类似于~\glssymbol{EM}~算法的学习算法,其中我们轮流迭代两步,一步是用~\glssymbol{MAP}~推断估计出$\Vh^*$,另一步是更新$\Vtheta$来增大$\log p(\Vh^*,\Vv)$。
从~\glssymbol{EM}~算法角度看,这也是对$\CalL$的一种形式的\gls{coordinate_ascent},交替迭代时通过推断来优化关于$q$的$\CalL$以及通过参数更新来优化关于$\Vtheta$的$\CalL$。
作为一个整体,这个算法的正确性可以得到保证,因为$\CalL$是$\log p(\Vv)$的下界。
在~\glssymbol{MAP}~推断中,这个保证是无效的,因为~\gls{dirac_distribution}的微分熵趋近于负无穷,使得这个界会无限地松。
然而,人为加入一些$\Vmu$的噪声会使得这个界又有了意义。
% 628


\glssymbol{MAP}~推断作为\gls{feature_extractor}以及一种学习机制被广泛地应用在了\gls{DL}中。
它主要用于\gls{sparse_coding}模型中。
% 628


我们回过头来看\secref{sec:sparse_coding}中的\gls{sparse_coding},\gls{sparse_coding}是一种在\gls{hidden_unit}上加上了诱导稀疏性的先验知识的\gls{linear_factor}。
一个常用的选择是可分解的Laplace先验,表示为
\begin{align}
	p(h_i) = \frac{\lambda}{2}  \exp(-\lambda \vert h_i \vert).
\end{align}
可见的节点是由一个线性变化加上噪声生成的:
\begin{align}
p(\Vv\mid\Vh) = \CalN(\Vv;{\MW}\Vh + \Vb,\beta^{-1}{\MI}).
\end{align}
% 628 end


分布$p(\Vh\mid\Vv)$难以计算,甚至难以表达。
每一对$h_i$, $h_j$变量都是$\Vv$的母节点。
这也意味着当$\Vv$可被观察时,\gls{graphical_models}包含了一条连接$h_i$和$h_j$的活跃路径。
因此$p(\Vh \mid\Vv)$中所有的\gls{hidden_unit}都包含在了一个巨大的\gls{clique}中。
如果是高斯模型,那么这些相互作用关系可以通过协方差矩阵来高效地建模。
然而稀疏型先验使得这些相互作用关系并不服从高斯分布。
% 629 head


分布$p(\Vx\mid\Vh)$的难处理性导致了对数似然及其梯度也很难得到。
因此我们不能使用精确的\gls{MLE}来进行学习。
取而代之的是,我们通过~\glssymbol{MAP}~推断以及最大化由以$\Vh$为中心的\,\gls{dirac_distribution}所定义而成的~\glssymbol{ELBO}~来学习模型参数。
% 629


如果我们将训练集中所有的向量$\Vh$拼成矩阵$\MH$,并将所有的向量$\Vv$拼起来组成矩阵$\MV$,那么\gls{sparse_coding}问题意味着最小化
\begin{align}
	J(\MH,\MW) = \sum_{i,j}^{}\vert H_{i,j}\vert + \sum_{i,j}^{}\Big(\MV - \MH \MW^{\top}\Big)^2_{i,j}.
\end{align}
为了避免如极端小的$\MH$和极端大的$\MW$这样的病态的解,大多数\gls{sparse_coding}的应用包含了\gls{weight_decay}或者对$\MH$列范数的限制。
% 629


我们可以通过交替迭代,分别关于$\MH$和$\MW$最小化$J$的方式来最小化$J$。
且两个子问题都是凸的。
事实上,关于$\MW$的最小化问题就是一个\gls{linear_regression}问题。
然而关于这两个变量同时最小化$J$的问题通常并不是凸的。
% 629


关于$\MH$的最小化问题需要某些特别设计的算法,例如特征符号搜索方法~\citep{HonglakLee-2007}。
% 629


\section{变分推断和变分学习}
\label{sec:variational_inference_and_learning}
% 629


我们已经说明过了为什么\gls{ELBO} $\CalL(\Vv,\Vtheta,q)$是$\log  p(\Vv;\Vtheta)$的一个下界、如何将推断看作是关于分布$q$最大化$\CalL$ 的过程以及如何将学习看作是关于参数$\Vtheta$最大化$\CalL$的过程。
我们也讲到了~\glssymbol{EM}~算法在给定了分布$q$的条件下能够进行\gls{large_learning_step},而基于~\glssymbol{MAP}~推断的学习算法则是学习一个$p(\Vh \mid \Vv)$的点估计而非推断整个完整的分布。
在这里我们介绍一些变分学习中更加通用的算法。
% 629

%  在一个关于q的有约束的分布族上
变分学习的核心思想就是在一个关于$q$的有约束的分布族上最大化$\CalL$。
选择这个分布族时应该考虑到计算$\SetE_q \log p(\Vh,\Vv)$的难易度。
一个典型的方法就是添加分布$q$如何分解的假设。
% 630 head


一种常用的变分学习的方法是加入一些限制使得$q$是一个\gls{factorial}分布:
\begin{align}
	q(\Vh\mid\Vv) = \prod_{i}^{}q(h_i \mid \Vv).
\end{align}
这被称为\firstgls{mean_field}方法。
更一般地说,我们可以通过选择分布$q$的形式来选择任何\gls{graphical_models}的结构,通过选择变量之间相互作用的多少来灵活地决定近似程度的大小。
这种完全通用的\gls{graphical_models}方法被称为\firstgls{structured_variational_inference} \citep{Saul96}。
% 630 


变分方法的优点是我们不需要为分布$q$设定一个特定的参数化形式。
我们设定它如何分解,之后通过解决优化问题来找出在这些分解限制下最优的概率分布。
对离散型\gls{latent_variable}来说,这意味着我们使用传统的优化技巧来优化描述分布$q$的有限个变量。
对连续型\gls{latent_variable}来说,这意味着我们使用一个被称为\gls{calculus_of_variations}的数学分支工具来解决函数空间上的优化问题。
然后决定哪一个函数来表示分布$q$。
\gls{calculus_of_variations}是``变分学习''或者``变分推断''这些名字的来因,尽管当\gls{latent_variable}是离散时\gls{calculus_of_variations}并没有用武之地。
当遇到连续型\gls{latent_variable}时,\gls{calculus_of_variations}不需要过多地人工选择模型,是一种很有用的工具。
我们只需要设定分布$q$如何分解,而不需要去猜测一个特定的能够精确近似原后验分布的分布$q$。
% 630 


因为$\CalL(\Vv,\Vtheta,q)$被定义成$\log p(\Vv;\Vtheta) - D_{\text{KL}} (q(\Vh\mid\Vv) \Vert  p(\Vh\mid\Vv;\Vtheta) )$,我们可以认为关于$q$最大化$\CalL$的问题等价于(关于$q$)最小化$D_{\text{KL}}(q(\Vh\mid\Vv)\Vert p(\Vh\mid\Vv))$。
在这种情况下,我们要用$q$来拟合$p$。
然而,与以前方法不同,我们使用~\gls{KL}的相反方向来拟合一个近似。
当我们使用\gls{MLE}来用模型拟合数据时,我们最小化$D_{\text{KL}}(p_{\text{data}} \Vert p_{\text{model}})$。
如\figref{fig:chap3_kl_direction_color}所示,这意味着\gls{maximum_likelihood}鼓励模型在每一个数据达到高概率的地方达到高概率,而基于优化的推断则鼓励了$q$在每一个真实后验分布概率低的地方概率较小。
这两种基于~\gls{KL}的方法都有各自的优点与缺点。
选择哪一种方法取决于在具体每一个应用中哪一种性质更受偏好。
在基于优化的推断问题中,从计算角度考虑,我们选择使用$D_{\text{KL}}(q(\Vh\mid\Vv)\Vert p(\Vh\mid\Vv))$。
具体地说,计算$D_{\text{KL}}(q(\Vh\mid\Vv)\Vert p(\Vh\mid\Vv))$涉及到了计算分布$q$下的期望。
所以通过将分布$q$设计得较为简单,我们可以简化求所需要的期望的计算过程。
\gls{KL}的相反方向需要计算真实后验分布下的期望。
因为真实后验分布的形式是由模型的选择决定的,所以我们不能设计出一种能够精确计算$D_{\text{KL}}(p(\Vh\mid\Vv) \Vert q(\Vh\mid\Vv))$的开销较小的方法。
% 631 head




\subsection{离散型\gls{latent_variable}}
\label{sec:discrete_latent_variables}
% 631  19.4.1

关于离散型\gls{latent_variable}的变分推断相对来说比较直接。
我们定义一个分布$q$,通常分布$q$的每个因子都由一些离散状态的可查询表格定义。
在最简单的情况中,$\Vh$是二值的并且我们做了\gls{mean_field}假定,分布$q$可以根据每一个$h_i$分解。
在这种情况下,我们可以用一个向量$\hat{\Vh}$来参数化分布$q$,$\hat{\Vh}$的每一个元素都代表一个概率,即$q(h_i = 1\mid \Vv) = \hat{h}_i$。
% 631


在确定了如何表示分布$q$以后,我们只需要优化它的参数。
在离散型\gls{latent_variable}模型中,这是一个标准的优化问题。
基本上分布$q$的选择可以通过任何优化算法解决,比如\gls{GD}算法。
% 631


因为它在许多学习算法的内循环中出现,所以这个优化问题必须可以很快求解。
为了追求速度,我们通常使用特殊设计的优化算法。
这些算法通常能够在极少的循环内解决一些小而简单的问题。
一个常见的选择是使用\gls{fixed_point_equation},换句话说,就是解关于$\hat{h}_i$的方程
\begin{equation}
\frac{\partial}{\partial \hat{h}_i}\CalL = 0.
\end{equation}
我们反复地更新$\hat{\Vh}$不同的元素直到满足收敛准则。
% 631


为了具体化这些描述,我们接下来会讲如何将变分推断应用到\firstgls{binary_sparse_coding}模型(这里我们所描述的模型是 \citet{henniges2010binary} 提出的,但是我们采用了传统、通用的\gls{mean_field}方法,而原文作者采用了一种特殊设计的算法)中。
数学推导过程非常详细,为希望完全了解我们描述过的变分推断和变分学习高级概念描述的读者所准备。
而对于并不计划推导或者实现变分学习算法的读者来说,可以放心跳过,直接阅读下一节,这并不会遗漏新的高级概念。
建议那些从事\gls{binary_sparse_coding}研究的读者可以重新看一下\secref{sec:useful_properties_of_common_functions}中描述的一些经常在概率模型中出现的有用的函数性质。
我们在推导过程中随意地使用了这些性质,并没有特别强调它们。
% 632  head


在\gls{binary_sparse_coding}模型中,输入$\Vv\in\SetR^n$,是由模型通过添加高斯噪声到$m$个或有或无的不同成分的和而生成的。
每一个成分可以是开或者关的,对应着\gls{hidden_unit}~$\Vh \in\{0,1\}^m$:
\begin{align}
p(h_i = 1) &= \sigma(b_i), \\
p(\Vv\mid\Vh) &= \CalN(\Vv;\MW \Vh,{\Vbeta}^{-1}),
\end{align}
其中$\Vb$是一个可以学习的\gls{bias_aff}集合,$\MW$是一个可以学习的权值矩阵,${\Vbeta}$是一个可以学习的对角精度矩阵。
% 632


使用\gls{maximum_likelihood}来训练这样一个模型需要对参数进行求导。
我们考虑对其中一个\gls{bias_aff}进行求导的过程:
\begin{align}
		& \frac{\partial}{\partial b_i} \log p(\Vv) \\
		= &  \frac{\frac{\partial}{\partial b_i} p(\Vv)}{p(\Vv)}\\
		= & \frac{\frac{\partial}{\partial b_i} \sum_{\Vh}^{} p(\Vh,\Vv)}{p(\Vv)}\\
		= &  \frac{\frac{\partial}{\partial b_i} \sum_{\Vh}^{} p(\Vh) p(\Vv\mid \Vh)  }{p(\Vv)}\\
		= &  \frac{ \sum_{\Vh}^{}  p(\Vv\mid \Vh) \frac{\partial}{\partial b_i} p(\Vh) }{p(\Vv)}\\
		= &  \sum_{\Vh}^{}  p(\Vh\mid \Vv) \frac{  \frac{\partial}{\partial b_i} p(\Vh) }{p(\Vh)}\\
		= & \SetE_{\RVh\sim p(\Vh\mid\Vv)} \frac{\partial}{\partial b_i}\log p(\Vh).
\end{align}
% 632 end


这需要计算$p(\Vh\mid\Vv)$下的期望。
不幸的是,$p(\Vh\mid\Vv)$是一个很复杂的分布。
关于$p(\Vh,\Vv)$和$p(\Vh\mid\Vv)$的图结构可以参考\figref{fig:bsc}。
\gls{hidden_unit}的后验分布对应的是关于\gls{hidden_unit}的完全图,所以相对于暴力算法,\gls{variable_elimination}算法并不能有助于提高计算期望的效率。
% 633 head


\begin{figure}[!htb]
\ifOpenSource
\centerline{\includegraphics{figure.pdf}}
\else
	\centerline{\includegraphics{Chapter19/figures/bsc}}
\fi
\caption{包含四个\gls{hidden_unit}的\gls{binary_sparse_coding}的图结构。
\emph{(左)}~$p(\Vh,\Vv)$的图结构。
要注意边是有向的,每两个\gls{hidden_unit}都是每个可见单元的\gls{coparent}。
\emph{(右)}~$p(\Vh,\Vv)$的图结构。
为了解释\gls{coparent}之间的活跃路径,后验分布所有\gls{hidden_unit}之间都有边。}
	\label{fig:bsc}
\end{figure}



取而代之的是,我们可以应用变分推断和变分学习来解决这个难点。
% 633


我们可以做一个\gls{mean_field}近似:
\begin{equation}
	q(\Vh\mid\Vv) = \prod_{i}^{}q(h_i\mid\Vv).
\end{equation}
% 633


\gls{binary_sparse_coding}中的\gls{latent_variable}是二值的,所以为了表示可分解的$q$我们假设对$m$个~\gls{bernoulli_distribution} $q(h_i\mid\Vv)$建模。
表示~\gls{bernoulli_distribution}的一种很自然的方法是使用一个概率向量$\hat{\Vh}$,满足$q(h_i\mid\Vv) = \hat{h}_i$。
为了避免计算中的误差,比如说计算$\log \hat{h}_i$时,我们对$\hat{h}_i$添加一个约束,即$\hat{h}_i$不等于$0$或者$1$。
% 633 


我们将会看到变分推断方程理论上永远不会赋予$\hat{h}_i\ $ $0$或者$1$。
然而在软件实现过程中,机器的舍入误差会导致$0$或者$1$的值。
在\gls{binary_sparse_coding}的软件实现中,我们希望使用一个没有限制的变分参数向量$\Vz$以及
通过关系$\hat{\Vh} = \sigma(\Vz)$来获得$\Vh$。
因此通过使用等式$\log \sigma(z_i) = -\zeta(-z_i)$来建立\,\ENNAME{sigmoid}\,函数和\,\ENNAME{softplus}\,函数的关系,我们可以放心地在计算机上计算$\log \hat{h}_i$。
% 633


在开始\gls{binary_sparse_coding}模型中变分学习的推导时,我们首先说明了\gls{mean_field}近似的使用可以使得学习过程更加简单。
% 633 end  


\gls{ELBO}可以表示为
\begin{align}
& \CalL(\Vv,\Vtheta,q)\\
 = & \SetE_{\RVh\sim q}[\log p(\Vh,\Vv)] + H(q)\\
 = & \SetE_{\RVh\sim q}[\log p(\Vh) + \log p(\Vv\mid\Vh) - \log q(\Vh\mid\Vv)]\\
= & \SetE_{\RVh\sim q}\Big[\sum_{i=1}^{m}\log p(h_i) + \sum_{i=1}^{n} \log p(v_i\mid\Vh) - \sum_{i=1}^{m}\log q(h_i\mid\Vv)\Big]\\
= &  \sum_{i=1}^{m}\Big[\hat{h}_i(\log \sigma(b_i) - \log \hat{h}_i) + (1 - \hat{h}_i)(\log \sigma(-b_i) - \log (1-\hat{h}_i))\Big] \\
& +  \SetE_{\RVh\sim q} \Bigg[ \sum_{i=1}^{n}\log \sqrt{\frac{\beta_i}{2\pi}}\exp(-\frac{\beta_i}{2}(v_i - \MW_{i,:}\Vh)^2)\Bigg] \\
= &  \sum_{i=1}^{m}\Big[\hat{h}_i(\log \sigma(b_i) - \log \hat{h}_i) + (1 - \hat{h}_i)(\log \sigma(-b_i) - \log (1-\hat{h}_i))\Big] \\
& + \frac{1}{2} \sum_{i=1}^{n} \Bigg[ \log {\frac{\beta_i}{2\pi}} - \beta_i \Bigg(v_i^2 - 2 v_i \MW_{i,:}\hat{\Vh} + \sum_{j}\Big[W^2_{i,j}\hat{h}_j + \sum_{k \neq j}W_{i,j}W_{i,k}\hat{h}_j\hat{h}_k \Big] \Bigg) \Bigg]. 
\label{eqn:1936}
\end{align}
% 634 head 
尽管这些方程从美学观点来看有些不尽如人意。
他们展示了$\CalL$可以被表示为少量简单的代数运算。
因此\gls{ELBO}$\,\CalL$是易于处理的。
我们可以把$\CalL$看作是难以处理的对数似然函数的一个替代。
% 634  head 


原则上说,我们可以使用关于$\Vv$和$\Vh$的\gls{GA}。
这会成为一个推断和学习算法的完美组合。%\footnote{译者注:推断算法和学习算法的组合}
但是,由于两个原因,我们往往不这么做。
第一点,对每一个$\Vv$我们需要存储$\hat{\Vh}$。
我们通常更加偏向于那些不需要为每一个样本都准备内存的算法。
如果我们需要为每一个样本都存储一个动态更新的向量,使得算法很难处理几十亿的样本。
第二个原因就是为了能够识别$\Vv$的内容,我们希望能够有能力快速提取特征$\hat{\Vh}$。
在实际应用场景中,我们需要在有限时间内计算出$\hat{\Vh}$。
% 634


由于以上两个原因,我们通常不会采用\gls{GD}来计算\gls{mean_field}参数$\hat{\Vh}$。
取而代之的是,我们使用\gls{fixed_point_equation}来快速估计。
% 634


\gls{fixed_point_equation}的核心思想是我们寻找一个关于$\Vh$的\gls{local_maximum},满足$\nabla_{\Vh}\CalL(\Vv,\Vtheta,\hat{\Vh}) = 0$。
我们无法同时高效地计算所有$\hat{\Vh}$的元素。
然而,我们可以解决单个变量的问题:
\begin{equation}
\label{eqn:1937}
\frac{\partial}{\partial \hat{h}_i} \CalL(\Vv,\Vtheta,\hat{\Vh}) = 0 .
\end{equation}
% 634 end  


我们可以迭代地将这个解应用到$i = 1,\ldots,m$,然后重复这个循环直到我们满足了收敛准则。
常见的收敛准则包含了当整个循环所改进的$\CalL$不超过预设的\gls{tolerance}量时停止,或者是循环中改变的$\hat{\Vh}$不超过某个值时停止。
% 635  head

在很多不同的模型中,迭代的\gls{mean_field}\gls{fixed_point_equation}是一种能够提供快速变分推断的通用算法。
为了使它更加具体,我们详细地讲一下如何推导出\gls{binary_sparse_coding}模型的更新过程。
% 635


首先,我们给出了对$\hat{h}_i$的导数表达式。
为了得到这个表达式,我们将\eqnref{eqn:1936}代入到\eqnref{eqn:1937}的左边:
\begin{align}
& \frac{\partial}{\partial \hat{h}_i} \CalL (\Vv,\Vtheta,\hat{\Vh})    \\
= & \frac{\partial}{\partial \hat{h}_i} \Bigg[\sum_{j=1}^{m}\Big[\hat{h}_j (\log \sigma(b_j) - \log \hat{h}_j) + (1 - \hat{h}_j)(\log \sigma(-b_j) - \log (1-\hat{h}_j))\Big] \\
& + \frac{1}{2} \sum_{j=1}^{n} \Bigg[ \log {\frac{\beta_j}{2\pi}} - \beta_j \Bigg(v_j^2 - 2 v_j \MW_{j,:}\hat{\Vh} + \sum_{k}\Bigg[W^2_{j,k}\hat{h}_k + \sum_{l \neq k}W_{j,k}W_{j,l}\hat{h}_k\hat{h}_l \Bigg] \Bigg) \Bigg]\Bigg] \\
= & \log \sigma(b_i) - \log \hat{h}_i - 1 + \log (1 - \hat{h}_i ) + 1 - \log \sigma (- b_i ) \\
 & + \sum_{j=1}^{n} \Bigg[\beta_j \Bigg(v_j W_{j,i} - \frac{1}{2} W_{j,i}^2 - \sum_{k \neq i} \MW_{j,k}\MW_{j,i} \hat{h}_k\Bigg) \Bigg]\\
 = & b_i - \log \hat{h}_i + \log (1 - \hat{h}_i ) + \Vv^{\top} {\Vbeta} \MW_{:,i} - \frac{1}{2} \MW_{:,i}^{\top} {\Vbeta}\MW_{:,i} -\sum_{j\neq i}\MW^{\top}_{:,j}{\Vbeta}\MW_{:,i}\hat{h}_j.
 \label{eqn:1943}
\end{align}
% 635  

为了应用固定点更新的推断规则,我们通过令\eqnref{eqn:1943}等于$0$来解$\hat{h}_i$:
\begin{align}
\label{eqn:1944}
\hat{h}_i = \sigma\Bigg(b_i + \Vv^{\top} {\Vbeta} \MW_{:,i} - \frac{1}{2} \MW_{:,i}^{\top} {\Vbeta} \MW_{:,i} - \sum_{j \neq i }  \MW_{:,j}^{\top} {\Vbeta}  \MW_{:,i} \hat{h}_j \Bigg).
\end{align}
% 635 end

此时,我们可以发现\gls{graphical_models}中的推断和\gls{RNN}之间存在着紧密的联系。
具体地说,\gls{mean_field}\gls{fixed_point_equation}定义了一个\gls{RNN}。
这个神经网络的任务就是完成推断。
我们已经从模型描述的角度介绍了如何推导这个网络,但是直接训练这个推断网络也是可行的。
有关这种思路的一些想法在\chapref{chap:deep_generative_models}中有所描述。
% 636  head 


在\gls{binary_sparse_coding}模型中,我们可以发现\eqnref{eqn:1944}中描述的\gls{recurrent_network}连接包含了根据相邻\gls{hidden_unit}变化值来反复更新当前\gls{hidden_unit}的操作。   %?? 连接  换成 (和图模型推断的相关联系)     
输入层通常给\gls{hidden_unit}发送一个固定的信息$\Vv^{\top}\beta\MW$,然而\gls{hidden_unit}不断地更新互相传送的信息。
具体地说,当$\hat{h}_i$和$\hat{h}_j$两个单元的权重向量平行时,它们会互相抑制。
这也是一种形式的竞争——两个解释输入的\gls{hidden_unit}之间,只有一个解释得更好的才被允许继续保持活跃。
在\gls{binary_sparse_coding}的后验分布中,\gls{mean_field}近似试图捕获到更多的\gls{explaining_away}相互作用,从而产生了这种竞争。   %?? 捕获  换成  反应,
事实上,\gls{explaining_away}效应会产生一个\gls{multimodal}的后验分布,以致于如果我们从后验分布中采样,一些样本在一个单元是活跃的,其他的样本在另一个单元活跃,只有很少的样本能够两者都处于活跃状态。
不幸的是,\gls{explaining_away}作用无法通过\gls{mean_field}中\gls{factorial}分布$q$来建模,因此建模时\gls{mean_field}近似只能选择一个\gls{mode}。
这个现象的一个例子可以参考\figref{fig:chap3_kl_direction_color}。
% 636




我们将\eqnref{eqn:1944}重写成等价的形式来揭示一些深层的含义:
\begin{align}
\hat{h}_i = \sigma\Bigg(b_i + \Big(\Vv - \sum_{j\neq i} \MW_{:,j}\hat{h}_j\Big)^{\top} {\Vbeta}\MW_{:,i} - \frac{1}{2} \MW_{:,i}^{\top} {\Vbeta} \MW_{:,i}\Bigg). 
\end{align}
在这种新的形式中,我们可以将$\Vv - \sum_{j\neq i} \MW_{:,j}\hat{h}_j$看作是输入,而不是$\Vv$。
因此,我们可以把第$i$个单元视作给定其他单元编码时给$\Vv$中的剩余误差编码。
由此我们可以将\gls{sparse_coding}视作是一个迭代的\gls{AE},将输入反复地编码解码,试图在每一轮迭代后都能修复重构中的误差。
% 636


在这个例子中,我们已经推导出了每一次更新单个结点的更新规则。
如果能够同时更新更多的结点,那会更令人满意。
某些\gls{graphical_models},比如\gls{DBM},我们可以同时解出$\hat{\Vh}$中的许多元素。
不幸的是,\gls{binary_sparse_coding}并不适用这种块更新。
取而代之的是,我们使用一种被称为\firstgls{damping}的启发式技巧来实现块更新。
在\gls{damping}方法中,对$\hat{\Vh}$中的每一个元素我们都可以解出最优值,然后对于所有的值都在这个方向上移动一小步。
这个方法不能保证每一步都能增加$\CalL$,但是对于许多模型都很有效。
关于在\gls{message_passing}算法中如何选择同步程度以及使用\gls{damping}策略可以参考 \citet{koller-book2009} 。
 % 637 head 




\subsection{\glsentrytext{calculus_of_variations}}
\label{sec:calculus_of_variations}
% p 637 head   19.4.2 

在继续介绍变分学习之前,我们有必要简单地介绍一种变分学习中重要的数学工具:\firstgls{calculus_of_variations}。
% p 637


许多\gls{ML}的技巧是基于寻找一个输入向量$\Vtheta\in\SetR^n$来最小化函数$J(\Vtheta)$,使得它取到最小值。
这个步骤可以利用多元微积分以及线性代数的知识找到满足$\nabla_{\Vtheta} J(\Vtheta) = 0$的\gls{critical_points}来完成。
在某些情况下,我们希望能够解一个函数$f(\Vx)$,比如当我们希望找到一些随机变量的\gls{PDF}时。
正是\gls{calculus_of_variations}能够让我们完成这个目标。
% p 637



函数$f$的函数被称为\firstgls{functional} $J[f]$。
正如我们许多情况下对一个函数求关于以向量的元素为变量的\gls{partial_derivatives}一样,我们可以使用\firstgls{functional_derivative},
即在任意特定的$\Vx$值,对一个\gls{functional} $J[f]$求关于函数$f(\Vx)$的导数,这也被称为\firstgls{variational_derivative}。
\gls{functional} $J$的关于函数$f$在点$\Vx$处的\gls{functional_derivative}被记作$\frac{\delta}{\delta f(x)}J$。
% p 637



完整正式的\gls{functional_derivative}的推导不在本书的范围之内。
对于我们的目标而言,了解可微分函数$f(\Vx)$以及带有连续导数的可微分函数$g(y,\Vx)$就足够了:
\begin{align}
	\frac{\delta}{\delta f(\Vx)} \int g(f(\Vx),\Vx)d\Vx = \frac{\partial}{\partial y}g(f(\Vx),\Vx).
\end{align}
为了使上述等式更加直观,我们可以把$f(\Vx)$看作是一个有着无穷不可数多元素的向量,由一个实数向量$\Vx$表示。
在这里(看作是一个不完全的介绍),这种关系式中描述的\gls{functional_derivative}和向量$\Vtheta\in\SetR^n$的导数相同:
\begin{align}
	\frac{\partial}{\partial \theta_i}\sum_{j}^{}g(\theta_j,j) = \frac{\partial}{\partial \theta_i}g(\theta_i,i).
\end{align}
在其他\gls{ML}文献中的许多结果则使用了更为通用的\firstgls{euler_lagrange_eqn},
它能够使得$g$不仅依赖于$f$的导数而且也依赖于$f$的值。
但是在本书中我们不需要这个通用版本。
% p 637  end


为了关于一个向量优化某个函数,我们求出了这个函数关于这个向量的梯度,然后找这个梯度中每一个元素都为$0$的点。
类似地,我们可以通过寻找一个函数使得\gls{functional_derivative}的每个点都等于$0$从而来优化一个\gls{functional}。
% p 638 head


下面介绍一个该过程如何运行的例子,我们考虑寻找一个定义在$x\in\SetR$上的有最大\gls{differential_entropy}的\gls{PDF}。
我们回过头来看一下一个概率分布$p(x)$的熵,定义如下:
\begin{align}
	H[p] = - \SetE_x \log p(x).
\end{align}
对于连续的值,这个期望可以被看作一个积分:
\begin{align}
	H[p] = - \int p(x) \log p(x) dx.
\end{align}
% p 638


我们不能简单地仅仅关于函数$p(x)$最大化$H[p]$,因为那样的话结果可能不是一个概率分布。
为了解决这个问题,我们需要使用一个\gls{lagrange_multi}来添加一个分布$p(x)$积分值为$1$的约束。
同样地,当方差增大时,熵也会无限制地增加。
因此,寻找哪一个分布有最大熵这个问题是没有意义的。
但是,在给定固定的方差$\sigma^2$时,我们可以寻找一个最大熵的分布。
最后,这个问题还是\gls{underdetermined},因为在不改变熵的条件下一个分布可以被随意地改变。
为了获得一个唯一的解,我们再加一个约束:分布的均值必须为$\mu$。
那么这个问题的拉格朗日\gls{functional}如下:
\begin{align}
\label{eqn:1950}
&		\CalL[p] =  \lambda_1 \Big(\int p(x)dx - 1\Big)  + \lambda_2 (\SetE[x] - \mu) +  \lambda_3 (\SetE[(x-\mu)^2] - \sigma^2)  + H[p]\\
& =  \int \Big(\lambda_1 p(x) + \lambda_2 p(x)x + \lambda_3 p(x)(x-\mu)^2 - p(x)\log p(x) \Big)dx - \lambda_1 - \mu \lambda_2 - \sigma^2\lambda_3.
\end{align}
% p 638   


为了关于$p$最小化\gls{lagrange_multi},我们令\gls{functional_derivative}等于$0$:
\begin{align}
\label{eqn:1952}
	\forall x,\ \  \frac{\delta}{\delta p(x)} \CalL = \lambda_1 + \lambda_2 x + \lambda_3(x-\mu)^2 - 1 - \log p(x) = 0 .
\end{align}
% p 638  end 


这个条件告诉我们$p(x)$的\gls{functional}形式。
通过代数运算重组上述方程,我们可以得到
\begin{align}
\label{eqn:1953}
	p(x) = \exp\big(\lambda_1 + \lambda_2 x + \lambda_3 (x-\mu)^2  - 1\big).
\end{align}
% p 638 end 


我们并没有直接假设$p(x)$取这种形式,而是通过最小化\gls{functional}从理论上得到了这个$p(x)$的表达式。
为了解决这个最小化问题,我们需要选择$\lambda$的值来确保所有的约束都能够满足。
我们有很大的自由去选择$\lambda$。
因为只要满足约束,拉格朗日关于$\lambda$这个变量的梯度就为$0$。
为了满足所有的约束,我们可以令$\lambda_1 = 1 - \log \sigma\sqrt{2\pi}$,$\lambda_2 = 0$, $\lambda_3 = - \frac{1}{2\sigma^2}$,从而得到
\begin{align}
\label{eqn:1954}
	p(x) = \CalN(x;\mu,\sigma^2).
\end{align}
这也是当我们不知道真实的分布时总是使用\gls{normal_distribution}的一个原因。
因为\gls{normal_distribution}拥有最大的熵,我们通过这个假定来保证了最小可能量的结构。
% 639 head



当寻找熵的拉格朗日\gls{functional}的\gls{critical_points}并且给定一个固定的方差时,我们只能找到一个对应最大熵的\gls{critical_points}。
那最小化熵的\gls{PDF}是什么样的呢?
为什么我们无法发现对应着\gls{minimum}的第二个\gls{critical_points}呢?
原因是没有一个特定的函数能够达到最小的熵值。
当函数把越多的概率密度加到$x = \mu + \sigma$和$x = \mu - \sigma$两个点上,越少的概率密度到其他点上时,它们的熵值会减少,而方差却不变。
然而任何把所有的权重都放在这两点的函数的积分都不为$1$,不是一个有效的概率分布。
所以不存在一个最小熵的\gls{PDF},就像不存在一个最小的正实数一样。
然而,我们发现存在一个收敛的概率分布的序列,收敛到权重都在两个点上。
这种情况能够退化为混合\gls{dirac_distribution}。
因为\,\gls{dirac_distribution}并不是一个单独的\gls{PDF},所以\,\gls{dirac_distribution}或者混合\,\gls{dirac_distribution}并不能对应函数空间的一个点。
所以对我们来说,当寻找一个\gls{functional_derivative}为$0$的函数空间的点时,这些分布是不可见的。
这就是这种方法的局限之处。
诸如\,\gls{dirac_distribution}这样的分布可以通过其他方法被找到,比如可以先猜测一个解,然后证明它是满足条件的。
% 639 



\subsection{连续型\gls{latent_variable}}
\label{sec:continuous_latent_variables}
% 639 end            19.4.3 


当我们的\gls{graphical_models}包含连续型\gls{latent_variable}时,我们仍然可以通过最大化$\CalL$进行变分推断和变分学习。
然而,我们需要使用\gls{calculus_of_variations}来实现关于$q(\Vh\mid\Vv)$最大化$\CalL$。
% 639 end  


% 640 head  
在大多数情况下,研究者并不需要解决任何\gls{calculus_of_variations}的问题。
取而代之的是,\gls{mean_field}固定点迭代更新有一个通用的方程。
如果我们做了\gls{mean_field}近似:
\begin{align}
\label{eqn:1955}
	q(\Vh\mid\Vv) = \prod_i q(h_i \mid\Vv),
\end{align}
% 640 mid
并且对任何的$j\neq i$固定$q(h_j\mid\Vv)$,那么只需要满足分布$p$中任何联合分布变量的概率值不为$0$,我们就可以通过归一化下面这个未归一的分布
\begin{align}
	\label{eqn:1956}
	\tilde{q}(h_i \mid\Vv) = \exp \big(\SetE_{\RVh_{-i}\sim q(\RVh_{-i}\mid\Vv)}	\log \tilde{p}(\Vv,\Vh)\big)
\end{align}
% 640 mid
来得到最优的$q(h_i\mid\Vv)$。
在这个方程中计算期望就能得到正确的$q(h_i\mid\Vv)$的表达式。
我们只有在希望提出一种新形式的变分学习算法时才需要使用\gls{calculus_of_variations}来直接推导$q$的函数形式。
\eqnref{eqn:1956}给出了适用于任何概率模型的\gls{mean_field}近似。
% 640  




% 640  
\eqnref{eqn:1956}是一个\gls{fixed_point_equation},对每一个$i$它都被迭代地反复使用直到收敛。
然而,它还包含着更多的信息。
它还包含了最优解取到的\gls{functional}形式,无论我们是否能够通过\gls{fixed_point_equation}来解出它。
这意味着我们可以利用方程中的\gls{functional}形式,把其中一些值当成参数,然后通过任何我们想用的优化算法来解决这个问题。


% 640  
我们拿一个简单的概率模型作为例子,其中\gls{latent_variable}满足$\Vh\in\SetR^2$,可见变量只有一个$v$。
假设$p(\Vh) = \CalN(\Vh;0,\MI)$以及$p(v\mid\Vh) = \CalN(v;\Vw^{\top}\Vh;1)$,我们可以积掉$\Vh$来简化这个模型,结果是关于$v$的\gls{gaussian_distribution}。
这个模型本身并不有趣。
只是为了说明\gls{calculus_of_variations}如何应用在概率建模之中,我们才构造了这个模型。



% 640  end
忽略归一化常数时,真实的后验分布如下:
\begin{align}
	\label{eqn:1957}
   & p(\Vh\mid\Vv)\\
 \propto & p(\Vh, \Vv)\\
 = & p(h_1) p(h_2) p(\Vv\mid\Vh)\\
 \propto & \exp \big(-\frac{1}{2} [h_1^2 + h_2^2 + (v-h_1w_1 - h_2w_2)^2]\big)\\
 = & \exp \big( - \frac{1}{2} [h_1^2 + h_2^2 + v^2 + h_1^2w_1^2 + h_2^2w_2^2 - 2vh_1w_1 - 2vh_2w_2 + 2h_1w_1h_2w_2] \big).
\end{align}
在上式中,我们发现由于带有$h_1,h_2$乘积项的存在,真实的后验并不能关于$h_1,h_2$分解。
% 641 head 



应用\eqnref{eqn:1956},我们可以得到
\begin{align}
\label{eqn:1962}
& \tilde{q}(h_1\mid\Vv)\\ 
= & \exp\big(\SetE_{\RSh_2\sim q(\RSh_2\mid\Vv)} \log \tilde{p}(\Vv,\Vh)\big) \\
= & \exp\Big( -\frac{1}{2} \SetE_{\RSh_2\sim q(\RSh_2\mid\Vv)} [h_1^2 + h_2^2 + v^2 + h_1^2w_1^2 + h_2^2w_2^2 \\
& -2vh_1w_1 - 2vh_2w_2 + 2h_1w_1h_2w_2]\Big).
\end{align}
% 641 
从这里,我们可以发现其中我们只需要从$q(h_2\mid \Vv)$中获得两个有效值:
$\SetE_{\RSh_2\sim q(\RSh\mid\Vv)}[h_2]$和$\SetE_{\RSh_2\sim q(\RSh\mid\Vv)}[h_2^2]$。
把这两项记作$\langle h_2 \rangle$和$\langle h_2^2 \rangle$,我们可以得到:
\begin{align}
\label{eqn:1966}
\tilde{q}(h_1\mid\Vv) = & \exp(-\frac{1}{2} [h_1^2 + \langle h_2^2 \rangle  + v^2 + h_1^2w_1^2 + \langle h_2^2 \rangle w_2^2 
\\ &	-2vh_1w_1 - 2v\langle h_2 \rangle w_2 + 2h_1w_1\langle h_2 \rangle w_2]).	
\end{align}
% 641 


从这里,我们可以发现$\tilde{q}$的\gls{functional}形式满足\gls{gaussian_distribution}。
因此,我们可以得到$q(\Vh\mid\Vv) = \CalN(\Vh;\Vmu,\Vbeta^{-1})$,其中$\Vmu$和对角的$\Vbeta$是变分参数,我们可以使用任何方法来优化它。
%有必要再强调一下,我们并没有假设$q$是一个\gls{gaussian_distribution},这个高斯的形式是使用\gls{calculus_of_variations}来最大化关于$\CalL$的分布$q$\footnote{此处似乎有笔误。}推导出的。    %?? 什么笔误啊。。。  最大化L关于q 还是 最大化q关于L
有必要再强调一下,我们并没有假设$q$是一个\gls{gaussian_distribution},这个高斯的形式是使用\gls{calculus_of_variations}来关于分布$q$最大化$\CalL$而推导出来的。
在不同的模型上应用相同的方法可能会得到不同\gls{functional}形式的分布$q$。
% 641

当然,上述模型只是为了说明情况的一个简单例子。
\gls{DL}中关于变分学习中连续型变量的实际应用可以参考~\citet{Goodfeli-et-al-TPAMI-Deep-PrePrint-2013-small}。
% 641



\subsection{学习和推断之间的相互作用}
\label{sec:interactions_between_learning_and_inference}
% 641 end  19.4.4  


在学习算法中使用\gls{approximate_inference}会影响学习的过程,反过来学习的过程也会影响推断算法的准确性。
% 641 end


具体来说,训练算法倾向于朝使得\gls{approximate_inference}算法中的近似假设变得更加真实的方向来适应模型。
当训练参数时,变分学习增加
\begin{align}
\label{eqn:1968}
	\SetE_{\RVh \sim q}\log p(\Vv,\Vh).
\end{align}
% 642 
对于一个特定的$\Vv$,对于$q(\Vh \mid \Vv)$中概率很大的$\Vh$它增加了$p(\Vh\mid\Vv)$;对于$q(\Vh \mid \Vv)$中概率很小的$\Vh$它减小了$p(\Vh\mid\Vv)$。
% 642

这种行为使得我们做的近似假设变得合理。 %这种行为使我们的近似假设成为自我实现。
如果我们用\gls{unimodal}近似后验来训练模型,那么所得具有真实后验的模型会比我们使用精确推断训练模型获得的模型更接近\gls{unimodal}。  %?? <bad>
% 642

	
因此,估计变分近似对模型的破坏程度是很困难的。
存在几种估计$\log p(\Vv)$的方式。
通常我们在训练模型之后估计$\log p(\Vv;\Vtheta)$,然后发现它和$\CalL(\Vv,\Vtheta,q)$的差距是很小的。
从这里我们可以得出结论,对于特定的从学习过程中获得的$\Vtheta$来说,变分近似是很准确的。
然而我们无法直接得到变分近似普遍很准确或者变分近似几乎不会对学习过程产生任何负面影响这样的结论。
为了准确衡量变分近似带来的危害,我们需要知道$\Vtheta^* = \max_{\Vtheta} \log p(\Vv;\Vtheta)$。
$\CalL(\Vv,\Vtheta,q)\approx \log p(\Vv;\Vtheta)$和$\log p(\Vv;\Vtheta)\ll \log p(\Vv;\Vtheta^*)$同时成立是有可能的。
如果存在$\max_q \CalL(\Vv,\Vtheta^*,q)\ll \log p(\Vv;\Vtheta^*)$,即在$\Vtheta^*$点处后验分布太过复杂使得$q$分布族无法准确描述,那么学习过程永远无法到达$\Vtheta^*$。
这样的一类问题是很难发现的,因为只有在我们有一个能够找到$\Vtheta^*$的较好的学习算法时,才能确定地进行上述的比较。
% 642



\section{\glsentrytext{learned}\glsentrytext{approximate_inference}}
\label{sec:learned_approximate_inference}
% 642  end    19.5

我们已经看到了推断可以被视作一个增加函数$\CalL$值的优化过程。
显式地通过迭代方法(比如\gls{fixed_point_equation}或者基于梯度的优化算法)来进行优化的过程通常是代价很高且耗时巨大的。
通过学习一个近似推断,许多推断算法避免了这种代价。
具体地说,我们可以将优化过程视作将一个输入$\Vv$投影到一个近似分布$q^* = \arg\max_q\  \CalL(\Vv,q)$的一个$f$的函数。
一旦我们将多步的迭代优化过程看作是一个函数,我们可以用一个近似函数为$\hat{f}(\Vv;{\Vtheta})$的\gls{NN}来近似它。


\subsection{\glsentrytext{wake_sleep}算法}  
%?? wake_sleep 翻译成  醒眠  你斟酌下
\label{sec:wake_sleep}
% 643 head 19.5.1

训练一个可以用$\Vv$来推断$\Vh$的模型的一个主要难点在于我们没有一个\gls{supervised}训练集来训练模型。
给定一个$\Vv$,我们无法获知一个合适的$\Vh$。
从$\Vv$到$\Vh$的映射依赖于模型族的选择,并且在学习过程中随着${\Vtheta}$的改变而变化。
\firstgls{wake_sleep}算法~\citep{Hinton95,Frey96}通过从模型分布中抽取$\Vv$和$\Vh$的样本来解决这个问题。
例如,在\gls{directed_model}中,这可以通过执行从$\Vh$开始并在$\Vv$结束的\gls{ancestral_sampling}来高效地完成。
然后这个推断网络可以被训练来执行反向的映射:预测哪一个$\Vh$产生了当前的$\Vv$。
%<bad>这种方法的主要缺点是我们将只能够训练推断网络在模型下具有高概率的$\Vv$值。
这种方法的主要缺点是我们将只能在那些在当前模型上有较高概率的$\Vv$值上训练推断网络。
在学习早期,模型分布与数据分布偏差较大,因此推断网络将不具有在类似数据的样本上学习的机会。
% 643 mid


在\secref{sec:stochastic_maximum_likelihood_and_contrastive_divergence}中,我们看到睡眠做梦在人类和动物中作用的一个可能解释是,做梦可以提供\gls{monte_carlo}训练算法用于近似\gls{undirected_model}中对数\gls{partition_function}负梯度的\gls{negative_phase}样本。
生物做梦的另一个可能解释是它提供来自$p(\Vh,\Vv)$的样本,这可以用于训练推断网络在给定$\Vv$的情况下预测$\Vh$。
在某些意义上,这种解释比\gls{partition_function}的解释更令人满意。
如果\gls{monte_carlo}算法仅使用梯度的\gls{positive_phase}运行几个步骤,然后仅对梯度的\gls{negative_phase}运行几个步骤,那么结果通常不会很好。
人类和动物通常连续清醒几个小时,然后连续睡着几个小时。
这个时间表如何支持\gls{undirected_model}的\gls{monte_carlo}训练尚不清楚。
然而,基于最大化$\CalL$的学习算法可以通过长时间调整改进$q$和长期调整${\Vtheta}$来实现。
如果生物做梦的作用是训练网络来预测$q$,那么这解释了动物如何能够保持清醒几个小时
(它们清醒的时间越长,$\CalL$和$\log p(\Vv)$之间的差距越大, 但是$\CalL$仍然是下限)
并且睡眠几个小时(\gls{generative_model}本身在睡眠期间不被修改), 而不损害它们的内部模型。
当然,这些想法纯粹是猜测性的,没有任何确定的证据表明做梦实现了这些目标之一。
做梦也可以通过从动物的过渡模型(用来训练动物策略)采样合成经验来服务于\gls{RL}而不是概率建模。
也许睡眠可以服务于一些\gls{ML}社区尚未发现的其他目的。
% 644 head



\subsection{\glsentrytext{learned}推断的其他形式}
\label{sec:other_forms_of_learned_inference}
% 644    19.5.2 

这种\gls{learned}\gls{approximate_inference}策略已经被应用到了其他模型中。
\citet{Salakhutdinov+Larochelle-2010}证明了在\gls{learned}推断网络中的单遍传递相比于在\gls{DBM}中的迭代\gls{mean_field}\gls{fixed_point_equation}能够得到更快的推断。
其训练过程是基于运行推断网络的,然后运行一步\gls{mean_field}来改进其估计,并训练推断网络来输出这个更精细的估计以代替其原始估计。
% 644


我们已经在\secref{sec:predictive_sparse_decomposition}中看到,预测性的稀疏分解模型训练一个浅层\gls{encoder}网络,从而预测输入的\gls{sparse_coding}。
这可以被看作是\gls{AE}和\gls{sparse_coding}之间的混合。
为模型设计概率语义是可能的,其中\gls{encoder}可以被视为执行\gls{learned}近似~\glssymbol{MAP}~推断。
由于其浅层的\gls{encoder},PSD不能实现我们在\gls{mean_field}推断中看到的单元之间的那种竞争。
然而,该问题可以通过训练深度\gls{encoder}实现\gls{learned}\gls{approximate_inference}来补救,如ISTA技术~\citep{Gregor+LeCun-ICML2010}。
% 644


近来\gls{learned}\gls{approximate_inference}已经成为了\gls{VAE}形式的\gls{generative_model}中的主要方法之一 \citep{Kingma-arxiv2013,Rezende-et-al-ICML2014}。
在这种优美的方法中,不需要为推断网络构造显式的目标。
反之,推断网络仅仅被用来定义$\CalL$,然后调整推断网络的参数来增大$\CalL$。
我们将在\secref{sec:variational_autoencoders}中详细介绍这种模型。
% 644

我们可以使用\gls{approximate_inference}来训练和使用很多不同的模型。
其中许多模型将在下一章中描述。
% 644

\chapter{深度\glsentrytext{generative_model}}
\label{chap:deep_generative_models}
在本章中,我们介绍几种具体的\gls{generative_model},这些模型可以使用\chapref{chap:structured_probabilistic_models_for_deep_learning}至\chapref{chap:approximate_inference}中出现的技术构建和训练。
所有这些模型在某种程度上都代表了多个变量的概率分布。
有些模型允许显式地计算概率分布函数。
其他模型则不允许直接评估概率分布函数,但支持隐式获取分布知识的操作,如从分布中采样。
这些模型中的一部分使用\chapref{chap:structured_probabilistic_models_for_deep_learning}中的\gls{graphical_model}语言,从图和\gls{factor}的角度描述为\gls{structured_probabilistic_models}。
其他的不能简单地从因子角度描述,但仍然代表概率分布。


\section{\glsentrytext{BM}}
\label{sec:boltzmann_machines}
\gls{BM}最初作为一种广义的``\gls{connectionism}''引入,用来学习二值向量上的任意概率分布~\citep{Fahlman83,Ackley85,Hinton-Boltzmann,Hinton86a}。
\gls{BM}的变体(包含其他类型的变量)早已超过了原始\gls{BM}的流行程度。
在本节中,我们简要介绍二值\gls{BM}并讨论训练模型和进行\gls{inference}时出现的问题。

我们在$d$维二值随机向量$\Vx \in \{0, 1\}^d$上定义\gls{BM}。
\gls{BM}是一种\gls{energy_based_model}(\secref{sec:energy_based_models}),意味着我们可以使用\gls{energy_function}定义\gls{joint_probability_distribution}:
\begin{align}\label{eq:201px}
 P(\Vx) = \frac{\exp(-E(\Vx))}{Z},
\end{align}
其中$E(\Vx)$是\gls{energy_function},$Z$是确保$\sum_{\Vx} P(\Vx)=1$的\gls{partition_function}。
\gls{BM}的\gls{energy_function}如下
给出:
\begin{align}\label{eq:202E}
 E(\Vx) = -\Vx^\top \MU \Vx - \Vb^\top \Vx,
\end{align}
其中$\MU$是模型参数的``权重''矩阵,$\Vb$是\gls{bias_aff}向量。

% -- 645 --

在一般设定下,给定一组训练样本,每个样本都是$n$维的。
\eqnref{eq:201px}描述了观察到的变量的\gls{joint_probability_distribution}。
虽然这种情况显然可行,但它限制了观察到的变量和权重矩阵描述的变量之间相互作用的类型。
具体来说,这意味着一个单元的概率由其他单元值的\gls{linear_model}(\gls{logistic_regression})给出。

当不是所有变量都能被观察到时,\gls{BM}变得更强大。
在这种情况下,\gls{latent_variable}类似于\gls{MLP}中的\gls{hidden_unit},并模拟可见单元之间的高阶交互。
正如添加\gls{hidden_unit}将\gls{logistic_regression}转换为~\glssymbol{MLP},导致~\glssymbol{MLP}~成为函数的\gls{universal_approximator},具有\gls{hidden_unit}的\gls{BM}不再局限于建模变量之间的线性关系。
相反,\gls{BM}变成了离散变量上\gls{PMF}的\gls{universal_approximator} \citep{LeRoux-Bengio-2008}。


正式地,我们将单元$\Vx$分解为两个子集:可见单元$\Vv$和\gls{latent}(或隐藏)单元$\Vh$。
\gls{energy_function}变为
\begin{align}
 E(\Vv, \Vh) = -\Vv^\top \MR \Vv - \Vv^\top \MW \Vh - \Vh^\top \MS \Vh - \Vb^\top \Vv - \Vc^\top \Vh.
\end{align}
\paragraph{\gls{BM}的学习}\gls{BM}的学习算法通常基于最大似然。
所有\gls{BM}都具有难以处理的\gls{partition_function},因此最大似然梯度必须使用\chapref{chap:confronting_the_partition_function}中的技术来近似。

% -- 646 --

\gls{BM}有一个有趣的性质,当基于最大似然的学习规则训练时,连接两个单元的特定权重的更新仅取决于这两个单元在不同分布下收集的统计信息:$P_{\text{model}}(\Vv)$和$\hat{P}_{\text{data}}(\Vv) P_{\text{model}}(\Vh  \mid  \Vv)$。
网络的其余部分参与\gls{shaping}这些统计信息,但权重可以在完全不知道网络其余部分或这些统计信息如何产生的情况下更新。
这意味着学习规则是``局部''的,这使得\gls{BM}的学习似乎在某种程度上是生物学合理的。
我们可以设想每个神经元都是\gls{BM}中随机变量的情况,那么连接两个随机变量的轴突和树突只能通过观察与它们物理上实际接触细胞的激发模式来学习。
特别地,\gls{positive_phase}期间,经常同时激活的两个单元之间的连接会被加强。
这是~\ENNAME{Hebbian}~学习规则\citep{Hebb49}的一个例子 ,经常总结为好记的短语——``fire together, wire together''。
\ENNAME{Hebbian}~学习规则是生物系统学习中最古老的假设性解释之一,直至今天仍然有重大意义 \citep{DelGuidice-et-al-2009}。


不仅仅使用局部统计信息的其他学习算法似乎需要假设更多的学习机制。
例如,对于大脑在\gls{MLP}中实现的\gls{back_propagation},似乎需要维持一个辅助通信的网络,并借此向后传输梯度信息。
已经有学者\citep{Hinton-DL2007,Bengio-arxiv2015} 提出生物学上可行(和近似)的\gls{back_propagation}实现方案,但仍然有待验证,\citet{Bengio-arxiv2015} 还将梯度的\gls{back_propagation}关联到类似于\gls{BM}(但具有连续\gls{latent_variable})的能量模型中的\gls{inference}。

从生物学的角度看,\gls{BM}学习中的\gls{negative_phase}阶段有点难以解释。
正如\secref{sec:stochastic_maximum_likelihood_and_contrastive_divergence}所主张的,人类在睡眠时做梦可能是一种形式的\gls{negative_phase}采样。
尽管这个想法更多的只是猜测。

\section{\glsentrytext{RBM}}
\label{sec:restricted_boltzmann_machines}
\gls{RBM}以\firstgls{harmonium}之名\citep{Smolensky86}面世之后,成为了深度概率模型中最常见的组件之一。
我们之前在\secref{sec:example_the_restricted_boltzmann_machine}简要介绍了~\glssymbol{RBM}。
在这里我们回顾以前的内容并探讨更多的细节。
\glssymbol{RBM}~是包含一层可观察变量和单层\gls{latent_variable}的无向概率\gls{graphical_model}。
\glssymbol{RBM}~可以堆叠起来(一个在另一个的顶部)形成更深的模型。
\figref{fig:chap20_dbn}展示了一些例子。
特别地, \figref{fig:chap20_dbn}a显示~\glssymbol{RBM}~本身的图结构。
它是一个二分图,观察层或\gls{latent_layer}中的任何单元之间不允许存在连接。

% -- 647 --

\begin{figure}[!htb]
\ifOpenSource
\centerline{\includegraphics{figure.pdf}}
\else
\centering
\begin{tabular}{cc}
\includegraphics{Chapter20/figures/squished_rbm.pdf} &
\includegraphics{Chapter20/figures/dbn.pdf}\\
(a)&(b)
\end{tabular}
\begin{tabular}{c}
 \includegraphics{Chapter20/figures/dbm_simple.pdf}\\
(c)
\end{tabular}
\fi
\caption{可以用\gls{RBM}构建的模型示例。
(a)\gls{RBM}本身是基于二分图的无向\gls{graphical_model},在图的一部分具有可见单元,另一部分具有\gls{hidden_unit}。可见单元之间没有连接,\gls{hidden_unit}之间也没有任何连接。通常每个可见单元连接到每个\gls{hidden_unit},但也可以构造稀疏连接的~\glssymbol{RBM},如卷积~\glssymbol{RBM}。
(b)\gls{DBN}是涉及有向和无向连接的混合\gls{graphical_model}。
与~\glssymbol{RBM}~一样,它也没有层内连接。
然而,\glssymbol{DBN}~具有多个\gls{hidden_layer},因此\gls{hidden_unit}之间的连接在分开的层中。
\gls{DBN}所需的所有局部条件概率分布都直接复制~\glssymbol{RBM}~的局部条件概率分布。
或者,我们也可以用完全无向图表示\gls{DBN},但是它需要层内连接来捕获父节点间的依赖关系。
(c)\gls{DBM}是具有几层\gls{latent_variable}的无向\gls{graphical_model}。
与~\glssymbol{RBM}~和~\glssymbol{DBN}~一样,\glssymbol{DBM}~也缺少层内连接。
\glssymbol{DBM}~与~\glssymbol{RBM}~的联系不如~\glssymbol{DBN}~紧密。
当从~\glssymbol{RBM}~堆栈初始化~\glssymbol{DBM}~时,有必要对~\glssymbol{RBM}~的参数稍作修改。
某些种类的~\glssymbol{DBM}~可以直接训练,而不用先训练一组~\glssymbol{RBM}。
}
\label{fig:chap20_dbn}
\end{figure}

% -- 648 --

我们从二值版本的\gls{RBM}开始,但如我们之后所见,这还可以扩展为其他类型的可见和\gls{hidden_unit}。

更正式地说,令观察层由一组$n_v$个二值随机变量组成,我们统称为向量$\RVv$。
我们将$n_h$个二值随机变量的\gls{latent}或\gls{hidden_layer}记为$\Vh$。

就像普通的\gls{BM},\gls{RBM}也是\gls{energy_based_model},其\gls{joint_probability_distribution}由\gls{energy_function}指定:
\begin{align}
 P(\RVv = \Vv, \RVh = \Vh) = \frac{1}{Z} \exp(-E(\Vv, \Vh)).
\end{align}
RBM的\gls{energy_function}由下给出
\begin{align} \label{eq:205e}
 E(\Vv, \Vh) = -\Vb^\top \Vv - \Vc^\top \Vh - \Vv^\top \MW \Vh,
\end{align}
其中$Z$是被称为\gls{partition_function}的归一化常数:
\begin{align}
 Z = \sum_{\Vv} \sum_{\Vh} \exp \{-E(\Vv, \Vh) \}
\end{align}
从\gls{partition_function} $Z$的定义显而易见,计算$Z$的朴素方法(对所有状态进行穷举求和)计算上可能是难以处理的,除非有巧妙设计的算法可以利用概率分布中的规则来更快地计算$Z$。
在\gls{RBM}的情况下, \citet{long10rbm}正式证明\gls{partition_function} $Z$是难解的。
难解的\gls{partition_function} $Z$意味着归一化\gls{joint_probability_distribution} $P(\Vv)$也难以评估。


\subsection{条件分布}
\label{sec:conditional_distributions_chap20}
虽然$P(\Vv)$难解,但~\glssymbol{RBM}~的二分图结构具有非常特殊的性质,其条件分布$P(\RVh \mid \RVv)$和$P(\RVv\mid\RVh)$是\gls{factorial}的,并且计算和采样是相对简单的。

% -- 649 --

从联合分布中导出条件分布是直观的:
\begin{align}
 P(\Vh \mid \Vv) &= \frac{P(\Vh, \Vv)}{P(\Vv)} \\
 &= \frac{1}{P(\Vv)} \frac{1}{Z} \exp \big\{ \Vb^\top \Vv + \Vc^\top \Vh + \Vv^\top\MW \Vh \big\} \\
 &=  \frac{1}{Z^\prime} \exp \big\{ \Vc^\top \Vh + \Vv^\top\MW \Vh \big\} \\
 &=  \frac{1}{Z^\prime} \exp \Big\{ \sum_{j=1}^{n_h}\Vc_j^\top \Vh_j 
 + \sum_{n_h}^{j=1} \Vv^\top\MW_{:,j} \Vh_j \Big\} \\
 &=  \frac{1}{Z^\prime} \prod_{j=1}^{n_h} \exp \big\{ \Vc_j^\top \Vh_j + \Vv^\top\MW_{:,j} \Vh_j \big\} .
\end{align}
由于我们相对可见单元$\RVv$计算条件概率,相对于分布$P(\RVh  \mid  \RVv)$我们可以将它们视为常数。
条件分布$ P(\RVh  \mid  \RVv) $ \gls{factorial}相乘的本质,我们可以将向量$\Vh$上的联合概率写成单独元素$h_j$上(未归一化)分布的乘积。
现在原问题变成了对单个二值$h_j$上的分布进行归一化的简单问题。
\begin{align}
 P(h_j=1  \mid  \Vv) &= \frac{\tilde{P}(h_j=1 \mid \Vv)}{\tilde{P}(h_j=0 \mid \Vv) + \tilde{P}(h_j=1 \mid \Vv)} \\
 &= \frac{\exp \{ c_j + \Vv^\top \MW_{:,j} \} }{\exp\{ 0 \} + \exp \{ c_j + \Vv^\top \MW_{:,j}\}} \\
 &= \sigma (c_j + \Vv^\top \MW_{:,j}).
\end{align}
现在我们可以将关于\gls{hidden_layer}的完全条件分布表达为\gls{factorial}形式:
\begin{align}
 P(\Vh  \mid  \Vv) = \prod_{j=1}^{n_h} \sigma \big( (2\Vh-1) \odot (\Vc + \MW^\top \Vv) \big)_j .
\end{align}

类似的推导将显示我们感兴趣的另一条件分布,$P(\Vv  \mid  \Vh)$也是\gls{factorial}形式的分布:
\begin{align}
  P(\Vv  \mid  \Vh) = \prod_{i=1}^{n_v} \sigma \big( (2\Vv-1) \odot (\Vb + \MW \Vh) \big)_i .
\end{align}


\subsection{训练\glsentrytext{RBM}}
\label{sec:training_restricted_boltzmann_machines}

%<BAD>因为~\glssymbol{RBM}~允许以高效~\glssymbol{mcmc}~采样(\gls{block_gibbs_sampling}的形式)对$\tilde{P}(\Vv)$进行高效评估和求导,所以可以简单地使用\chapref{chap:confronting_the_partition_function}中描述的任意训练具有难解\gls{partition_function}模型的技术。
因为~\glssymbol{RBM}~允许高效计算$\tilde{P}(\Vv)$的估计和微分,并且还允许高效地(以\gls{block_gibbs_sampling}的形式)进行\glssymbol{mcmc}~采样,所以我们很容易使用\chapref{chap:confronting_the_partition_function}中训练具有难以计算\gls{partition_function}的模型的技术来训练~\glssymbol{RBM}。
这包括~\glssymbol{contrastive_divergence}、\,\glssymbol{SML}(\glssymbol{persistent_contrastive_divergence})、\gls{ratio_matching}等。
与深度学习中使用的其他\gls{undirected_model}相比,\glssymbol{RBM}~可以相对直接地训练,因为我们可以以闭解形式计算$P(\RVh  \mid  \Vv)$。
其他一些深度模型,如\gls{DBM},同时具备难处理的\gls{partition_function}和难以推断的难题。

% -- 650 --

\section{\glsentrytext{DBN}}
\label{sec:deep_belief_networks}

\firstall{DBN}是第一批成功应用深度架构训练的非卷积模型之一\citep{Hinton06,hinton2007learning}。
2006年\gls{DBN}的引入开始了当前深度学习的复兴。
在引入\gls{DBN}之前,深度模型被认为太难以优化。
具有凸\gls{objective_function}的\gls{kernel_machines}引领了研究前沿。
\gls{DBN}在MNIST数据集上表现超过内核化支持向量机,以此证明深度架构是能够成功的\citep{Hinton06}。
尽管现在与其他无监督或生成学习算法相比,\gls{DBN}大多已经失去了青睐并很少使用,但它们在深度学习历史中的重要作用仍应该得到承认。

\gls{DBN}是具有若干\gls{latent_variable}层的\gls{generative_model}。
\gls{latent_variable}通常是二值的,而可见单元可以是二值或实数。
尽管构造连接比较稀疏的~\glssymbol{DBN}~是可能的,但在一般的模型中,每层的每个单元连接到每个相邻层中的每个单元(没有层内连接)。
顶部两层之间的连接是无向的。
而所有其他层之间的连接是有向的,箭头指向最接近数据的层。
见\figref{fig:chap20_dbn}b的例子。


具有$l$个\gls{hidden_layer}的~\glssymbol{DBN}~包含$l$个权重矩阵:$\MW^{(1)},\ldots, \MW^{(l)}$。
同时也包含$l+1$个\gls{bias_aff}向量:
$\Vb^{(0)},\ldots,\Vb^{(l)}$,其中$\Vb^{(0)}$是可见层的\gls{bias_aff}。
\glssymbol{DBN}~表示的概率分布由下式给出:
\begin{align}
 P(\Vh^{(l)}, \Vh^{(l-1)}) \propto& \exp \big( \Vb^{(l)^\top} \Vh^{(l)} +  \Vb^{(l-1)^\top} \Vh^{(l-1)}
 + \Vh^{(l-1)^\top} \MW^{(l)} \Vh^{(l)} \big), \\
 P(h_i^{(k)} = 1  \mid  \Vh^{(k+1)}) &= \sigma \big( b_i^{(k)} + \MW_{:,i}^{(k+1)^\top} \Vh^{(k+1)} 
                                                          \big)~ \forall i,  \forall k \in 1, \ldots, l-2, \\
P(v_i = 1  \mid  \Vh^{(1)}) &=  \sigma \big( b_i^{(0)} + \MW_{:,i}^{(1)^\top} \Vh^{(1)} 
                                                          \big)~ \forall i.
\end{align}
在实值可见单元的情况下,替换
\begin{align}
 \RVv \sim \CalN \big( \Vv; \Vb^{(0)} + \MW^{(1)^\top} \Vh^{(1)}, \Vbeta^{-1} \big)
\end{align}
为便于处理,$\Vbeta$为对角形式。
至少在理论上,推广到其他指数族的可见单元是直观的。
只有一个\gls{hidden_layer}的~\glssymbol{DBN}~只是一个~\glssymbol{RBM}。

% -- 651 --

为了从~\glssymbol{DBN}~中生成样本,我们先在顶部的两个\gls{hidden_layer}上运行几个~\gls{gibbs_sampling}步骤。
这个阶段主要从~\glssymbol{RBM}(由顶部两个\gls{hidden_layer}定义)中采一个样本。
然后,我们可以对模型的其余部分使用单次\gls{ancestral_sampling},以从可见单元绘制样本。

\gls{DBN}引发许多与\gls{directed_model}和\gls{undirected_model}同时相关的问题。

% What ?
由于每个有向层内的\gls{explaining_away}效应,并且由于无向连接的两个\gls{hidden_layer}之间的相互作用,\gls{DBN}中的\gls{inference}是难解的。
评估或最大化对数似然的标准\gls{ELBO}也是难以处理的,因为\gls{ELBO}基于大小等于网络宽度的\gls{clique}的期望。

评估或最大化对数似然,不仅需要面对边缘化\gls{latent_variable}时难以处理的\gls{inference}问题,而且还需要处理顶部两层\gls{undirected_model}内难处理的\gls{partition_function}问题。

为训练\gls{DBN},我们可以先使用\gls{contrastive_divergence}或\gls{SML}方法训练~\glssymbol{RBM}~以最大化$ \SetE_{\RVv \sim p_{\text{data}}} \log p(\Vv)$。
\,\glssymbol{RBM}~的参数定义了~\glssymbol{DBN}~第一层的参数。
然后,第二个~\glssymbol{RBM}~训练为近似最大化
\begin{align}
 \SetE_{\RVv \sim p_{\text{data}}}  \SetE_{\RVh^{(1)} \sim p^{(1)}(\Vh^{(1)}  \mid  \Vv)}  \log p^{(2)}(\Vh^{(1)}) ,
\end{align}
其中$p^{(1)}$是第一个~\glssymbol{RBM}~表示的概率分布,$p^{(2)}$是第二个~\glssymbol{RBM}~表示的概率分布。
换句话说,第二个~\glssymbol{RBM}~被训练为模拟由第一个~\glssymbol{RBM}~的\gls{hidden_unit}采样定义的分布,而第一个~\glssymbol{RBM}~由数据驱动。
这个过程能无限重复,从而向~\glssymbol{DBN}~添加任意多层,其中每个新的~\glssymbol{RBM}~对前一个~\glssymbol{RBM}~的样本建模。
每个~\glssymbol{RBM}~定义~\glssymbol{DBN}~的另一层。
这个过程可以被视为提高数据在~\glssymbol{DBN}~下似然概率的变分下界\citep{Hinton06}。


在大多数应用中,对~\glssymbol{DBN}~进行贪心逐层训练后,不需要再花功夫对其进行联合训练。
然而,使用\gls{wake_sleep}算法对其进行生成\gls{fine_tuning}是可能的。

% -- 652 --

训练好的~\glssymbol{DBN}~可以直接用作\gls{generative_model},但是~\glssymbol{DBN}~的大多数兴趣来自于它们改进分类模型的能力。
我们可以从~\glssymbol{DBN}~获取权重,并使用它们定义~\glssymbol{MLP}:
\begin{align}
 \Vh^{(1)} &= \sigma \big( b^{(1)} + \Vv^\top \MW^{(1)} \big), \\
 \Vh^{(l)} &= \sigma \big( b_i^{(l)} + \Vh^{(l-1)^\top}\MW^{(l)} \big) ~\forall l \in 2, \ldots, m.
\end{align}
利用~\glssymbol{DBN}~的生成训练后获得的权重和\gls{bias_aff}初始化该~\glssymbol{MLP}~之后,我们可以训练该~\glssymbol{MLP}~来执行分类任务。
这种~\glssymbol{MLP}~的额外训练是判别性\gls{fine_tuning}的示例。


与\chapref{chap:approximate_inference}中从基本原理导出的许多\gls{inference}方程相比,这种特定选择的~\glssymbol{MLP}~有些随意。
这个~\glssymbol{MLP}~是一个启发式选择,似乎在实践中效果不错,并在文献中一贯使用。
许多近似\gls{inference}技术是由它们在一些约束下,并在对数似然上找到最大\emph{紧}变分下界的能力所驱动的。
我们可以使用~\glssymbol{DBN}~中~\glssymbol{MLP}~定义的\gls{hidden_unit}的期望,构造对数似然的变分下界,但这对于\gls{hidden_unit}上的\emph{任何}概率分布都是如此,并没有理由相信该~\glssymbol{MLP}~提供了一个特别的紧界。
特别地,\glssymbol{MLP}~忽略了~\glssymbol{DBN}~\gls{graphical_model}中许多重要的相互作用。
\glssymbol{MLP}~将信息从可见单元向上传播到最深的\gls{hidden_unit},但不向下或侧向传播任何信息。
\glssymbol{DBN}~\gls{graphical_model}解释了同一层内所有\gls{hidden_unit}之间的相互作用以及层之间的自顶向下的相互作用。


虽然~\glssymbol{DBN}~的对数似然是难处理的,但它可以使用~\glssymbol{AIS}~近似\citep{Salakhutdinov+Murray-2008}。
通过近似,可以评估其作为\gls{generative_model}的质量。


术语``\gls{DBN}''通常不正确地用于指代任意种类的\gls{DNN},甚至没有\gls{latent_variable}意义的网络。
这个术语应特指最深层中具有无向连接,而在所有其他连续层之间存在向下有向连接的模型。

这个术语也可能导致一些混乱,因为术语``信念网络''有时指纯粹的\gls{directed_model},而\gls{DBN}包含一个无向层。
\gls{DBN}也与动态贝叶斯网络(dynamic Bayesian networks) \citep{Dean+Kanazawa-1989}共享首字母缩写~\glssymbol{DBN},动态贝叶斯网络表示\gls{markov_chain}的\gls{bayesian_network}。

% -- 653 --

\section{\glsentrytext{DBM}}
\label{sec:deep_boltzmann_machines}

\firstall{DBM} \citep{SalHinton09}是另一种深度\gls{generative_model}。
与\glsacr{DBN}不同的是,它是一个完全无向的模型。
与~\glssymbol{RBM}~不同的是,\glssymbol{DBM}~有几层\gls{latent_variable}(\glssymbol{RBM}~只有一层)。
但是像~\glssymbol{RBM}~一样,每一层内的每个变量是相互独立的,并条件于相邻层中的变量。
见\figref{fig:chap20_dbm_simple}中的图结构。
\gls{DBM}已经被应用于各种任务,包括文档建模\citep{srivastava2013modeling}。

\begin{figure}[!htb]
\ifOpenSource
\centerline{\includegraphics{figure.pdf}}
\else
\centerline{\includegraphics{Chapter20/figures/dbm_simple}}
\fi
\caption{具有一个可见层(底部)和两个\gls{hidden_layer}的\gls{DBM}的\gls{graphical_model}。
仅在相邻层的单元之间存在连接。
没有层内连接。}
\label{fig:chap20_dbm_simple}
\end{figure}

与~\glssymbol{RBM}~和~\glssymbol{DBN}~一样,\glssymbol{DBM}~通常仅包含二值单元 (正如我们为简化模型的演示而假设的),但很容易就能扩展到实值可见单元。

\glssymbol{DBM}~是\gls{energy_based_model},这意味着模型变量的\gls{joint_probability_distribution}由\gls{energy_function} $E$参数化。
在一个\gls{DBM}包含一个可见层$\Vv$和三个\gls{hidden_layer} $\Vh^{(1)},\Vh^{(2)}$和$\Vh^{(3)}$的情况下,联合概率由下式给出:
\begin{align}
 P(\Vv, \Vh^{(1)},  \Vh^{(2)},  \Vh^{(3)}) = \frac{1}{Z(\Vtheta)} 
 \exp \big( -E(\Vv, \Vh^{(1)},  \Vh^{(2)},  \Vh^{(3)}; \Vtheta) \big).
\end{align}
为简化表示,下式省略了\gls{bias_aff}参数。
\glssymbol{DBM}~\gls{energy_function}定义如下:
\begin{align}
    E(\Vv, \Vh^{(1)}, \Vh^{(2)}, \Vh^{(3)}; \Vtheta)  = -\Vv^\top \MW^{(1)}\Vh^{(1)} 
 - \Vh^{(1)^\top}\MW^{(2)}\Vh^{(2)}- \Vh^{(2)^\top}\MW^{(3)}\Vh^{(3)}.
\end{align}

与~\glssymbol{RBM}~的\gls{energy_function}(\eqnref{eq:205e})相比,\glssymbol{DBM}~\gls{energy_function}以权重矩阵($\MW^{(2)}$和$\MW^{(3)}$)的形式表示\gls{hidden_unit}(\gls{latent_variable})之间的连接。
正如我们将看到的,这些连接对模型行为以及我们如何在模型中进行\gls{inference}都有重要的影响。

% -- 654 --

\begin{figure}[!htb]
\ifOpenSource
\centerline{\includegraphics{figure.pdf}}
\else
\centerline{\includegraphics{Chapter20/figures/dbm_bipartite}}
\fi
\caption{\gls{DBM},重新排列后显示为二分图结构。}
\label{fig:chap20_dbm_bipartite}
\end{figure}

与全连接的\gls{BM}(每个单元连接到其他每个单元)相比,\glssymbol{DBM}~提供了类似于~\glssymbol{RBM}~的一些优点。

具体来说, 如\figref{fig:chap20_dbm_bipartite}所示,\glssymbol{DBM}~的层可以组织成一个二分图,其中奇数层在一侧,偶数层在另一侧。
容易发现,当我们条件于偶数层中的变量时,奇数层中的变量变得条件独立。   
当然,当我们条件于奇数层中的变量时,偶数层中的变量也会变得条件独立。

\glssymbol{DBM}~的二分图结构意味着我们可以应用之前用于~\glssymbol{RBM}~条件分布的相同式子来确定~\glssymbol{DBM}~中的条件分布。
在给定相邻层值的情况下,层内的单元彼此条件独立,因此二值变量的分布可以由~\ENNAME{Bernoulli}~参数(描述每个单元的激活概率)完全描述。
在具有两个\gls{hidden_layer}的示例中,激活概率由下式给出:
\begin{align}
 P(v_i=1  \mid  \Vh^{(1)}) &= \sigma \big( \MW_{i,:}^{(1)}\Vh^{(1)} \big), \\
 P(h_i^{(1)}=1  \mid  \Vv, \Vh^{(2)}) &= \sigma \big( \Vv^\top  \MW_{:,i}^{(1)}
 + \MW_{i,:}^{(2)}\Vh^{(2)} \big) ,
\end{align}
和
\begin{align}
P(h_k^{(2)} =1  \mid  \Vh^{(1)}) = \sigma \big(\Vh^{(1)\top} \MW_{:,k}^{(2)} \big).
\end{align}

% -- 655 --

二分图结构使~\gls{gibbs_sampling}能在\gls{DBM}中高效采样。
\gls{gibbs_sampling}的方法是一次只更新一个变量。
\glssymbol{RBM}~允许所有可见单元以一个块的方式更新,而所有\gls{hidden_unit}在另一个块上更新。
我们可以简单地假设具有$l$层的~\glssymbol{DBM}~需要$l+1$次更新,每次迭代更新由某层单元组成的块。
然而,我们可以仅在两次迭代中更新所有单元。
\gls{gibbs_sampling}可以将更新分成两个块,一块包括所有偶数层(包括可见层),另一个包括所有奇数层。
由于~\glssymbol{DBM}~二分连接模式,给定偶数层,关于奇数层的分布是\gls{factorial}的,因此可以作为块同时且独立地采样。
类似地,给定奇数层,可以同时且独立地将偶数层作为块进行采样。
高效采样对使用\gls{SML}算法的训练尤其重要。


\subsection{有趣的性质}
\gls{DBM}具有许多有趣的性质。

\glssymbol{DBM}~在~\glssymbol{DBN}~之后开发。
与~\glssymbol{DBN}~相比,\glssymbol{DBM}~的后验分布$P(\Vh  \mid  \Vv)$更简单。
有点违反直觉的是,这种后验分布的简单性允许更加丰富的后验近似。
在~\glssymbol{DBN}~的情况下,我们使用启发式的近似\gls{inference}过程进行分类,其中我们可以通过~\glssymbol{MLP}(使用~\ENNAME{sigmoid}~激活函数并且权重与原始~\glssymbol{DBN}~相同)中的向上传播猜测\gls{hidden_unit}合理的\gls{meanfield}期望值。
\emph{任何}分布$Q(\Vh)$可用于获得对数似然的变分下界。
因此这种启发式的过程让我们能够获得这样的下界。
但是,该界没有以任何方式显式优化,所以该界可能是远远不紧的。
特别地,$Q$的启发式估计忽略了相同层内\gls{hidden_unit}之间的相互作用以及更深层中\gls{hidden_unit}对更接近输入的\gls{hidden_unit}自顶向下的反馈影响。
因为~\glssymbol{DBN}~中基于启发式~\glssymbol{MLP}~的\gls{inference}过程不能考虑这些相互作用,所以得到的$Q$想必远不是最优的。
\glssymbol{DBM}~中,在给定其他层的情况下,层内的所有\gls{hidden_unit}都是条件独立的。
这种层内相互作用的缺失使得通过\gls{fixed_point_equation}优化变分下界并找到真正最佳的\gls{meanfield}期望(在一些数值容差内)变得可能的。

% -- 656 --

使用适当的\gls{meanfield}允许~\glssymbol{DBM}~的近似\gls{inference}过程捕获自顶向下反馈相互作用的影响。
这从神经科学的角度来看是有趣的,因为根据已知,人脑使用许多自上而下的反馈连接。
由于这个性质,\glssymbol{DBM}~已被用作真实神经科学现象的计算模型 \citep{series2010hallucinations,reichert2011neuronal}。


\glssymbol{DBM}~一个不理想的特性是从中采样是相对困难的。 
\glssymbol{DBN}~只需要在其顶部的一对层中使用~\glssymbol{mcmc}~采样。
其他层仅在采样过程末尾涉及,并且只需在一个高效的\gls{ancestral_sampling}过程。
要从~\glssymbol{DBM}~生成样本,必须在所有层中使用~\glssymbol{mcmc},并且模型的每一层都参与每个\gls{markov_chain}转移。


\subsection{\glssymbol{DBM}\glsentrytext{meanfield}\gls{inference}}
\label{sec:dbm_mean_field_inference}
给定相邻层,一个~\glssymbol{DBM}~层上的条件分布是\gls{factorial}的。
在有两个\gls{hidden_layer}的~\glssymbol{DBM}~的示例中,这些分布是$P(\Vv  \mid  \Vh^{(1)}), P(\Vh^{(1)}  \mid  \Vv, \Vh^{(2)})$和$P(\Vh^{(2)}  \mid  \Vh^{(1)})$。
因为层之间的相互作用,\emph{所有}\gls{hidden_layer}上的分布通常不是\gls{factorial}的。
在有两个\gls{hidden_layer}的示例中,由于$\Vh^{(1)}$和$\Vh^{(2)}$之间的交互权重$\MW^{(2)}$使得这些变量相互依赖, $ P(\Vh^{(1)}  \mid  \Vv, \Vh^{(2)})$不是\gls{factorial}的。


与~\glssymbol{DBN}~的情况一样,我们还是要找出近似~\glssymbol{DBM}~后验分布的方法。
然而,与~\glssymbol{DBN}~不同,\glssymbol{DBM}~在其\gls{hidden_unit}上的后验分布(复杂的) 很容易用变分近似来近似(如\secref{sec:variational_inference_and_learning}所讨论),具体是一个\gls{meanfield}近似。
\gls{meanfield}近似是变分\gls{inference}的简单形式,其中我们将近似分布限制为完全\gls{factorial}的分布。
在~\glssymbol{DBM}~的情况下,\gls{meanfield}方程捕获层之间的双向相互作用。
在本节中,我们推导出由~\cite{SalHinton09}最初引入的迭代近似\gls{inference}过程。

% -- 657 --

在\gls{inference}的变分近似中,我们通过一些相当简单的分布族近似特定目标分布——在这里指给定可见单元时\gls{hidden_unit}的后验分布。
在\gls{meanfield}近似的情况下,近似族是\gls{hidden_unit}条件独立的分布集合。


我们现在为具有两个\gls{hidden_layer}的示例推导\gls{meanfield}方法。
令$Q(\Vh^{(1)}, \Vh^{(2)}  \mid  \Vv)$为$P(\Vh^{(1)}, \Vh^{(2)}  \mid  \Vv)$的近似。
\gls{meanfield}假设意味着
\begin{align}
 Q(\Vh^{(1)}, \Vh^{(2)}  \mid  \Vv) = \prod_j Q(h_j^{(1)} \mid  \Vv) \prod_k Q(\Vh_k^{(2)}  \mid  \Vv).
\end{align}

\gls{meanfield}近似试图找到这个分布族中最适合真实后验$P(\Vh^{(1)}, \Vh^{(2)}  \mid  \Vv)$的成员。
重要的是,每次我们使用$\Vv$的新值时,必须再次运行\gls{inference}过程以找到不同的分布$Q$。


我们可以设想很多方法来衡量$Q(\Vh  \mid  \Vv)$与$P(\Vh  \mid  \Vv)$的拟合程度。
\gls{meanfield}方法是最小化
\begin{align}
 \text{KL}(Q \mid  \mid P) = \sum_{\Vh} Q(\Vh^{(1)}, \Vh^{(2)}  \mid  \Vv) 
 \log \Big( \frac{Q(\Vh^{(1)}, \Vh^{(2)}  \mid  \Vv)}{P(\Vh^{(1)}, \Vh^{(2)}  \mid  \Vv)} \Big).
\end{align}

一般来说,除了要保证独立性假设,我们不必提供参数形式的近似分布。
变分近似过程通常能够恢复近似分布的函数形式。
然而,在二值\gls{hidden_unit}(我们在这里推导的情况)的\gls{meanfield}假设的情况下,不会由于预先固定模型的参数而损失一般性。

我们将$Q$作为~\gls{bernoulli_distribution}的乘积进行参数化,即我们将$\Vh^{(1)}$每个元素的概率与一个参数相关联。
具体来说,对于每个$j$,$\hat h_j^{(1)} = Q(h_j^{(1)}=1 \mid  \Vv)$,其中$\hat h_j^{(1)} \in [0,1]$。
另外,对于每个$k$,$\hat h_k^{(2)} = Q(h_k^{(2)}=1 \mid  \Vv)$,其中$\hat h_k^{(2)} \in [0,1]$。
因此,我们有以下近似后验:
\begin{align}
 Q(\Vh^{(1)}, \Vh^{(2)}  \mid  \Vv) &=  \prod_j Q(h_j^{(1)} \mid  \Vv) \prod_k Q(h_k^{(2)}  \mid  \Vv) \\
 &= \prod_j (\hat h_j^{(1)})^{h_j^{(1)}}(1-\hat h_j^{(1)})^{(1-h_j^{(1)})} \times
 \prod_k (\hat h_k^{(2)})^{h_k^{(2)}}(1-\hat h_k^{(2)})^{(1-h_k^{(2)})} .
\end{align}
当然,对于具有更多层的~\glssymbol{DBM},近似后验的参数化可以通过明显的方式扩展,即利用图的二分结构,遵循~\gls{gibbs_sampling}相同的调度,同时更新所有偶数层,然后同时更新所有奇数层。

% -- 658 --

现在我们已经指定了近似分布$Q$的函数族,但仍然需要指定用于选择该函数族中最适合$P$的成员的过程。
最直接的方法是使用\eqnref{eqn:1956}指定的\gls{meanfield}方程。
这些方程是通过求解变分下界导数为零的位置而导出。
他们以抽象的方式描述如何优化任意模型的变分下界(只需对$Q$求期望)。

应用这些一般的方程,我们得到以下更新规则(再次忽略\gls{bias_aff}项):
\begin{align} \label{eq:2033h1}
 h_j^{(1)} &= \sigma  \Big(  \sum_i v_i \MW_{i,j}^{(1)} 
 + \sum_{k^{\prime}} \MW_{j,k^{\prime}}^{(2)} \hat h_{k^{\prime}}^{(2)}  \Big), ~\forall j ,\\
 \label{eq:2034h2}
 \hat h_{k}^{(2)} &=  \sigma  \Big(  \sum_{j^{\prime}} \MW_{j^{\prime},k}^{(2)}
 \hat h_{j^{\prime}}^{(1)}  \Big), ~\forall k.
\end{align}
在该方程组的不动点处,我们具有变分下界$\CalL(Q)$的局部最大值。
因此,这些不动点更新方程定义了迭代算法,其中我们交替更新$h_{j}^{(1)} $ (使用\eqnref{eq:2033h1})和$h_{k}^{(2)} $ (使用\eqnref{eq:2034h2})。
对于诸如MNIST的小问题,少至10次迭代就足以找到用于学习的近似\gls{positive_phase}梯度,而50次通常足以获得要用于高精度分类的单个特定样本的高质量表示。
将近似变分\gls{inference}扩展到更深的~\glssymbol{DBM}~是直观的。


\subsection{\glssymbol{DBM}~的参数学习}
\label{sec:dbm_parameter_learning}

\glssymbol{DBM}~中的学习必须面对难解\gls{partition_function}的挑战(使用\chapref{chap:confronting_the_partition_function}中的技术),以及难解后验分布的挑战(使用\chapref{chap:approximate_inference}中的技术)。

如\secref{sec:dbm_mean_field_inference}中所描述的,变分\gls{inference}允许构建近似难处理的$P(\Vh  \mid  \Vv)$的分布$Q(\Vh \mid \Vv)$。
然后通过最大化$\CalL(\Vv, Q, \Vtheta)$(难处理的对数似然的变分下界$\log P(\Vv; \Vtheta)$)学习。

% -- 659 --

对于具有两个\gls{hidden_layer}的\gls{DBM},$\CalL$由下式给出
\begin{align}
 \CalL(Q, \Vtheta) = \sum_i \sum_{j^{\prime}} v_i W_{i,j^{\prime}}^{(1)} 
 \hat h_{j^{\prime}}^{(1)} +  \sum_{j^{\prime}} \sum_{k^{\prime}} \hat h_{j^{\prime}}^{(1)}
 W_{j^{\prime}, k^{\prime}}^{(2)} \hat h_{k^{\prime}}^{(2)} - \log Z(\Vtheta) + \CalH(Q).
\end{align}
该表达式仍然包含对数\gls{partition_function} $ \log Z(\Vtheta) $。
由于\gls{DBM}包含\gls{RBM}作为组件,用于计算\gls{RBM}的\gls{partition_function}和采样的困难同样适用于\gls{DBM}。
这意味着评估\gls{BM}的\gls{PMF}需要近似方法,如\gls{AIS}。
同样,训练模型需要近似对数\gls{partition_function}的梯度。
见\chapref{chap:confronting_the_partition_function}对这些方法的一般性描述。
\glssymbol{DBM}~通常使用\gls{SML}训练。
\chapref{chap:confronting_the_partition_function}中描述的许多其他技术都不适用。
诸如\gls{pseudolikelihood}的技术需要评估非归一化概率的能力,而不是仅仅获得它们的变分下界。
对于\gls{DBM},\gls{contrastive_divergence}是缓慢的,因为它们不能在给定可见单元时对\gls{hidden_unit}进行高效采样——反而,每当需要新的\gls{negative_phase}样本时,\gls{contrastive_divergence}将需要\gls{burn_in}一条\gls{markov_chain}。
% 660 mid

非变分版本的\gls{SML}算法已经在\secref{sec:stochastic_maximum_likelihood_and_contrastive_divergence}讨论过。
\algref{alg:sml_dbm}给出了应用于~\glssymbol{DBM}~的变分\gls{SML}算法。
回想一下,我们描述的是~\glssymbol{DBM}~的简化变体(缺少\gls{bias_aff}参数); 很容易推广到包含\gls{bias_aff}参数的情况。

\begin{algorithm}%[!ht]
\caption{用于训练具有两个\gls{hidden_layer}的\glssymbol{DBM}的变分\gls{SML}算法} 
\label{alg:sml_dbm}
\begin{algorithmic}
\STATE 设步长 $\epsilon$ 为一个小正数
\STATE 设定\gls{gibbs_steps} $k$,大到足以让$p(\Vv,\Vh^{(1)},\Vh^{(2)}; \Vtheta + \epsilon \Delta_{\Vtheta})$的\gls{markov_chain}能\gls{burn_in} (从来自 $p(\Vv,\Vh^{(1)},\Vh^{(2)}; \Vtheta)$ 的样本开始)。 
\STATE 初始化三个矩阵,$\tilde{\MV}$, $\tilde{\MH}^{(1)}$ 和 $\tilde{\MH}^{(2)}$ 每个都将 $m$行设为随机值(例如,来自~\gls{bernoulli_distribution},边缘分布大致与模型匹配)。  
\WHILE{没有收敛(学习循环)} 
\STATE 从训练数据采包含$m$个样本的\gls{minibatch},并将它们排列为设计矩阵$\MV$的行。
\STATE 初始化矩阵 $\hat{\MH}^{(1)}$ 和 $\hat{\MH}^{(2)}$,使其大致符合模型的边缘分布。 % ??
\WHILE{没有收敛(\gls{meanfield}\gls{inference}循环)}
        \STATE $\hat{\MH}^{(1)} \leftarrow \sigmoid \left(
          \MV \MW^{(1)} + \hat{\MH}^{(2)} \MW^{(2) \top} \right)$.
        \STATE $\hat{\MH}^{(2)} \leftarrow \sigmoid \left(
          \hat{\MH}^{(1)} \MW^{(2)} \right)$.
\ENDWHILE
\STATE $\Delta_{\MW^{(1)}} \leftarrow \frac{1}{m} \MV^\top \hat{\MH}^{(1)}$
\STATE $\Delta_{\MW^{(2)}} \leftarrow \frac{1}{m} \hat{\MH}^{(1)\ \top} \hat{\MH}^{(2)}$

\FOR{$l=1$ to $k$ (\gls{gibbs_sampling})}
\STATE Gibbs block 1:
   \STATE $\forall i, j, \tilde{V}_{i,j} \text{ 采自 } P(\tilde{V}_{i,j} = 1) =
    \sigmoid \left( 
     \MW_{j,:}^{(1)} 
     \left( \tilde{\MH}_{i,:}^{(1)} \right)^\top
     \right)$.
   \STATE $\forall i, j, \tilde{H}^{(2)}_{i,j} \text{ 采自 } P(\tilde{H}^{(2)}_{i,j} = 1) = 
   \sigmoid \left(\tilde{\MH}_{i,:}^{(1)} \MW_{:,j}^{(2)} 
    \right)$.
\STATE Gibbs block 2:
   \STATE $\forall i, j, \tilde{H}^{(1)}_{i,j} \text{ 采自 } P(\tilde{H}^{(1)}_{i,j} = 1) = \sigmoid \left( \tilde{\MV}_{i,:}
     \MW_{:,j}^{(1)} + \tilde{\MH}_{i,:}^{(2)} \MW_{j,:}^{(2) \top} \
   \right)$.
\ENDFOR
\STATE $\Delta_{\MW^{(1)}} \leftarrow \Delta_{\MW^{(1)}} - \frac{1}{m} \MV^\top \tilde{\MH}^{(1)}$
\STATE $\Delta_{\MW^{(2)}} \leftarrow \Delta_{\MW^{(2)}} - \frac{1}{m} \tilde{\MH}^{(1) \top} \tilde{\MH}^{(2)}$
\STATE $\MW^{(1)} \leftarrow \MW^{(1)} + \epsilon \Delta_{\MW^{(1)}}$
(这是大概的描述,实践中使用的算法更高效,如具有衰减\gls{learning_rate}的动量)
\STATE $\MW^{(2)} \leftarrow \MW^{(2)} + \epsilon \Delta_{\MW^{(2)}}$
\ENDWHILE
\end{algorithmic}
\end{algorithm}


\subsection{\glsentrytext{layer_wise_pretraining}}
\label{sec:layer_wise_pretraining}

不幸的是,随机初始化后使用\gls{SML}训练(如上所述)的~\glssymbol{DBM}~通常导致失败。
在一些情况下,模型不能学习如何充分地表示分布。
在其他情况下,\glssymbol{DBM}~可以很好地表示分布,但是没有比仅使用~\glssymbol{RBM}~获得更高的似然。
除第一层之外,所有层都具有非常小权重的~\glssymbol{DBM}~与~\glssymbol{RBM}~表示大致相同的分布。

如\secref{sec:jointly_training_deep_boltzmann_machines} 所述,目前已经开发了允许联合训练的各种技术。
然而,克服~\glssymbol{DBM}~的联合训练问题最初和最流行的方法是\gls{greedy_layer_wise_pretraining}。
在该方法中,\glssymbol{DBM}~的每一层被单独视为~\glssymbol{RBM},进行训练。
第一层被训练为对输入数据进行建模。
每个后续~\glssymbol{RBM}~被训练为对来自前一~\glssymbol{RBM}~后验分布的样本进行建模。
在以这种方式训练了所有~\glssymbol{RBM}~之后,它们可以被组合成~\glssymbol{DBM}。
然后可以用~\glssymbol{persistent_contrastive_divergence}~训练~\glssymbol{DBM}。
通常,\glssymbol{persistent_contrastive_divergence}~训练将仅使模型的参数、由数据上的对数似然衡量的性能、或区分输入的能力发生微小的变化。
见\figref{fig:chap20_standard_dbm_color}展示的训练过程。

\begin{figure}[!htb]
\ifOpenSource
\centerline{\includegraphics{figure.pdf}}
\else
\centerline{\includegraphics{Chapter20/figures/standard_dbm_color}}
\fi
\caption{用于分类MNIST数据集的\gls{DBM}训练过程~\citep{SalHinton09,Srivastava14}。
(a)使用\glssymbol{contrastive_divergence}近似最大化$\log P(\Vv)$来训练\glssymbol{RBM}。
(b)训练第二个\glssymbol{RBM},使用\glssymbol{contrastive_divergence}-$k$近似最大化$\log P(\Vh^{(1)}, \RSy)$来建模$\Vh^{(1)}$和目标类$\RSy$,其中$\Vh^{(1)}$采自第一个\glssymbol{RBM}条件于数据的后验。 在学习期间将$k$从$1$增加到$20$。
(c)将两个\glssymbol{RBM}组合为\glssymbol{DBM}。
使用$k = 5$的\gls{SML}训练,近似最大化$\log P(\RVv, \RSy)$。
(d)将$\RSy$从模型中删除。
定义新的一组特征$\Vh^{(1)}$和$\Vh^{(2)}$,可在缺少$\RSy$的模型中运行\gls{meanfield}\gls{inference}后获得。% ??
使用这些特征作为\glssymbol{MLP}的输入,其结构与\gls{meanfield}的额外轮相同,并且具有用于估计$\RSy$的额外输出层。
初始化\glssymbol{MLP}的权重与\glssymbol{DBM}的权重相同。
使用\gls{SGD}和\gls{dropout}训练\glssymbol{MLP}近似最大化$\log P(\RSy \mid \RVv)$。
图来自~\citet{Goodfellow-et-al-NIPS2013}。
}
\label{fig:chap20_standard_dbm_color}
\end{figure}

% -- 662 head

这种\gls{greedy_layer_wise_training}过程不仅仅是\gls{coordinate_ascent}。
因为我们在每个步骤优化参数的一个子集,它与\gls{coordinate_ascent}具有一些传递相似性。
这两种方法是不同的,因为\gls{greedy_layer_wise_training}过程中,我们在每个步骤都使用了不同的\gls{objective_function}。


\glssymbol{DBM}~的\gls{greedy_layer_wise_pretraining}与~\glssymbol{DBN}~的\gls{greedy_layer_wise_pretraining}不同。
每个单独的~\glssymbol{RBM}~的参数可以直接复制到相应的~\glssymbol{DBN}。
在~\glssymbol{DBM}~的情况下,\glssymbol{RBM}~的参数在包含到~\glssymbol{DBM}~中之前必须修改。
\glssymbol{RBM}~栈的中间层仅使用自底向上的输入进行训练,但在栈组合形成~\glssymbol{DBM}~后,该层将同时具有自底向上和自顶向下的输入。
为了解释这种效应,\citet{SalHinton09}提倡在将其插入~\glssymbol{DBM}~之前,将所有~\glssymbol{RBM}(顶部和底部~\glssymbol{RBM}~除外)的权重除2。
另外,必须使用每个可见单元的两个``副本''来训练底部~\glssymbol{RBM},并且两个副本之间的权重约束为相等。
这意味着在向上传播时,权重能有效地加倍。
类似地,顶部~\glssymbol{RBM}~应当使用最顶层的两个副本来训练。

为了使用\gls{DBM}获得最好结果,我们需要修改标准的~\glssymbol{SML}~算法,即在联合~\glssymbol{persistent_contrastive_divergence}~训练步骤的\gls{negative_phase}期间使用少量的\gls{meanfield} \citep{SalHinton09}。
具体来说,应当相对于其中所有单元彼此独立的\gls{meanfield}分布来计算能量梯度的期望。
这个\gls{meanfield}分布的参数应该通过运行一次\gls{meanfield}\gls{fixed_point_equation}获得。
\citet{Goodfellow-et-al-NIPS2013}比较了在\gls{negative_phase}中使用和不使用部分\gls{meanfield}的中心化~\glssymbol{DBM}~的性能。


\subsection{联合训练\glsentrytext{DBM}}
\label{sec:jointly_training_deep_boltzmann_machines}

经典~\glssymbol{DBM}~需要\gls{greedy_unsupervised_pretraining},并且为了更好的分类,需要在它们提取的隐藏特征之上,使用独立的基于~\glssymbol{MLP}~的分类器。
这种方法有一些不理想的性质。
因为我们不能在训练第一个~\glssymbol{RBM}~时评估完整~\glssymbol{DBM}~的属性,所以在训练期间难以跟踪性能。
因此,直到相当晚的训练过程,我们都很难知道我们的超参数表现如何。
\glssymbol{DBM}~的软件实现需要很多不同的模块,如用于单个~\glssymbol{RBM}~的~\glssymbol{contrastive_divergence}~训练、完整~\glssymbol{DBM}~的~\glssymbol{persistent_contrastive_divergence}~训练以及基于\gls{back_propagation}的~\glssymbol{MLP}~训练。
最后,\gls{BM}顶部的~\glssymbol{MLP}~失去了\gls{BM}概率模型的许多优点,例如当某些输入值丢失时仍能够进行\gls{inference}的优点。

% -- 662 --

主要有两种方法可以处理\gls{DBM}的联合训练问题。
第一个是\textbf{中心化\gls{DBM}}(centered deep Boltzmann machine) \citep{Montavon2012},通过\gls{reparametrization}模型使其在开始学习过程时\gls{cost_function}的~\gls{hessian}~具有更好的条件数。
这个模型不用经过\gls{greedy_layer_wise_pretraining}阶段就能训练。
这个模型在测试集上获得出色的对数似然,并能产生高质量的样本。
不幸的是,作为分类器,它仍然不能与适当\gls{regularize}的~\glssymbol{MLP}~竞争。
联合训练\gls{DBM}的第二种方式是使用\firstall{MPDBM} \citep{Goodfellow-et-al-NIPS2013}。
该模型的训练\gls{criterion}允许\gls{back_propagation}算法,以避免使用~\glssymbol{mcmc}~估计梯度的问题。
不幸的是,新的\gls{criterion}不会导致良好的似然性或样本,但是相比~\glssymbol{mcmc}~方法,它确实会导致更好的分类性能和良好的\gls{inference}缺失输入的能力。

如果我们回到\gls{BM}的一般观点,即包括一组权重矩阵$\MU$和\gls{bias_aff} $\Vb$的单元$\Vx$,\gls{BM}中心化技巧是最容易描述的。
回顾\eqnref{eq:202E},\gls{energy_function}由下式给出
\begin{align}
 E(\Vx) = -\Vx^\top \MU \Vx - \Vb^\top \Vx.
\end{align}
在权重矩阵$\MU$中使用不同的稀疏模式,我们可以实现不同架构的\gls{BM},如~\glssymbol{RBM}~或具有不同层数的~\glssymbol{DBM}。
将$\Vx$分割成可见和\gls{hidden_unit}并将$\MU$中不相互作用的单元的归零可以实现这些架构。
中心化\gls{BM}引入了一个向量$\Vmu$,并从所有状态中减去:
\begin{align}
    E^{\prime}(\Vx; \MU, \Vb) = -(\Vx - \Vmu)^\top \MU (\Vx - \Vmu) - (\Vx - \Vmu)^\top \Vb.
\end{align}
通常$\Vmu$在开始训练时固定为一个超参数。
当模型初始化时,通常选择为$\Vx - \Vmu \approx 0$。
这种\gls{reparametrization}不改变模型可表示的概率分布的集合,但它确实改变了应用于似然的\gls{SGD}的动态。
具体来说,在许多情况下,这种\gls{reparametrization}导致更好条件数的~\gls{hessian}~矩阵。
\citet{melchior2013center}通过实验证实了~\gls{hessian}~矩阵条件数的改善,并观察到中心化技巧等价于另一个\gls{BM}学习技术——\textbf{增强梯度}(enhanced gradient) \citep{ICML2011Cho_98-small}。
即使在困难的情况下,例如训练多层的\gls{DBM},\gls{hessian}~矩阵条件数的改善也能使学习成功。

% -- 664 --

联合训练\gls{DBM}的另一种方法是\glsacr{MPDBM},它将\gls{meanfield}方程视为定义一系列用于近似求解每个可能\gls{inference}问题的\gls{recurrent_network}~\citep{Goodfellow-et-al-NIPS2013}。
模型被训练为使每个\gls{recurrent_network}获得对相应\gls{inference}问题的准确答案,而不是训练模型来最大化似然。
训练过程如\figref{fig:chap20_multi_prediction_training}所示。
它包括随机采一个训练样本、随机采样\gls{inference}网络的输入子集,然后训练\gls{inference}网络来预测剩余单元的值。

\begin{figure}[!htb]
\ifOpenSource
\centerline{\includegraphics{figure.pdf}}
\else
\centerline{\includegraphics{Chapter20/figures/multi_prediction_training}}
\fi
\caption{\gls{DBM}多预测训练过程的示意图。
每一行指示相同训练步骤内\gls{minibatch}中的不同样本。
每列表示\gls{meanfield}\gls{inference}过程中的\gls{time_step}。
对于每个样本,我们对数据变量的子集进行采样,作为\gls{inference}过程的输入。
这些变量以黑色阴影表示条件。
然后我们运行\gls{meanfield}\gls{inference}过程,箭头指示过程中的哪些变量会影响其他变量。
在实际应用中,我们将\gls{meanfield}展开为几个步骤。
在此示意图中,我们只展开为两个步骤。
虚线箭头表示获得更多步骤需要如何展开该过程。
未用作\gls{inference}过程输入的数据变量成为目标,以灰色阴影表示。
我们可以将每个样本的\gls{inference}过程视为\gls{recurrent_network}。
为了使其在给定输入后能产生正确的目标,我们使用\gls{GD}和\gls{back_propagation}训练这些\gls{recurrent_network}。
这可以训练~\glssymbol{MPDBM}~\gls{meanfield}过程产生准确的估计。
图改编自~\citet{Goodfellow-et-al-NIPS2013}。
}
\label{fig:chap20_multi_prediction_training}
\end{figure}

这种用于近似\gls{inference},通过\gls{computational_graph}进行\gls{back_propagation}的一般原理已经应用于其他模型~\citep{Stoyanov2011,brakel13a}。
在这些模型和~\glssymbol{MPDBM}~中,最终\gls{loss}不是似然的下界。
相反,最终\gls{loss}通常基于近似\gls{inference}网络对缺失值施加的近似条件分布。
这意味着这些模型的训练有些启发式。
如果我们检查由~\glssymbol{MPDBM}~学习出来的\gls{BM}表示$p(\Vv)$,在~\gls{gibbs_sampling}产生较差样本的意义下,它倾向于有些缺陷。

通过\gls{inference}图的\gls{back_propagation}有两个主要优点。
首先,它以模型真正使用的方式训练模型 —— 使用近似\gls{inference}。
这意味着在~\glssymbol{MPDBM}~中,进行如填充缺失的输入或执行分类(尽管存在缺失的输入)的近似\gls{inference}比在原始~\glssymbol{DBM}~中更准确。
原始~\glssymbol{DBM}~不会自己做出准确的分类器; 使用原始~\glssymbol{DBM}~的最佳分类结果是基于~\glssymbol{DBM}~提取的特征训练独立的分类器,而不是通过使用~\glssymbol{DBM}~中的\gls{inference}来计算关于类标签的分布。
\glssymbol{MPDBM}~中的\gls{meanfield}\gls{inference}作为分类器,不需要进行特殊修改就获得良好的表现。
通过近似\gls{inference}\gls{back_propagation}的另一个优点是\gls{back_propagation}计算\gls{loss}的精确梯度。
对于优化而言,比~\glssymbol{SML}~训练中具有偏差和方差的近似梯度更好。
这可能解释了为什么~\glssymbol{MPDBM}~可以联合训练,而~\glssymbol{DBM}~需要\gls{greedy_layer_wise_pretraining}。
近似\gls{inference}图\gls{back_propagation}的缺点是它不提供一种优化对数似然的方法,而提供\gls{generalized_pseudolikelihood}的启发式近似。

\glssymbol{MPDBM}~启发了对NADE框架的扩展NADE-$k$~\citep{Raiko-et-al-2014} ,我们将在\secref{sec:nade}中描述。
% 665 end

\glssymbol{MPDBM}~与~\gls{dropout}~有一定联系。
\gls{dropout}~在许多不同的\gls{computational_graph}之间共享相同的参数,每个图之间的差异是包括还是排除每个单元。
\glssymbol{MPDBM}~还在许多\gls{computational_graph}之间共享参数。
在~\glssymbol{MPDBM}~的情况下,图之间的差异是每个输入单元是否被观察到。
当没有观察到单元时,\glssymbol{MPDBM}~不会像~\gls{dropout}~那样将其完全删除。
相反,\glssymbol{MPDBM}~将其视为要\gls{inference}的\gls{latent_variable}。
我们可以想象将~\gls{dropout}~应用到~\glssymbol{MPDBM},即额外去除一些单元而不是将它们变为\gls{latent_variable}。

% 667 head
\section{实值数据上的\glsentrytext{BM}}
\label{sec:boltzmann_machines_for_real_valued_data}
虽然\gls{BM}最初是为二值数据而开发的,但是许多应用,例如图像和音频建模似乎需要表示实值上概率分布的能力。
在一些情况下,我们可以将区间$[0,1]$中的实值数据视为表示二值变量的期望。
例如, \citet{Hinton-PoE-2000}将训练集中灰度图像的像素值视为定义$[0,1]$间的概率值。
每个像素定义二值变量为$1$的概率,并且二值像素的采样
都彼此独立。
这是评估灰度图像数据集上二值模型的常见过程。
然而,这种方法理论上并不特别令人满意,并且以这种方式独立采样的二值图像具有噪声表象。
在本节中,我们介绍概率密度定义在实值数据上的\gls{BM}。


\subsection{\gls{GBRBM}}
\label{sec:gaussian_bernoulli_rbms}
\gls{RBM}可以用于许多指数族的条件分布 \citep{Welling05}。
其中,最常见的是具有二值\gls{hidden_unit}和实值可见单元的~\glssymbol{RBM},其中可见单元上的条件分布是\gls{gaussian_distribution}(均值为\gls{hidden_unit}的函数)。

有很多方法可以参数化~\gls{GBRBM}。
首先,我们可以选择协方差矩阵或精度矩阵来参数化\gls{gaussian_distribution}。
这里,我们介绍选择精度矩阵的情况。
我们可以通过简单的修改获得协方差的形式。
我们希望条件分布为
\begin{align}
 p(\Vv  \mid  \Vh) = \CalN(\Vv; \MW\Vh, \Vbeta^{-1}).
\end{align}
通过扩展未归一化的对数条件分布可以找到需要添加到\gls{energy_function}中的项:
\begin{align} \label{eq:2039log}
 \log \CalN(\Vv; \MW\Vh, \Vbeta^{-1}) = -\frac{1}{2}(\Vv - \MW\Vh)^\top \Vbeta (\Vv - \MW\Vh) + 
 f(\Vbeta) .
\end{align}

% -- 667 --

此处$f$封装所有的参数,但不包括模型中的随机变量。
因为$f$的唯一作用是归一化分布,并且我们选择的任何可作为\gls{partition_function}的\gls{energy_function}都能起到这个作用,所以我们可以忽略$f$。

如果我们在\gls{energy_function}中包含\eqnref{eq:2039log}中涉及$\Vv$的所有项(其符号被翻转),并且不添加任何其他涉及$\Vv$的项,那么我们的\gls{energy_function}就能表示想要的条件分布$p(\Vv  \mid  \Vh)$。

其他条件分布比较自由,如$p(\Vh  \mid  \Vv)$。
注意\eqnref{eq:2039log}包含一项
\begin{align}
 \frac{1}{2}\Vh^\top\MW^\top \Vbeta \MW \Vh .
\end{align}
因为该项包含$h_i h_j$项,它不能被全部包括在内。
这些对应于\gls{hidden_unit}之间的边。
如果我们包括这些项,我们将得到一个\gls{linear_factor},而不是\gls{RBM}。
当设计我们的\gls{BM}时,我们简单地省略这些$h_i h_j$交叉项。
省略这些项不改变条件分布$p(\Vv  \mid  \Vh)$,因此\eqnref{eq:2039log}仍满足。
然而,我们仍然可以选择是否包括仅涉及单个$h_i$的项。
如果我们假设精度矩阵是对角的,就能发现对于每个\gls{hidden_unit} $h_i$,我们有一项
\begin{align}
 \frac{1}{2} h_i \sum_j \beta_j W_{j,i}^2.
\end{align}
在上面,我们使用了$h_i^2 = h_i$的事实(因为$h_i \in \{ 0, 1\}$)。
如果我们在\gls{energy_function}中包含此项(符号被翻转),则当该单元的权重较大且以高精度连接到可见单元时,\gls{bias_aff} $h_i$将自然被关闭。
是否包括该\gls{bias_aff}项不影响模型可以表示的分布族(假设我们包括\gls{hidden_unit}的\gls{bias_aff}参数),但是它确实会影响模型的学习动态。
包括该项可以帮助\gls{hidden_unit}(即使权重在幅度上快速增加时)保持合理激活。

因此,在~\gls{GBRBM}~上定义\gls{energy_function}的一种方式:
\begin{align}
 E(\Vv, \Vh) = \frac{1}{2} \Vv^\top (\Vbeta \odot \Vv) -  (\Vv \odot \Vbeta)^\top\MW \Vh - \Vb^\top \Vh,
\end{align}
但我们还可以添加额外的项或者通过方差而不是精度参数化能量。

% -- 668 --

在这个推导中,我们没有在可见单元上添加\gls{bias_aff}项,但添加这样的\gls{bias_aff}是容易的。
\gls{GBRBM}~参数化一个最终变化的来源是如何处理精度矩阵的选择。
它可以被固定为常数(可能基于数据的边缘精度估计)或学习出来。
它也可以是标量乘以单位矩阵,或者是一个对角矩阵。
在此情况下,由于一些操作需要对矩阵\gls{invert},我们通常不允许非对角的精度矩阵,因为高斯分布的一些操作需要对矩阵\gls{invert},一个对角矩阵可以非常容易地被\gls{invert}。
在接下来的章节中,我们将看到其他形式的\gls{BM},它们允许对协方差结构建模,并使用各种技术避免对精度矩阵\gls{invert}。


\subsection{条件协方差的\glsentrytext{undirected_model}}
\label{sec:undirected_models_of_conditional_covariance}

虽然\gls{gaussian_rbm}~已成为实值数据的标准能量模型, \citet{Ranzato2010a}认为\gls{gaussian_rbm}~感应\gls{bias_aff}不能很好地适合某些类型的实值数据中存在的统计变化,特别是自然图像。
问题在于自然图像中的许多信息内容嵌入于像素之间的协方差而不是原始像素值中。
换句话说,图像中的大多数有用信息在于像素之间的关系,而不是其绝对值。
由于\gls{gaussian_rbm}~仅对给定\gls{hidden_unit}的输入条件均值建模,所以它不能捕获条件协方差信息。
为了回应这些评论,已经有学者提出了替代模型,设法更好地考虑实值数据的协方差。
这些模型包括\firstall{mcrbm}\footnote{术语``mcRBM''根据字母M-C-R-B-M发音;``mc''不是``McDonald's''中的``Mc''的发音。}、\firstall{mpot}模型和\firstall{ssrbm}。


\paragraph{\gls{mcrbm}} \glssymbol{mcrbm}~使用\gls{hidden_unit}独立地编码所有可观察单元的条件均值和协方差。
\glssymbol{mcrbm}~的\gls{hidden_layer}分为两组单元:均值单元和协方差单元。
建模条件均值的那组单元是简单的\gls{gaussian_rbm}。
另一半是\firstall{crbm} \citep{Ranzato2010a},对条件协方差的结构进行建模(如下所述)。

% -- 669 --

具体来说,在二值均值的单元$\Vh^{(m)}$和二值协方差单元$\Vh^{(c)}$的情况下,\glssymbol{mcrbm}~模型被定义为
两个\gls{energy_function}的组合:
\begin{align}
 E_{\text{mc}}(\Vx, \Vh^{(m)}, \Vh^{(c)}) = E_{\text{m}}(\Vx, \Vh^{(m)}) + E_{\text{c}}(\Vx, \Vh^{(c)}),
\end{align}
其中$E_{\text{m}}$为标准的~\gls{GBRBM}~\gls{energy_function}\footnote{这个版本的~\gls{GBRBM}~\gls{energy_function}假定图像数据的每个像素具有零均值。考虑非零像素均值时,可以简单地将像素偏移添加到模型中。},
\begin{align} \label{eq:2044e}
E_{\text{m}}(\Vx, \Vh^{(m)}) = \frac{1}{2}\Vx^\top \Vx - \sum_j \Vx^\top \MW_{:,j} h_j^{(m)} - \sum_j 
 b_j^{(m)} h_j^{(m)},
\end{align}
$E_{\text{c}}$是~\glssymbol{crbm}~建模条件协方差信息的\gls{energy_function}:
\begin{align}
 E_{\text{c}}(\Vx, \Vh^{(c)}) = \frac{1}{2} \sum_j h_j^{(c)} \big( \Vx^\top \Vr^{(j)}\big)^2 - \sum_j 
 b_j^{(c)} h_j^{(c)}.
\end{align}
参数$\Vr^{(j)}$与$h_j^{(c)}$关联的协方差权重向量对应,$\Vb^{(c)}$是一个协方差\gls{bias_aff}向量。
组合后的\gls{energy_function}定义联合分布,
\begin{align}
 p_{\text{mc}}(\Vx, \Vh^{(m)}, \Vh^{(c)}) = \frac{1}{Z} \exp \Big\{ -E_{\text{mc}}(\Vx, \Vh^{(m)}, 
 \Vh^{(c)}) \Big\},
\end{align}
以及给定$\Vh^{(m)}$和$\Vh^{(c)}$后,关于观察数据相应的条件分布(为一个多元\gls{gaussian_distribution}):
\begin{align}
 p_{\text{mc}}(\Vx \mid \Vh^{(m)}, \Vh^{(c)})  = \CalN \Bigg( \Vx \,; \MC_{\Vx \mid \Vh}^{\,\text{mc}} \Bigg(
\sum_j \MW_{:,j}h_j^{(m)} \Bigg), \MC_{\Vx \mid \Vh}^{\,\text{mc}}
 \Bigg).
\end{align}
注意协方差矩阵$\MC_{\Vx \mid \Vh}^{\,\text{mc}} = \Big( \sum_j h_j^{(c)} \Vr^{(j)} \Vr^{(j)T} + \MI
\Big)^{-1}$是非对角的,且$\MW$是与建模条件均值的\gls{gaussian_rbm}~相关联的权重矩阵。
由于非对角的条件协方差结构,难以通过\gls{contrastive_divergence}或\gls{persistent_contrastive_divergence}来训练~\glssymbol{mcrbm}。
\glssymbol{contrastive_divergence}~和~\glssymbol{persistent_contrastive_divergence}~需要从$\Vx,\Vh^{(m)},\Vh^{(c)}$的联合分布中采样,这在标准\glssymbol{RBM}~中可以通过~\gls{gibbs_sampling}在条件分布上采样实现。
但是,在~\glssymbol{mcrbm}~中,从$ p_{\text{mc}}(\Vx  \mid \Vh^{(m)}, \Vh^{(c)}) $中抽样需要在学习的每个迭代计算$(\MC^{\,\text{mc}})^{-1}$。
这对于更大的观察数据可能是不切实际的计算负担。
\citet{Ranzato2010b-short}通过使用~\glssymbol{mcrbm}~自由能上的哈密尔顿(混合)\gls{monte_carlo}~\citep{Neal93b}直接从边缘$p(\Vx)$采样,避免了直接从条件$  p_{\text{mc}}(\Vx  \mid \Vh^{(m)}, \Vh^{(c)}) $抽样。

 % -- 670 --
 
\paragraph{\gls{mpot}}
\glsacr{mpot}模型~\citep{ranzato+mnih+hinton:2010-short}以类似~\glssymbol{mcrbm}~扩展~\glssymbol{crbm}~的方式扩展PoT模型~\citep{Welling2003a-small}。
通过添加类似\gls{gaussian_rbm}~中\gls{hidden_unit}的非零高斯均值来实现。
与~\glssymbol{mcrbm}~一样,观察值上的PoT条件分布是多元高斯(具有非对角的协方差)分布; 然而,不同于~\glssymbol{mcrbm}~,隐藏变量的互补条件分布是由条件独立的~\gls{gamma_distribution}给出。
\gls{gamma_distribution} $\CalG(k, \theta)$是关于正实数且均值为$k\theta$的概率分布。
我们只需简单地了解~\gls{gamma_distribution}就足以理解~\glssymbol{mpot}~模型的基本思想。

\glssymbol{mpot}~的\gls{energy_function}为:
\begin{align}
 &E_{\text{mPoT}}(\Vx, \Vh^{(m)}, \Vh^{(c)}) \\
 &= E_{\text{m}}(\Vx, \Vh^{(m)}) + \sum_j \Big( h_j^{(c)} \big( 1+\frac{1}{2}(\Vr^{(j)T}\Vx)^2  \big)
 +(1-\gamma_j)\log h_j^{(c)} \Big),
\end{align}
其中$\Vr^{(j)}$是与单元$h_j^{(c)}$相关联的协方差权重向量,$E_m(\Vx, \Vh^{(m)})$如\eqnref{eq:2044e}所定义。

正如~\glssymbol{mcrbm}~一样,\glssymbol{mpot}~模型\gls{energy_function}指定一个多元\gls{gaussian_distribution},其中关于$\Vx$的条件分布具有非对角的协方差。
\glssymbol{mpot}~模型中的学习(也像~\glssymbol{mcrbm} ) 由于无法从非对角高斯条件分布$p_{\text{mPoT}}(\Vx  \mid  \Vh^{(m)}, \Vh^{(c)}) $采样而变得复杂。
因此~\citet{ranzato+mnih+hinton:2010-short} 也倡导通过哈密尔顿(混合)\gls{monte_carlo}~\citep{Neal93b}直接采样$p(\Vx)$。


\paragraph{\gls{ssrbm}} \firstall{ssrbm}\citep{Courville+al-2011}提供对实值数据的协方差结构建模的另一种方法。
与~\glssymbol{mcrbm}~相比,\glssymbol{ssrbm}~具有既不需要矩阵求逆也不需要哈密尔顿\gls{monte_carlo}方法的优点。
%作为自然图像的模型,\glssymbol{ssrbm}感兴趣的是。
就像~\glssymbol{mcrbm}~和~\glssymbol{mpot}~模型,\glssymbol{ssrbm}~的二值\gls{hidden_unit}通过使用辅助实值变量来编码跨像素的条件协方差。

% -- 671 --

\gls{ssrbm}~有两类\gls{hidden_unit}:二值\,\textbf{尖峰}(spike)单元$\RVh$和实值\,\textbf{平板}(slab)单元$\RVs$。
条件于\gls{hidden_unit}的可见单元均值由$(\Vh \odot \Vs)\MW^\top$给出。
换句话说,每一列$\MW_{:,i}$定义当$h_i=1$时可出现在输入中的分量。
相应的尖峰变量$\RSh_i$确定该分量是否存在。
如果存在的话,相应的平板变量$\RSs_i$确定该分量的强度。
当尖峰变量激活时,相应的平板变量将沿着$\MW_{:,i}$定义的轴的输入增加方差。
这允许我们对输入的协方差建模。
幸运的是,使用~\gls{gibbs_sampling}的\gls{contrastive_divergence}和\gls{persistent_contrastive_divergence}仍然适用。
此处无需对任何矩阵求逆。

形式上,\glssymbol{ssrbm}~模型通过其\gls{energy_function}定义:
\begin{align}
 E_{\text{ss}}(\Vx, \Vs, \Vh) &= - \sum_i \Vx^\top \MW_{:,i} s_i h_i + \frac{1}{2} \Vx^\top
 \Bigg( \VLambda + \sum_i \VPhi_i h_i \Bigg) \Vx \\
 &+ \frac{1}{2} \sum_i \alpha_i s_i^2 - \sum_i \alpha_i \mu_i s_i h_i - \sum_i b_i h_i 
 + \sum_i \alpha_i \mu_i^2 h_i,
 \end{align}
其中$b_i$是尖峰$h_i$的\gls{bias_aff},$\VLambda$是观测值$\Vx$上的对角精度矩阵。
参数$\alpha_i > 0$是实值平板变量$\Vs_i$的标量精度参数。
参数$\VPhi_i$是定义$\Vx$上的$\Vh$调制二次惩罚的非负对角矩阵。
每个$\mu_i$是平板变量$s_i$的均值参数。


利用\gls{energy_function}定义的联合分布,能相对容易地导出~\glssymbol{ssrbm}~条件分布。
例如,通过边缘化平板变量$\Vs$,给定二值尖峰变量$\Vh$,关于观察量的条件分布由下式给出
\begin{align}
 p_{\text{ss}} (\Vx  \mid  \Vh) &= \frac{1}{P(\Vh)} \frac{1}{Z} \int \exp\{ -E(\Vx, \Vs, \Vh) \} d\Vs \\
 &= \CalN \Bigg( \Vx\,; \MC_{\Vx \mid \Vh}^{\,\text{ss}} \sum_i \MW_{:,i}\mu_i h_i, 
  \MC_{\Vx \mid \Vh}^{\,\text{ss}} \Bigg)
\end{align}
其中$ \MC_{\Vx \mid \Vh}^{\,\text{ss}} = (\VLambda + \sum_i \VPhi_i h_i 
-\sum_i \alpha_i^{-1} h_i \MW_{:,i}\MW_{:,i}^\top)^{-1}$。
最后的等式只有在协方差矩阵$\MC_{\Vx \mid \Vh}^{\,\text{ss}} $正定时成立。

由尖峰变量选通意味着$\RVh \odot \RVs$上的真实边缘分布是稀疏的。
这不同于\gls{sparse_coding},其中来自模型的样本在编码中``几乎从不''(在测度理论意义上)包含零,并且需要\glssymbol{MAP}\gls{inference}来强加稀疏性。

% -- 672 --

相比~\glssymbol{mcrbm}~和~\glssymbol{mpot}~模型,\glssymbol{ssrbm}~以明显不同的方式参数化观察量的条件协方差。
\glssymbol{mcrbm}~和~\glssymbol{mpot}~都通过 $\big( \sum_j h_j^{(c)} \Vr^{(j)} \Vr^{(j)\top} + \MI \big)^{-1}$建模观察量的协方差结构,使用 $\Vh_j > 0$的\gls{hidden_unit}的激活来对方向$\Vr^{(j)}$的条件协方差施加约束。
相反,\glssymbol{ssrbm}~使用隐藏尖峰激活$h_i = 1$来指定观察结果的条件协方差,以沿着由相应权重向量指定的方向捏合精度矩阵。
\glssymbol{ssrbm}~条件协方差与一个不同模型给出的类似:概率主成分分析的乘积(PoPPCA)\citep{Williams2002}。
在\gls{overcomplete}的设定下,\glssymbol{ssrbm}~参数化的稀疏激活仅允许在稀疏激活$h_i$的所选方向上有显著方差(高于由$\VLambda^{-1}$给出的近似方差)。
在~\glssymbol{mcrbm}~或~\glssymbol{mpot}~模型中,\gls{overcomplete}的表示意味着,捕获观察空间中特定方向上的变化需要在该方向上的正交投影下去除潜在的所有约束。
这表明这些模型不太适合于\gls{overcomplete}设定。

\gls{ssrbm}~的主要缺点是参数的一些设置会对应于非正定的协方差矩阵。
这种协方差矩阵会在离均值更远的值上放置更大的未归一化概率,导致所有可能结果上的积分发散。
通常这个问题可以通过简单的启发式技巧来避免。
理论上还没有任何令人满意的解决方法。
使用约束优化来显式地避免概率未定义的区域(不过分保守是很难做到的),并且这还会阻止模型到达参数空间的高性能区域。

定性地,\glssymbol{ssrbm}~的卷积变体能产生自然图像的优秀样本。
\figref{fig:chap16_fig-ssrbm}中展示了一些样例。

\glssymbol{ssrbm}~允许几个扩展,
包括平板变量的高阶交互和平均池化\citep{courville-al-ieee14} 使得模型能够在标注数据稀缺时为分类器学习到出色的特征。
向\gls{energy_function}添加一项能防止\gls{partition_function}在\gls{sparse_coding}模型下变得不确定,如尖峰和平板\gls{sparse_coding}~\citep{Goodfeli-et-al-TPAMI-Deep-PrePrint-2013},也称为S3C。

% -- 673 --

\section{\glsentrytext{convolutional_bm}}
\label{sec:convolutional_boltzmann_machines}

如\chapref{chap:convolutional_networks}所示,超高维度输入(如图像)会对机器学习模型的计算、 内存和统计要求造成很大的压力。
通过使用小核的离散卷积来替换矩阵乘法是解决具有空间平移不变性或时间结构的输入问题的标准方式。
\citet{Desjardins-2008} 表明这种方法应用于~\glssymbol{RBM}~时效果很好。

深度卷积网络通常需要池化操作,使得每个连续层的空间大小减小。
前馈\gls{convolutional_network}通常使用池化函数,例如池化元素的最大值。
目前尚不清楚如何将其推广到\gls{energy_based_model}的设定中。
我们可以在$n$个二值检测器单元$\RVd$上引入二值池化单元$\RSp$,强制$p = \max_i d_i$,并且当违反约束时将\gls{energy_function}设置为$\infty$。
因为它需要评估$2^n$个不同的能量设置来计算归一化常数,这种方式不能很好地扩展。
对于小的$3\times3$池化区域,每个池化单元需要评估$2^9 = 512$个\gls{energy_function}!


~\citet{HonglakL2009} 针对这个问题,开发了一个称为\,\textbf{概率最大池化}(probabilistic max pooling)的解决方案(不要与``随机池化''混淆,``随机池化''是用于隐含地构建卷积前馈网络集成的技术)。
概率最大池化背后的策略是约束检测器单元,使得一次最多只有一个可以处于活动状态。
这意味着仅存在$n + 1$个总状态($n$个检测器单元中某一个状态为开和一个对应于所有检测器单元关闭的附加状态)。
当且仅当检测器单元中的一个开启时,池化单元打开。
所有单元的状态关闭时,能量被分配为零。
我们可以认为这是在用包含$n + 1$个状态的单个变量来描述模型,或者等价地具有$n + 1$个变量的模型,除了$n+1$个联合分配的变量之外的能量赋为$\infty$。

虽然高效的概率最大池化确实能强迫检测器单元互斥,这在某些情景下可能是有用的\gls{regularize}约束而在其他情景下是对模型容量有害的限制。
它也不支持重叠池化区域。
从前馈\gls{convolutional_network}获得最佳性能通常需要重叠的池化区域,因此这种约束可能大大降低了\gls{convolutional_bm}的性能。

\citet{HonglakL2009} 证明概率最大池化可以用于构建卷积\gls{DBM}\footnote{该论文将模型描述为``\gls{DBN}'',但因为它可以被描述为纯无向模型(具有易处理逐层\gls{meanfield}不动点更新),所以它最适合\gls{DBM}的定义。}。
该模型能够执行诸如填补输入缺失部分的操作。
虽然这种模型在理论上有吸引力,让它在实践中工作是具有挑战性的,作为分类器通常不如通过\gls{supervised}训练的传统\gls{convolutional_network}。

% -- 674 --

许多卷积模型对于许多不同空间大小的输入同样有效。
对于\gls{BM}, 由于各种原因很难改变输入尺寸。
\gls{partition_function}随着输入大小的改变而改变。
此外,许多\gls{convolutional_network}按与输入大小成比例地缩放池化区域来实现尺寸不变性,但缩放\gls{BM}池化区域是不优雅的。
传统的卷积神经网络可以使用固定数量的池化单元并且动态地增加它们池化区域的大小,以此获得可变大小输入的固定尺寸的表示。
对于\gls{BM},大型池化区域的计算成本比朴素方法高很多。
 \citet{HonglakL2009} 的方法使得每个检测器单元在相同的池化区域中互斥,解决了计算问题,但仍然不允许大小可变的池化区域。
例如,假设我们在学习边缘检测器时,检测器单元上具有$2 \times 2$的概率最大池化。
这强制约束在每个$2 \times 2$的区域中只能出现这些边中的一条。
如果我们随后在每个方向上将输入图像的大小增加50\%, 则期望边缘的数量会相应地增加。
相反,如果我们在每个方向上将池化区域的大小增加50\%到$3 \times 3 $,则互斥性约束现在指定这些边中的每一个在$3 \times3$区域中仅可以出现一次。
当我们以这种方式增长模型的输入图像时, 模型会生成密度较小的边。
当然,这些问题只有在模型必须使用可变数量的池化,以便产出固定大小的输出向量时才会出现。
只要模型的输出是可以与输入图像成比例缩放的特征图,使用概率最大池化的模型仍然可以接受可变大小的输入图像。

图像边界处的像素也带来一些困难, 由于\gls{BM}中的连接是对称的事实而加剧。
如果我们不隐式地补零输入, 则将会导致比可见单元更少的\gls{hidden_unit},并且图像边界处的可见单元将不能被良好地建模,因为它们位于较少\gls{hidden_unit}的接受场中。
然而,如果我们隐式地补零输入,则边界处的\gls{hidden_unit}将由较少的输入像素驱动,并且可能在需要时无法激活。

% -- 675 --

\section{用于结构化或序列输出的\glsentrytext{BM}}
\label{sec:boltzmann_machines_for_structured_or_sequential_outputs}

在结构化输出场景中,我们希望训练可以从一些输入$\Vx$映射到一些输出$\Vy$的模型,$\Vy$的不同条目彼此相关,并且必须遵守一些约束。
例如,在语音合成任务中,$\Vy$是波形,并且整个波形听起来必须像连贯的发音。

表示$\Vy$中的条目之间关系的自然方式是使用概率分布$p(\RVy  \mid  \Vx)$。
扩展到建模条件分布的\gls{BM}可以支持这种概率模型。

使用\gls{BM}条件建模的相同工具不仅可以用于结构化输出任务,还可以用于序列建模。 
在后一种情况下,模型必须估计变量序列上的概率分布$p(\RVx^{(1)}, \dots, \RVx^{(\tau)})$,而不仅仅是将输入$\Vx$映射到输出$\Vy$。
为完成这个任务,条件\gls{BM}可以表示$p(\RVx^{(\tau)}  \mid  \RVx^{(1)}, \dots, \RVx^{(\tau-1)})$形式的\gls{factor}。

视频游戏和电影工业中一个重要序列建模任务是建模用于渲染3-D人物骨架关节角度的序列。 
这些序列通常通过记录角色移动的运动捕获系统收集。
人物运动的概率模型允许生成新的(之前没见过的)但真实的动画。
为了解决这个序列建模任务,\citet{Taylor+2007} 针对小的$m$引入了条件~\glssymbol{RBM}~建模$p(\Vx^{(t)}  \mid  \Vx^{(t-1)}, \dots, \Vx^{(t-m)})$。
该模型是$p(\Vx^{(t)})$上的~\glssymbol{RBM},其\gls{bias_aff}参数是$\Vx$前面$m$个值的线性函数。
当我们条件于$\Vx^{(t-1)}$的不同值和更早的变量时, 我们会得到一个关于$\RVx$的新~\glssymbol{RBM}。
\glssymbol{RBM}~关于$\RVx$的权重不会改变,但是条件于不同的过去值, 我们可以改变~\glssymbol{RBM}~中的不同\gls{hidden_unit}处于活动状态的概率。
通过激活和去激活\gls{hidden_unit}的不同子集,我们可以对$\RVx$上诱导的概率分布进行大的改变。
条件~\glssymbol{RBM}~的其他变体 \citep{Mnih-2011} 和使用条件~\glssymbol{RBM}~进行序列建模的其他变体是可能的 \citep{TaylorHintonICML2009,SutskeverHintonTaylor2009-small,Boulanger-et-al-ICML2012}。

另一个序列建模任务是对构成歌曲音符序列的分布进行建模。
\citet{Boulanger-et-al-ICML2012} 引入了\textbf{RNN-RBM}序列模型并应用于这个任务。
RNN-RBM由~\glssymbol{RNN}(产生用于每个\gls{time_step}的~\glssymbol{RBM}~参数)组成,是帧序列$\Vx^{(t)}$的\gls{generative_model}。
与之前只有~\glssymbol{RBM}~的\gls{bias_aff}参数会在一个\gls{time_step}到下一个发生变化的方法不同,RNN-RBM使用~\glssymbol{RNN}~来产生~\glssymbol{RBM}~的所有参数(包括权重)。
为了训练模型,我们需要能够通过~\glssymbol{RNN}~\gls{back_propagation}\gls{loss_function}的梯度。
\gls{loss_function}不直接应用于~\glssymbol{RNN}~输出。
相反,它应用于~\glssymbol{RBM}。
这意味着我们必须使用\gls{contrastive_divergence}或相关算法关于~\glssymbol{RBM}~参数进行近似的微分。
然后才可以使用通常的\gls{BPTT}算法通过~\glssymbol{RNN}~\gls{back_propagation}该近似梯度。

% -- 676 --

\section{其他\glsentrytext{BM}}
\label{sec:other_boltzmann_machines}
\gls{BM}的许多其他变种是可能的。

\gls{BM}可以用不同的训练\gls{criterion}扩展。
我们专注于训练为大致最大化生成标准$\log p(\Vv)$的\gls{BM}。
相反,旨在最大化$\log p(y \mid \Vv)$来训练判别的~\glssymbol{RBM}~也是有可能的\citep{Larochelle+Bengio-2008-small}。
当使用生成性和判别性标准的线性组合时,该方法通常表现最好。
不幸的是,至少使用现有的方法来看,\glssymbol{RBM}~似乎并不如~\glssymbol{MLP}~那样的监督学习器强大。

在实践中使用的大多数\gls{BM}在其\gls{energy_function}中仅具有二阶相互作用,意味着它们的\gls{energy_function}是许多项的和,并且每个单独项仅包括两个随机变量之间的乘积。
这种项的一个例子是$v_iW_{i,j}h_j$。
我们还可以训练高阶\gls{BM}~\citep{sejnowski1987higher} ,其中\gls{energy_function}项涉及许多变量的乘积。
\gls{hidden_unit}和两个不同图像之间的三向交互可以建模从一个视频帧到下一个帧的空间变换 \citep{Memisevic+Hinton-2007,Memisevic+Hinton-2010}。
通过\gls{one_hot}类别变量的乘法可以根据存在哪个类来改变可见单元和\gls{hidden_unit}之间的关系\citep{Nair2009}。
使用高阶交互的一个最近的示例是具有两组\gls{hidden_unit}的\gls{BM},一组同时与可见单元$\Vv$和类别标签$y$交互,另一组仅与输入值$\Vv$交互\citep{luo2011learning}。 % ?
这可以被解释为鼓励一些\gls{hidden_unit}学习使用与类相关的特征来建模输入,而且还学习额外的\gls{hidden_unit}(不需要根据样本类别,学习逼真$\Vv$样本所需的繁琐细节)。
高阶交互的另一个用途是选通一些特征。
\citet{Sohn-et-al-ICML2013} 介绍了一个带有三阶交互的\gls{BM},以及与每个可见单元相关的二进制掩码变量。
当这些掩码变量设置为零时,它们消除可见单元对\gls{hidden_unit}的影响。
这允许将与分类问题不相关的可见单元从估计类别的\gls{inference}路径中移除。

更一般地说,\gls{BM}框架是一个丰富的模型空间,允许比迄今为止已经探索的更多的模型结构。
开发新形式的\gls{BM}相比于开发新的神经网络层需要更多细心和创造力,因为它通常很难找到一个能保持\gls{BM}所需的所有不同条件分布的可解性的\gls{energy_function}。
尽管这需要努力,该领域仍对创新开放。

% -- 677 --

\section{通过随机操作的\glsentrytext{back_propagation}}
\label{sec:back_propagation_through_random_operations}


传统的神经网络对一些输入变量$\Vx$施加确定性变换。
当开发\gls{generative_model}时,我们经常希望扩展\gls{NN}以实现$\Vx$的随机变换。
这样做的一个直接方法是使用额外输入$\Vz$(从一些简单的概率分布采样得到,如均匀或\gls{gaussian_distribution})来增强神经网络。
神经网络在内部仍可以继续执行确定性计算,但是函数$f(\Vx,\Vz)$对于不能访问$\Vz$的观察者来说将是随机的。
假设$f$是连续可微的,我们可以像往常一样使用\gls{back_propagation}计算训练所需的梯度。

作为示例,让我们考虑从均值$\mu$和方差$\sigma^2$的\gls{gaussian_distribution}中采样$\RSy$的操作:
\begin{align}
 \RSy \sim \CalN(\mu, \sigma^2).
\end{align}
因为$\RSy$的单个样本不是由函数产生的,而是由一个采样过程产生,它的输出会随我们的每次查询变化,所以取$\RSy$相对于其分布的参数$\mu$和$\sigma^2$的导数似乎是违反直觉的。
然而,我们可以将采样过程重写,对基本随机变量$\RSz \sim \CalN(z;0,1)$进行转换以从期望的分布获得样本:
\begin{align}
 y = \mu + \sigma z.
\end{align}

现在我们将其视为具有额外输入$\RSz$的确定性操作,可以通过采样操作来\gls{back_propagation}。
至关重要的是,额外输入是一个随机变量,其分布不是任何我们想对其计算导数的变量的函数。
如果我们可以用相同的$\RSz$值再次重复采样操作,结果会告诉我们$\mu$或$\sigma$的微小变化将会如何改变输出。

% -- 678 --

能够通过该采样操作\gls{back_propagation}允许我们将其并入更大的图中。
我们可以在采样分布的输出之上构建图元素。
例如,我们可以计算一些\gls{loss_function}$J(y)$的导数。
我们还可以构建这样的图元素,其输出是采样操作的输入或参数。
例如,我们可以通过$\mu = f(\Vx; \Vtheta)$和$\sigma = g(\Vx; \Vtheta)$构建更大的图。
在这个增强图中,我们可以通过这些函数的\gls{back_propagation}导出$\nabla_{\Vtheta} J(y)$。

在该高斯采样示例中使用的原理能更广泛地应用。
我们可以将任何形为$p(\RSy;\Vtheta)$或$p(\RSy \mid \Vx;\Vtheta)$的概率分布表示为$p(\RSy \mid \Vomega)$,其中$\Vomega$是同时包含参数$\Vtheta$和输入$\Vx$的变量(如果适用的话)。
给定从分布$p(\RSy \mid \Vomega)$采样的值$\RSy$(其中$\Vomega$可以是其他变量的函数),我们可以将
\begin{align}
 \RVy \sim p(\RVy  \mid  \Vomega)
\end{align}
重写为
\begin{align}
 \Vy = f(\Vz; \Vomega),
\end{align}
其中$\Vz$是随机性的来源。
只要$f$是几乎处处连续可微的,我们就可以使用传统工具(例如应用于$f$的\gls{back_propagation}算法)计算$\RSy$相对于$\Vomega$的导数。
至关重要的是,$\Vomega$不能是$\Vz$的函数,且$\Vz$不能是$\Vomega$的函数。
这种技术通常被称为\firstgls{reparametrization_trick}、\textbf{随机\gls{back_propagation}}(stochastic back-propagation)或\textbf{扰动分析}(perturbation analysis)。


要求$f$是连续可微的,当然需要$\Vy$是连续的。
如果我们希望通过产生离散值样本的采样过程进行\gls{back_propagation},则可以使用\gls{RL}算法(如~\ENNAME{REINFORCE}~算法\citep{Williams-1992}的变体)来估计$\Vomega$上的梯度,这将在\secref{sec:back_propagating_through_discrete_stochastic_operations}中讨论。

在神经网络应用中,我们通常选择从一些简单的分布中采样$\Vz$,如单位均匀分布或单位\gls{gaussian_distribution},并通过网络的确定性部分重塑其输入来实现更复杂的分布。

通过随机操作扩展梯度或优化的想法可追溯到二十世纪中叶~\citep{Price-1958,Bonnet-1964},并且首先在\gls{RL}~\citep{Williams-1992}的情景下用于机器学习。
最近,它已被应用于变分近似~\citep{Opper+Archambeau-2009} 和随机生成神经网络~\citep{bengio2013estimating,Kingma-arxiv2013,Kingma+Welling-arxiv2014,Kingma+Welling-ICLR2014,Rezende-et-al-ICML2014,Goodfellow-et-al-NIPS2014-small}。
许多网络,如\gls{DAE}或使用~\gls{dropout}~的\gls{regularize}网络,也被自然地设计为将噪声作为输入,而不需要任何特殊的\gls{reparametrization}就能使噪声独立于模型。

% -- 679 --

\subsection{通过离散随机操作的\glsentrytext{back_propagation}}
\label{sec:back_propagating_through_discrete_stochastic_operations}

当模型发射离散变量$\Vy$时,\gls{reparametrization_trick}不再适用。
假设模型采用输入$\Vx$和参数$\Vtheta$,两者都封装在向量$\Vomega$中,并且将它们与随机噪声$\Vz$组合以产生$\Vy$:
\begin{align}
 \Vy = f(\Vz;\Vomega).
\end{align}
因为$\Vy$是离散的,$f$必须是一个阶跃函数。
阶跃函数的导数在任何点都是没用的。
在每个阶跃边界, 导数是未定义 的,但这是一个小问题。
大问题是导数在阶跃边界之间的区域几乎处处为零。
因此,任何\gls{cost_function} $J(\Vy)$的导数无法给出如何更新模型参数$\Vtheta$的任何信息。

\ENNAME{REINFORCE}~算法(REward Increment $=$ nonnegative Factor $\times$ Offset Reinforcement $\times$ Characteristic Eligibility)提供了定义一系列简单而强大解决方案的框架\citep{Williams-1992}。
其核心思想是,即使$J(f(\Vz;\Vomega))$是具有无用导数的阶跃函数,期望代价$\SetE_{\RVz \sim p(\RVz)} J(f(\Vz;\Vomega))$通常是服从\gls{GD}的光滑函数。
虽然当 $\Vy$是高维(或者是许多离散随机决策组合的结果)时,该期望通常是难解的,但我们可以使用\gls{monte_carlo}平均进行无偏估计。
梯度的随机估计可以与~\glssymbol{SGD}~或其他基于随机梯度的优化技术一起使用。

通过简单地微分期望成本,我们可以推导出~\ENNAME{REINFORCE}~最简单的版本:
\begin{align}
 \SetE_{\Vz}[J(\Vy)] &= \sum_{\Vy} J(\Vy) p(\Vy), \\
 \frac{\partial \SetE[J(\Vy)]}{\partial \Vomega} &= \sum_{\Vy} J(\Vy)
 \frac{\partial p(\Vy)}{\partial \Vomega} \label{eq:sum_exp} \\
 &=  \sum_{\Vy} J(\Vy) p(\Vy) \frac{\partial\log p(\Vy)}{\partial \Vomega} 
 \label{eq:sum_log} \\
 & \approx \frac{1}{m} \sum_{\Vy^{(i)} \sim p(\Vy), i=1}^m 
 J(\Vy^{(i)}) \frac{\partial\log p(\Vy^{(i)})}{\partial \Vomega}. \label{eq:sum_prob}
\end{align}
\eqnref{eq:sum_exp}依赖于$J$不直接引用$\Vomega$的假设。
放松这个假设来扩展该方法是简单的。
\eqnref{eq:sum_log}利用对数的导数规则,
$\frac{\partial\log p(\Vy)}{\partial \Vomega} = \frac{1}{p(\Vy)}
\frac{\partial p(\Vy)}{\partial \Vomega}$。
\eqnref{eq:sum_prob}给出了该梯度的无偏\gls{monte_carlo}估计。

% -- 680 --

在本节中我们写的$p(\Vy)$,可以等价地写成$p(\Vy  \mid  \Vx)$。
这是因为$p(\Vy)$由$\Vomega$参数化,并且如果$\Vx$存在,$\Vomega$包含$\Vtheta$和$\Vx$两者。

简单~\ENNAME{REINFORCE}~估计的一个问题是其具有非常高的方差,需要采$\Vy$的许多样本才能获得对梯度的良好估计,或者等价地,如果仅绘制一个样本,\glssymbol{SGD}~将收敛得非常缓慢并将需要较小的\gls{learning_rate}。
通过使用\firstgls{variance_reduction}方法~\citep{Wilson-1984,LEcuyer-1994},可以地减少该估计的方差。
想法是修改估计量,使其预期值保持不变,但方差减小。
在~\ENNAME{REINFORCE}~的情况下提出的\gls{variance_reduction}方法,涉及计算用于偏移$J(\Vy)$的\,\textbf{基线}(baseline)。
注意,不依赖于$\Vy$的任何偏移$b(\Vw)$都不会改变估计梯度的期望,因为
\begin{align}
 E_{p(\Vy)} \Bigg[ \frac{\partial\log p(\Vy)}{\partial \Vomega}  \Bigg] &=
 \sum_{\Vy} p(\Vy) \frac{\partial\log p(\Vy)}{\partial \Vomega} \\
 &= \sum_{\Vy} \frac{\partial p(\Vy)}{\partial \Vomega} \\
 &= \frac{\partial}{\partial \Vomega} \sum_{\Vy} p(\Vy) = 
 \frac{\partial}{\partial \Vomega} 1 = 0,
\end{align}
这意味着
\begin{align}
 E_{p(\Vy)} \Bigg[ (J(\Vy) - b(\Vomega))\frac{\partial\log p(\Vy)}{\partial \Vomega}  \Bigg] &= 
 E_{p(\Vy)} \Bigg[ J(\Vy) \frac{\partial\log p(\Vy)}{\partial \Vomega} \Bigg]
 - b(\Vomega) E_{p(\Vy)} \Bigg[ \frac{\partial\log p(\Vy)}{\partial \Vomega}  \Bigg] \\
 &= E_{p(\Vy)} \Bigg[ J(\Vy) \frac{\partial\log p(\Vy)}{\partial \Vomega} \Bigg] .
\end{align}
此外,我们可以通过计算$(J(\Vy) - b(\Vomega))\frac{\partial\log p(\Vy)}{\partial \Vomega} $
关于$p(\Vy)$的方差,并关于$b(\Vomega)$最小化获得最优$b(\Vomega)$。
我们发现这个最佳基线$b^*(\Vomega)_i$对于向量$\Vomega$的每个元素$\omega_i$是不同的:
\begin{align} \label{eq:est_b}
 b^*(\Vomega)_i = \frac{E_{p(\Vy)} \Big[ J(\Vy)
 \frac{\partial\log p(\Vy)^2}{\partial \omega_i}  \Big]}
{E_{p(\Vy)} \Big[\frac{\partial\log p(\Vy)^2}{\partial \omega_i}\Big] }.
\end{align}
相对于$\omega_i$的梯度估计则变为
\begin{align}
 (J(\Vy) - b(\Vomega)_i)\frac{\partial\log p(\Vy)}{\partial \omega_i},
\end{align}
其中$ b(\Vomega)_i$ 估计上述$ b^*(\Vomega)_i$。
获得估计$b$通常需要将额外输出添加到神经网络,并训练新输出对$\Vomega$的每个元素估计$E_{p(\Vy)} 
[J(\Vy)\frac{\partial\log p(\Vy)^2}{\partial \omega_i}]$和
$E_{p(\Vy)}[\frac{\partial\log p(\Vy)^2}{\partial \omega_i}]$。
这些额外的输出可以用\gls{mean_squared_error}目标训练,对于给定的$\Vomega$,从$p(\Vy)$采样$\Vy$时,分别用$J(\Vy)\frac{\partial\log p(\Vy)^2}{\partial \omega_i}$和 $\frac{\partial\log p(\Vy)^2}{\partial \omega_i}$作目标。
然后可以将这些估计代入\eqnref{eq:est_b}就能恢复估计$b$。
\citet{Mnih+Gregor-ICML2014} 倾向于使用通过目标$J(\Vy)$训练的单个共享输出(跨越$\Vomega$的所有元素$i$),并使用$b(\Vomega) \approx E_{p(\Vy)} [J(\Vy)]$作为基线。

% -- 681 --

在\gls{RL}背景下引入的\gls{variance_reduction}方法\citep{Sutton-et-al-2000,Weaver+Tao-UAI2001}, \citet{Dayan-1990}推广了二值奖励的前期工作。
可以参考 \citet{bengio2013estimating}、\citet{Mnih+Gregor-ICML2014}、\citet{Ba+Mnih-arxiv2014}、\citet{Mnih2014}或 \citet{Xu-et-al-ICML2015} 中在深度学习的背景下使用减少方差的~\ENNAME{REINFORCE}~算法的现代例子。
除了使用与输入相关的基线$b(\Vomega)$,\citet{Mnih+Gregor-ICML2014}发现可以在训练期间调整$(J(\Vy) - b(\Vomega))$的尺度(即除以训练期间的移动平均估计的标准差 ),即作为一种适应性\gls{learning_rate},可以抵消训练过程中该量大小发生的重要变化的影响。
\citet{Mnih+Gregor-ICML2014} 称之为启发式\textbf{方差归一化}(variance normalization)。

基于~\ENNAME{REINFORCE}~的估计器可以被理解为将$\Vy$的选择与$J(\Vy)$的对应值相关联来估计梯度。
如果在当前参数化下不太可能出现$\Vy$的良好值,则可能需要很长时间来偶然获得它,并且获得所需信号的配置应当被加强。

\section{有向生成网络}
\label{sec:directed_generative_nets}

如\chapref{chap:structured_probabilistic_models_for_deep_learning}所讨论的,\gls{directed_graphical_model}构成了一类突出的\gls{graphical_model}。
虽然\gls{directed_graphical_model}在更大的机器学习社群中非常流行,但在较小的深度学习社群中,大约直到2013年它们都掩盖在\gls{undirected_model}(如~\glssymbol{RBM})的光彩之下。

在本节中,我们回顾一些传统上与深度学习社群相关的标准\gls{directed_graphical_model}。

我们已经描述过部分有向的模型——\gls{DBN}。
我们还描述过可以被认为是浅度有向\gls{generative_model}的\gls{sparse_coding}模型。
尽管在样本生成和密度估计方面表现不佳,在深度学习的背景下它们通常被用作特征学习器。
我们接下来描述多种深度完全有向的模型。

% -- 682 --

\subsection{\glsentrytext{sigmoid_bn}}
\label{ sec:sigmoid _belief_networks}

\gls{sigmoid_bn} \citep{Neal-1990}是一种具有特定条件概率分布的\gls{directed_graphical_model}的简单形式。
一般来说,我们可以将~\gls{sigmoid_bn}视为具有二值向量的状态$\Vs$,其中状态的每个元素都受其祖先影响:
\begin{align}
 p(s_i) = \sigma \Bigg( \sum_{j<i} W_{j,i} s_j + b_i \Bigg).
\end{align}

\gls{sigmoid_bn}最常见的结构是被分为许多层的结构,其中\gls{ancestral_sampling}通过一系列多个\gls{hidden_layer}进行,然后最终生成可见层。
这种结构与\gls{DBN}非常相似,但它们在采样过程开始时的单元彼此独立,而不是从\gls{RBM}采样。
这种结构由于各种原因而令人感兴趣。
一个原因是该结构是可见单元上概率分布的通用近似,即在足够深的情况下,可以任意良好地近似二值变量的任何概率分布(即使各个层的宽度受限于可见层的维度 )\citep{Sutskever+Hinton-2008}。


虽然生成可见单元的样本在~\gls{sigmoid_bn}中是非常高效的,但是其他大多数操作不是很高效。
给定可见单元,对\gls{hidden_unit}的\gls{inference}是难解的。
因为变分下界涉及对包含整个层的团求期望,\gls{meanfield}\gls{inference}也是难以处理的。
这个问题一直困难到足以限制有向离散网络的普及。


在~\gls{sigmoid_bn}中执行\gls{inference}的一种方法是构造专用于~\gls{sigmoid_bn}的不同下界 \citep{Saul+96}。
这种方法只适用于非常小的网络。
另一种方法是使用\gls{learned}\gls{inference}机制,如\secref{sec:learned_approximate_inference}中描述的。
\gls{helmholtz_machine} \citep{Dayan-et-al-1995,dayan1996varieties} 结合了一个~\gls{sigmoid_bn}与一个预测\gls{hidden_unit}上\gls{meanfield}分布参数的\gls{inference}网络。
\gls{sigmoid_bn}的现代方法\citep{Gregor-et-al-ICML2014,Mnih+Gregor-ICML2014} 仍然使用这种\gls{inference}网络的方法。
因为\gls{latent_variable}的离散本质,这些技术仍然是困难的。
人们不能简单地通过\gls{inference}网络的输出\gls{back_propagation},而必须使用相对不可靠的机制即通过离散采样过程进行\gls{back_propagation}(如\secref{sec:back_propagating_through_discrete_stochastic_operations}所述)。
最近基于\gls{importance_sampling}、重加权的\gls{wake_sleep}\citep{Bornschein+Bengio-ICLR2015-small} 或双向~\gls{helmholtz_machine}~\citep{Bornschein-et-al-arxiv2015-small} 的方法使得我们可以快速训练~\gls{sigmoid_bn},并在基准任务上达到最好的表现。

\gls{sigmoid_bn}的一种特殊情况是没有\gls{latent_variable}的情况。
在这种情况下学习是高效的,因为没有必要将\gls{latent_variable}边缘化到似然之外。
一系列称为\gls{auto_regressive_network}的模型将这个完全可见的\gls{BN}\gls{generalize}到其他类型的变量(除二值变量)和其他结构(除对数线性关系)的条件分布。
\gls{auto_regressive_network}将在\secref{sec:auto_regressive_networks}中描述。


\subsection{可微\glsentrytext{generator_network}}
\label{sec:differentiable_generator_networks}

许多\gls{generative_model}基于使用可微\firstgls{generator_network}的想法。
这种模型使用可微函数$g(\Vz;\Vtheta^{(g)})$将\gls{latent_variable} $\RVz$的样本变换为样本$\RVx$或样本$\RVx$上的分布,可微函数通常可以由神经网络表示。
这类模型包括将\gls{generator_network}与\gls{inference}网络配对的\gls{VAE}、将\gls{generator_network}与判别器网络配对的\gls{GAN},以及孤立地训练\gls{generator_network}的技术。


\gls{generator_network}本质上仅是用于生成样本的参数化计算过程,其中的体系结构提供了从中采样的可能分布族以及选择这些族内分布的参数。

作为示例,从具有均值$\Vmu$和协方差$\VSigma$的正态分布绘制样本的标准过程是将来自零均值和单位协方差的正态分布的样本$\Vz$馈送到非常简单的\gls{generator_network}中。 这个\gls{generator_network}只包含一个仿射层:
\begin{align}
 \Vx = g(\Vz) = \Vmu + \ML \Vz ,
\end{align}
其中$\ML$由$\VSigma$的~\ENNAME{Cholesky}~分解给出。


伪随机数发生器也可以使用简单分布的非线性变换。
例如,\textbf{逆变换采样}(inverse transform sampling)\citep{devroye2013non}从$U(0,1)$中采一个标量$z$,并且对标量$x$应用非线性变换。 % ??
在这种情况下,$g(z)$由累积分布函数$F(x) = \int_{-\infty}^{x} p(v) dv$的反函数给出。
如果我们能够指定$p(x)$,在$x$上积分,并取所得函数的反函数,我们不用通过机器学习就能从$p(x)$进行采样。

为了从更复杂的分布(难以直接指定、难以积分或难以求所得积分的反函数)中生成样本,我们使用\gls{feedforward_network}来表示非线性函数$g$的参数族,并使用训练数据来\gls{inference}参数以选择所期望的函数。

我们可以认为$g$提供了变量的非线性变化,将$\RVz$上的分布变换成$\RVx$上想要的分布。

回顾\eqnref{eqn:3.47},对于可求反函数的、可微的、连续的$g$,
\begin{align}
 p_z(\Vz) = p_x(g(\Vz)) \Big | \det (\frac{\partial g}{\partial \Vz}) \Big |.
\end{align}
这隐含地对$\RVx$施加概率分布:
\begin{align}
 p_x(\Vx) = \frac{p_z(g^{-1}(\Vx))}{ | \det (\frac{\partial g}{\partial \Vz}) |}.
\end{align}
当然,取决于$g$的选择,这个公式可能难以评估,因此我们经常需要使用间接学习$g$的方法,而不是直接尝试最大化$\log p(\Vx)$。

在某些情况下,我们使用$g$来定义$\Vx$上的条件分布,而不是使用$g$直接提供$\Vx$的样本。
例如,我们可以使用一个\gls{generator_network},其最后一层由~\ENNAME{sigmoid}~输出组成,可以提供~\gls{bernoulli_distribution}的平均参数:
\begin{align}
 p(\RVx_i = 1  \mid  \Vz) = g(\Vz)_i .
\end{align}
在这种情况下, 我们使用$g$来定义$p(\Vx  \mid  \RVz)$时, 我们通过边缘化$\Vz$来对$\Vx$施加分布:
\begin{align}
 p(\Vx) = \SetE_{\Vz} p(\Vx  \mid  \Vz).
\end{align}

两种方法都定义了一个分布$p_g(\Vx)$, 并允许我们使用\secref{sec:back_propagation_through_random_operations}中的\gls{reparametrization_trick}来训练$p_g$的各种评估\gls{criterion}。

% -- 685 --

表示\gls{generator_network}的两种不同方法(发出条件分布的参数相对直接发射样品 )具有互补的优缺点。
当\gls{generator_network}在$\Vx$上定义条件分布时,它不但能生成连续数据,也能生成离散数据。
当\gls{generator_network}直接提供采样时,它只能产生连续的数据(我们可以在前向传播中引入离散化,但这样做意味着模型不再能够使用\gls{back_propagation}进行训练)。
直接采样的优点是,我们不再被迫使用条件分布(可以容易地写出来并由人类设计者进行代数操作的形式)。

基于可微\gls{generator_network}的方法是由分类可微前馈网络中\gls{GD}的成功应用而推动的。
在\gls{supervised_learning}的背景中,基于梯度训练学习的深度前馈网络在给定足够的\gls{hidden_unit}和足够的训练数据的情况下,在实践中似乎能保证成功。
这个同样的方案能成功转移到\gls{generative_modeling}上吗?

\gls{generative_modeling}似乎比分类或回归更困难,因为学习过程需要优化难以处理的\gls{criterion}。
在可微\gls{generator_network}的情况中,\gls{criterion}是难以处理的,因为数据不指定\gls{generator_network}的输入$\Vz$和输出$\Vx$。
在\gls{supervised_learning}的情况下,输入$\Vx$和输出$\Vy$同时给出,并且优化过程只需学习如何产生指定的映射。
在生成建模的情况下,学习过程需要确定如何以有用的方式排布$\Vz$空间,以及额外的如何从$\Vz$映射到$\Vx$。

\citet{dosovitskiy2015learning}研究了一个简化问题,其中$\Vz$和$\Vx$之间的对应关系已经给出。
具体来说,训练数据是计算机渲染的椅子图。
\gls{latent_variable} $\Vz$是渲染引擎的参数,描述了椅子模型的选择、椅子的位置以及影响图像渲染的其他配置细节。
使用这种合成的生成数据,\gls{convolutional_network}能够学习将图像内容的描述$\Vz$映射到渲染图像的近似$\Vx$。
这表明当现代可微\gls{generator_network}具有足够的模型容量时,足以成为良好的\gls{generative_model},并且现代优化算法具有拟合它们的能力。
困难在于当每个$\Vx$的$\Vz$的值不是固定的且在每次训练前是未知时,如何训练\gls{generator_network}。


在接下来的章节中,我们讨论仅给出$\Vx$的训练样本,训练可微\gls{generator_network}的几种方法。

% -- 686 --

\subsection{\glsentrytext{VAE}}
\label{sec:variational_autoencoders}
\firstall{VAE}\citep{Kingma-arxiv2013,Rezende-et-al-ICML2014}是一个使用学好的近似\gls{inference}的\gls{directed_model}, 可以纯粹地使用基于梯度的方法进行训练。


为了从模型生成样本,\glssymbol{VAE}~首先从编码分布$p_{\text{model}}(\Vz)$中采样$\Vz$。
然后使样本通过可微\gls{generator_network} $g(\Vz)$。
最后,从分布$p_{\text{model}}(\Vx;g(\Vz)) = p_{\text{model}}(\Vx  \mid  \Vz)$ 中采样$\Vx$。
然而在训练期间,近似\gls{inference}网络(或\gls{encoder})$q(\Vz  \mid  \Vx)$用于获得$\Vz$,而$p_{\text{model}}(\Vx  \mid  \Vz)$则被视为\gls{decoder}网络。


\gls{VAE}背后的关键思想是,它们可以通过最大化与数据点$\Vx$相关联的变分下界$\CalL(q)$来训练:
\begin{align}
\CalL(q) &= \SetE_{\Vz \sim q(\Vz  \mid  \Vx)} \log p_{\text{model}} (\Vz, \Vx)
+ \CalH(q(\RVz  \mid  \Vx)) \label{eq:var_lower}  \\
&= \SetE_{\Vz \sim q(\Vz  \mid  \Vx)} \log p_{\text{model}} (\Vx  \mid  \Vz)
- D_{\text{KL}}(q(\RVz  \mid  \Vx) ~||~ p_{\text{model}}(\RVz)) \label{eq:var_kl} \\
& \leq \log p_{\text{model}}(\Vx).
\end{align}
在\eqnref{eq:var_lower}中,我们将第一项视为\gls{latent_variable}的近似后验下可见和隐藏变量的联合对数似然性(正如~\glssymbol{EM}~一样,不同的是我们使用近似而不是精确后验)。
第二项则可视为近似后验的熵。
当$q$被选择为\gls{gaussian_distribution},其中噪声被添加到预测平均值时,最大化该熵项促使该噪声标准偏差的增加。
更一般地,这个熵项鼓励变分后验将高概率质量置于可能已经产生$\Vx$的许多$\Vz$值上,而不是坍缩到单个估计最可能值的点。
在\eqnref{eq:var_kl}中,我们将第一项视为在其他\gls{AE}中出现的重构对数似然。
第二项试图使近似后验分布$q(\RVz  \mid  \Vx)$和模型先验$p_{\text{model}}(\Vz)$彼此接近。


变分\gls{inference}和学习的传统方法是通过优化算法\gls{inference} $q$,通常是迭代不动点方程(\secref{sec:variational_inference_and_learning})。
这些方法是缓慢的,并且通常需要以闭解形式计算$\SetE_{\RVz \sim q} \log p_{\text{model}} (\Vz, \Vx)$。 % mistake ???
\gls{VAE}背后的主要思想是训练产生$q$参数的参数编码器(有时也称为\gls{inference}网络或识别模型)。
只要$\Vz$是连续变量,我们就可以通过从$q(\Vz  \mid  \Vx) = q(\Vz; f(\Vx; \Vtheta))$中采样$\Vz$的样本\gls{back_propagation},以获得相对于$\Vtheta$的梯度。
学习则仅包括相对于\gls{encoder}和\gls{decoder}的参数最大化$\CalL$。
$\CalL$中的所有期望都可以通过\gls{monte_carlo}采样来近似。

% -- 687 --

\gls{VAE}方法是优雅的, 理论上令人愉快的,并且易于实现。
它也获得了出色的结果,是\gls{generative_modeling}中的最先进方法之一。
它的主要缺点是从在图像上训练的\gls{VAE}中采样的样本往往有些模糊。
这种现象的原因尚不清楚。
一种可能性是模糊性是最大似然的固有效应,因为我们需要最小化$D_{\text{KL}}(p_{\text{data}} ||p_{\text{model}} )$。
如\figref{fig:chap3_kl_direction_color}所示,这意味着模型将为训练集中出现的点分配高的概率,但也可能为其他点分配高的概率。
还有其他原因可以导致模糊图像。
模型选择将概率质量置于模糊图像而不是空间的其他部分的部分原因是实际使用的\gls{VAE}通常在$p_{\text{model}}(\Vx; g(\Vz))$使用\gls{gaussian_distribution}。
最大化这种分布似然性的下界与训练具有\gls{mean_squared_error}的传统\gls{AE}类似,这意味着它倾向于忽略由少量像素表示的特征或其中亮度变化微小的像素。
如~\citet{Theis2015d}和~\citet{Huszar-arXiv2015}指出的,该问题不是~\glssymbol{VAE}~特有的,而是与优化对数似然或$D_{\text{KL}}(p_{\text{data}} ||p_{\text{model}} )$的\gls{generative_model}共享的。
现代~\glssymbol{VAE}~模型另一个麻烦的问题是,它们倾向于仅使用$\Vz$维度中的小子集,就像\gls{encoder}不能够将具有足够局部方向的输入空间变换到边缘分布与分解前匹配的空间。


\glssymbol{VAE}~框架可以直接扩展到大范围的模型架构。
相比\gls{BM},这是关键的优势,因为\gls{BM}需要非常仔细地设计模型来保持易解性。
\glssymbol{VAE}~可以与广泛的可微算子族一起良好工作。
一个特别复杂的~\glssymbol{VAE}~是\textbf{深度循环注意写者}(DRAW)模型\citep{Gregor2015}。
DRAW使用一个循环编码器和循环解码器并结合\gls{attention_mechanism}。
DRAW模型的生成过程包括顺序访问不同的小图像块并绘制这些点处的像素值。
我们还可以通过在~\glssymbol{VAE}~框架内使用循环编码器和解码器来定义变分~\glssymbol{RNN}~\citep{Chung-et-al-NIPS2015}来扩展~\glssymbol{VAE}~以生成序列。
从传统~\glssymbol{RNN}~生成样本仅在输出空间涉及非确定性操作。
而变分~\glssymbol{RNN}~还具有由~\glssymbol{VAE}~\gls{latent_variable}捕获的潜在更抽象层的随机变化性。

% -- 688 --

\glssymbol{VAE}~框架已不仅仅扩展到传统的变分下界,还有\,\textbf{重要加权\gls{AE}}(importance-weighted autoencoder)\citep{burda2015importance}的目标:
\begin{align}
 \CalL_k(\Vx, q) = \SetE_{\RVz^{(1)},\dots,\RVz^{(k)} \sim q(\Vz  \mid  \Vx)}
 \Bigg[ \log \frac{1}{k} \sum_{i=1}^k 
 \frac{p_{\text{model}}(\Vx, \Vz^{(i)})}{q(\Vz^{(i)}  \mid  \Vx)} \Bigg].
\end{align}
这个新的目标在$k=1$时等同于传统的下界$\CalL$。
然而,它也可以被解释为基于提议分布$q(\Vz  \mid  \Vx)$中$\Vz$的\gls{importance_sampling}而形成的真实$\log p_{\text{model}}(\Vx)$估计。
重要加权\gls{AE}目标也是$\log p_{\text{model}}(\Vx)$的下界,并且随着$k$增加而变得更紧。


\gls{VAE}与~\glssymbol{MPDBM}~和其他涉及通过近似\gls{inference}图的\gls{back_propagation}方法有一些有趣的联系 \citep{Goodfellow-et-al-NIPS2013,Stoyanov2011,brakel13a}。
这些以前的方法需要诸如\gls{meanfield}不动点方程的\gls{inference}过程来提供\gls{computational_graph}。
\gls{VAE}被定义为任意\gls{computational_graph},这使得它能适用于更广泛的概率模型族,因为它不需要将模型的选择限制到具有易处理的\gls{meanfield}不动点方程的那些模型。
\gls{VAE}还具有增加模型对数似然边界的优点,而~\glssymbol{MPDBM}~和相关模型的\gls{criterion}更具启发性,并且除了使近似\gls{inference}的结果准确外很少有概率的解释。
\gls{VAE}的一个缺点是它仅针对一个问题学习\gls{inference}网络,即给定$\Vx$ \gls{inference} $\Vz$。
较老的方法能够在给定任何其他变量子集的情况下对任何变量子集执行近似\gls{inference},因为\gls{meanfield}不动点方程指定如何在所有这些不同问题的\gls{computational_graph}之间共享参数。


\gls{VAE}的一个非常好的特性是,同时训练参数编码器与\gls{generator_network}的组合迫使模型学习一个编码器可以捕获的可预测的坐标系。
这使得它成为一个优秀的\gls{manifold_learning}算法。
\figref{fig:chap20_kingma-vae-2d-faces-manifold}展示了由\gls{VAE}学到的低维流形的例子。
图中所示的情况之一,算法发现了存在于面部图像中两个独立的变化因素:旋转角和情绪表达。

% -- 689 --

\begin{figure}[!htb]
\ifOpenSource
\centerline{\includegraphics{figure.pdf}}
\else
\centerline{
\includegraphics[width=0.44\figwidth]{Chapter20/figures/kingma-vae-2d-faces-manifold.pdf}
\includegraphics[width=0.55\figwidth]{Chapter20/figures/kingma-vae-2d-mnist-manifold.pdf}
}
\fi
\caption{由\gls{VAE}学习的高维\gls{manifold}在2维坐标系中的示例\citep{Kingma+Welling-ICLR2014}。
我们可以在纸上直接绘制两个可视化的维度,因此可以使用2维\gls{latent}编码训练模型来了解模型的工作原理(即使我们认为数据\gls{manifold}的固有维度要高得多)。
图中所示的图像不是来自训练集的样本,而是仅仅通过改变2维``编码''$\Vz$,由模型$p(\Vx \mid \Vz)$实际生成的图像$\Vx$(每个图像对应于``编码''$\Vz$位于2维均匀网格的不同选择)。
\emph{(左)}~\ENNAME{Frey}~人脸\gls{manifold}的2维映射。 其中一个维度(水平)已发现大致对应于面部的旋转,而另一个(垂直)对应于情绪表达。
\emph{(右)}~MNIST~\gls{manifold}的2维映射。
}
\label{fig:chap20_kingma-vae-2d-faces-manifold}
\end{figure}

\subsection{\glsentrytext{GAN}}
\label{sec:generative_adversarial_networks}

\firstall{GAN}\citep{Goodfellow-et-al-NIPS2014-small}是基于可微\gls{generator_network}的另一种\gls{generative_modeling}方法。


\gls{GAN}基于博弈论场景,其中\gls{generator_network}必须与对手竞争。
\gls{generator_network}直接产生样本$\Vx = g(\Vz; \Vtheta^{(g)})$。
其对手,\firstgls{discriminator_network},试图区分从训练数据抽取的样本和从生成器抽取的样本。
判别器发出由$d(\Vx; \Vtheta^{(d)})$给出的概率值,指示$\Vx$是真实训练样本而不是从模型抽取的伪造样本的概率。

% -- 690 --


形式化表示\gls{GAN}中学习的最简单方式是零和游戏,其中函数$v(\Vtheta^{(g)}, \Vtheta^{(d)})$确定判别器的收益。
生成器接收$-v(\Vtheta^{(g)}, \Vtheta^{(d)})$作为它自己的收益。
在学习期间,每个玩家尝试最大化自己的收益,因此收敛在
\begin{align}
 g^* = \underset{g}{\argmin} \, \underset{d}{\max}~ v(g, d).
\end{align}
$v$的默认选择是
\begin{align}
 v(\Vtheta^{(g)}, \Vtheta^{(d)}) = \SetE_{\RVx \sim p_{\text{data}}} 
 \log d(\Vx) + \SetE_{\Vx \sim p_{\text{model}}} \log (1 - d(\Vx)).
\end{align}
这驱使判别器试图学习将样品正确地分类为真的或伪造的。
同时,生成器试图欺骗分类器以让其相信样本是真实的。
在收敛时,生成器的样本与实际数据不可区分,并且判别器处处都输出$\frac{1}{2}$。
然后就可以丢弃判别器。


设计~\glssymbol{GAN}~的主要动机是学习过程既不需要近似\gls{inference}也不需要\gls{partition_function}梯度的近似。
当$\max_d v(g,d)$在$\Vtheta^{(g)}$中是凸的(例如,在\gls{PDF}的空间中直接执行优化的情况)时,该过程保证收敛并且是渐近一致的。


不幸的是,在实践中由神经网络表示的$g$和$d$以及$\max_d v(g, d)$不凸时,\glssymbol{GAN}~中的学习可能是困难的。
\citet{Goodfellow-ICLR2015}认为不收敛可能会引起~\glssymbol{GAN}~的\gls{underfitting}问题。
一般来说,同时对两个玩家的成本梯度下降不能保证达到平衡。
例如,考虑价值函数$v(a,b) = ab$,其中一个玩家控制$a$并产生成本$ab$,而另一玩家控制$b$并接收成本$-ab$。
如果我们将每个玩家建模为无穷小的梯度步骤,每个玩家以另一个玩家为代价降低自己的成本,则$a$和$b$进入稳定的圆形轨迹,而不是到达原点处的平衡点。
注意,极小极大化游戏的平衡不是$v$的局部最小值。
相反,它们是同时最小化的两个玩家成本的点。
这意味着它们是$v$的鞍点,相对于第一个玩家的参数是局部最小值,而相对于第二个玩家的参数是局部最大值。
两个玩家可以永远轮流增加然后减少$v$,而不是正好停在玩家没有能力降低其成本的鞍点。
目前不知道这种不收敛的问题会在多大程度上影响~\glssymbol{GAN}。
 
\citet{Goodfellow-ICLR2015}确定了另一种替代的形式化收益公式,其中博弈不再是零和,每当判别器最优时, 具有与最大似然学习相同的预期梯度。
因为最大似然训练收敛,这种~\glssymbol{GAN}~博弈的重述在给定足够的样本时也应该收敛。
不幸的是,这种替代的形式化似乎并没有提高实践中的收敛,可能是由于判别器的次优性或围绕期望梯度的高方差。

% -- 691 --

在真实实验中,\glssymbol{GAN}~博弈的最佳表现形式既不是零和也不等价于最大似然,而是\citet{Goodfellow-et-al-NIPS2014-small}引入的带有启发式动机的不同形式化。
在这种最佳性能的形式中,生成器旨在增加判别器发生错误的对数概率,而不是旨在降低判别器进行正确预测的对数概率。
这种重述仅仅是观察的结果,即使在判别器确信地拒绝所有生成器样本的情况下,它也能导致生成器代价函数的导数相对于判别器的对数保持很大。

稳定~\glssymbol{GAN}~学习仍然是一个开放的问题。
幸运的是,当仔细选择模型架构和超参数时,\glssymbol{GAN}~学习效果很好。
\citet{radford2015unsupervised}设计了一个深度卷积~\glssymbol{GAN}(DCGAN),在图像合成的任务上表现非常好,并表明其\gls{latent}的表示空间能捕获到变化的重要因素,如\figref{fig:chap15_generative_glasses}所示。
\figref{fig:chap20_lsun_small}展示了DCGAN生成器生成的图像示例。
 
\begin{figure}[!htb]
\ifOpenSource
\centerline{\includegraphics{figure.pdf}}
\else
\centering
$\vcenter{\hbox{
\includegraphics[width=0.45\figwidth]{Chapter20/figures/lsun_bedrooms_small.png}
}}
\vcenter{\hbox{
\includegraphics[width=0.45\figwidth]{Chapter20/figures/lsun_small.png}
}}$
\fi
\caption{在LSUN数据集上训练后,由~\glssymbol{GAN}~生成的图像。
\emph{(左)}由DCGAN模型生成的卧室图像,经\citet{radford2015unsupervised}许可转载。
\emph{(右)}由LAPGAN模型生成的教堂图像,经 \citet{denton2015deep}许可转载。
}
\label{fig:chap20_lsun_small}
\end{figure}

\glssymbol{GAN}~学习问题也可以通过将生成过程分成许多级别的细节来简化。
我们可以训练有条件的~\glssymbol{GAN}~\citep{mirza2014conditional},并学习从分布$p(\Vx \mid \Vy)$中采样,而不是简单地从边缘分布$p(\Vx)$中采样。
\citet{denton2015deep} 表明一系列的条件~\glssymbol{GAN}~可以被训练为首先生成非常低分辨率的图像,然后增量地向图像添加细节。
由于使用拉普拉斯金字塔来生成包含不同细节水平的图像,这种技术被称为~\ENNAME{LAPGAN}~模型。
\ENNAME{LAPGAN}~生成器不仅能够欺骗判别器网络,而且能够欺骗人类观察者,实验主体将高达40%的网络输出识别为真实数据。
请看\figref{fig:chap20_lsun_small}中~\ENNAME{LAPGAN}~生成器生成的图像示例。


\glssymbol{GAN}~训练过程中一个不寻常的能力是它可以拟合向训练点分配零概率的概率分布。
\gls{generator_network}学习跟踪其点在某种程度上类似于训练点的流形,而不是最大化特定点的对数概率。
有点矛盾的是,这意味着模型可以将负无穷大的对数似然分配给测试集,同时仍然表示人类观察者判断为能捕获生成任务本质的流形。
这不是明显的优点或缺点,并且只要向\gls{generator_network}最后一层所有生成的值添加高斯噪声,就可以保证\gls{generator_network}向所有点分配非零概率。
以这种方式添加高斯噪声的\gls{generator_network}从相同分布的采样,即使用\gls{generator_network}参数化条件\gls{gaussian_distribution}的均值所获得的分布。

% -- 692 --

\gls{dropout}~似乎在判别器网络中很重要。
特别地,在计算\gls{generator_network}的梯度时,单元应当被随机地丢弃。
使用权重除以二的确定性版本的判别器的梯度似乎不是那么有效。
同样,从不使用~\gls{dropout}~似乎会产生不良的结果。


虽然~\glssymbol{GAN}~框架被设计为用于可微\gls{generator_network},但是类似的原理可以用于训练其他类型的模型。
例如,\textbf{自监督提升}( self-supervised boosting)可以用于训练~\glssymbol{RBM}~生成器以欺骗\gls{logistic_regression}判别器~\citep{welling2002self}。

% -- 693 --

\subsection{\glsentrytext{generative_moment_matching_network}}
\label{sec:generative_moment_matching_networks}

\firstgls{generative_moment_matching_network}\citep{Li-et-al-2015,dziugaite2015training}是另一种基于可微\gls{generator_network}的\gls{generative_model}。
与~\glssymbol{VAE}~和~\glssymbol{GAN}~不同,它们不需要将\gls{generator_network}与任何其他网络配对,如不需要与用于~\glssymbol{VAE}~的\gls{inference}网络配对,也不需要与~\glssymbol{GAN}~的判别器网络。


\gls{generative_moment_matching_network}使用称为\firstgls{moment_matching}的技术训练。
\gls{moment_matching}背后的基本思想是以如下的方式训练生成器——令模型生成的样本的许多统计量尽可能与训练集中的样本相似。
在此情景下,\firstgls{moment}是对随机变量不同幂的期望。
例如,第一\gls{moment}是均值,第二\gls{moment}是平方值的均值,以此类推。
多维情况下,随机向量的每个元素可以被升高到不同的幂, 因此使得\gls{moment}可以是任意数量的形式 %?
\begin{align}
 \SetE_{\Vx} \prod_i x_i^{n_i},
\end{align}
其中 $\Vn = [n_1, n_2, \dots, n_d]^\top$是一个非负整数的向量。


在第一次检查时,这种方法似乎在计算上是不可行的。
例如,如果我们想匹配形式为$x_ix_j$的所有\gls{moment},那么我们需要最小化在$\Vx$的维度上是二次的多个值之间的差。
此外,甚至匹配所有第一和第二\gls{moment}将仅足以拟合多变量\gls{gaussian_distribution},其仅捕获值之间的线性关系。
我们使用神经网络的野心是捕获复杂的非线性关系,这将需要更多的\gls{moment}。
\glssymbol{GAN}~通过使用动态更新的判别器避免了穷举所有\gls{moment}的问题,该判别器自动将其注意力集中在\gls{generator_network}最不匹配的统计量上。


相反,我们可以通过最小化一个被称为\firstall{MMD}\citep{scholkopf2002learning,gretton2012kernel}的\gls{cost_function}来训练\gls{generative_moment_matching_network}。
该\gls{cost_function}通过向核函数定义的特征空间隐式映射,在无限维空间中测量第一\gls{moment}的误差,使得对无限维向量的计算变得可行。
当且仅当所比较的两个分布相等时,\glssymbol{MMD}~代价为零。

从可视化方面看,来自\gls{generative_moment_matching_network}的样本有点令人失望。
幸运的是,它们可以通过将\gls{generator_network}与\gls{AE}组合来改进。
首先,训练\gls{AE}以重构训练集。
接下来, \gls{AE}的编码器用于将整个训练集转换到编码空间。
然后训练\gls{generator_network}以生成编码样本, 这些编码样本可以经解码器映射到视觉上令人满意的样本。

与~\glssymbol{GAN}~不同,\gls{cost_function}仅关于一批同时来自训练集和\gls{generator_network}的实例定义。
我们不可能将训练更新作为一个训练样本或仅来自\gls{generator_network}的一个样本的函数。
这是因为必须将\gls{moment}计算为许多样本的经验平均值。
当批量大小太小时,\glssymbol{MMD}~可能低估采样分布的真实变化量。
有限的批量大小都不足以大到完全消除这个问题, 但是更大的批量大小减少了低估的量。
当批量大小太大时,训练过程就会慢得不可行,因为计算单个小梯度步长必须一下子处理许多样本。

与~\glssymbol{GAN}~一样,即使\gls{generator_network}为训练点分配零概率,仍可以使用~\glssymbol{MMD}~训练\gls{generator_network}。

% -- 694 --

\subsection{卷积生成网络}
\label{sec:convolutional_generative_networks}
当生成图像时,将卷积结构的引入\gls{generator_network}通常是有用的(见~\citet{Goodfellow-et-al-NIPS2014-small}或~\citet{dosovitskiy2015learning}的例子)。
为此,我们使用卷积算子的``转置'',如\secref{sec:variants_of_the_basic_convolution_function}所述。
这种方法通常能产生更逼真的图像,并且比不使用\gls{parameter_sharing}的全连接层使用更少的参数。


用于识别任务的\gls{convolutional_network}具有从图像到网络顶部的某些概括层(通常是类标签)的信息流。
当该图像通过网络向上流动时, 随着图像的表示变得对于有害变换保持不变,信息也被丢弃。
在\gls{generator_network}中,情况恰恰相反。
要生成图像的表示通过网络传播时必须添加丰富的详细信息, 最后产生图像的最终表示,这个最终表示当然是带有所有细节的精细图像本身(具有对象位置、姿势、纹理以及明暗)。
在卷积识别网络中丢弃信息的主要机制是池化层。
而\gls{generator_network}似乎需要添加信息。
由于大多数池化函数不可逆,我们不能将池化层求逆后放入\gls{generator_network}。
更简单的操作是仅仅增加表示的空间大小。
似乎可接受的方法是使用~\citet{dosovitskiy2015learning}引入的``去池化''。
该层对应于某些简化条件下\gls{max_pooling}的逆操作。
首先,\gls{max_pooling}操作的步幅被约束为等于池化区域的宽度。
其次,每个池化区域内的最大输入被假定为左上角的输入。
最后,假设每个池化区域内所有非最大的输入为零。
这些是非常强和不现实的假设,但它们允许我们对\gls{max_pooling}算子求逆。
逆去池化的操作分配一个零张量,然后将每个值从输入的空间坐标$i$复制到输出的空间坐标$i \times k$。
整数值$k$定义池化区域的大小。
即使驱动去池化算子定义的假设是不现实的, 后续层也能够学习补偿其不寻常的输出, 所以由整体模型生成的样本在视觉上令人满意。

% -- 695 --

\subsection{\glsentrytext{auto_regressive_network}}
\label{sec:auto_regressive_networks}
\gls{auto_regressive_network}是没有\gls{latent}随机变量的有向概率模型。
这些模型中的条件概率分布由神经网络表示(有时是极简单的神经网络,例如\gls{logistic_regression})。
这些模型的图结构是完全图。
它们可以通过概率的链式法则分解观察变量上的联合概率,从而获得形如$P(x_d \mid x_{d-1},\dots, x_1)$条件概率的乘积。
这样的模型被称为\,\textbf{完全可见的贝叶斯网络}(fully-visible Bayes networks, FVBN),并成功地以许多形式使用,首先是对每个条件分布\gls{logistic_regression}~\citep{Frey98} ,然后是带有\gls{hidden_unit}的神经网络~\citep{Bengio+Bengio-NIPS2000,Larochelle+Murray-2011-small}。
在某些形式的\gls{auto_regressive_network}中, 例如在\secref{sec:nade}中描述的~\glssymbol{NADE}~\citep{Larochelle+Murray-2011-small},我们可以引入\gls{parameter_sharing}的一种形式,它能带来统计优点(较少的唯一参数)和计算优势 (较少计算量)。
这是深度学习中反复出现的主题——\emph{特征重用}的另一个实例。


\subsection{\glsentrytext{linear_auto_regressive_network}}
\label{sec:linear_auto_regressive_networks}

\gls{auto_regressive_network}的最简单形式是没有\gls{hidden_unit}、没有参数或特征共享的形式。
每个$P(x_i \mid x_{i-1},\dots, x_1)$被参数化为\gls{linear_model}(对于实值数据的\gls{linear_regression},对于二值数据的\gls{logistic_regression},对于离散数据的\ENNAME{softmax}回归)。
这个模型由~\citet{Frey98} 引入,当有$d$个变量要建模时,该模型有$\CalO(d^2)$个参数。
如\figref{fig:chap20_fvbn}所示。

\begin{figure}[!htb]
\ifOpenSource
\centerline{\includegraphics{figure.pdf}}
\else
\centerline{\includegraphics{Chapter20/figures/fvbn}}
\fi
\caption{完全可见的\gls{BN}从前$i-1$个变量预测第$i$个变量。
\emph{(上)}~FVBN的\gls{directed_graphical_model}。
\emph{(下)}对数FVBN相应的\gls{computational_graph},其中每个预测由线性预测器作出。
}
\label{fig:chap20_fvbn}
\end{figure}

如果变量是连续的,\gls{linear_auto_regressive_network}只是表示多元\gls{gaussian_distribution}的另一种方式, 只能捕获观察变量之间线性的成对相互作用。
 
\gls{linear_auto_regressive_network}本质上是线性分类方法在\gls{generative_modeling}上的推广。
因此,它们具有与线性分类器相同的优缺点。
像线性分类器一样,它们可以用凸\gls{loss_function}训练,并且有时允许闭解形式(如在高斯情况下)。
像线性分类器一样,模型本身不提供增加其容量的方法,因此必须使用其他技术(如输入的基扩展或核技巧)来提高容量。

% -- 696 --

\subsection{\glsentrytext{neural_auto_regressive_network}}
\label{sec:neural_auto_regressive_networks}

\gls{neural_auto_regressive_network} \citep{Bengio+Bengio-trnn2000,Bengio+Bengio-NIPS2000}具有与逻辑\gls{auto_regressive_network}相同的从左到右的\gls{graphical_model}(\figref{fig:chap20_fvbn}),但在该\gls{graphical_model}结构内采用不同的条件分布参数。
新的参数化更强大,它可以根据需要随意增加容量,并允许近似任意联合分布。
新的参数化还可以引入深度学习中常见的参数共享和特征共享原理来改进\gls{generalize}能力。
设计这些模型的动机是避免传统表格\gls{graphical_model}引起的\gls{curse_of_dimensionality},并与\figref{fig:chap20_fvbn}共享相同的结构。
在表格离散概率模型中,每个条件分布由概率表表示,其中所涉及的变量的每个可能配置都具有一个条目和一个参数。
通过使用神经网络,可以获得两个优点:
\begin{enumerate}
 \item 通过具有$(i-1) \times k$个输入和$k$个输出的神经网络(如果变量是离散的并有$k$个值,使用\gls{one_hot}编码)参数化每个$P(x_i  \mid  x_{i-1}, \dots, x_1)$,让我们不需要指数量级参数(和样本)的情况下就能估计条件概率, 然而仍然能够捕获随机变量之间的高阶依赖性。

 \item 不需要对预测每个$x_i$使用不同的神经网络,如\figref{fig:chap20_neural_autoregressive}所示的从左到右连接,允许将所有神经网络合并成一个。
 等价地,它意味着为预测$x_i$所计算的\gls{hidden_layer}特征可以重新用于预测$x_{i+k}~(k > 0)$。
 因此\gls{hidden_unit}被\emph{组织}成第$i$组中的所有单元仅依赖于输入值 $x_1, \dots, x_i$的特定的组。
 用于计算这些\gls{hidden_unit}的参数被联合优化以改进对序列中所有变量的预测。
 这是\emph{重用原理}的一个实例,这是从循环和\gls{convolutional_network}架构到多任务和\gls{transfer_learning}的场景中反复出现的深度学习原理。

\end{enumerate}

% -- 697 --
\begin{figure}[!htb]
\ifOpenSource
\centerline{\includegraphics{figure.pdf}}
\else
\centerline{\includegraphics{Chapter20/figures/neural_autoregressive}}
\fi
\caption{\gls{neural_auto_regressive_network}从前$i-1$个变量预测第$i$个变量$x_i$,但经参数化后,作为$x_1,\dots,x_i$函数的特征(表示为$h_i$的\gls{hidden_unit}的组)可以在预测所有后续变量$x_{i+1},x_{i+2},\dots,x_{d}$时重用。
}
\label{fig:chap20_neural_autoregressive}
\end{figure}


如在\secref{sec:linear_units_for_gaussian_output_distributions}中讨论的,使神经网络的输出预测$x_i$条件分布的\emph{参数},每个$P(x_i  \mid  x_{i-1}, \dots, x_1)$就可以表示一个条件分布。
虽然原始\gls{neural_auto_regressive_network}最初是在纯粹离散多变量数据(带有~\ENNAME{sigmoid}~输出的~\ENNAME{Bernoulli}~变量或~\ENNAME{softmax}~输出的~\ENNAME{Multinoulli}~变量)的背景下评估,但我们可以自然地将这样的模型扩展到连续变量或同时涉及离散和连续变量的联合分布。

% -- 698 --

\subsection{\glssymbol{NADE}}
\label{sec:nade}

\firstall{NADE}是最近非常成功的\gls{neural_auto_regressive_network}的一种形式 \citep{Larochelle+Murray-2011-small}。
与~\citet{Bengio+Bengio-NIPS2000}的原始\gls{neural_auto_regressive_network}中的连接相同,但~\glssymbol{NADE}~引入了附加的\gls{parameter_sharing}方案,如\figref{fig:chap20_NADE}所示。
不同组$j$的\gls{hidden_unit}的参数是共享的。

从第$i$个输入$x_i$到第$j$组\gls{hidden_unit}的第$k$个元素$h_k^{(j)} ~(j \geq i)$的权重$W_{j,k,i}^{'}$是组内共享的:
\begin{align}
 W_{j,k,i}^{'} = W_{k,i}.
\end{align}
其余$j<i$的权重为零。


\begin{figure}[!htb]
\ifOpenSource
\centerline{\includegraphics{figure.pdf}}
\else
\centerline{\includegraphics{Chapter20/figures/NADE}}
\fi
\caption{\gls{NADE}(\glssymbol{NADE})的示意图。
\gls{hidden_unit}被组织在组$\Vh^{(j)}$中,使得只有输入$x_1,\dots,x_i$参与计算$\Vh^{(i)}$和预测$P(x_j \mid x_{j-1},\dots,x_1)$(对于$j> i$)。
\glssymbol{NADE}~使用特定的权重共享模式区别于早期的\gls{neural_auto_regressive_network}:$W_{j,k,i}^{'} = W_{k,i}$被共享于所有从$x_i$到任何$j \geq i$组中第$k$个单元的权重(在图中使用相同的线型表示复制权重的每个实例)。 
注意向量$(W_{1,i}, W_{2,i},\dots,W_{n,i})$记为$\MW_{:,i}$。
}
\label{fig:chap20_NADE}
\end{figure}

% -- 699 --

\citet{Larochelle+Murray-2011-small} 选择了这种共享方案,使得~\glssymbol{NADE}~模型中的正向传播与在\gls{meanfield}\gls{inference}中执行的计算大致相似,以填充~\glssymbol{RBM}~中缺失的输入。
这个\gls{meanfield}\gls{inference}对应于运行具有共享权重的循环网络,并且该\gls{inference}的第一步与~\glssymbol{NADE}~中的相同。
使用~\glssymbol{NADE}~的唯一区别是,连接\gls{hidden_unit}到输出的输出权重独立于连接输入单元和\gls{hidden_unit}的权重进行参数化。
在~\glssymbol{RBM}~中,隐藏到输出的权重是输入到隐藏权重的转置。
\glssymbol{NADE}~架构可以扩展为不仅仅模拟\gls{meanfield}循环\gls{inference}的一个\gls{time_step},而是$k$步。
这种方法称为~\glssymbol{NADE}-$k$~\citep{Raiko-et-al-2014}。


如前所述,\gls{auto_regressive_network}可以被扩展成处理连续数据。
用于参数化连续密度的特别强大和通用的方法是混合权重为$\alpha_i$(组$i$的系数或先验概率),每组条件均值为$\mu_i$和每组条件方差为$\sigma_i^2$的高斯混合体。
一个称为~\ENNAME{RNADE}~的模型\citep{Benigno-et-al-NIPS2013-small} 使用这种参数化将~\glssymbol{NADE}~扩展到实值。
与其他混合密度网络一样,该分布的参数是网络的输出,由~\ENNAME{softmax}~单元产生混合的权量概率以及参数化的方差,因此可使它们为正的。
由于条件均值$\mu_i$和条件方差$\sigma_i^2$之间的相互作用,\gls{SGD}在数值上可能会表现不好。
为了减少这种困难, \citet{Benigno-et-al-NIPS2013-small}在后向传播阶段使用伪梯度代替平均值上的梯度。

另一个非常有趣的神经自回归架构的扩展摆脱了为观察到的变量选择任意顺序的需要\citep{Uria+al-ICML2014}。
在\gls{auto_regressive_network}中,该想法是训练网络以能够通过随机采样顺序来处理任何顺序,并将信息提供给指定哪些输入被观察的\gls{hidden_unit}(在条件条的右侧),以及哪些是被预测并因此被认为是缺失的(在条件条的左侧)。
这是不错的性质,因为它允许人们非常高效地使用训练好的\gls{auto_regressive_network}来\emph{执行任何\gls{inference}问题}(即从给定任何变量的子集,从任何子集上的概率分布预测或采样)。
最后,由于变量的许多顺序是可能的(对于$n$个变量是$n!$),并且变量的每个顺序$o$产生不同的$p(\RVx \mid o)$,我们可以组成许多$o$值模型的集成:
\begin{align}
 p_{\text{ensemble}}(\RVx) = \frac{1}{k} \sum_{i=1}^k p(\RVx  \mid  o^{(i)}).
\end{align}
这个集成模型通常能更好地\gls{generalize},并且为测试集分配比单个排序定义的单个模型更高的概率。

% -- 700 --

在同一篇文章中,作者提出了深度版本的架构,但不幸的是,这立即使计算成本像原始\gls{neural_auto_regressive_network}一样高\citep{Bengio+Bengio-NIPS2000}。
第一层和输出层仍然可以在$\CalO(nh)$的乘法-加法操作中计算,如在常规~\glssymbol{NADE}~中,其中$h$是\gls{hidden_unit}的数量(\figref{fig:chap20_NADE}和\figref{fig:chap20_neural_autoregressive}中的组$h_i$的大小),而它在~\citet{Bengio+Bengio-NIPS2000}中是$\CalO(n^2h)$。
然而,对于其他\gls{hidden_layer}的计算量是$\CalO(n^2 h^2)$(假设在每个层存在$n$组$h$个\gls{hidden_unit},且在$l$层的每个``先前''组参与预测$l+1$层处的``下一个''组)。
如在~\citet{Uria+al-ICML2014}中, 使$l+1$层上的第$i$个组仅取决于第$i$个组,$l$层处的计算量将减少到$\CalO(nh^2)$,但仍然比常规~\glssymbol{NADE}~差$h$倍。


\section{从\gls{AE}采样}
\label{sec:drawing_samples_from_autoencoders}

在\chapref{chap:autoencoders}中,我们看到许多种学习数据分布的\gls{AE}。
\gls{score_matching}、 \gls{DAE}和\gls{CAE}之间有着密切的联系。
这些联系表明某些类型的\gls{AE}以某些方式学习数据分布。
我们还没有讨论如何从这样的模型中采样。

某些类型的\gls{AE},例如\gls{VAE},明确地表示概率分布并且允许直接的\gls{ancestral_sampling}。
而大多数其他类型的\gls{AE}则需要~\glssymbol{mcmc}~采样。

\gls{CAE}被设计为恢复数据流形切面的估计。
这意味着使用注入噪声的重复编码和解码将引起沿着流形表面的随机游走\citep{Rifai-icml2012-small,Mesnil-et-al-LW2012}。
这种流形扩散技术是\gls{markov_chain}的一种。

更一般的\gls{markov_chain}还可以从任何\gls{DAE}中采样。

% -- 701 --

\subsection{与任意\gls{DAE}相关的\gls{markov_chain}}
\label{sec:markov_chain_associated_with_any_denoising_autoencoder}

上述讨论留下了一个开放问题——注入什么噪声和从哪获得\gls{markov_chain}(可以根据\gls{AE}估计的分布生成样本)。
\citet{Bengio-et-al-NIPS2013-small}展示了如何构建这种用于\textbf{广义\gls{DAE}}(generalized denoising autoencoder)的\gls{markov_chain}。
广义\gls{DAE}由去噪分布指定,给定损坏输入后,对干净输入的估计进行采样。

根据估计分布生成的\gls{markov_chain}的每个步骤由以下子步骤组成,如\figref{fig:chap20_dae_markov_chain}所示:
\begin{enumerate}
 \item 从先前状态$\Vx$开始,注入损坏噪声,从$C(\tilde{\Vx}  \mid  \Vx)$中采样$\tilde{\Vx}$。
 \item 将$\tilde{\Vx}$编码为$\Vh = f(\tilde{\Vx})$。
 \item 解码$\Vh$以获得$p(\RVx  \mid  \Vomega = g(\Vh)) = p(\RVx  \mid  \tilde{\Vx})$的参数$\Vomega = g(\Vh)$。
 \item 从$p(\RVx  \mid  \Vomega = g(\Vh)) = p(\RVx  \mid  \tilde{\Vx})$采样下一状态$\Vx$。
\end{enumerate}
\citet{Bengio-et-al-ICML-2014} 表明,如果\gls{AE} $p(\RVx  \mid  \tilde{\Vx})$形成对应真实条件分布的一致估计量,则上述\gls{markov_chain}的平稳分布形成数据生成分布$\RVx$的一致估计量(虽然是隐式的)。

\begin{figure}[!htb]
\ifOpenSource
\centerline{\includegraphics{figure.pdf}}
\else
\centerline{\includegraphics{Chapter20/figures/dae_markov_chain}}
\fi
\caption{\gls{markov_chain}的每个步骤与训练好的\gls{DAE}相关联,根据由去噪对数似然\gls{criterion}隐式训练的概率模型生成样本。
每个步骤包括:(a)通过损坏过程$C$向状态$\Vx$注入噪声产生$\tilde{\Vx}$,(b)用函数$f$对其编码,产生$\Vh = f(\tilde{\Vx})$,(c)用函数$g$解码结果, 产生用于\gls{reconstruction}分布的参数$\Vomega$,(d)给定$\Vomega$,从\gls{reconstruction}分布$p(\RVx \mid \Vomega = g(f(\tilde{\Vx})))$采样新状态。
在典型的平方\gls{reconstruction_error}情况下,$g(\Vh) = \hat{\Vx}$,并估计$\SetE [ \Vx \mid \tilde{\Vx}]$,损坏包括添加高斯噪声,并且从$p(\RVx | \Vomega)$的采样包括第二次向\gls{reconstruction} $\hat{\Vx}$添加高斯噪声。
后者的噪声水平应对应于\gls{reconstruction}的\gls{mean_squared_error},而注入的噪声是控制混合速度以及估计器平滑经验分布程度的超参数\citep{Vincent-NC-2011-small}。
在这所示的例子中,只有$C$和$p$条件是随机步骤($f$和$g$是确定性计算),我们也可以在\gls{AE}内部注入噪声,如\gls{GSN}~\citep{Bengio-et-al-ICML-2014}。
}
\label{fig:chap20_dae_markov_chain}
\end{figure}

% -- 702 --

\subsection{夹合与条件采样}
\label{sec:clamping_and_conditional_sampling}

与\gls{BM}类似,\gls{DAE}及其推广(例如下面描述的~\glssymbol{GSN})可用于从条件分布$p(\RVx_f  \mid  \RVx_o)$中采样,只需夹合\emph{观察}单元$\RVx_f$并在给定$\RVx_f$和采好的\gls{latent_variable}(如果有的话)下仅重采样\emph{自由}单元$\RVx_o$。
例如,\glssymbol{MPDBM}~可以被解释为\gls{DAE}的一种形式,并且能够采样丢失的输入。
\glssymbol{GSN}~随后将~\glssymbol{MPDBM}~中的一些想法推广以执行相同的操作~\citep{Bengio-et-al-ICML-2014}。
\citet{Alain-et-al-arxiv2015} 从~\citet{Bengio-et-al-ICML-2014}的命题1中发现了一个缺失条件,即转移算子(由从链的一个状态到下一个状态的随机映射定义)应该满足\firstgls{detailed_balance}的属性, 表明无论转移算子正向或反向运行,\gls{markov_chain}都将保持平衡。

在\figref{fig:chap20_inpainting1500-half_croppedhalf__with_nearest}中展示了夹合一半像素(图像的右部分)并在另一半上运行\gls{markov_chain}的实验。

\begin{figure}[!htb]
\ifOpenSource
\centerline{\includegraphics{figure.pdf}}
\else
\centerline{\includegraphics[width=0.8\textwidth]{Chapter20/figures/inpainting1500-half_croppedhalf__with_nearest}}
\fi
\caption{在每步仅重采样左半部分,夹合图像的右半部分并运行\gls{markov_chain}的示意图。
这些样本来自\gls{reconstruction}~MNIST数字的~\glssymbol{GSN}(每个\gls{time_step}使用回退过程)。
}
\label{fig:chap20_inpainting1500-half_croppedhalf__with_nearest}
\end{figure}

\subsection{回退训练过程}
\label{sec:walk_back_training_procedure}
回退训练过程由~\citet{Bengio-et-al-NIPS2013-small} 等人提出,作为一种加速\gls{DAE}生成训练收敛的方法。
不像执行一步编码-解码重建,该过程有代替的多个随机编码-解码步骤组成(如在生成\gls{markov_chain}中),以训练样本初始化(正如在\secref{sec:stochastic_maximum_likelihood_and_contrastive_divergence}中描述的\gls{contrastive_divergence}算法),并惩罚最后的概率重建(或沿途的所有重建)。

训练$k$个步骤与训练一个步骤是等价的(在实现相同稳态分布的意义上),但是实际上可以更有效地去除来自数据的伪模式。

% -- 703 --

\section{\glsentrytext{GSN}}
\label{sec:generative_stochastic_networks}
\firstall{GSN} \citep{Bengio-et-al-ICML-2014} 是\gls{DAE}的推广,除可见变量(通常表示为$\RVx$)之外,在生成\gls{markov_chain}中还包括\gls{latent_variable} $\RVh$。

\glssymbol{GSN}~由两个条件概率分布参数化, 指定\gls{markov_chain}的一步:
\begin{enumerate}
 \item $p(\RVx^{(k)}  \mid \RVh^{(k)} )$指示在给定当前\gls{latent}状态下如何产生下一个可见变量。
 这种``重建分布''也可以在\gls{DAE}、 \glssymbol{RBM}、\glssymbol{DBN}~和~\glssymbol{DBM}~中找到。
 \item $p(\RVh^{(k)}  \mid \RVh^{(k-1)}, \RVx^{(k-1)})$指示在给定先前的\gls{latent}状态和可见变量下如何更新\gls{latent}状态变量。
\end{enumerate}

\gls{DAE}和~\glssymbol{GSN}~不同于经典的概率模型(有向或无向),它们自己参数化生成过程而不是通过可见和\gls{latent_variable}的联合分布的数学形式。
相反,后者\emph{如果存在则隐式}地定义为生成\gls{markov_chain}的稳态分布。
存在稳态分布的条件是温和的,并且需要与标准~\glssymbol{mcmc}~方法相同的条件(见\secref{sec:markov_chain_monte_carlo_methods})。
这些条件是保证链混合的必要条件,但它们可能被某些过渡分布的选择(例如,如果它们是确定性的)所违反。

% -- 704 --

我们可以想象~\glssymbol{GSN}~不同的训练\gls{criterion}。
由~\citet{Bengio-et-al-ICML-2014} 提出和评估的只对可见单元上对数概率的重建,如应用于\gls{DAE}。
通过将$\RVx^{(0)} = \Vx$ 夹合到观察到的样本并且在一些后续\gls{time_step}处使生成$\Vx$的概率最大化,即最大化$\log p(\RVx^{(k)} = \Vx  \mid  \RVh^{(k)})$, 其中给定$\RVx^{(0)} = \Vx$后,$\RVh^{(k)}$从链中采样。
为了估计相对于模型其他部分的$\log p(\RVx^{(k)} = \Vx  \mid  \RVh^{(k)})$的梯度,\citet{Bengio-et-al-ICML-2014}使用了在\secref{sec:back_propagation_through_random_operations}中介绍的\gls{reparametrization_trick}。

回退训练过程(在\secref{sec:walk_back_training_procedure}中描述)可以用来改善训练~\glssymbol{GSN}~的收敛性~\citep{Bengio-et-al-ICML-2014} 。


\subsection{判别性\,\glssymbol{GSN}}
\label{sec:discriminant_gsns}
\glssymbol{GSN}的原始公式\citep{Bengio-et-al-ICML-2014} 用于\gls{unsupervised_learning}和对观察数据$\RVx$的$p(\RVx)$的隐式建模,但是我们可以修改框架来优化$p(\RVy  \mid  \Vx)$。


例如, \citet{Zhou+Troyanskaya-ICML2014} 以如下方式推广~\glssymbol{GSN},只\gls{back_propagation}输出变量上的重建对数概率,并保持输入变量固定。
他们将这种方式成功应用于建模序列(蛋白质二级结构),并在\gls{markov_chain}的转换算子中引入(一维)卷积结构。
重要的是要记住,对于\gls{markov_chain}的每一步,我们需要为每个层生成新序列,并且该序列用于在下一\gls{time_step}计算其他层的值(例如下面一个和上面一个)的输入。

因此,\gls{markov_chain}确实不只是输出变量(与更高层的\gls{hidden_layer}相关联),并且输入序列仅用于条件化该链,其中\gls{back_propagation}使得它能够学习输入序列如何条件化由\gls{markov_chain}隐含表示的输出分布。
因此这是在结构化输出中使用~\glssymbol{GSN}~的一个例子。

\citet{Zohrer+Pernkopf-NIPS2014-small} 引入了一个混合模型,通过简单地添加(使用不同的权重)监督和非监督成本即$\RVy$和$\RVx$的重建对数概率, 组合了监督目标(如上面的工作)和无监督目标(如原始的~\glssymbol{GSN})。
\citet{Larochelle+Bengio-2008-small}以前在~\glssymbol{RBM}~中就提出了这样的混合标准。
他们展示了在这种方案下分类性能的提升。

% -- 705 --

\section{其他生成方案}
\label{sec:other_generation_schemes}
目前为止我们已经描述的方法,使用~\glssymbol{mcmc}~采样、\gls{ancestral_sampling}或两者的一些混合来生成样本。
虽然这些是\gls{generative_modeling}中最流行的方法,但它们绝不是唯一的方法。


\citet{Sohl-Dickstein-et-al-ICML2015} 开发了一种基于非平衡热力学学习\gls{generative_model}的\,\textbf{扩散反演}(diffusion inversion)训练方案。
该方法基于我们希望从中采样的概率分布具有结构的想法。
这种结构会被递增地使概率分布具有更多熵的扩散过程逐渐破坏。
为了形成\gls{generative_model},我们可以反过来运行该过程,通过训练模型逐渐将结构恢复到非结构化分布。
通过迭代地应用使分布更接近目标分布的过程,我们可以逐渐接近该目标分布。
在涉及许多迭代以产生样本的意义上,这种方法类似于~\glssymbol{mcmc}~方法。
然而,模型被定义为由链的最后一步产生的概率分布。
在这个意义上,没有由迭代过程诱导的近似。
\citet{Sohl-Dickstein-et-al-ICML2015} 介绍的方法也非常接近于\gls{DAE}的生成解释(\secref{sec:markov_chain_associated_with_any_denoising_autoencoder})。
与\gls{DAE}一样, 扩散反演训练一个尝试概率地撤消添加的噪声效果的转移算子。
不同之处在于,扩散反演只需要消除扩散过程的一个步骤,而不是一直返回到一个干净的数据点。
这解决了\gls{DAE}的普通重建对数似然目标中存在的以下两难问题:小噪声的情况下学习者只能看到数据点附近的配置, 而在大噪声的情况下,\gls{DAE}被要求做几乎不可能的工作(因为去噪分布是高度复杂和\gls{multimodal}的)。
利用扩散反演目标,学习者可以更精确地学习数据点周围的密度形状,以及去除可能在远离数据点处出现的假性模式。

样本生成的另一种方法是\firstall{ABC}框架\citep{rubin1984bayesianly}。
在这种方法中,样本被拒绝或修改以使样本选定函数的\gls{moment}匹配期望分布的那些\gls{moment}。
虽然这个想法与\gls{moment_matching}一样使用样本的\gls{moment},但它不同于\gls{moment_matching},因为它修改样本本身,而不是训练模型来自动发出具有正确\gls{moment}的样本。
\citet{BachmanP15} 展示了如何在深度学习的背景下使用~\glssymbol{ABC}~中的想法,即使用~\glssymbol{ABC}~来\gls{shaping}~\glssymbol{GSN}~的~\glssymbol{mcmc}~轨迹。

我们期待更多其他等待发现的\gls{generative_modeling}方法。

% -- 706 --

\section{评估\glsentrytext{generative_model}}
\label{sec:evaluating_generative_models}

研究\gls{generative_model}的研究者通常需要将一个\gls{generative_model}与另一个\gls{generative_model}比较,通常是为了证明新发明的\gls{generative_model}比之前存在的模型更能捕获一些分布。

这可能是一个困难且微妙的任务。
通常,我们不能实际评估模型下数据的对数概率,但仅可以评估一个近似。
在这些情况下,重要的是思考和沟通清楚正在测量什么。
例如,假设我们可以评估模型A对数似然的随机估计和模型B对数似然的确定性下界。
如果模型A得分高于模型B,哪个更好?
如果我们关心确定哪个模型具有分布更好的内部表示,我们实际上不能说哪个更好,除非我们有一些方法来确定模型B的边界有多松。
然而,如果我们关心在实践中该模型能用得多好,例如执行异常检测,则基于特定于感兴趣的实际任务的准则,可以公平地说模型是更好的,例如基于排名测试样例和排名标准,如\gls{precision}和\gls{recall}。

评估\gls{generative_model}的另一个微妙之处是,评估指标往往是自身困难的研究问题。 
可能很难确定模型是否被公平比较。
例如,假设我们使用~\glssymbol{AIS}~来估计$\log Z$以便为我们刚刚发明的新模型计算$\log \tilde{p}(\Vx) - \log Z$。
\glssymbol{AIS}~计算经济的实现可能无法找到模型分布的几种模式并低估$Z$,这将导致我们高估$\log p(\Vx)$。
因此可能难以判断高似然估计是否是良好模型或不好的~\glssymbol{AIS}~实现导致的结果。
 
机器学习的其他领域通常允许在数据预处理中有一些变化。
例如,当比较\gls{object_recognition}算法的准确性时,通常可接受的是对每种算法略微不同地预处理输入图像(基于每种算法具有何种输入要求)。
而因为预处理的变化,会导致\gls{generative_modeling}的不同,甚至非常小和微妙的变化也是完全不可接受的。
对输入数据的任何更改都会改变要捕获的分布,并从根本上改变任务。
例如,将输入乘以$0.1$将人为地将概率增加$10$倍。

% -- 707 --

预处理的问题通常在基于MNIST数据集上的\gls{generative_model}产生,MNIST数据集是非常受欢迎的\gls{generative_modeling}基准之一。
MNIST由灰度图像组成。
一些模型将MNIST图像视为实向量空间中的点,而其他模型将其视为二值。
还有一些将灰度值视为二值样本的概率。
我们必须将实值模型仅与其他实值模型比较,二值模型仅与其他二值模型进行比较。
否则,测量的似然性不在相同的空间。
对于二值模型,对数似然可以最多为零,而对于实值模型,它可以是任意高的,因为它是关于密度的测度。
在二值模型中,比较使用完全相同的二值化模型是重要的。
例如,我们可以将$0.5$设为阈值后,将灰度像素二值化为0或1,或者通过由灰度像素强度给出样本为1的概率来采一个随机样本。
如果我们使用随机二值化,我们可能将整个数据集二值化一次,或者我们可能为每个训练步骤采不同的随机样例,然后采多个样本进行评估。
这三个方案中的每一个都会产生极不相同的似然数,并且当比较不同的模型时,两个模型使用相同的二值化方案来训练和评估是重要的。
事实上,应用单个随机二值化步骤的研究者共享包含随机二值化结果的文件,使得基于二值化步骤的不同输出的结果没有差别。

因为从数据分布生成真实样本是\gls{generative_model}的目标之一,所以实践者通常通过视觉检查样本来评估\gls{generative_model}。
在最好的情况下,这不是由研究人员本身,而是由不知道样品来源的实验受试者完成 \citep{denton2015deep}。
不幸的是,非常差的概率模型可能会产生非常好的样本。
验证模型是否仅复制一些训练示例的常见做法如\figref{fig:chap16_fig-ssrbm}所示。
该想法是根据在$\Vx$空间中的欧几里得距离,为一些生成的样本显示它们在训练集中的最近邻。
此测试旨在检测模型\gls{overfitting}训练集并仅再现训练实例的情况。
甚至可能同时\gls{underfitting}和\gls{overfitting},但仍然能产生单独看起来好的样本。
想象一下,\gls{generative_model}用狗和猫的图像训练时,但只是简单地学习来重现狗的训练图像。
这样的模型明显\gls{overfitting},因为它不能产生不在训练集中的图像,但是它也\gls{underfitting},因为它不给猫的训练图像分配概率。
然而,人类观察者将判断狗的每个个体图像都是高质量的。
在这个简单的例子中,对于能够检查许多样本的人类观察者来说,确定猫的不存在是容易的。
在更实际的设定中,在具有数万个模式的数据上训练后的\gls{generative_model}可以忽略少数模式,并且人类观察者不能容易地检查或记住足够的图像以检测丢失的变化。

% -- 708 --

由于样本的视觉质量不是可靠的标准,所以当计算可行时,我们通常还评估模型分配给测试数据的对数似然。
不幸的是,在某些情况下,似然性似乎不可能测量我们真正关心的模型的任何属性。
例如,MNIST的实值模型可以将任意低的方差分配给从不改变的背景像素,获得任意高的似然。
即使这不是一个非常有用的事情,检测这些常量特征的模型和算法可以获得无限的奖励。
实现接近负无穷代价的可能性存在于任何实值的最大似然问题中,但是对于MNIST的\gls{generative_model}问题尤为严重, 因为许多输出值是不需要预测的。
这强烈地表明需要开发评估\gls{generative_model}的其他方法。


\citet{Theis2015d} 回顾了评估\gls{generative_model}所涉及的许多问题,包括上述的许多想法。
他们强调了\gls{generative_model}有许多不同的用途,并且指标的选择必须与模型的预期用途相匹配。
例如,一些\gls{generative_model}更好地为大多数真实的点分配高概率,而其他\gls{generative_model}擅长于不将高概率分配给不真实的点。
这些差异可能源于\gls{generative_model}是设计为最小化$D_{\text{KL}}(p_{\text{data}} || p_{\text{model}})$还是$D_{\text{KL}}(p_{\text{model}} || p_{\text{data}})$,如\figref{fig:chap3_kl_direction_color}所示。
不幸的是,即使我们将每个指标的使用限制在最适合的任务上,目前使用的所有指标仍存在严重的缺陷。
因此,\gls{generative_modeling}中最重要的研究课题之一不仅仅是如何提升\gls{generative_model},事实上还包括了设计新的技术来衡量我们的进步。

% -- 709 --

\section{结论}
\label{sec:conclusion}

为了让模型理解表示在给定训练数据中的大千世界,训练具有\gls{hidden_unit}的\gls{generative_model}是一种有力方法。
通过学习模型$p_{\text{model}}(\Vx)$和表示$p_{\text{model}}(\Vh  \mid  \Vx)$,\gls{generative_model}可以解答$\Vx$输入变量之间关系的许多\gls{inference}问题, 并且可以在层次的不同层对$\Vh$求期望来提供表示$\Vx$的许多不同方式。
\gls{generative_model}承诺为~\glssymbol{AI}~系统提供它们需要理解的、所有不同直观概念的框架,让它们有能力在面对不确定性的情况下推理这些概念。
我们希望我们的读者能够找到增强这些方法的新途径,并继续探究学习和智能背后原理的旅程。

