
\section{均值不等式初探}
\label{sec:mean-value-inequality}

均值不等式是中学数学里的一个重要的基本不等式,其所体现的量与量之间的大
小关系简单而深刻,本文从中略窥端倪。

完整的均值不等式可以这样表述:设$a$与$b$是两个正实数,那么成立着如下的
不等式:
\begin{equation}
  \label{eq-mean-value-inequality}
  \frac{2}{\frac{1}{a}+\frac{1}{b}}\leqslant \sqrt{ab} \leqslant
  \frac{a+b}{2} \leqslant \sqrt{\frac{a^2+b^2}{2}}
\end{equation}
式中的三个小于等于号都只在$a=b$时才能成立。

上述不等式中,四个项从左至右依次被称为调和平均数、几何平均数、算术平均
数和平方平均数。

显然这四个平均数都具有对称性,即$a$和$b$地位等价,而且在$a\neq b$时它
们都位于两者之间,而在$a=b$时都等于共同值。

四个项都具有形式$f^{-1}(\frac{f(a)+f(b)}{2})$,这表明这四个项分别
由四个函数来决定:$M_{f(x)}(a,b)$,第一项$f(x)=\frac{1}{x}$,第二项...嗯,$f(x)=\ln{x}$,
第三项$f(x)=x$,第四项$f(x)=x^2$,由此可以将完整的均值不等式写成下面的
形式:
\begin{equation}
  \label{eq:mean-value-inequality-with-function}
  M_{\frac{1}{x}}(a,b) \leqslant M_{\ln{x}}(a,b) \leqslant M_x(a,b)
  \leqslant M_{x^2}(a,b)
\end{equation}

这就是说,这四个项从构成形式上只相差一个具体的函数,因此它们之所以会有
如\ref{eq-mean-value-inequality}中的大小关系,完全是因为相应函数的性质所引起的。

从图象上看,对于函数$f(x)$,如果连接$(a,f(a))$和$(b,f(b))$这两个点,再
作直线$y=\frac{f(a)+f(b)}{2}$,则其与函数曲线交点的横坐标即是
$f^{-1}(\frac{f(a)+f(b)}{2})$,因此,若函数曲线向左凸的越多,则由其决
定的平均数越小,反之则越大,而算术平均数则为一个分水岭。

既然算术平均数为一个分水岭,假若某个函数所决定的平均数小于算术平均数,
即$f^{-1}(\frac{f(a)+f(b)}{2}) \leqslant \frac{a+b}{2}$,又假若这个函数又是严格单调增加的,那
么两边取函数值即得$\frac{f(a)+f(b)}{2} \leqslant f(\frac{a+b}{2})$,这
表明它是上凸函数,同样,如果它是严格单调减少的函数,则它是下凸函数。可
以得出个结论:严格增加的上凸函数和严格减少的下凸函数,它们决定的平均数
要小于算术平均数,而严格减少的上凸函数和严格增加的下凸函数,所决定的平
均数要大于算术平均数。

接下来引入排序不等式,先看均值不等式的一个变形:
\begin{equation}
  a^2+b^2 \geqslant 2ab
\end{equation}
为了看清其中所包含的排序的特性,将其改写为下面形状:
\begin{equation}
  aa+bb \geqslant ab+ba
\end{equation}
可以看到,仅仅是换了一下两个乘子的位置,就导致运算结果变小了。

现在将其一般化,设有四个数分为两组,$x_1, x_2$和$y_1,y_2$,且每组
内都已经按大小排好序:$x_1 \leqslant x_2$,$y_1 \leqslant y_2$,则只要
作差就可以证明成立着下面的不等式:
\begin{equation}
  x_1y_1+x_2y_2 \geqslant x_1y_2+x_2y_1
\end{equation}
这个结论容易推广到每组多个数的情况,设有两组实数$x_i$和
$y_i(i=0,1,\ldots,n)$,假定每组数都已经按大小排好序(两组数大小方向相同),
则有
\begin{equation}
  \label{eq:sort-equality}
  \sum_{i=0}^nx_iy_i \geqslant \sum_{i=0}^nx_iy_{r_i} \geqslant \sum_{i=0}x_iy_{n-i}
\end{equation}
用通俗语言说就是:同序和最大,反序和最小,乱序和居中,上式中${r_i}$是
$0,1,\ldots,n$的任意一个排序。

排序不等式体现了乘法运算中一个事实,即两组数对应相乘再相加,要得到大的结果,
应把大数与大数相乘,小数与小数相乘,反之,要得到小的结果,则应把大数与
小数相乘,小数则与大数相乘。

物理学中匀变速直线运动中有一个有意思的问题:中间时刻的瞬时速度和中间位
移处的瞬间速度,两者谁大?有了均值不等式,这个问题就迎刃而解了,中间时
刻的瞬时速度是起点与终点两个速度的算术平均数,而中间位移处的瞬间速度则
为平方平均数,结论就是很自然的了。