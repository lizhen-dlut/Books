\documentclass{ctexart}
\usepackage[left=1cm,right=1cm,bottom=1.5cm]{geometry}
\usepackage{amsmath}
\usepackage{amssymb} 
\usepackage{array}

\usepackage{enumerate}
\usepackage{enumitem}
\usepackage{fancyhdr}
\usepackage{fontenc} 
\usepackage{graphicx,calc}

\usepackage{ifthen}
\usepackage{inputenc} 
\usepackage{lastpage}
\usepackage{lipsum}
\usepackage{multicol}
\usepackage{multirow}
\usepackage{tikz}
\usetikzlibrary{calc}

\usepackage{hyperref}

\newcommand{\productName}{\uline{``乐孕安''}}
\newcommand{\DocumentTitle}{\kaishu{\LARGE 乐土精准医疗 \\
	\vspace{1ex}
	 \Large\productName 调查问卷}}
\newcommand{\doumentAuthor}{张洋}

\hypersetup{
	pdftitle= {\DocumentTitle} ,
	pdfauthor= {\doumentAuthor}, 
	pdfkeywords = {乐孕安, 测试题}, 
	pdfcreator={\doumentAuthor},
	pdfproducer={\doumentAuthor},
	pdfstartview={FitH},
	hidelinks
} 


\newlength{\lengthOptionA}
\newlength{\lengthOptionB}
\newlength{\lengthOptionC}
\newlength{\lengthOptionD}
\newlength{\lhalf}
\newlength{\lquarter}
\newlength{\lmax}
\setlength{\lhalf}{0.5\textwidth}
\setlength{\lquarter}{0.25\textwidth}

\newcommand{\choiceOptions}[4]
{
%	\hfill \\[.5pt]%
	\settowidth{\lengthOptionA}{A.~#1~~~}
	\settowidth{\lengthOptionB}{B.~#2~~~}
	\settowidth{\lengthOptionC}{C.~#3~~~}
	\settowidth{\lengthOptionD}{D.~#4~~~}
	
	\ifthenelse{\lengthtest{\lengthOptionA>\lengthOptionB}}
		{\setlength{\lmax}{\lengthOptionA}}
		{\setlength{\lmax}{\lengthOptionB}}
	\ifthenelse{\lengthtest{\lengthOptionC>\lmax}}
		{\setlength{\lmax}{\lengthOptionC}}{}
	\ifthenelse{\lengthtest{\lengthOptionD>\lmax}}
		{\setlength{\lmax}{\lengthOptionD}}{}
	
	\ifthenelse{\lengthtest{\lmax>\lhalf}}
		{\noindent{}A.~#1 \\ B.~#2 \\ C.~#3 \\ D.~#4}
		{
			\ifthenelse{\lengthtest{\lmax>\lquarter}}
				{\noindent\makebox[\lhalf][l]{A.~#1~~~}%
					\makebox[\lhalf][l]{B.~#2~~~}\\\noindent%
					\makebox[\lhalf][l]{C.~#3~~~}%
					\makebox[\lhalf][l]{D.~#4~~~}
				}%
				{\noindent\makebox[\lquarter][l]{A.~#1~~~}%
					\makebox[\lquarter][l]{B.~#2~~~}%
					\makebox[\lquarter][l]{C.~#3~~~}%
					\makebox[\lquarter][l]{D.~#4~~~}
				}
			}
}

\newcommand{\zongfenlana}
{%☆总分栏
	\begin{center}
		\setlength{\tabcolsep}{4mm}%列宽
		\renewcommand{\arraystretch}{1.5}%行高
		\begin{tabular}{*{12}{|c}|}%格式化列样式
			\hline
			题号&一&二&三&四&总分&总分人&复核人\\ \hline
			分数&&&&&&&\\
			\hline
		\end{tabular}
	\end{center}
\vskip-3mm
}

\newcommand{\examineeInfo}
{
	\ifthenelse{\isodd{\value{page}}}
	{
		\begin{tikzpicture}[remember picture,overlay]
		\path 	(current page.south east) coordinate (a0);
		\path 	(current page.north) coordinate (a0);
		\draw 	(a0)[shift={(0,-2)}] node (a1) [rotate=0,fill=gray!0,minimum height=1cm,minimum width=1cm]
		{\kaishu{}\zihao{4}
			区域\raisebox{-2pt}{\rule{35mm}{0.4pt}}%
			部门\raisebox{-2pt}{\rule{35mm}{0.4pt}}%
			姓名\raisebox{-2pt}{\rule{35mm}{0.4pt}}%
			工号\raisebox{-2pt}{\rule{35mm}{0.4pt}}}
		(a1)[shift={(0,-1.5)}] node [rotate=0,fill=gray!0,minimum height=1cm,minimum width=1cm]
		{\kaishu{}\zihao{4}...........................................%
			\raisebox{-0.6ex}{装}............................................%
			\raisebox{-0.6ex}{订}............................................%
			\raisebox{-0.6ex}{线}...........................................};
		\end{tikzpicture}
	}
	{
		\begin{tikzpicture}[remember picture,overlay]
		\path 	(current page.south east) coordinate (a0);
		\path 	(current page.north) coordinate (a0);
		\draw 	(a0)[shift={(0,-2)}] node (a1) [rotate=0,fill=gray!0,minimum height=1cm,minimum width=1cm]
		{\kaishu{}\zihao{4}
			区域\raisebox{-2pt}{\rule{35mm}{0.4pt}}%
			部门\raisebox{-2pt}{\rule{35mm}{0.4pt}}%
			姓名\raisebox{-2pt}{\rule{35mm}{0.4pt}}%
			工号\raisebox{-2pt}{\rule{35mm}{0.4pt}}}
		(a1)[shift={(0,-1.5)}] node [rotate=0,fill=gray!0,minimum height=1cm,minimum width=1cm]
		{\kaishu{}\zihao{4}...........................................%
			\raisebox{-0.6ex}{装}............................................%
			\raisebox{-0.6ex}{订}............................................%
			\raisebox{-0.6ex}{线}...........................................};
		\end{tikzpicture}
	}
}

\newcommand{\markArea}
{
	\setlength{\tabcolsep}{4mm}%列宽
	\renewcommand{\arraystretch}{1}%行高
	\begin{tabular}[c]{*{2}{|c}|}
		\hline
		评卷人&得分\\\hline
		&\\[2mm]
		\hline
	\end{tabular}
}

\pagestyle{fancy}

\lhead{\setlength{\unitlength}{1em}
	\begin{picture}(0,0)
	\put(0,0){\includegraphics[width=6em]{./figures/logo.pdf}}
	\end{picture}}
\chead{\textbf{\productName 调查问卷}} 
\rhead{\textbf{姓名:\TextField[name=name,width=5em,charsize=0pt,align=1]{\mbox{}}}}

\begin{document}
	\begin{Form}[action=mailto:zhangyang@cheerlandgroup.com, encoding=html]

	
	%\examineeInfo
	
	\begin{center}
		{\DocumentTitle}
	\end{center}

%	\zongfenlana
	
%	\begin{center}
%		\begin{tabular}{p{15em}p{15em}p{15em}}
%			\multicolumn{1}{c}{ \Submit[name=Submit]{\large 提\qquad 交}} & & \multicolumn{1}{c}{\Reset[name=Reset]{\large 重\qquad  置}}
%		\end{tabular}
%	\end{center}
	
	\newcommand{\answerField}[1][\theenumi]{\uline{\TextField[name=#1,charsize=0pt,width=4em,align=1]{\mbox{}}}。}
	
	\newcommand{\blockAnswer}[1]{答:
		
		\TextField[name=#1,charsize=0pt, height=5cm, width=0.9\textwidth, multiline=true]{\ }}
	
	\newcommand{\glossaryDefinitionAnswer}[1]{
		\TextField[name=#1,width=0.9\textwidth,charsize=0pt,multiline=true]{}}
	
	\newcommand{\fillBlankField}[1]{\TextField[name=#1,width=0.3\textwidth,charsize=0pt]{}}
	
	\begin{enumerate}
		%\item[\kaishu{一}]{\makebox[2mm][r]{、}\kaishu{}选择题}
		
		\item 对于染色体异常检测产品,您更关心\answerField{} 
			\choiceOptions{能够检测的片段大小:比如核型检测的片段大小约在3M$ \sim $ 10M}{检测的断点精度:即是否满足PCR、FISH等检测需求}{检测的变异类型是否全面:数目变异、缺失、重复、倒位、易位}{其他}

		\item 目前来说,Array-CGH可以检测的染色体结构变异包括\answerField{} 
			\choiceOptions{染色体缺失}{染色体重复}{染色体倒位}{染色体易位}
			
		\item 若有一产品,其相对于现有产品来说,最大的优势是可以检测出样本中存在的染色体的倒位和易位,您认为临床上该产品价值如何\answerField{} 
			\choiceOptions{有比较高的价值}{有价值,但价值较低}{基本上无价值}{保留意见}
		
		\item 若有一产品,其相对于现有产品来说,最大的优势是检测精度更高,满足PCR和FISH等下游实验的要求,您认为临床上该产品价值如何\answerField{} 
			\choiceOptions{有比较高的价值}{有价值,但价值较低}{基本上无价值}{保留意见}
						
		\item 若一产品,其可用来检测染色体的结构变异(数目变异、缺失、重复、\uline{倒位}、\uline{易位}),其片段精度在100KB,断点精度为1KB,您认为临床上是否有应用价值\answerField{} 
			\choiceOptions{有}{无}{临床说不清楚,但是可以考虑科研}{保留意见}

		\item 若一产品,其可用来检测染色体的结构变异(数目变异、缺失、重复、\uline{倒位}、\uline{易位}),其片段精度在100KB,断点精度为1KB,对其检测的结果,临床实践中,您是否有意愿通过PCR、FISH等方法来验证\answerField{} 
			\choiceOptions{有}{无}{视情况而定}{保留意见}
						
		\item 对于不明原因的不孕不育患者,您认为其原因更可能是\answerField{} 
			\choiceOptions{心理原因}{遗传物质异常}{免疫方面的原因}{保留意见}
			
		\item 我们认为,不明原因不孕不育患者的诊断过程中,应优先考虑染色体问题,您是否赞同\answerField{} 
			\choiceOptions{赞同}{不赞同}{部分赞同}{保留意见}
		
		\item 临床诊疗中,了解染色体变异的大致区域即可满足需求,您是否赞同\answerField{} 
			\choiceOptions{赞同}{不赞同}{部分赞同}{保留意见}
			
		\item 临床实践中,不同染色体变异在人群中分布情况是\answerField{} 
			\choiceOptions{}{不赞同}{部分赞同}{保留意见}
		
		\item 您是否了解过\productName\ \answerField{} 
			\choiceOptions{没有}{有}{部分了解}{保留意见}
		
		\item 您认为可以用来在临床上检测染色体结构变异的产品有哪些\answerField{} 
			\choiceOptions{核型分析}{NIPT}{CNV-Seq}{\productName}
			
		\item 如果您需要了解\productName,您更希望我们提供哪些材料\answerField{} 
			\choiceOptions{产品宣传手册}{发表的相关文章}{与其他技术相比的技术优势}{其他材料}
		
		\item 对于成熟的高通量测序产品,在临床实践中,您认为最重要的问题是\answerField{} 
			\choiceOptions{产品的科学性}{是否能够解决世纪问题}{是否得到相关部门批准}{价格和周期}		
	\end{enumerate}
\end{Form}
\end{document}