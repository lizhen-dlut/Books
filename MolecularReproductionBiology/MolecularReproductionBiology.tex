\documentclass[11pt,a4paper]{ctexbook}
\input{../preamble}
\newcommand{\terminologyItem}[1]{\textbf{\uwave{#1}}}

\newcommand{\myKeypoint}[1]{\textbf{\uline{#1}}}
% !Mode:: "TeX:UTF-8"
\newglossaryentry{DL}
{
  name=深度学习,
  description={deep learning},
  sort={deep learning},
}

\newglossaryentry{knowledge_base}
{
  name=知识库,
  description={knowledge base},
  sort={knowledge base},
}

\newglossaryentry{ML}
{
  name=机器学习,
  description={machine learning},
  sort={machine learning},
}

\newglossaryentry{ML_model}
{
  name=机器学习模型,
  description={machine learning model},
  sort={machine learning model},
}

\newglossaryentry{ebv}
{
  name=基本单位向量,
  description={elementary basis vectors},
  sort={elementary basis vectors},
}

\newglossaryentry{logistic_regression}
{
  name=逻辑回归,
  description={logistic regression},
  sort={logistic regression},
}

\newglossaryentry{regression}
{
  name=回归,
  description={regression},
  sort={regression},
}

\newglossaryentry{AI}
{
  name=人工智能,
  description={artificial intelligence},
  sort={artificial intelligence},
  symbol={AI}
}

\newglossaryentry{naive_bayes}
{
  name=朴素贝叶斯,
  description={naive Bayes},
  sort={naive Bayes},
}

\newglossaryentry{representation}
{
  name=表示,
  description={representation},
  sort={representation},
}

\newglossaryentry{representation_learning}
{
  name=表示学习,
  description={representation learning},
  sort={representation learning},
}

\newglossaryentry{AE}
{
  name=自编码器,
  description={autoencoder},
  sort={autoencoder},
}

\newglossaryentry{encoder}
{
  name=编码器,
  description={encoder},
  sort={encoder},
}

\newglossaryentry{decoder}
{
  name=解码器,
  description={decoder},
  sort={decoder},
}

\newglossaryentry{MLP}
{
  name=多层感知机,
  description={multilayer perceptron},
  sort={multilayer perceptron},
  symbol={MLP}
}

\newglossaryentry{cybernetics}
{
  name=控制论,
  description={cybernetics},
  sort={cybernetics},
}

\newglossaryentry{connectionism}
{
  name=联结主义,
  description={connectionism},
  sort={connectionism},
}

\newglossaryentry{ANN}
{
  name=人工神经网络,
  description={artificial neural network},
  sort={artificial neural network},
  symbol={ANN}
}

\newglossaryentry{NN}
{
  name=神经网络,
  description={neural network},
  sort={neural network},
}

\newglossaryentry{SGD}
{
  name=随机梯度下降,
  description={stochastic gradient descent},
  sort={stochastic gradient descent},
  symbol={SGD}
}

\newglossaryentry{linear_model}
{
  name=线性模型,
  description={linear model},
  sort={linear model},
}

\newglossaryentry{mode}
{
  name=峰值,
  description={mode},
  sort={mode},
}

\newglossaryentry{unimodal}
{
  name=单峰值,
  description={unimodal},
  sort={unimodal},
}

\newglossaryentry{modality}
{
  name=模态,
  description={modality},
  sort={modality},
}

\newglossaryentry{multimodal}
{
  name=多峰值,
  description={multimodal},
  sort={multimodal},
}

\newglossaryentry{linear_regression}
{
  name=线性回归,
  description={linear regression},
  sort={linear regression},
}

\newglossaryentry{ReLU}
{
  name=整流线性单元,
  description={rectified linear unit},
  sort={rectified linear unit},
  symbol={ReLU}
}

\newglossaryentry{distributed_representation}
{
  name=分布式表示,
  description={distributed representation},
  sort={distributed representation},
}

\newglossaryentry{nondistributed_representation}
{
  name=非分布式表示,
  description={nondistributed representation},
  sort={nondistributed representation},
}

\newglossaryentry{nondistributed}
{
  name=非分布式,
  description={nondistributed},
  sort={nondistributed},
}

\newglossaryentry{hidden_unit}
{
  name=隐藏单元,
  description={hidden unit},
  sort={hidden unit},
}

\newglossaryentry{LSTM}
{
  name=长短期记忆,
  description={long short-term memory},
  sort={long short-term memory},
  symbol={LSTM}
}

\newglossaryentry{DBN}
{
  name=深度信念网络,
  description={deep belief network},
  sort={deep belief network},
  symbol={DBN}
}

\newglossaryentry{RNN}
{
  name=循环神经网络,
  description={recurrent neural network},
  sort={recurrent neural network},
  symbol={RNN}
}

\newglossaryentry{recurrence}
{
  name=循环,
  description={recurrence},
  sort={recurrence},
}

\newglossaryentry{RL}
{
  name=强化学习,
  description={reinforcement learning},
  sort={reinforcement learning},
}

\newglossaryentry{inference}
{
  name=推断,
  description={inference},
  sort={inference},
}

\newglossaryentry{overflow}
{
  name=上溢,
  description={overflow},
  sort={overflow},
}

\newglossaryentry{underflow}
{
  name=下溢,
  description={underflow},
  sort={underflow},
}

\newglossaryentry{softmax}
{
  name=softmax函数,
  description={softmax function},
  sort={softmax function},
}

\newglossaryentry{softmax_chap15}
{
  name=softmax,
  description={softmax},
  sort={softmax},
}

\newglossaryentry{underestimation}
{
  name=欠估计,
  description={underestimation},
  sort={underestimation},
}

\newglossaryentry{overestimation}
{
  name=过估计,
  description={overestimation},
  sort={overestimation},
}

\newglossaryentry{softmax_unit}
{
  name=softmax单元,
  description={softmax unit},
  sort={softmax unit},
}

\newglossaryentry{softmax_chap11}
{
  name=softmax,
  description={softmax},
  sort={softmax},
}

\newglossaryentry{multinoulli}
{
  name=Multinoulli分布,
  description={multinoulli distribution},
  sort={multinoulli distribution},
}

\newglossaryentry{poor_conditioning}
{
  name=病态条件,
  description={poor conditioning},
  sort={poor conditioning},
}

\newglossaryentry{objective_function}
{
  name=目标函数,
  description={objective function},
  sort={objective function},
}

\newglossaryentry{objective}
{
  name=目标,
  description={objective},
  sort={objective},
}

\newglossaryentry{criterion}
{
  name=准则,
  description={criterion},
  sort={criterion},
}

\newglossaryentry{cost_function}
{
  name=代价函数,
  description={cost function},
  sort={cost function},
}

\newglossaryentry{cost}
{
  name=代价,
  description={cost},
  sort={cost},
}

\newglossaryentry{loss_function}
{
  name=损失函数,
  description={loss function},
  sort={loss function},
}

\newglossaryentry{prcurve}
{
  name=PR曲线,
  description={PR curve},
  sort={PR curve},
}

\newglossaryentry{fscore}
{
  name=F分数,
  description={F-score},
  sort={F-score},
}

\newglossaryentry{loss}
{
  name=损失,
  description={loss},
  sort={loss},
}

\newglossaryentry{error_function}
{
  name=误差函数,
  description={error function},
  sort={error function},
}

\newglossaryentry{GD}
{
  name=梯度下降,
  description={gradient descent},
  sort={gradient descent},
}

\newglossaryentry{local_descent}
{
  name=局部下降,
  description={local descent},
  sort={local descent},
}

\newglossaryentry{steepest}
{
  name=最陡下降,
  description={steepest descent},
  sort={steepest descent},
}

\newglossaryentry{GA}
{
  name=梯度上升,
  description={gradient ascent},
  sort={gradient ascent},
}

\newglossaryentry{derivative}
{
  name=导数,
  description={derivative},
  sort={derivative},
}

\newglossaryentry{critical_points}
{
  name=临界点,
  description={critical point},
  sort={critical point},
}

\newglossaryentry{stationary_point}
{
  name=驻点,
  description={stationary point},
  sort={stationary point},
}

\newglossaryentry{local_minimum}
{
  name=局部极小点,
  description={local minimum},
  sort={local minimum},
}

\newglossaryentry{minimum}
{
  name=极小点,
  description={minimum},
  sort={minimum},
}

\newglossaryentry{local_minima}
{
  name=局部极小值,
  description={local minima},
  sort={local minima},
}

\newglossaryentry{minima}
{
  name=极小值,
  description={minima},
  sort={minima},
}

\newglossaryentry{global_minima}
{
  name=全局极小值,
  description={global minima},
  sort={global minima},
}

\newglossaryentry{local_maxima}
{
  name=局部极大值,
  description={local maxima},
  sort={local maxima},
}

\newglossaryentry{maxima}
{
  name=极大值,
  description={maxima},
  sort={maxima},
}

\newglossaryentry{local_maximum}
{
  name=局部极大点,
  description={local maximum},
  sort={local maximum},
}

\newglossaryentry{saddle_points}
{
  name=鞍点,
  description={saddle point},
  sort={saddle point},
}

\newglossaryentry{global_minimum}
{
  name=全局最小点,
  description={global minimum},
  sort={global minimum},
}

\newglossaryentry{partial_derivatives}
{
  name=偏导数,
  description={partial derivative},
  sort={partial derivative},
}

\newglossaryentry{gradient}
{
  name=梯度,
  description={gradient},
  sort={gradient},
}

\newglossaryentry{identifiable}
{
  name=可辨认的,
  description={identifiable},
  sort={identifiable},
}

\newglossaryentry{directional_derivative}
{
  name=方向导数,
  description={directional derivative},
  sort={directional derivative},
}

\newglossaryentry{line_search}
{
  name=线搜索,
  description={line search},
  sort={line search},
}

\newglossaryentry{example}
{
  name=样本,
  description={example},
  sort={example},
}

\newglossaryentry{hill_climbing}
{
  name=爬山,
  description={hill climbing},
  sort={hill climbing},
}

\newglossaryentry{ill_conditioning}
{
  name=病态,
  description={ill conditioning},
  sort={ill conditioning},
}

\newglossaryentry{jacobian}
{
  name=Jacobian,
  description={Jacobian},
  sort={Jacobian},
}

\newglossaryentry{hessian}
{
  name=Hessian,
  description={Hessian},
  sort={Hessian},
}

\newglossaryentry{second_derivative}
{
  name=二阶导数,
  description={second derivative},
  sort={second derivative},
}

\newglossaryentry{curvature}
{
  name=曲率,
  description={curvature},
  sort={curvature},
}

\newglossaryentry{taylor}
{
  name=泰勒,
  description={taylor},
  sort={taylor},
}

\newglossaryentry{second_derivative_test}
{
  name=二阶导数测试,
  description={second derivative test},
  sort={second derivative test},
}

\newglossaryentry{newton_method}
{
  name=牛顿法,
  description={Newton's method},
  sort={Newton's method},
}

\newglossaryentry{second_order_method}
{
  name=二阶方法,
  description={second-order method},
  sort={second-order method},
}

\newglossaryentry{first_order_method}
{
  name=一阶方法,
  description={first-order method},
  sort={first-order method},
}

\newglossaryentry{lipschitz}
{
  name=Lipschitz,
  description={Lipschitz},
  sort={Lipschitz},
}

\newglossaryentry{lipschitz_continuous}
{
  name=Lipschitz连续,
  description={Lipschitz continuous},
  sort={Lipschitz continuous},
}

\newglossaryentry{lipschitz_constant}
{
  name=Lipschitz常数,
  description={Lipschitz constant},
  sort={Lipschitz constant},
}

\newglossaryentry{convex_optimization}
{
  name=凸优化,
  description={Convex optimization},
  sort={Convex optimization},
}

\newglossaryentry{nonconvex}
{
  name=非凸,
  description={nonconvex},
  sort={nonconvex},
}

\newglossaryentry{nume_optimization}
{
  name=数值优化,
  description={numerical optimization},
  sort={numerical optimization},
}

\newglossaryentry{constrained_optimization}
{
  name=约束优化,
  description={constrained optimization},
  sort={constrained optimization},
}

\newglossaryentry{feasible}
{
  name=可行,
  description={feasible},
  sort={feasible},
}

\newglossaryentry{KKT}
{
  name=Karush–Kuhn–Tucker,
  description={Karush–Kuhn–Tucker},
  sort={Karush–Kuhn–Tucker},
  symbol={KKT}
}

\newglossaryentry{generalized_lagrangian}
{
  name=广义Lagrangian,
  description={generalized Lagrangian},
  sort={generalized Lagrangian},
}

\newglossaryentry{generalized_lagrange_function}
{
  name=广义Lagrange函数,
  description={generalized Lagrange function},
  sort={generalized Lagrange function},
}

\newglossaryentry{equality_constraints}
{
  name=等式约束,
  description={equality constraint},
  sort={equality constraint},
}

\newglossaryentry{inequality_constraints}
{
  name=不等式约束,
  description={inequality constraint},
  sort={inequality constraint},
}

\newglossaryentry{regularization}
{
  name=正则化,
  description={regularization},
  sort={regularization},
}

\newglossaryentry{regularizer}
{
  name=正则化项,
  description={regularizer},
  sort={regularizer},
}

\newglossaryentry{regularize}
{
  name=正则化,
  description={regularize},
  sort={regularize},
}

\newglossaryentry{generalization}
{
  name=泛化,
  description={generalization},
  sort={generalization},
}

\newglossaryentry{generalize}
{
  name=泛化,
  description={generalize},
  sort={generalize},
}

\newglossaryentry{underfitting}
{
  name=欠拟合,
  description={underfitting},
  sort={underfitting},
}

\newglossaryentry{overfitting}
{
  name=过拟合,
  description={overfitting},
  sort={overfitting},
}

\newglossaryentry{bias_sta}
{
  name=偏差,
  description={bias in statistics},
  sort={bias in statistics},
}

\newglossaryentry{BIAS}
{
  name=偏差,
  description={biass},
  sort={biass},
}

\newglossaryentry{bias_aff}
{
  name=偏置,
  description={bias in affine function},
  sort={bias in affine function},
}

\newglossaryentry{variance}
{
  name=方差,
  description={variance},
  sort={variance},
}

\newglossaryentry{ensemble}
{
  name=集成,
  description={ensemble},
  sort={ensemble},
}

\newglossaryentry{estimator}
{
  name=估计,
  description={estimator},
  sort={estimator},
}

\newglossaryentry{weight_decay}
{
  name=权重衰减,
  description={weight decay},
  sort={weight decay},
}

\newglossaryentry{ridge_regression}
{
  name=岭回归,
  description={ridge regression},
  sort={ridge regression},
}

\newglossaryentry{tikhonov_regularization}
{
  name=Tikhonov正则,
  description={Tikhonov regularization},
  sort={Tikhonov regularization},
}

\newglossaryentry{covariance}
{
  name=协方差,
  description={covariance},
  sort={covariance},
}

\newglossaryentry{sparse}
{
  name=稀疏,
  description={sparse},
  sort={sparse},
}

\newglossaryentry{feature_selection}
{
  name=特征选择,
  description={feature selection},
  sort={feature selection},
}

\newglossaryentry{feature_extractor}
{
  name=特征提取器,
  description={feature extractor},
  sort={feature extractor},
}

\newglossaryentry{MAP}
{
  name=最大后验,
  description={Maximum A Posteriori},
  sort={Maximum A Posteriori},
  symbol={MAP}
}

\newglossaryentry{pooling}
{
  name=池化,
  description={pooling},
  sort={pooling},
}

\newglossaryentry{dropout}
{
  name=Dropout,
  description={Dropout},
  sort={dropout},
}

\newglossaryentry{monte_carlo}
{
  name=蒙特卡罗,
  description={Monte Carlo},
  sort={Monte Carlo},
}

\newglossaryentry{early_stopping}
{
  name=提前终止,
  description={early stopping},
  sort={early stopping},
}

\newglossaryentry{CNN}
{
  name=卷积神经网络,
  description={convolutional neural network},
  sort={convolutional neural network},
  symbol={CNN}
}

\newglossaryentry{mcmc}
{
  name=马尔可夫链蒙特卡罗,
  description={Markov Chain Monte Carlo},
  symbol={MCMC},
  sort={Markov Chain Monte Carlo},
}

\newglossaryentry{tempering_transition}
{
  name=回火转移,
  description={tempered transition},
  sort={tempered transition},
}

\newglossaryentry{markov_chain}
{
  name=马尔可夫链,
  description={Markov Chain},
  sort={Markov Chain},
}

\newglossaryentry{harris_chain}
{
  name=哈里斯链,
  description={Harris Chain},
  sort={Harris Chain},
}

\newglossaryentry{minibatch}
{
  name=小批量,
  description={minibatch},
  sort={minibatch},
}

\newglossaryentry{importance_sampling}
{
  name=重要采样,
  description={Importance Sampling},
  sort={Importance Sampling},
}

\newglossaryentry{undirected_model}
{
  name=无向模型,
  description={undirected Model},
  sort={undirected Model},
}

\newglossaryentry{partition_function}
{
  name=配分函数,
  description={Partition Function},
  sort={Partition Function},
}

\newglossaryentry{law_of_large_numbers}
{
  name=大数定理,
  description={Law of large number},
  sort={Law of large number},
}

\newglossaryentry{central_limit_theorem}
{
  name=中心极限定理,
  description={central limit theorem},
  sort={central limit theorem},
}

\newglossaryentry{energy_based_model}
{
  name=基于能量的模型,
  description={Energy-based model},
  symbol={EBM},
  sort={Energy-based model},
}

\newglossaryentry{tempering}
{
  name=回火,
  description={tempering},
  sort={tempering},
}

\newglossaryentry{biased_importance_sampling}
{
  name=有偏重要采样,
  description={biased importance sampling},
  sort={biased importance sampling},
}

\newglossaryentry{VAE}
{
  name=变分自编码器,
  description={variational auto-encoder},
  sort={variational auto-encoder},
  symbol={VAE},
}

\newglossaryentry{CV}
{
  name=计算机视觉,
  description={Computer Vision},
  sort={Computer Vision},
}

\newglossaryentry{SR}
{
  name=语音识别,
  description={Speech Recognition},
  sort={Speech Recognition},
}

\newglossaryentry{NLP}
{
  name=自然语言处理,
  description={Natural Language Processing},
  sort={Natural Language Processing},
  symbol={NLP}
}

\newglossaryentry{RBM}
{
  name=受限玻尔兹曼机,
  description={Restricted Boltzmann Machine},
  sort={Restricted Boltzmann Machine},
  symbol={RBM}
}

\newglossaryentry{discriminative_RBM}
{
  name=判别RBM,
  description={discriminative RBM},
  sort={discriminative RBM},
}

\newglossaryentry{Boltzmann}
{
  name=玻尔兹曼,
  description={Boltzmann},
  sort={Boltzmann},
}

\newglossaryentry{BM}
{
  name=玻尔兹曼机,
  description={Boltzmann Machine},
  sort={Boltzmann Machine},
}

\newglossaryentry{DBM}
{
  name=深度玻尔兹曼机,
  description={Deep Boltzmann Machine},
  sort={Deep Boltzmann Machine},
  symbol={DBM}
}

\newglossaryentry{CBM}
{
  name=卷积玻尔兹曼机,
  description={Convolutional Boltzmann Machine},
  sort={Convolutional Boltzmann Machine},
  symbol={CBM}
}

\newglossaryentry{directed_model}
{
  name=有向模型,
  description={Directed Model},
  sort={Directed Model},
}

\newglossaryentry{ancestral_sampling}
{
  name=原始采样,
  description={Ancestral Sampling},
  sort={Ancestral Sampling},
}

\newglossaryentry{stochastic_matrix}
{
  name=随机矩阵,
  description={Stochastic Matrix},
  sort={Stochastic Matrix},
}

\newglossaryentry{stationary_distribution}
{
  name=平稳分布,
  description={Stationary Distribution},
  sort={Stationary Distribution},
}

\newglossaryentry{equilibrium_distribution}
{
  name=均衡分布,
  description={Equilibrium Distribution},
  sort={Equilibrium Distribution},
}

\newglossaryentry{index}
{
  name=索引,
  description={index of matrix},
  sort={index of matrix},
}

\newglossaryentry{burn_in}
{
  name=磨合,
  description={Burning-in},
  sort={Burning-in},
}

\newglossaryentry{mixing_time}
{
  name=混合时间,
  description={Mixing Time},
  sort={Mixing Time},
}

\newglossaryentry{mixing}
{
  name=混合,
  description={Mixing},
  sort={Mixing},
}

\newglossaryentry{gibbs_sampling}
{
  name=Gibbs采样,
  description={Gibbs Sampling},
  sort={Gibbs Sampling},
}

\newglossaryentry{block_gibbs_sampling}
{
  name=块吉布斯采样,
  description={block Gibbs Sampling},
  sort={block Gibbs Sampling},
}

\newglossaryentry{gibbs_steps}
{
  name=吉布斯步数,
  description={Gibbs steps},
  sort={Gibbs steps},
}

\newglossaryentry{bagging}
{
  name=Bagging,
  description={bootstrap aggregating},
  sort={bagging},
}

\newglossaryentry{mask}
{
  name=掩码,
  description={mask},
  sort={mask},
}

\newglossaryentry{batch_normalization}
{
  name=批标准化,
  description={batch normalization},
  sort={batch normalization},
}

\newglossaryentry{Batch_normalization}
{
  name=批标准化,
  description={Batch normalization},
  sort={Batch normalization},
}

\newglossaryentry{parameter_sharing}
{
  name=参数共享,
  description={parameter sharing},
  sort={parameter sharing},
}

\newglossaryentry{KL}
{
  name=KL散度,
  description={KL divergence},
  sort={KL},
}

\newglossaryentry{temperature}
{
  name=温度,
  description={temperature},
  sort={temperature},
}

\newglossaryentry{critical_temperatures}
{
  name=临界温度,
  description={critical temperatures},
  sort={critical temperatures},
}

\newglossaryentry{parallel_tempering}
{
  name=并行回火,
  description={parallel tempering},
  sort={parallel tempering},
}

\newglossaryentry{ASR}
{
  name=自动语音识别,
  description={Automatic Speech Recognition},
  sort={Automatic Speech Recognition},
  symbol={ASR}
}

\newglossaryentry{GP_GPU}
{
  name=通用GPU,
  description={general purpose GPU},
  sort={general purpose GPU},
}

\newglossaryentry{coalesced}
{
  name=级联,
  description={coalesced},
  sort={coalesced},
}

\newglossaryentry{warp}
{
  name=warp,
  description={warp},
  sort={warp},
}

\newglossaryentry{data_parallelism}
{
  name=数据并行,
  description={data parallelism},
  sort={data parallelism},
}

\newglossaryentry{model_parallelism}
{
  name=模型并行,
  description={model parallelism},
  sort={model parallelism},
}

\newglossaryentry{ASGD}
{
  name=异步随机梯度下降,
  description={Asynchoronous Stochastic Gradient Descent},
  sort={Asynchoronous Stochastic Gradient Descent},
}

\newglossaryentry{parameter_server}
{
  name=参数服务器,
  description={parameter server},
  sort={parameter server},
}

\newglossaryentry{model_compression}
{
  name=模型压缩,
  description={model compression},
  sort={model compression},
}

\newglossaryentry{dynamic_structure}
{
  name=动态结构,
  description={dynamic structure},
  sort={dynamic structure},
}

\newglossaryentry{conditional_computation}
{
  name=条件计算,
  description={conditional computation},
  sort={conditional computation},
}

\newglossaryentry{sphering}
{
  name=sphering,
  description={sphering},
  sort={sphering},
}

\newglossaryentry{GCN}
{
  name=全局对比度归一化,
  description={Global contrast normalization},
  sort={Global contrast normalization},
  symbol={GCN}
}

\newglossaryentry{LCN}
{
  name=局部对比度归一化,
  description={local contrast normalization},
  symbol={LCN},
  sort={local contrast normalization},
}

\newglossaryentry{HMM}
{
  name=隐马尔可夫模型,
  description={Hidden Markov Model},
  sort={Hidden Markov Model},
  symbol={HMM}
}

\newglossaryentry{GMM}
{
  name=高斯混合模型,
  description={Gaussian Mixture Model},
  sort={Gaussian Mixture Model},
  symbol={GMM}
}

\newglossaryentry{transcribe}
{
  name=转录,
  description={transcribe},
  sort={transcribe},
}

\newglossaryentry{PCA}
{
  name=主成分分析,
  description={principal components analysis},
  sort={principal components analysis},
  symbol={PCA}
}

\newglossaryentry{FA}
{
  name=因子分析,
  description={factor analysis},
  sort={factor analysis},
}

\newglossaryentry{ICA}
{
  name=独立成分分析,
  description={independent component analysis},
  sort={independent component analysis},
  symbol={ICA}
}

\newglossaryentry{tICA}
{
  name=地质ICA,
  description={topographic ICA},
  sort={topo independent component analysis},
}

\newglossaryentry{sparse_coding}
{
  name=稀疏编码,
  description={sparse coding},
  sort={sparse coding},
}

\newglossaryentry{fixed_point_arithmetic}
{
  name=定点运算,
  description={fixed-point arithmetic},
  sort={fixed-point arithmetic},
}

\newglossaryentry{float_point_arithmetic}
{
  name=浮点运算,
  description={float-point arithmetic},
  sort={float-point arithmetic},
}

\newglossaryentry{GPU}
{
  name=图形处理器,
  description={Graphics Processing Unit},
  sort={Graphics Processing Unit},
  symbol={GPU}
}

\newglossaryentry{generative_model}
{
  name=生成模型,
  description={generative model},
  sort={generative model},
}

\newglossaryentry{generative_modeling}
{
  name=生成式建模,
  description={generative modeling},
  sort={generative modeling},
}

\newglossaryentry{dataset_augmentation}
{
  name=数据集增强,
  description={dataset augmentation},
  sort={dataset augmentation},
}

\newglossaryentry{whitening}
{
  name=白化,
  description={whitening},
  sort={whitening},
}

\newglossaryentry{DNN}
{
  name=深度神经网络,
  description={DNN},
  sort={DNN},
}

\newglossaryentry{end_to_end}
{
  name=端到端的,
  description={end-to-end},
  sort={end-to-end},
}

\newglossaryentry{structured_probabilistic_models}
{
  name=结构化概率模型,
  description={structured probabilistic model},
  sort={structured probabilistic model},
}

\newglossaryentry{graphical_models}
{
  name=图模型,
  description={graphical model},
  sort={graphical model},
}

\newglossaryentry{directed_graphical_model}
{
  name=有向图模型,
  description={directed graphical model},
  sort={directed graphical model},
}

\newglossaryentry{dependency}
{
  name=依赖,
  description={dependency},
  sort={dependency},
}

\newglossaryentry{bayesian_network}
{
  name=贝叶斯网络,
  description={Bayesian network},
  sort={Bayesian network},
}

\newglossaryentry{model_averaging}
{
  name=模型平均,
  description={model averaging},
  sort={model averaging},
}

\newglossaryentry{boosting}
{
  name=Boosting,
  description={Boosting},
  sort={Boosting},
}

\newglossaryentry{weight_scaling_inference_rule}
{
  name=权重比例推断规则,
  description={weight scaling inference rule},
  sort={weight scaling inference rule},
}

\newglossaryentry{statement}
{
  name=声明,
  description={statement},
  sort={statement},
}

\newglossaryentry{quantum_mechanics}
{
  name=量子力学,
  description={quantum mechanics},
  sort={quantum mechanics},
}

\newglossaryentry{subatomic}
{
  name=亚原子,
  description={subatomic},
  sort={subatomic},
}

\newglossaryentry{fidelity}
{
  name=逼真度,
  description={fidelity},
  sort={fidelity},
}

\newglossaryentry{degree_of_belief}
{
  name=信任度,
  description={degree of belief},
  sort={degree of belief},
}

\newglossaryentry{frequentist_probability}
{
  name=频率派概率,
  description={frequentist probability},
  sort={frequentist probability},
}

\newglossaryentry{subsample}
{
  name=子采样,
  description={subsample},
  sort={subsample},
}

\newglossaryentry{bayesian_probability}
{
  name=贝叶斯概率,
  description={Bayesian probability},
  sort={Bayesian probability},
}

\newglossaryentry{likelihood}
{
  name=似然,
  description={likelihood},
  sort={likelihood},
}

\newglossaryentry{RV}
{
  name=随机变量,
  description={random variable},
  sort={random variable},
}

\newglossaryentry{PD}
{
  name=概率分布,
  description={probability distribution},
  sort={probability distribution},
}

\newglossaryentry{PMF}
{
  name=概率质量函数,
  description={probability mass function},
  sort={probability mass function},
  symbol={PMF}
}

\newglossaryentry{joint_probability_distribution}
{
  name=联合概率分布,
  description={joint probability distribution},
  sort={joint probability distribution},
}

\newglossaryentry{normalized}
{
  name=归一化的,
  description={normalized},
  sort={normalized},
}

\newglossaryentry{uniform_distribution}
{
  name=均匀分布,
  description={uniform distribution},
  sort={uniform distribution},
}

\newglossaryentry{PDF}
{
  name=概率密度函数,
  description={probability density function},
  sort={probability density function},
  symbol={PDF}
}

\newglossaryentry{cumulative_function}
{
  name=累积函数,
  description={cumulative function},
  sort={cumulative function},
}

\newglossaryentry{marginal_probability_distribution}
{
  name=边缘概率分布,
  description={marginal probability distribution},
  sort={marginal probability distribution},
}

\newglossaryentry{sum_rule}
{
  name=求和法则,
  description={sum rule},
  sort={sum rule},
}

\newglossaryentry{conditional_probability}
{
  name=条件概率,
  description={conditional probability},
  sort={conditional probability},
}

\newglossaryentry{intervention_query}
{
  name=干预查询,
  description={intervention query},
  sort={intervention query},
}

\newglossaryentry{causal_modeling}
{
  name=因果模型,
  description={causal modeling},
  sort={causal modeling},
}

\newglossaryentry{causal_factor}
{
  name=因果因子,
  description={causal factor},
  sort={causal factor},
}

\newglossaryentry{chain_rule}
{
  name=链式法则,
  description={chain rule},
  sort={chain rule},
}

\newglossaryentry{product_rule}
{
  name=乘法法则,
  description={product rule},
  sort={product rule},
}

\newglossaryentry{independent}
{
  name=相互独立的,
  description={independent},
  sort={independent},
}

\newglossaryentry{conditionally_independent}
{
  name=条件独立的,
  description={conditionally independent},
  sort={conditionally independent},
}

\newglossaryentry{expectation}
{
  name=期望,
  description={expectation},
  sort={expectation},
}

\newglossaryentry{expected_value}
{
  name=期望值,
  description={expected value},
  sort={expected value},
}

\newglossaryentry{example:chap5}
{
  name=样本,
  description={example},
  sort={example},
}

\newglossaryentry{feature}
{
  name=特征,
  description={feature},
  sort={feature},
}

\newglossaryentry{accuracy}
{
  name=准确率,
  description={accuracy},
  sort={accuracy},
}

\newglossaryentry{error_rate}
{
  name=错误率,
  description={error rate},
  sort={error rate},
}

\newglossaryentry{training_set}
{
  name=训练集,
  description={training set},
  sort={training set},
}

\newglossaryentry{explanatory_factor}
{
  name=解释因子,
  description={explanatory factort},
  sort={explanatory factor},
}

\newglossaryentry{underlying}
{
  name=潜在,
  description={underlying},
  sort={underlying},
}

\newglossaryentry{underlying_cause}
{
  name=潜在成因,
  description={underlying cause},
  sort={underlying cause},
}

\newglossaryentry{test_set}
{
  name=测试集,
  description={test set},
  sort={test set},
}

\newglossaryentry{performance_measures}
{
  name=性能度量,
  description={performance measures},
  sort={performance measures},
}

\newglossaryentry{experience}
{
  name=经验,
  description={experience, E},
  sort={experience, E},
}

\newglossaryentry{unsupervised}
{
  name=无监督,
  description={unsupervised},
  sort={unsupervised},
}

\newglossaryentry{supervised}
{
  name=监督,
  description={supervised},
  sort={supervised},
}

\newglossaryentry{semi_supervised}
{
  name=半监督,
  description={semi-supervised},
  sort={semi-supervised},
}

\newglossaryentry{supervised_learning}
{
  name=监督学习,
  description={supervised learning},
  sort={supervised learning},
}

\newglossaryentry{unsupervised_learning}
{
  name=无监督学习,
  description={unsupervised learning},
  sort={unsupervised learning},
}

\newglossaryentry{dataset}
{
  name=数据集,
  description={dataset},
  sort={dataset},
}

\newglossaryentry{setting}
{
  name=情景,
  description={setting},
  sort={setting},
}

\newglossaryentry{data_points}
{
  name=数据点,
  description={data point},
  sort={data point},
}

\newglossaryentry{label}
{
  name=标签,
  description={label},
  sort={label},
}

\newglossaryentry{labeled}
{
  name=标注,
  description={labeled},
  sort={labeled},
}

\newglossaryentry{unlabeled}
{
  name=未标注,
  description={unlabeled},
  sort={unlabeled},
}

\newglossaryentry{target}
{
  name=目标,
  description={target},
  sort={target},
}

\newglossaryentry{reinforcement_learning}
{
  name=强化学习,
  description={reinforcement learning},
  sort={reinforcement learning},
}

\newglossaryentry{design_matrix}
{
  name=设计矩阵,
  description={design matrix},
  sort={design matrix},
}

\newglossaryentry{parameters}
{
  name=参数,
  description={parameter},
  sort={parameter},
}

\newglossaryentry{weights}
{
  name=权重,
  description={weight},
  sort={weight},
}

\newglossaryentry{mean_squared_error}
{
  name=均方误差,
  description={mean squared error},
  sort={mean squared error},
  symbol={MSE}
}

\newglossaryentry{normal_equations}
{
  name=正规方程,
  description={normal equation},
  sort={normal equation},
}

\newglossaryentry{training_error}
{
  name=训练误差,
  description={training error},
  sort={training error},
}

\newglossaryentry{generalization_error}
{
  name=泛化误差,
  description={generalization error},
  sort={generalization error},
}

\newglossaryentry{test_error}
{
  name=测试误差,
  description={test error},
  sort={test error},
}

\newglossaryentry{hypothesis_space}
{
  name=假设空间,
  description={hypothesis space},
  sort={hypothesis space},
}

\newglossaryentry{capacity}
{
  name=容量,
  description={capacity},
  sort={capacity},
}

\newglossaryentry{representational_capacity}
{
  name=表示容量,
  description={representational capacity},
  sort={representational capacity},
}

\newglossaryentry{effective_capacity}
{
  name=有效容量,
  description={effective capacity},
  sort={effective capacity},
}

\newglossaryentry{linear_threshold_units}
{
  name=线性阀值单元,
  description={linear threshold units},
  sort={linear threshold units},
}

\newglossaryentry{nonparametric}
{
  name=非参数,
  description={non-parametric},
  sort={non-parametric},
}

\newglossaryentry{nearest_neighbor_regression}
{
  name=最近邻回归,
  description={nearest neighbor regression},
  sort={nearest neighbor regression},
}

\newglossaryentry{nearest_neighbor}
{
  name=最近邻,
  description={nearest neighbor},
  sort={nearest neighbor},
}

\newglossaryentry{bayes_error}
{
  name=贝叶斯误差,
  description={Bayes error},
  sort={Bayes error},
}

\newglossaryentry{no_free_lunch_theorem}
{
  name=没有免费午餐定理,
  description={no free lunch theorem},
  sort={no free lunch theorem},
}

\newglossaryentry{validation_set}
{
  name=验证集,
  description={validation set},
  sort={validation set},
}

\newglossaryentry{benchmarks}
{
  name=基准,
  description={bechmark},
  sort={bechmark},
}

\newglossaryentry{baseline}
{
  name=基准,
  description={baseline},
  sort={beseline}
}

\newglossaryentry{point_estimator}
{
  name=点估计,
  description={point estimator},
  sort={point estimator},
}

\newglossaryentry{estimator:chap5}
{
  name=估计量,
  description={estimator},
  sort={estimator},
}

\newglossaryentry{statistics}
{
  name=统计量,
  description={statistics},
  sort={statistics},
}

\newglossaryentry{unbiased}
{
  name=无偏,
  description={unbiased},
  sort={unbiased},
}

\newglossaryentry{biased}
{
  name=有偏,
  description={biased},
  sort={biased},
}

\newglossaryentry{asynchronous}
{
  name=异步,
  description={asynchronous},
  sort={asynchronous},
}

\newglossaryentry{asymptotically_unbiased}
{
  name=渐近无偏,
  description={asymptotically unbiased},
  sort={asymptotically unbiased},
}

\newglossaryentry{sample_mean}
{
  name=样本均值, %采样均值?
  description={sample mean},
  sort={sample mean},
}

\newglossaryentry{sample_variance}
{
  name=样本方差, %采样方差?
  description={sample variance},
  sort={sample variance},
}

\newglossaryentry{unbiased_sample_variance}
{
  name=无偏样本方差, %采样方差?
  description={unbiased sample variance},
  sort={unbiased sample variance},
}

\newglossaryentry{standard_error}
{
  name=标准差,
  description={standard error},
  symbol={SE},
  sort={standard error},
}

\newglossaryentry{consistency}
{
  name=一致性,
  description={consistency},
  sort={consistency},
}

\newglossaryentry{almost_sure}
{
  name=几乎必然,
  description={almost sure},
  sort={almost sure},
}

\newglossaryentry{almost_sure_convergence}
{
  name=几乎必然收敛,
  description={almost sure convergence},
  sort={almost sure convergence},
}

\newglossaryentry{statistical_efficiency}
{
  name=统计效率,
  description={statistic efficiency},
  sort={statistic efficiency},
}

\newglossaryentry{parametric_case}
{
  name=有参情况,
  description={parametric case},
  sort={parametric case},
}

\newglossaryentry{frequentist_statistics}
{
  name=频率派统计,
  description={frequentist statistics},
  sort={frequentist statistics},
}

\newglossaryentry{bayesian_statistics}
{
  name=贝叶斯统计,
  description={Bayesian statistics},
  sort={Bayesian statistics},
}

\newglossaryentry{prior_probability_distribution}
{
  name=先验概率分布,
  description={prior probability distribution},
  sort={prior probability distribution},
}

\newglossaryentry{maximum_a_posteriori}
{
  name=最大后验,
  description={maximum a posteriori},
  sort={maximum a posteriori},
}

\newglossaryentry{maximum_likelihood_estimation}
{
  name=最大似然估计,
  description={maximum likelihood estimation},
  sort={maximum likelihood estimation},
}

\newglossaryentry{maximum_likelihood}
{
  name=最大似然,
  description={maximum likelihood},
  sort={maximum likelihood},
}

\newglossaryentry{kernel_trick}
{
  name=核技巧,
  description={kernel trick},
  sort={kernel trick},
}

\newglossaryentry{kernel}
{
  name=核函数,
  description={kernel function},
  sort={kernel function},
}

\newglossaryentry{gaussian_kernel}
{
  name=高斯核,
  description={Gaussian kernel},
  sort={Gaussian kernel},
}

\newglossaryentry{kernel_machines}
{
  name=核机器,
  description={kernel machine},
  sort={kernel machine},
}

\newglossaryentry{kernel_methods}
{
  name=核方法,
  description={kernel method},
  sort={kernel method},
}

\newglossaryentry{support_vectors}
{
  name=支持向量,
  description={support vector},
  sort={support vector},
}

\newglossaryentry{SVM}
{
  name=支持向量机,
  description={support vector machine},
  symbol={SVM}
}

\newglossaryentry{phoneme}
{
  name=音素,
  description={phoneme},
  sort={phoneme},
}

\newglossaryentry{acoustic}
{
  name=声学,
  description={acoustic},
  sort={acoustic},
}

\newglossaryentry{phonetic}
{
  name=语音,
  description={phonetic},
  sort={phonetic},
}

\newglossaryentry{mixture_of_experts}
{
  name=专家混合体,
  description={mixture of experts},
  sort={mixture of experts},
}

\newglossaryentry{gauss_mixture}
{
  name=高斯混合体,
  description={Gaussian mixtures},
  sort={Gaussian mixtures},
}

\newglossaryentry{hard_mixture_of_experts}
{
  name=硬专家混合体, 
  description={hard mixture of experts},
  sort={hard mixture of experts},
}

\newglossaryentry{gater}
{
  name=选通器,
  description={gater},
  sort={gater},
}

\newglossaryentry{expert_network}
{
  name=专家网络,
  description={expert network},
  sort={expert network},
}

\newglossaryentry{attention_mechanism}
{
  name=注意力机制,
  description={attention mechanism},
  sort={attention mechanism},
}

\newglossaryentry{fast_dropout}
{
  name=快速Dropout,
  description={fast dropout},
  sort={fast dropout},
}

\newglossaryentry{dropout_boosting}
{
  name=Dropout Boosting,
  description={Dropout Boosting},
  sort={dropout boosting},
}

\newglossaryentry{adversarial_example}
{
  name=对抗样本,
  description={adversarial example},
  sort={adversarial example},
}

\newglossaryentry{adversarial}
{
  name=对抗,
  description={adversarial},
  sort={adversarial},
}

\newglossaryentry{virtual_adversarial_example}
{
  name=虚拟对抗样本,
  description={virtual adversarial example},
  sort={virtual adversarial example},
}

\newglossaryentry{adversarial_training}
{
  name=对抗训练,
  description={adversarial training},
  sort={adversarial training},
}

\newglossaryentry{virtual_adversarial_training}
{
  name=虚拟对抗训练,
  description={virtual adversarial training},
  sort={virtual adversarial training}
}

\newglossaryentry{tangent_distance}
{
  name=切面距离,
  description={tangent distance},
  sort={tangent distance},
}

\newglossaryentry{tangent_prop}
{
  name=正切传播,
  description={tangent prop},
  sort={tangent prop},
}

\newglossaryentry{tangent_propagation}
{
  name=正切传播,
  description={tangent propagation}
}

\newglossaryentry{double_backprop}
{
  name=双反向传播,
  description={double backprop},
  sort={double backprop},
}

\newglossaryentry{EM}
{
  name=期望最大化,
  description={expectation maximization},
  sort={expectation maximization},
  symbol={EM}
}

\newglossaryentry{mean_field}
{
  name=均值场,
  description={mean-field},
  sort={mean-field},
}

\newglossaryentry{ELBO}
{
  name=证据下界,
  description={evidence lower bound},
  sort={evidence lower bound},
  symbol={ELBO}
}

\newglossaryentry{variational_free_energy}
{
  name=变分自由能,
  description={variational free energy},
  sort={variational free energy},
}

\newglossaryentry{structured_variational_inference}
{
  name=结构化变分推断,
  description={structured variational inference},
  sort={structured variational inference},
}

\newglossaryentry{variational_inference}
{
  name=变分推断,
  description={variational inference},
  sort={variational inference},
}

\newglossaryentry{binary_sparse_coding}
{
  name=二值稀疏编码,
  description={binary sparse coding},
  sort={binary sparse coding},
}

\newglossaryentry{feedforward_network}
{
  name=前馈网络,
  description={feedforward network},
  sort={feedforward network},
}

\newglossaryentry{transition}
{
  name=转移,
  description={transition},
  sort={transition},
}

\newglossaryentry{reconstruction}
{
  name=重构,
  description={reconstruction},
  sort={reconstruction},
}

\newglossaryentry{GSN}
{
  name=生成随机网络,
  description={generative stochastic network},
  sort={generative stochastic network},
  symbol={GSN}
}

\newglossaryentry{score_matching}
{
  name=得分匹配,
  description={score matching},
  sort={score matching},
}

\newglossaryentry{factorial}
{
  name=因子,
  description={factorial},
  sort={factorial},
}

\newglossaryentry{factorized}
{
  name=分解的,
  description={factorized},
  sort={factorized},
}

\newglossaryentry{meanfield}
{
  name=均匀场,
  description={meanfield},
  sort={meanfield},
}

\newglossaryentry{MLE}
{
  name=最大似然估计,
  description={maximum likelihood estimation},
  sort={maximum likelihood estimation},
}

\newglossaryentry{PPCA}
{
  name=概率PCA,
  description={probabilistic PCA},
  sort={probabilistic PCA},
  symbol={PPCA}
}

\newglossaryentry{SGA}
{
  name=随机梯度上升,
  description={Stochastic Gradient Ascent},
  sort={Stochastic Gradient Ascent},
}

\newglossaryentry{clique}
{
  name=团,
  description={clique},
  sort={clique},
}

\newglossaryentry{dirac_distribution}
{
  name=Dirac分布,
  description={dirac distribution},
  sort={dirac distribution},
}

\newglossaryentry{fixed_point_equation}
{
  name=不动点方程,
  description={fixed point equation},
  sort={fixed point equation},
}

\newglossaryentry{calculus_of_variations}
{
  name=变分法,
  description={calculus of variations},
  sort={calculus of variations},
}

\newglossaryentry{wake_sleep}
{
  name=醒眠,
  description={wake sleep},
  sort={wake sleep},
}

\newglossaryentry{BN}
{
  name=信念网络,
  description={belief network},
  sort={belief network},
}

\newglossaryentry{MRF}
{
  name=马尔可夫随机场,
  description={Markov random field},
  sort={Markov random field},
  symbol={MRF}
}

\newglossaryentry{markov_network}
{
  name=马尔可夫网络,
  description={Markov network},
  sort={Markov network},
}

\newglossaryentry{log_linear_model}
{
  name=对数线性模型,
  description={log-linear model},
  sort={log-linear model},
}

\newglossaryentry{product_of_expert}
{
  name=专家之积,
  description={product of expert},
  sort={product of expert},
}

\newglossaryentry{free_energy}
{
  name=自由能,
  description={free energy},
  sort={free energy},
}

\newglossaryentry{harmony}
{
  name=harmony,
  description={harmony},
  sort={harmony},
}

\newglossaryentry{separation}
{
  name=分离,
  description={separation},
  sort={separation},
}

\newglossaryentry{separate}
{
  name=分离的,
  description={separate},
  sort={separate},
}

\newglossaryentry{dseparation}
{
  name=d-分离,
  description={d-separation},
  sort={d-separation},
}

\newglossaryentry{local_conditional_probability_distribution}
{
  name=局部条件概率分布,
  description={local conditional probability distribution},
  sort={local conditional probability distribution},
}

\newglossaryentry{conditional_probability_distribution}
{
  name=条件概率分布,
  description={conditional probability distribution}
}

\newglossaryentry{boltzmann_distribution}
{
  name=玻尔兹曼分布,
  description={Boltzmann distribution},
  sort={Boltzmann distribution},
}

\newglossaryentry{gibbs_distribution}
{
  name=吉布斯分布,
  description={Gibbs distribution},
  sort={Gibbs distribution},
}

\newglossaryentry{energy_function}
{
  name=能量函数,
  description={energy function},
  sort={energy function},
}

\newglossaryentry{immorality}
{
  name=不道德,
  description={immorality},
  sort={immorality},
}

\newglossaryentry{moralization}
{
  name=道德化,
  description={moralization},
  sort={moralization},
}

\newglossaryentry{moralized_graph}
{
  name=道德图,
  description={moralized graph},
  sort={moralized graph},
}

\newglossaryentry{standard_deviation}
{
  name=标准差,
  description={standard deviation},
  symbol={SD},
  sort={standard deviation},
}

\newglossaryentry{correlation}
{
  name=相关系数,
  description={correlation},
  sort={correlation},
}

\newglossaryentry{standard_normal_distribution}
{
  name=标准正态分布,
  description={standard normal distribution},
  sort={standard normal distribution},
}

\newglossaryentry{covariance_matrix}
{
  name=协方差矩阵,
  description={covariance matrix},
  sort={covariance matrix},
}

\newglossaryentry{bernoulli_distribution}
{
  name=Bernoulli分布,
  description={Bernoulli distribution},
  sort={Bernoulli distribution},
}

\newglossaryentry{bernoulli_output_distribution}
{
  name=Bernoulli输出分布,
  description={Bernoulli output distribution},
  sort={Bernoulli output distribution},
}

\newglossaryentry{multinoulli_distribution}
{
  name=Multinoulli分布,
  description={multinoulli distribution},
  sort={multinoulli distribution},
}

\newglossaryentry{multinoulli_output_distribution}
{
  name=Multinoulli输出分布,
  description={multinoulli output distribution},
  sort={multinoulli output distribution},
}

\newglossaryentry{categorical_distribution}
{
  name=范畴分布,
  description={categorical distribution},
  sort={categorical distribution},
}

\newglossaryentry{multinomial_distribution}
{
  name=多项式分布,
  description={multinomial distribution},
  sort={multinomial distribution},
}

\newglossaryentry{normal_distribution}
{
  name=正态分布,
  description={normal distribution},
  sort={normal distribution},
}

\newglossaryentry{gaussian_distribution}
{
  name=高斯分布,
  description={Gaussian distribution},
  sort={Gaussian distribution},
}

\newglossaryentry{precision}
{
  name=精度,
  description={precision},
  sort={precision},
}

\newglossaryentry{multivariate_normal_distribution}
{
  name=多维正态分布,
  description={multivariate normal distribution},
  sort={multivariate normal distribution},
}

\newglossaryentry{precision_matrix}
{
  name=精度矩阵,
  description={precision matrix},
  sort={precision matrix},
}

\newglossaryentry{isotropic}
{
  name=各向同性,
  description={isotropic},
  sort={isotropic},
}

\newglossaryentry{exponential_distribution}
{
  name=指数分布,
  description={exponential distribution},
  sort={exponential distribution},
}

\newglossaryentry{indicator_function}
{
  name=指示函数,
  description={indicator function},
  sort={indicator function},
}

\newglossaryentry{laplace_distribution}
{
  name=Laplace分布,
  description={Laplace distribution},
  sort={Laplace distribution},
}

\newglossaryentry{dirac_delta_function}
{
  name=Dirac delta函数,
  description={Dirac delta function},
  sort={Dirac delta function},
}

\newglossaryentry{generalized_function}
{
  name=广义函数,
  description={generalized function},
  sort={generalized function},
}

\newglossaryentry{empirical_distribution}
{
  name=经验分布,
  description={empirical distribution},
  sort={empirical distribution},
}

\newglossaryentry{empirical_frequency}
{
  name=经验频率,
  description={empirical frequency},
  sort={empirical frequency},
}

\newglossaryentry{mixture_distribution}
{
  name=混合分布,
  description={mixture distribution},
  sort={mixture distribution},
}

\newglossaryentry{latent_variable}
{
  name=潜变量,
  description={latent variable},
  sort={latent variable},
}

\newglossaryentry{hidden_variable}
{
  name=隐藏变量,
  description={hidden variable},
  sort={hidden variable},
}

\newglossaryentry{prior_probability}
{
  name=先验概率,
  description={prior probability},
  sort={prior probability},
}

\newglossaryentry{posterior_probability}
{
  name=后验概率,
  description={posterior probability},
  sort={posterior probability},
}

\newglossaryentry{universal_approximator}
{
  name=万能近似器,
  description={universal approximator},
  sort={universal approximator},
}

\newglossaryentry{universal_function_approximator}
{
  name=万能函数近似器,
  description={universal function approximator},
  sort={universal function approximator},
}

\newglossaryentry{logistic_sigmoid}
{
  name=logistic sigmoid,
  description={logistic sigmoid},
  sort={logistic sigmoid},
}

\newglossaryentry{sigmoid}
{
  name=sigmoid,
  description={sigmoid},
  sort={sigmoid},
}

\newglossaryentry{saturate}
{
  name=饱和,
  description={saturate},
  sort={saturate},
}

\newglossaryentry{softplus_function}
{
  name=softplus函数,
  description={softplus function},
  sort={softplus function},
}

\newglossaryentry{logit}
{
  name=分对数,
  description={logit},
  sort={logit},
}

\newglossaryentry{positive_part_function}
{
  name=正部函数,
  description={positive part function},
  sort={positive part function},
}

\newglossaryentry{negative part function}
{
  name=负部函数,
  description={negative part function},
  sort={negative part function},
}

\newglossaryentry{bayes_rule}
{
  name=贝叶斯规则,
  description={Bayes' rule},
  sort={Bayes' rule},
}

\newglossaryentry{measure_theory}
{
  name=测度论,
  description={measure theory},
  sort={measure theory},
}

\newglossaryentry{measure_zero}
{
  name=零测度,
  description={measure zero},
  sort={measure zero},
}

\newglossaryentry{almost_everywhere}
{
  name=几乎处处,
  description={almost everywhere},
  sort={almost everywhere},
}

\newglossaryentry{jacobian_matrix}
{
  name=Jacobian矩阵,
  description={Jacobian matrix},
  sort={Jacobian matrix},
}

\newglossaryentry{self_information}
{
  name=自信息,
  description={self-information},
  sort={self-information},
}

\newglossaryentry{nats}
{
  name=奈特,
  description={nats},
  sort={nats},
}

\newglossaryentry{bits}
{
  name=比特,
  description={bit},
  sort={bit},
}

\newglossaryentry{shannons}
{
  name=香农,
  description={shannons},
  sort={shannons},
}

\newglossaryentry{Shannon_entropy}
{
  name=香农熵,
  description={Shannon entropy},
  sort={Shannon entropy},
}

\newglossaryentry{differential_entropy}
{
  name=微分熵,
  description={differential entropy},
  sort={differential entropy},
}

\newglossaryentry{differential_equation}
{
  name=微分方程,
  description={differential equation},
  sort={differential equation},
}

\newglossaryentry{KL_divergence}
{
  name=KL散度,
  description={Kullback-Leibler (KL) divergence},
  sort={Kullback-Leibler (KL) divergence},
}

\newglossaryentry{cross_entropy}
{
  name=交叉熵,
  description={cross-entropy},
  sort={cross-entropy},
}

\newglossaryentry{entropy}
{
  name=熵,
  description={entropy},
  sort={entropy},
}

\newglossaryentry{factorization}
{
  name=分解,
  description={factorization},
  sort={factorization},
}

\newglossaryentry{structured_probabilistic_model}
{
  name=结构化概率模型,
  description={structured probabilistic model},
  sort={structured probabilistic model},
}

\newglossaryentry{graphical_model}
{
  name=图模型,
  description={graphical model},
  sort={graphical model},
}

\newglossaryentry{backoff}
{
  name=回退,
  description={back-off},
  sort={back-off},
}

\newglossaryentry{directed}
{
  name=有向,
  description={directed},
  sort={directed},
}

\newglossaryentry{undirected}
{
  name=无向,
  description={undirected},
  sort={undirected},
}

\newglossaryentry{undirected_graphical_model}
{
  name=无向图模型,
  description={undirected graphical model}
}

\newglossaryentry{proportional}
{
  name=成比例,
  description={proportional},
  sort={proportional},
}

\newglossaryentry{description}
{
  name=描述,
  description={description},
  sort={description},
}

\newglossaryentry{decision_tree}
{
  name=决策树,
  description={decision tree},
  sort={decision tree},
}

\newglossaryentry{factor_graph}
{
  name=因子图,
  description={factor graph},
  sort={factor graph},
}

\newglossaryentry{structure_learning}
{
  name=结构学习,
  description={structure learning},
  sort={structure learning},
}

\newglossaryentry{loopy_belief_propagation}
{
  name=环状信念传播,
  description={loopy belief propagation},
  sort={loopy belief propagation},
}

\newglossaryentry{harmonium}
{
  name=簧风琴,
  description={harmonium},
  sort={harmonium},
}

\newglossaryentry{convolutional_network}
{
  name=卷积网络,
  description={convolutional network},
  sort={convolutional network},
}

\newglossaryentry{convolutional_net}
{
  name=卷积网络,
  description={convolutional net},
  sort={convolutional net},
}

\newglossaryentry{main_diagonal}
{
  name=主对角线,
  description={main diagonal},
  sort={main diagonal},
}

\newglossaryentry{transpose}
{
  name=转置,
  description={transpose},
  sort={transpose},
}

\newglossaryentry{broadcasting}
{
  name=广播,
  description={broadcasting},
  sort={broadcasting},
}

\newglossaryentry{matrix_product}
{
  name=矩阵乘积,
  description={matrix product},
  sort={matrix product},
}

\newglossaryentry{adagrad}
{
  name=AdaGrad,
  description={AdaGrad},
  sort={AdaGrad},
}

\newglossaryentry{element_wise_product}
{
  name=元素对应乘积,
  description={element-wise product},
  sort={element-wise product},
}

\newglossaryentry{hadamard_product}
{
  name=Hadamard乘积,
  description={Hadamard product},
  sort={Hadamard product},
}

\newglossaryentry{clique_potential}
{
  name=团势能,
  description={clique potential},
  sort={clique potential},
}

\newglossaryentry{factor}
{
  name=因子,
  description={factor},
  sort={factor},
}

\newglossaryentry{unnormalized_probability_function}
{
  name=未归一化概率函数,
  description={unnormalized probability function},
  sort={unnormalized probability function},
}

\newglossaryentry{recurrent_network}
{
  name=循环网络,
  description={recurrent network},
  sort={recurrent network},
}

\newglossaryentry{vanish_explode_gradient}
{
  name=梯度消失与爆炸问题,
  description={vanishing and exploding gradient problem},
  sort={vanishing and exploding gradient problem },
}

\newglossaryentry{vanish_gradient}
{
  name=梯度消失,
  description={vanishing gradient},
  sort={vanishing gradient},
}

\newglossaryentry{explode_gradient}
{
  name=梯度爆炸,
  description={exploding gradient},
  sort={exploding gradient},
}

\newglossaryentry{computational_graph}
{
  name=计算图,
  description={computational graph},
  sort={computational graph},
}

\newglossaryentry{unfolding}
{
  name=展开,
  description={unfolding},
  sort={unfolding},
}

\newglossaryentry{invert}
{
  name=求逆,
  description={invert},
  sort={invert},
}

\newglossaryentry{time_step}
{
  name=时间步,
  description={time step},
  sort={time step},
}

\newglossaryentry{n_gram}
{
  name=$n$-gram,
  description={n-gram},
  sort={n-gram},
}

\newglossaryentry{curse_of_dimensionality}
{
  name=维数灾难,
  description={curse of dimensionality},
  sort={curse of dimensionality},
}

\newglossaryentry{smoothness_prior}
{
  name=平滑先验,
  description={smoothness prior},
  sort={smoothness prior},
}

\newglossaryentry{local_constancy_prior}
{
  name=局部不变性先验,
  description={local constancy prior},
  sort={local constancy prior},
}

\newglossaryentry{local_kernel}
{
  name=局部核,
  description={local kernel},
  sort={local kernel},
}

\newglossaryentry{manifold}
{
  name=流形,
  description={manifold},
  sort={manifold},
}

\newglossaryentry{manifold_tangent_classifier}
{
  name=流形正切分类器,
  description={manifold tangent classifier}
}

\newglossaryentry{manifold_learning}
{
  name=流形学习,
  description={manifold learning},
  sort={manifold learning},
}

\newglossaryentry{manifold_hypothesis}
{
  name=流形假设,
  description={manifold hypothesis},
  sort={manifold hypothesis},
}

\newglossaryentry{loop}
{
  name=环,
  description={loop},
  sort={loop},
}

\newglossaryentry{chord}
{
  name=弦,
  description={chord},
  sort={chord},
}

\newglossaryentry{chordal_graph}
{
  name=弦图,
  description={chordal graph},
  sort={chordal graph},
}

\newglossaryentry{triangulated_graph}
{
  name=三角形化图,
  description={triangulated graph},
  sort={triangulated graph},
}

\newglossaryentry{triangulate}
{
  name=三角形化,
  description={triangulate},
  sort={triangulate},
}

\newglossaryentry{risk}
{
  name=风险,
  description={risk},
  sort={risk},
}

\newglossaryentry{empirical_risk}
{
  name=经验风险,
  description={empirical risk},
  sort={empirical risk},
}

\newglossaryentry{empirical_risk_minimization}
{
  name=经验风险最小化,
  description={empirical risk minimization},
  sort={empirical risk minimization},
}

\newglossaryentry{surrogate_loss_function}
{
  name=代理损失函数,
  description={surrogate loss function},
  sort={surrogate loss function},
}

\newglossaryentry{batch}
{
  name=批量,
  description={batch},
  sort={batch},
}

\newglossaryentry{deterministic}
{
  name=确定性,
  description={deterministic},
  sort={deterministic},
}

\newglossaryentry{stochastic}
{
  name=随机,
  description={stochastic},
  sort={stochastic},
}

\newglossaryentry{online}
{
  name=在线,
  description={online},
  sort={online},
}

\newglossaryentry{minibatch_stochastic}
{
  name=小批量随机, %小批量
  description={minibatch stochastic},
  sort={minibatch stochastic},
}

\newglossaryentry{stream}
{
  name=流,
  description={stream},
  sort={stream},
}

\newglossaryentry{model_identifiability}
{
  name=模型可辨识性,
  description={model identifiability},
  sort={model identifiability},
}

\newglossaryentry{weight_space_symmetry}
{
  name=权重空间对称性,
  description={weight space symmetry},
  sort={weight space symmetry},
}

\newglossaryentry{saddle_free_newton_method}
{
  name=无鞍牛顿法,
  description={saddle-free Newton method},
  sort={saddle-free Newton method},
}

\newglossaryentry{gradient_clipping} 
{
  name=梯度截断,
  description={gradient clipping},
  sort={gradient clipping},
}

\newglossaryentry{power_method}
{
  name=幂方法,
  description={power method},
  sort={power method},
}

\newglossaryentry{linear_factor}
{
  name=线性因子模型,
  description={linear factor model},
  sort={linear factor model},
}

\newglossaryentry{forward_propagation}
{
  name=前向传播,
  description={forward propagation},
  sort={forward propagation},
}

\newglossaryentry{backward_propagation}
{
  name=反向传播,
  description={backward propagation},
  sort={backward propagation},
}

\newglossaryentry{unfolded_graph}
{
  name=展开图,
  description={unfolded graph},
  sort={unfolded graph},
}

\newglossaryentry{BPTT}
{
  name=通过时间反向传播,
  description={back-propagation through time},
  sort={back-propagation through time},
  symbol={BPTT}
}

\newglossaryentry{teacher_forcing}
{
  name=导师驱动过程,
  description={teacher forcing},
  sort={teacher forcing},
}

\newglossaryentry{stationary}
{
  name=平稳的,
  description={stationary},
  sort={stationary},
}

\newglossaryentry{deep_feedforward_network}
{
  name=深度前馈网络,
  description={deep feedforward network},
  sort={deep feedforward network},
}

\newglossaryentry{feedforward_neural_network}
{
  name=前馈神经网络,
  description={feedforward neural network},
  sort={feedforward neural network},
}

\newglossaryentry{feedforward}
{
  name=前向,
  description={feedforward},
  sort={feedforward},
}

\newglossaryentry{feedback}
{
  name=反馈,
  description={feedback},
  sort={feedback},
}

\newglossaryentry{network}
{
  name=网络,
  description={network},
  sort={network},
}

\newglossaryentry{first_layer}
{
  name=第一层,
  description={first layer},
  sort={first layer},
}

\newglossaryentry{second_layer}
{
  name=第二层,
  description={second layer},
  sort={second layer},
}

\newglossaryentry{depth}
{
  name=深度,
  description={depth},
  sort={depth},
}

\newglossaryentry{output_layer}
{
  name=输出层,
  description={output layer},
  sort={output layer},
}

\newglossaryentry{hidden_layer}
{
  name=隐藏层,
  description={hidden layer},
  sort={hidden layer},
}

\newglossaryentry{width}
{
  name=宽度,
  description={width},
  sort={width},
}

\newglossaryentry{unit}
{
  name=单元,
  description={unit},
  sort={unit},
}

\newglossaryentry{activation_function}
{
  name=激活函数,
  description={activation function},
  sort={activation function},
}

\newglossaryentry{dbd}
{
  name=delta-bar-delta,
  description={delta-bar-delta},
  sort={delta-bar-delta},
}

\newglossaryentry{gaussian_output_distribution}
{
  name=高斯输出分布,
  description={Gaussian output distribution},
  sort={Gaussian output distribution},
}

\newglossaryentry{back_propagation}
{
  name=反向传播,
  description={back propagation},
  sort={back propagation},
}

\newglossaryentry{back_propagate}
{
  name=反向传播,
  description={back propagate},
  sort={back propagate},
}

\newglossaryentry{BP}
{
  name=反向传播,
  description={backprop},
  sort={backprop},
  symbol={BP}
}

\newglossaryentry{functional}
{
  name=泛函,
  description={functional},
  sort={functional},
}

\newglossaryentry{mean_absolute_error}
{
  name=平均绝对误差,
  description={mean absolute error},
  sort={mean absolute error},
}

\newglossaryentry{winner_take_all}
{
  name=赢者通吃,
  description={winner-take-all},
  sort={winner-take-all},
}

\newglossaryentry{heteroscedastic}
{
  name=异方差,
  description={heteroscedastic},
  sort={heteroscedastic},
}

\newglossaryentry{mixture_density_network}
{
  name=混合密度网络,
  description={mixture density network},
  sort={mixture density network},
}

\newglossaryentry{clip_gradients}
{
  name=梯度截断,
  description={clip gradient},
  sort={clip gradient},
}

\newglossaryentry{absolute_value_rectification}
{
  name=绝对值整流,
  description={absolute value rectification},
  sort={absolute value rectification},
}

\newglossaryentry{leaky_ReLU}   %有没有更好的翻译
{
  name=渗漏整流线性单元,
  description={Leaky ReLU},
  sort={Leaky ReLU},
  symbol={Leaky ReLU}
}

\newglossaryentry{PReLU}
{
  name=参数化整流线性单元,
  description={parametric ReLU},
  sort={parametric ReLU},
  symbol={PReLU}
}

\newglossaryentry{maxout_unit}
{
  name=maxout单元,
  description={maxout unit},
  sort={maxout unit},
}

\newglossaryentry{maxout}
{
  name=maxout,
  description={maxout},
  symbol={maxout}
}

\newglossaryentry{catastrophic_forgetting}
{
  name=灾难遗忘,
  description={catastrophic forgetting},
  sort={catastrophic forgetting},
}

\newglossaryentry{RBF}
{
  name=径向基函数,
  description={radial basis function},
  sort={radial basis function},
  symbol={RBF}
}

\newglossaryentry{softplus}
{
  name=softplus,
  description={softplus},
  sort={softplus},
}

\newglossaryentry{hard_tanh}
{
  name=硬双曲正切函数,
  description={hard tanh},
  sort={hard tanh},
}

\newglossaryentry{architecture}
{
  name=架构,
  description={architecture},
  sort={architecture},
}

\newglossaryentry{universal_approximation_theorem}
{
  name=万能近似定理, %万能逼近定理?
  description={universal approximation theorem},
  sort={universal approximation theorem},
}

\newglossaryentry{operation}
{
  name=操作,
  description={operation},
  sort={operation},
}

\newglossaryentry{symbol}
{
  name=符号,
  description={symbol},
  sort={symbol},
}

\newglossaryentry{numeric_value}
{
  name=数值,
  description={numeric value},
  sort={numeric value},
}

\newglossaryentry{dynamic_programming}
{
  name=动态规划,
  description={dynamic programming},
  sort={dynamic programming},
}

\newglossaryentry{automatic_differentiation}
{
  name=自动微分,
  description={automatic differentiation},
  sort={automatic differentiation},
}

\newglossaryentry{reverse_mode_accumulation}
{
  name=反向模式累加,
  description={reverse mode accumulation},
  sort={reverse mode accumulation},
}

\newglossaryentry{forward_mode_accumulation}
{
  name=前向模式累加,
  description={forward mode accumulation},
  sort={forward mode accumulation},
}

\newglossaryentry{Krylov_methods}
{
  name=Krylov方法,
  description={Krylov method},
  sort={Krylov method},
}

\newglossaryentry{parallel_distributed_processing}
{
  name=并行分布式处理,
  description={Parallel Distributed Processing},
  sort={Parallel Distributed Processing},
}

\newglossaryentry{sparse_activation}
{
  name=稀疏激活,
  description={sparse activation},
  sort={sparse activation},
}

\newglossaryentry{damping}
{
  name=衰减,
  description={damping},
  sort={damping},
}

\newglossaryentry{learned}
{
  name=学成,
  description={learned},
  sort={learned},
}

\newglossaryentry{message_passing}
{
  name=信息传输,
  description={message passing},
  sort={message passing},
}

\newglossaryentry{functional_derivative}
{
  name=泛函导数,
  description={functional derivative},
  sort={functional derivative},
}

\newglossaryentry{variational_derivative}
{
  name=变分导数,
  description={variational derivative},
  sort={variational derivative},
}

\newglossaryentry{excess_error}
{
  name=额外误差,
  description={excess error},
  sort={excess error},
}

\newglossaryentry{momentum}
{
  name=动量,
  description={momentum},
  sort={momentum},
}

\newglossaryentry{nmomentum}
{
  name=Nesterov 动量,
  description={Nesterov momentum},
  sort={Nesterov momentum},
}

\newglossaryentry{chaos}
{
  name=混沌,
  description={chaos},
  sort={chaos},
}

\newglossaryentry{normalized_initialization}
{
  name=标准初始化, % ??
  description={normalized initialization},
  sort={normalized initialization},
}

\newglossaryentry{sparse_initialization}
{
  name=稀疏初始化,
  description={sparse initialization},
  sort={sparse initialization},
}

\newglossaryentry{conjugate_directions}
{
  name=共轭方向,
  description={conjugate directions},
  sort={conjugate directions},
}

\newglossaryentry{conjugate}
{
  name=共轭,
  description={conjugate},
  sort={conjugate},
}

\newglossaryentry{conditional_independent}
{
  name=条件独立,
  description={conditionally independent},
  sort={conditionally independent},
}

\newglossaryentry{ensemble_learning}
{
  name=集成学习,
  description={ensemble learning},
  sort={ensemble learning},
}

\newglossaryentry{NICE}
{
  name=非线性独立成分估计,
  description={nonlinear independent components estimation},
  sort={nonlinear independent components estimation},
  symbol={NICE}
}

\newglossaryentry{ISA}
{
  name=独立子空间分析,
  description={independent subspace analysis},
  sort={independent subspace analysis},
}

\newglossaryentry{SFA}
{
  name=慢特征分析,
  description={slow feature analysis},
  sort={slow feature analysis},
  symbol={SFA}
}

\newglossaryentry{slow_principle}
{
  name=慢性原则,
  description={slowness principle},
  sort={slowness principle},
}

\newglossaryentry{rectified_linear}
{
  name=整流线性,
  description={rectified linear},
  sort={rectified linear},
}

\newglossaryentry{rectified_linear_transformation}
{
  name=整流线性变换,
  description={rectified linear transformation},
  sort={rectified linear transformation},
}

\newglossaryentry{rectifier_network}
{
  name=整流网络,
  description={rectifier network},
  sort={rectifier network},
}

\newglossaryentry{coordinate_descent}
{
  name=坐标下降,
  description={coordinate descent},
  sort={coordinate descent},
}

\newglossaryentry{coordinate_ascent}
{
  name=坐标上升,
  description={coordinate ascent},
  sort={coordinate ascent},
}

\newglossaryentry{block_coordinate_descent}
{
  name=块坐标下降,
  description={block coordinate descent},
  sort={block coordinate descent},
}

\newglossaryentry{pretraining}
{
  name=预训练,
  description={pretraining},
  sort={pretraining},
}

\newglossaryentry{unsupervised_pretraining}
{
  name=无监督预训练,
  description={unsupervised pretraining},
  sort={unsupervised pretraining},
}

\newglossaryentry{greedy_layer_wise_unsupervised_pretraining}
{
  name=贪心逐层无监督预训练,
  description={greedy layer-wise unsupervised pretraining},
  sort={greedy layer-wise unsupervised pretraining},
}

\newglossaryentry{greedy_layer_wise_pretraining}
{
  name=贪心逐层预训练,
  description={greedy layer-wise pretraining},
  sort={greedy layer-wise pretraining},
}

\newglossaryentry{greedy_layer_wise_training}
{
  name=贪心逐层训练,
  description={greedy layer-wise training},
  sort={greedy layer-wise training},
}

\newglossaryentry{layer_wise}
{
  name=逐层的,
  description={layer-wise},
  sort={layer-wise},
}

\newglossaryentry{greedy_algorithm}
{
  name=贪心算法,
  description={greedy algorithm},
  sort={greedy algorithm},
}

\newglossaryentry{greedy}
{
  name=贪心,
  description={greedy},
  sort={greedy},
}

\newglossaryentry{fine_tuning}
{
  name=精调,
  description={fine-tuning},
  sort={fine-tuning},
}

\newglossaryentry{greedy_supervised_pretraining}
{
  name=贪心监督预训练,
  description={greedy supervised pretraining},
  sort={greedy supervised pretraining},
}

\newglossaryentry{continuation_method}
{
  name=延拓法,
  description={continuation method},
  sort={continuation method},
}

\newglossaryentry{curriculum_learning}
{
  name=课程学习,
  description={curriculum learning},
  sort={curriculum learning},
}

\newglossaryentry{shaping}
{
  name=塑造, % ? 整形
  description={shaping},
  sort={shaping},
}

\newglossaryentry{stochastic_curriculum}
{
  name=随机课程,
  description={stochastic curriculum},
  sort={stochastic curriculum},
}

\newglossaryentry{recall}
{
  name=召回率,
  description={recall},
  sort={recall},
}

\newglossaryentry{coverage}
{
  name=覆盖,
  description={coverage},
  sort={coverage},
}

\newglossaryentry{hyperparameter_optimization}
{
  name=超参数优化,
  description={hyperparameter optimization},
  sort={hyperparameter optimization},
}

\newglossaryentry{hyperparameter}
{
  name=超参数,
  description={hyperparameter},
  sort={hyperparameter}
}

\newglossaryentry{grid_search}
{
  name=网格搜索,
  description={grid search},
  sort={grid search},
}

\newglossaryentry{finite_difference}
{
  name=有限差分,
  description={finite difference},
  sort={finite difference},
}

\newglossaryentry{centered_difference}
{
  name=中心差分,
  description={centered difference},
  sort={centered difference},
}

\newglossaryentry{e_step}
{
  name=E步,
  description={expectation step},
  sort={expectation step},
  symbol={E step}
}

\newglossaryentry{m_step}
{
  name=M步,
  description={maximization step},
  sort={maximization step},
  symbol={M step}
}

\newglossaryentry{euler_lagrange_eqn}
{
  name=欧拉-拉格朗日方程,
  description={Euler-Lagrange Equation},
  sort={Euler-Lagrange Equation},
}

\newglossaryentry{lagrange_multi}
{
  name=拉格朗日乘子,
  description={Lagrange multiplier},
  sort={Lagrange multiplier},
}

\newglossaryentry{ESN}
{
  name=回声状态网络,
  description={echo state network},
  sort={echo state network},
  symbol={ESN}
}

\newglossaryentry{liquid_state_machines}
{
  name=流体状态机,
  description={liquid state machine},
  sort={liquid state machine},
}

\newglossaryentry{reservoir_computing}
{
  name=储层计算,
  description={reservoir computing},
  sort={reservoir computing},
}

\newglossaryentry{spectral_radius}
{
  name=谱半径,
  description={spectral radius},
  sort={spectral radius},
}

\newglossaryentry{contractive}
{
  name=收缩,
  description={contractive},
  sort={contractive},
}

\newglossaryentry{long_term_dependency}
{
  name=长期依赖,
  description={long-term dependency},
  sort={long-term dependency},
}

\newglossaryentry{skip_connection}
{
  name=跳跃连接,
  description={skip connection},
  sort={skip connection},
}

\newglossaryentry{leaky_unit}
{
  name=渗漏单元,
  description={leaky unit},
  sort={leaky unit},
}

\newglossaryentry{gated_rnn}
{
  name=门控RNN,
  description={gated RNN},
  sort={gated RNN},
}

\newglossaryentry{gated}
{
  name=门控,
  description={gated},
  sort={gated},
}

\newglossaryentry{convolution}
{
  name=卷积,
  description={convolution},
  sort={convolution},
}

\newglossaryentry{input}
{
  name=输入,
  description={input},
  sort={input},
}

\newglossaryentry{input_distribution}
{
  name=输入分布,
  description={input distribution},
  sort={input distribution},
}

\newglossaryentry{output}
{
  name=输出,
  description={output},
  sort={output},
}

\newglossaryentry{feature_map}
{
  name=特征映射,
  description={feature map},
  sort={feature map},
}

\newglossaryentry{flip}
{
  name=翻转,
  description={flip},
  sort={flip},
}

\newglossaryentry{cross_correlation}
{
  name=互相关函数,
  description={cross-correlation},
  sort={cross-correlation},
}

\newglossaryentry{Toeplitz_matrix}
{
  name=Toeplitz矩阵,
  description={Toeplitz matrix},
  sort={Toeplitz matrix},
}

\newglossaryentry{doubly_block_circulant_matrix}
{
  name=双重分块循环矩阵,
  description={doubly block circulant matrix},
  sort={doubly block circulant matrix},
}

\newglossaryentry{sparse_interactions}
{
  name=稀疏交互,
  description={sparse interactions},
  sort={sparse interactions},
}

\newglossaryentry{equivariant_representations}
{
  name=等变表示,
  description={equivariant representations},
  sort={equivariant representations},
}

\newglossaryentry{sparse_connectivity}
{
  name=稀疏连接,
  description={sparse connectivity},
  sort={sparse connectivity},
}

\newglossaryentry{sparse_weights}
{
  name=稀疏权重,
  description={sparse weights},
  sort={sparse weights},
}

\newglossaryentry{receptive_field}
{
  name=接受域,
  description={receptive field},
  sort={receptive field},
}

\newglossaryentry{tied_weights}
{
  name=绑定的权重,
  description={tied weights},
  sort={tied weights},
}

\newglossaryentry{equivariance}
{
  name=等变,
  description={equivariance},
  sort={equivariance},
}

\newglossaryentry{detector_stage}
{
  name=探测级,
  description={detector stage},
  sort={detector stage},
}

\newglossaryentry{symbolic_representation}
{
  name=符号表示,
  description={symbolic representation},
  sort={symbolic representation},
}

\newglossaryentry{pooling_funciton}
{
  name=池化函数,
  description={pooling function},
  sort={pooling function},
}

\newglossaryentry{max_pooling}
{
  name=最大池化,
  description={max pooling},
  sort={max pooling},
}

\newglossaryentry{pool}
{
  name=池,
  description={pool},
  sort={max pool},
}

\newglossaryentry{invariant}
{
  name=不变,
  description={invariant},
  sort={invariant},
}

\newglossaryentry{permutation_invariant}
{
  name=置换不变性,
  description={permutation invariant},
  sort={permutation invariant},
}

\newglossaryentry{stride}
{
  name=步幅,
  description={stride},
  sort={stride},
}

\newglossaryentry{downsampling}
{
  name=降采样,
  description={downsampling},
  sort={downsampling},
}

\newglossaryentry{valid}
{
  name=有效,
  description={valid},
  sort={valid},
}

\newglossaryentry{same}
{
  name=相同,
  description={same},
  sort={same},
}

\newglossaryentry{full}
{
  name=全,
  description={full},
  sort={full},
}

\newglossaryentry{unshared_convolution}
{
  name=非共享卷积,
  description={unshared convolution},
  sort={unshared convolution},
}

\newglossaryentry{tiled_convolution}
{
  name=平铺卷积,
  description={tiled convolution},
  sort={tiled convolution},
}

\newglossaryentry{recurrent_convolutional_network}
{
  name=循环卷积网络,
  description={recurrent convolutional network},
  sort={recurrent convolutional network},
}

\newglossaryentry{Fourier_transform}
{
  name=傅立叶变换,
  description={Fourier transform},
  sort={Fourier transform},
}

\newglossaryentry{separable}
{
  name=可分离的,
  description={separable},
  sort={separable},
}

\newglossaryentry{primary_visual_cortex}
{
  name=初级视觉皮层,
  description={primary visual cortex},
  sort={primary visual cortex},
}

\newglossaryentry{simple_cells}
{
  name=简单细胞,
  description={simple cell},
  sort={simple cell},
}

\newglossaryentry{complex_cells}
{
  name=复杂细胞,
  description={complex cell},
  sort={complex cell},
}

\newglossaryentry{fovea}
{
  name=中央凹,
  description={fovea},
  sort={fovea},
}

\newglossaryentry{saccade}
{
  name=扫视,
  description={saccade},
  sort={saccade},
}

\newglossaryentry{TDNNs}
{
  name=时延神经网络,
  description={time delay neural network},
  sort={time delay neural network},
  symbol={TDNN}
}

\newglossaryentry{reverse_correlation}
{
  name=反向相关,
  description={reverse correlation},
  sort={reverse correlation},
}

\newglossaryentry{Gabor_function}
{
  name=Gabor函数,
  description={Gabor function},
  sort={Gabor function},
}

\newglossaryentry{quadrature_pair}
{
  name=象限对,
  description={quadrature pair},
  sort={quadrature pair},
}

\newglossaryentry{gated_recurrent_unit}
{
  name=门控循环单元,
  description={gated recurrent unit},
  sort={gated recurrent unit},
  symbol={GRU}
}

\newglossaryentry{gated_recurrent_net}
{
  name=门控循环网络,
  description={gated recurrent net},
  sort={gated recurrent net}, 
}

\newglossaryentry{forget_gate}
{
  name=遗忘门,
  description={forget gate},
  sort={forget gate},
}

\newglossaryentry{clipping_gradient}
{
  name=截断梯度,
  description={clipping the gradient},
  sort={clipping the gradient},
}

\newglossaryentry{memory_network}
{
  name=记忆网络,
  description={memory network},
  sort={memory network},
}

\newglossaryentry{NTM}
{
  name=神经网络图灵机,
  description={neural Turing machine},
  sort={neural Turing machine},
  symbol={NTM}
}

\newglossaryentry{fine_tune}
{
  name=精调,
  description={fine-tune},
  sort={fine-tune},
}

\newglossaryentry{explaining_away}
{
  name=相消解释,
  description={explaining away},
  sort={explaining away},
}

\newglossaryentry{explaining_away_effect}
{
  name=相消解释作用,
  description={explaining away effect},
  sort={explaining away effect},
}

\newglossaryentry{common_cause}
{
  name=共因,
  description={common cause},
  sort={common cause},
}

\newglossaryentry{code}
{
  name=编码,
  description={code},
  sort={code},
}

\newglossaryentry{recirculation}
{
  name=再循环,
  description={recirculation},
  sort={recirculation},
}

\newglossaryentry{undercomplete}
{
  name=欠完备,
  description={undercomplete},
  sort={undercomplete},
}

\newglossaryentry{complete_graph}
{
  name=完全图,
  description={complete graph},
  sort={complete graph},
}

\newglossaryentry{underdetermined}
{
  name=欠定的,
  description={underdetermined},
  sort={underdetermined},
}

\newglossaryentry{overcomplete}
{
  name=过完备,
  description={overcomplete},
  sort={overcomplete},
}

\newglossaryentry{denoising}
{
  name=去噪,
  description={denoising},
  sort={denoising},
}

\newglossaryentry{denoise}
{
  name=去噪,
  description={denoise}
}

\newglossaryentry{DAE}
{
  name=去噪自编码器,
  description={denoising autoencoder},
  sort={denoising autoencoder},
  symbol={DAE}
}

\newglossaryentry{CAE}
{
  name=收缩自编码器,
  description={contractive autoencoder},
  sort={contractive autoencoder},
  symbol={CAE}
}

\newglossaryentry{reconstruction_error}
{
  name=重构误差,
  description={reconstruction error},
  sort={reconstruction error},
}

\newglossaryentry{gradient_field}
{
  name=梯度场,
  description={gradient field},
  sort={gradient field},
}

\newglossaryentry{denoising_score_matching}
{
  name=去噪得分匹配,
  description={denoising score matching},
  sort={denoising score matching},
}

\newglossaryentry{score}
{
  name=得分,
  description={score},
  sort={score},
}

\newglossaryentry{tangent_plane}
{
  name=切平面,
  description={tangent plane},
  sort={tangent plane},
}

\newglossaryentry{nearest_neighbor_graph}
{
  name=最近邻图,
  description={nearest neighbor graph},
  sort={nearest neighbor graph},
}

\newglossaryentry{embedding}
{
  name=嵌入,
  description={embedding},
  sort={embedding},
}

\newglossaryentry{PSD}
{
  name=预测稀疏分解,
  description={predictive sparse decomposition},
  sort={predictive sparse decomposition},
  symbol={PSD}
}

\newglossaryentry{learned_approximate_inference}
{
  name=学习近似推断,
  description={learned approximate inference},
  sort={learned approximate inference},
}

\newglossaryentry{approximate_inference}
{
  name=近似推断,
  description={approximate inference},
  sort={approximate inference},
}

\newglossaryentry{information_retrieval}
{
  name=信息检索,
  description={information retrieval},
  sort={information retrieval},
}

\newglossaryentry{semantic_hashing}
{
  name=语义哈希,
  description={semantic hashing},
  sort={semantic hashing},
}

\newglossaryentry{dimensionality_reduction}
{
  name=降维,
  description={dimensionality reduction},
  sort={dimensionality reduction},
}

\newglossaryentry{helmholtz_machine}
{
  name=Helmholtz机,
  description={Helmholtz machine},
  sort={Helmholtz machine},
}

\newglossaryentry{contrastive_divergence}
{
  name=对比散度,
  description={contrastive divergence},
  symbol={CD},
  sort={contrastive divergence},
}

\newglossaryentry{language_model}
{
  name=语言模型,
  description={language model},
  sort={language model},
}

\newglossaryentry{token}
{
  name=标记,
  description={token},
  sort={token},
}

\newglossaryentry{unigram}
{
  name=一元语法,
  description={unigram},
  sort={unigram},
}

\newglossaryentry{bigram}
{
  name=二元语法,
  description={bigram},
  sort={bigram},
}

\newglossaryentry{trigram}
{
  name=三元语法,
  description={trigram},
  sort={trigram},
}

\newglossaryentry{smoothing}
{
  name=平滑,
  description={smoothing},
  sort={smoothing},
}

\newglossaryentry{NLM}
{
  name=神经语言模型,
  description={Neural Language Model},
  sort={Neural Language Model},
  symbol={NLM}
}

\newglossaryentry{cascade}
{
  name=级联,
  description={cascade},
  sort={cascade},
}

\newglossaryentry{policy_gradient}
{
  name=策略梯度,
  description={policy gradient},
  sort={policy gradient},
}

\newglossaryentry{DGM}
{
  name=深度生成模型,
  description={deep generative model},
  sort={deep generative model},
}

\newglossaryentry{model}
{
  name=模型,
  description={model},
  sort={model},
}

\newglossaryentry{layer}
{
  name=层,
  description={layer},
  sort={layer},
}

\newglossaryentry{greedy_unsupervised_pretraining}
{
  name=贪心无监督预训练,
  description={greedy unsupervised pretraining},
  sort={greedy unsupervised pretraining},
}

\newglossaryentry{semi_supervised_learning}
{
  name=半监督学习,
  description={semi-supervised learning},
  sort={semi-supervised learning},
}

\newglossaryentry{supervised_model}
{
  name=监督模型,
  description={supervised model},
  sort={supervised model},
}

\newglossaryentry{word_embeddings}
{
  name=词嵌入,
  description={word embedding},
  sort={word embedding},
}

\newglossaryentry{one_hot}
{
  name=one-hot,
  description={one-hot},
  sort={one-hot},
}

\newglossaryentry{supervised_pretraining}
{
  name=监督预训练,
  description={supervised pretraining},
  sort={supervised pretraining},
}

\newglossaryentry{transfer_learning}
{
  name=迁移学习,
  description={transfer learning},
  sort={transfer learning},
}

\newglossaryentry{learner}
{
  name=学习器,
  description={learner},
  sort={learner},
}

\newglossaryentry{multitask_learning}
{
  name=多任务学习,
  description={multitask learning},
  sort={multitask learning},
}

\newglossaryentry{domain_adaption}
{
  name=领域自适应,
  description={domain adaption}  ,
  sort={domain adaption}  ,
}

\newglossaryentry{concept_drift}
{
  name=概念漂移,
  description={concept drift},
  sort={concept drift},
}

\newglossaryentry{one_shot_learning}
{
  name=一次学习,
  description={one-shot learning},
  sort={one-shot learning},
}

\newglossaryentry{zero_shot_learning}
{
  name=零次学习,
  description={zero-shot learning},
  sort={zero-shot learning},
}

\newglossaryentry{zero_data_learning}
{
  name=零数据学习,
  description={zero-data learning},
  sort={zero-data learning},
}

\newglossaryentry{multimodal_learning}
{
  name=多模态学习,
  description={multimodal learning},
  sort={multimodal learning},
}

\newglossaryentry{generative_adversarial_networks}
{
  name=生成式对抗网络,
  description={generative adversarial network},
  sort={generative adversarial network},
  symbol={GAN}
}

\newglossaryentry{generative_adversarial_framework}
{
  name=生成式对抗框架,
  description={generative adversarial framework},
  sort={generative adversarial framework},
}

\newglossaryentry{GAN}
{
  name=生成式对抗网络,
  description={generative adversarial network},
  sort={generative adversarial network},
  symbol={GAN}
}

\newglossaryentry{feedforward_classifier}
{
  name=前馈分类器,
  description={feedforward classifier},
  sort={feedforward classifier},
}

\newglossaryentry{sum_product_network}
{
  name=和-积网络,
  description={sum-product network},
  sort={sum-product network},
  symbol={SPN}
}

\newglossaryentry{deep_circuit}
{
  name=深度回路,
  description={deep circuit},
  sort={deep circuit},
}

\newglossaryentry{shadow_circuit}
{
  name=浅度回路,
  description={shadow circuit},
  sort={shadow circuit},
}

\newglossaryentry{linear_classifier}
{
  name=线性分类器,
  description={linear classifier},
  sort={linear classifier},
}

\newglossaryentry{positive_phase}
{
  name=正相,
  description={positive phase},
  sort={positive phase},
}

\newglossaryentry{negative_phase}
{
  name=负相,
  description={negative phase},
  sort={negative phase},
}

\newglossaryentry{leibniz_rule}
{
  name=莱布尼兹法则,
  description={Leibniz's rule},
  sort={Leibniz's rule},
}

\newglossaryentry{lebesgue_integrable}
{
  name=勒贝格可积,
  description={Lebesgue-integrable},
  sort={Lebesgue-integrable},
}

\newglossaryentry{spurious_modes}
{
  name=虚假模态,
  description={spurious modes},
  sort={spurious modes},
}

\newglossaryentry{SML}
{
  name=随机最大似然,
  symbol={SML},
  description={stochastic maximum likelihood},
  sort={stochastic maximum likelihood},
}

\newglossaryentry{persistent_contrastive_divergence}
{
  name=持续性对比散度,
  description={persistent contrastive divergence},
  sort={persistent contrastive divergence},
  symbol={PCD}
}

\newglossaryentry{FPCD}
{
  name=快速持续性对比散度,
  symbol={FPCD},
  description={fast persistent contrastive divergence},
  sort={fast persistent contrastive divergence},
}

\newglossaryentry{pseudolikelihood}
{
  name=伪似然,
  description={pseudolikelihood},
  sort={pseudolikelihood},
}

\newglossaryentry{generalized_pseudolikelihood_estimator}
{
  name=广义伪似然估计,
  description={generalized pseudolikelihood estimator},
  sort={generalized pseudolikelihood estimator},
}

\newglossaryentry{generalized_pseudolikelihood}
{
  name=广义伪似然,
  description={generalized pseudolikelihood},
  sort={generalized pseudolikelihood},
}

\newglossaryentry{GSM}
{
  name=广义得分匹配,
  description={generalized score matching},
  sort={generalized score matching},
  symbol={GSM}
}

\newglossaryentry{ratio_matching}
{
  name=比率匹配,
  description={ratio matching},
  sort={ratio matching},
}

\newglossaryentry{NCE}
{
  name=噪声对比估计,
  description={noise-contrastive estimation},
  sort={noise-contrastive estimation},
  symbol={NCE}
}

\newglossaryentry{noise_distribution}
{
  name=噪声分布,
  description={noise distribution},
  sort={noise distribution},
}

\newglossaryentry{noise}
{
  name=噪声,
  description={noise},
  sort={noise},
}

\newglossaryentry{self_contrastive_estimation}
{
  name=自对比估计,
  description={self-contrastive estimation},
  sort={self-contrastive estimation},
}

\newglossaryentry{iid_chap18}
{
  name=独立同分布,
  description={independent identically distributed},
  sort={independent identically distributed},
  symbol={i.i.d.}
}

\newglossaryentry{AIS}
{
  name=退火重要采样,
  description={annealed importance sampling},
  sort={annealed importance sampling},
  symbol={AIS}
}

\newglossaryentry{bridge_sampling}
{
  name=桥式采样,
  description={bridge sampling},
  sort={bridge sampling},
}

\newglossaryentry{linked_importance_sampling}
{
  name=链接重要采样,
  description={linked importance sampling},
  sort={linked importance sampling},
}

\newglossaryentry{ASIC}
{
  name=专用集成电路,
  description={application-specific integrated circuit},
  sort={application-specific integrated circuit},
  symbol={ASIC}
}

\newglossaryentry{FPGA}
{
  name=现场可编程门阵列,
  description={field programmable gated array},
  sort={field programmable gated array},
  symbol={FPGA}
}

\newglossaryentry{scalar}
{
  name=标量,
  description={scalar},
  sort={scalar},
}

\newglossaryentry{vector}
{
  name=向量,
  description={vector},
  sort={vector},
}

\newglossaryentry{matrix}
{
  name=矩阵,
  description={matrix},
  sort={matrix},
}

\newglossaryentry{tensor}
{
  name=张量,
  description={tensor},
  sort={tensor},
}

\newglossaryentry{dot_product}
{
  name=点积,
  description={dot product},
  sort={dot product},
}

\newglossaryentry{inner_product}
{
  name=内积,
  description={inner product},
  sort={inner product}
}

\newglossaryentry{square}
{
  name=方阵,
  description={square},
  sort={square},
}

\newglossaryentry{singular}
{
  name=奇异的,
  description={singular},
  sort={singular},
}

\newglossaryentry{norm}
{
  name=范数,
  description={norm},
  sort={norm},
}

\newglossaryentry{triangle_inequality}
{
  name=三角不等式,
  description={triangle inequality},
  sort={triangle inequality},
}

\newglossaryentry{euclidean_norm}
{
  name=欧几里得范数,
  description={Euclidean norm},
  sort={Euclidean norm},
}

\newglossaryentry{max_norm}
{
  name=最大范数,
  description={max norm},
  sort={max norm},
}

\newglossaryentry{frobenius_norm}
{
  name=Frobenius 范数,
  description={Frobenius norm},
  sort={Frobenius norm},
}

\newglossaryentry{diagonal_matrix}
{
  name=对角矩阵,
  description={diagonal matrix},
  sort={diagonal matrix},
}

\newglossaryentry{symmetric}
{
  name=对称,
  description={symmetric},
  sort={symmetric},
}

\newglossaryentry{unit_vector}
{
  name=单位向量,
  description={unit vector},
  sort={unit vector},
}

\newglossaryentry{unit_norm}
{
  name=单位范数,
  description={unit norm},
  sort={unit norm},
}

\newglossaryentry{orthogonal}
{
  name=正交,
  description={orthogonal},
  sort={orthogonal},
}

\newglossaryentry{orthogonal_matrix}
{
  name=正交矩阵,
  description={orthogonal matrix},
  sort={orthogonal matrix},
}

\newglossaryentry{orthonormal}
{
  name=标准正交,
  description={orthonormal},
  sort={orthonormal},
}

\newglossaryentry{eigendecomposition}
{
  name=特征分解,
  description={eigendecomposition},
  sort={eigendecomposition},
}

\newglossaryentry{eigenvector}
{
  name=特征向量,
  description={eigenvector},
  sort={eigenvector},
}

\newglossaryentry{eigenvalue}
{
  name=特征值,
  description={eigenvalue},
  sort={eigenvalue},
}

\newglossaryentry{decompose}
{
  name=分解,
  description={decompose},
  sort={decompose},
}

\newglossaryentry{P_D}
{
  name=正定,
  description={positive definite},
  sort={positive definite},
}

\newglossaryentry{ND}
{
  name=负定,
  description={negative definite},
  sort={negative definite},
}

\newglossaryentry{NSD}
{
  name=半负定,
  description={negative semidefinite},
  sort={negative semidefinite},
}

\newglossaryentry{P_SD}
{
  name=半正定,
  description={positive semidefinite},
  sort={positive semidefinite},
}

\newglossaryentry{SVD}
{
  name=奇异值分解,
  description={singular value decomposition},
  sort={singular value decomposition},
  symbol={SVD}
}

\newglossaryentry{Svalue}
{
  name=奇异值,
  description={singular value},
  sort={singular value},
}

\newglossaryentry{Svector}
{
  name=奇异向量,
  description={singular vector},
  sort={singular vector},
}

\newglossaryentry{left_Svector}
{
  name=左奇异向量,
  description={left singular vector},
  sort={left singular vector},
}

\newglossaryentry{right_Svector}
{
  name=右奇异向量,
  description={right singular vector},
  sort={right singular vector},
}

\newglossaryentry{left_Evector}
{
  name=左特征向量,
  description={left eigenvector},
  sort={left eigenvector},
}

\newglossaryentry{right_Evector}
{
  name=右特征向量,
  description={right eigenvector},
  sort={right eigenvector},
}

\newglossaryentry{Moore}
{
  name=Moore-Penrose 伪逆,
  description={Moore-Penrose pseudoinverse},
  sort={Moore-Penrose pseudoinverse},
}

\newglossaryentry{identity_matrix}
{
  name=单位矩阵,
  description={identity matrix},
  sort={identity matrix},
}

\newglossaryentry{matrix_inverse}
{
  name=矩阵逆,
  description={matrix inversion},
  sort={matrix inversion},
}

\newglossaryentry{origin}
{
  name=原点,
  description={origin},
  sort={origin},
}

\newglossaryentry{linear_combination}
{
  name=线性组合,
  description={linear combination},
  sort={linear combination},
}

\newglossaryentry{column_space}
{
  name=列空间,
  description={column space},
  sort={column space},
}

\newglossaryentry{range}
{
  name=值域,
  description={range},
  sort={range},
}

\newglossaryentry{linear_depend}
{
  name=线性相关,
  description={linear dependency},
  sort={linear dependency},
}

\newglossaryentry{linear_dependence}
{
  name=线性相关,
  description={linear dependence},
  sort={linear dependence}
}

\newglossaryentry{linearly_independent}
{
  name=线性无关,
  description={linearly independent},
  sort={linearly independent}
}

\newglossaryentry{column}
{
  name=列,
  description={column},
  sort={column},
}

\newglossaryentry{row}
{
  name=行,
  description={row},
  sort={row},
}

\newglossaryentry{span}
{
  name=生成子空间,
  description={span},
  sort={span},
}

\newglossaryentry{SLT}
{
  name=统计学习理论,
  description={statistical learning theory},
  sort={statistical learning theory},
}

\newglossaryentry{DGP}
{
  name=数据生成过程,
  description={data generating process},
  sort={data generating process},
}

\newglossaryentry{iid}
{
  name=独立同分布假设,
  description={i.i.d. assumption},
  sort={i.i.d. assumption},
}

\newglossaryentry{id}
{
  name=同分布的,
  description={identically distributed},
  sort={identically distributed},
}

\newglossaryentry{DGD}
{
  name=数据生成分布,
  description={data generating distribution},
  sort={data generating distribution},
}

\newglossaryentry{large_learning_step}
{
  name=大学习步骤,
  description={large learning step},
  sort={large learning step},
}

\newglossaryentry{variable_elimination}
{
  name=变量消去,
  description={variable elimination},
  sort={variable elimination},
}

\newglossaryentry{OR}
{
  name=奥卡姆剃刀,
  description={Occam's razor},
  sort={Occam's razor},
}

\newglossaryentry{VC}
{
  name=Vapnik-Chervonenkis维度,
  description={Vapnik-Chervonenkis dimension},
  sort={Vapnik-Chervonenkis dimension},
  symbol={VC}
}

\newglossaryentry{unsupervised_learning_algorithm}
{
  name=无监督学习算法,
  description={unsupervised learning algorithm},
  sort={unsupervised learning algorithm},
}

\newglossaryentry{supervised_learning_algorithm}
{
  name=监督学习算法,
  description={supervised learning algorithm},
  sort={supervised learning algorithm},
}

\newglossaryentry{word_embedding}
{
  name=词嵌入,
  description={word embedding},
  sort={word embedding},
}

\newglossaryentry{shortlist}
{
  name=短列表,
  description={shortlist},
  sort={shortlist},
}

\newglossaryentry{NMT}
{
  name=神经机器翻译,
  description={Neural Machine Translation},
  sort={Neural Machine Translation},
  symbol={NMT}
}

\newglossaryentry{machine_translation}
{
  name=机器翻译,
  description={machine translation},
  sort={machine translation}
}

\newglossaryentry{recommender_system}
{
  name=推荐系统,
  description={recommender system},
  sort={recommender system},
}

\newglossaryentry{proposal_distribution}
{
  name=提议分布,
  description={proposal distribution},
  sort={proposal distribution},
}

\newglossaryentry{bag_of_words}
{
  name=词袋,
  description={bag of words},
  sort={bag of words},
}

\newglossaryentry{collaborative_filtering}
{
  name=协同过滤,
  description={collaborative filtering},
  sort={collaborative filtering},
}

\newglossaryentry{exploration}
{
  name=探索,
  description={exploration},
  sort={exploration},
}

\newglossaryentry{exploitation}
{
  name=利用,
  description={exploitation},
  sort={exploitation},
}

\newglossaryentry{bandit}
{
  name=bandit ,
  description={bandit},
  sort={bandit},
}

\newglossaryentry{contextual_bandit}
{
  name=contextual bandit ,
  description={contextual bandit},
  sort={contextual bandit},
}

\newglossaryentry{policy}
{
  name=策略,
  description={policy},
  sort={policy},
}

\newglossaryentry{relation}
{
  name=关系,
  description={relation},
  sort={relation},
}

\newglossaryentry{binary_relation}
{
  name=二元关系,
  description={binary relation},
  sort={binary relation},
}

\newglossaryentry{attribute}
{
  name=属性,
  description={attribute},
  sort={attribute},
}

\newglossaryentry{relational_database}
{
  name=关系型数据库,
  description={relational database},
  sort={relational database},
}

\newglossaryentry{link_prediction}
{
  name=链接预测,
  description={link prediction},
  sort={link prediction},
}

\newglossaryentry{word_sense_disambiguation}
{
  name=词义消歧,
  description={word-sense disambiguation},
  sort={word-sense disambiguation},
}

\newglossaryentry{error_metric}
{
  name=误差度量,
  description={error metric},
  sort={error metric},
}

\newglossaryentry{performance_metrics}
{
  name=性能度量,
  description={performance metrics},
  sort={performance metrics},
}

\newglossaryentry{transcription_system}
{
  name=转录系统,
  description={transcription system},
  sort={transcription system},
}

\newglossaryentry{BFGS}
{
  name=BFGS,
  description={BFGS},
  sort={BFGS},
}

\newglossaryentry{LBFGS}
{
  name=L-BFGS,
  description={L-BFGS},
  sort={L-BFGS},
}

\newglossaryentry{CG}
{
  name=共轭梯度,
  description={conjugate gradient},
  sort={conjugate gradient},
  symbol={CG}
}

\newglossaryentry{nonlinear_CG}
{
  name=非线性共轭梯度,
  description={nonlinear conjugate gradients},
  sort={nonlinear conjugate gradients},
}

\newglossaryentry{bayesian_inference}
{
  name=贝叶斯推断,
  description={Bayesian inference},
  sort={Bayesian inference},
}

\newglossaryentry{online_learning}
{
  name=在线学习,
  description={online learning},
  sort={online learning},
}

\newglossaryentry{layer_wise_pretraining}
{
  name=逐层预训练,
  description={layer-wise pretraining},
  sort={layer-wise pretraining},
}

\newglossaryentry{MPDBM}
{
  name=多预测深度玻尔兹曼机,
  description={multi-prediction deep Boltzmann machine},
  sort={multi-prediction deep Boltzmann machine},
  symbol={MP-DBM}
}

\newglossaryentry{GBRBM}
{
  name=Gaussian-Bernoulli RBM,
  description={Gaussian-Bernoulli RBM},
  sort={Gaussian-Bernoulli RBM},
  symbol={GB-RBM}
}

\newglossaryentry{gaussian_rbm}
{
  name=高斯RBM,
  description={Gaussian RBM},
  sort={Gaussian RBM}
}

\newglossaryentry{mcrbm}
{
  name=均值和协方差RBM,
  description={mean and covariance RBM},
  sort={mean and covariance RBM},
  symbol={mcRBM},
}

\newglossaryentry{mcrbm2}
{
  name=均值-协方差RBM,
  description={mean-covariance restricted Boltzmann machine},
  sort={mean-covariance restricted Boltzmann machine},
  symbol={mcRBM2},
}

\newglossaryentry{crbm}
{
  name=协方差RBM,
  description={covariance RBM},
  sort={covariance RBM},
  symbol={cRBM},
}

\newglossaryentry{mpot}
{
  name=学生$t$分布均值乘积,
  description={mean product of Student t-distribution},
  sort={mean product of Student t-distribution},
  symbol={mPoT},
}

\newglossaryentry{ssrbm}
{
  name=尖峰和平板RBM,
  description={spike and slab RBM},
  sort={spike and slab RBM},
  symbol={ssRBM},
}

\newglossaryentry{gamma_distribution}
{
  name=Gamma分布,
  description={Gamma distribution},
  sort={Gamma distribution},
}

\newglossaryentry{convolutional_bm}
{
  name=卷积玻尔兹曼机 ,
  description={convolutional Boltzmann machine},
  sort={convolutional Boltzmann machine},
}

\newglossaryentry{reparametrization_trick}
{
  name=重参数化技巧,
  description={reparametrization trick},
  sort={reparametrization trick},
}

\newglossaryentry{reparametrization}
{
  name=重参数化,
  description={reparametrization},
  sort={reparametrization},
}

\newglossaryentry{variance_reduction}
{
  name=方差减小,
  description={variance reduction},
  sort={variance reduction},
}

\newglossaryentry{sigmoid_bn}
{
  name=sigmoid信念网络,
  description={sigmoid Belief Network},
  sort={sigmoid Belief Network},
}

\newglossaryentry{auto_regressive_network}
{
  name=自回归网络,
  description={auto-regressive network},
  sort={auto-regressive network}
}

\newglossaryentry{generator_network}
{
  name=生成器网络,
  description={generator network},
  sort={generator network}
}

\newglossaryentry{discriminator_network}
{
  name=判别器网络,
  description={discriminator network},
  sort={discriminator network},
}

\newglossaryentry{generative_moment_matching_network}
{
  name=生成矩匹配网络,
  description={generative moment matching network},
  sort={generative moment matching network},
}

\newglossaryentry{moment_matching}
{
  name=矩匹配,
  description={moment matching},
  sort={moment matching},
}

\newglossaryentry{moment}
{
  name=矩,
  description={moment},
  sort={moment},
}

\newglossaryentry{MMD}
{
  name=最大平均偏差,
  description={maximum mean discrepancy},
  sort={maximum mean discrepancy},
  symbol={MMD}
}

\newglossaryentry{linear_auto_regressive_network}
{
  name=线性自回归网络,
  description={linear auto-regressive network},
  sort={linear auto-regressive network}
}

\newglossaryentry{neural_auto_regressive_network}
{
  name=神经自回归网络,
  description={neural auto-regressive network},
  sort={neural auto-regressive network}
}

\newglossaryentry{NADE}
{
  name=神经自回归密度估计器,
  description={neural auto-regressive density estimator},
  sort={neural auto-regressive density estimator},
  symbol={NADE}
}

\newglossaryentry{detailed_balance}
{
  name=细致平衡,
  description={detailed balance},
  sort={detailed balance},
}

\newglossaryentry{ABC}
{
  name=近似贝叶斯计算,
  description={approximate Bayesian computation},
  sort={approximate Bayesian computationA},
  symbol={ABC}
}

\newglossaryentry{visible_layer}
{
  name=可见层,
  description={visible layer},
  sort={visible layer},
}

\newglossaryentry{infinite}
{
  name=无限,
  description={infinite},
  sort={infinite},
}

\newglossaryentry{deep_model}
{
  name=深度模型,
  description={deep model},
  sort={deep model},
}

\newglossaryentry{deep_network}
{
  name=深度网络,
  description={deep network},
  sort={deep network},
}

\newglossaryentry{tolerance}
{
  name=容差,
  description={tolerance},
  sort={tolerance},
}

\newglossaryentry{learning_rate}
{
  name=学习率,
  description={learning rate},
  sort={learning rate},
}

\newglossaryentry{ss}
{
  name=尖峰和平板,
  description={spike and slab},
  sort={spike and slab},
}

\newglossaryentry{context_specific_independence}
{
  name=特定环境下的独立,
  description={context-specific independences},
  sort={context-specific independences},
}

\newglossaryentry{coparent}
{
  name=共父,
  description={coparent},
  sort={coparent},
}

\newglossaryentry{srbm}
{
  name=半受限波尔兹曼机,
  description={semi-restricted Boltzmann Machine},
  sort={semi-restricted Boltzmann Machine},
}

\newglossaryentry{underfit_regime}
{
  name=欠拟合机制,
  description={underfitting regime},
  sort={underfitting regime},
}

\newglossaryentry{overfit_regime}
{
  name=过拟合机制,
  description={overfitting regime},
  sort={overfitting regime},
}

\newglossaryentry{optimal_capacity}
{
  name=最佳容量,
  description={optimal capacity},
  sort={optimal capacity},
}

\newglossaryentry{error_bar}
{
  name=误差条,
  description={error bar},
  sort={error bar},
}

\newglossaryentry{vstructure}
{
  name=V-结构,
  description={V-structure},
  sort={V-structure},
}

\newglossaryentry{collider}
{
  name=碰撞情况,
  description={the collider case},
  sort={the collider case},
}

\newglossaryentry{epochs}
{
  name=轮数,
  description={epochs},
  sort={epochs},
}

\newglossaryentry{epoch}
{
  name=轮,
  description={epoch},
  sort={epoch},
}

\newglossaryentry{logarithmic_scale}
{
  name=对数尺度,
  description={logarithmic scale},
  sort={logarithmic scale}
}

\newglossaryentry{random_search}
{
  name=随机搜索,
  description={random search},
  sort={random search}
}

\newglossaryentry{closed_form_solution}
{
  name=闭式解,
  description={closed form solution},
  sort={closed form solution}
}

\newglossaryentry{object_recognition}
{
  name=对象识别,
  description={object recognition},
  sort={object recognition}
}

\newglossaryentry{piecewise}
{
  name=分段,
  description={piecewise},
  sort={piecewise},
}

\newglossaryentry{alternative_splicing_dataset}
{
  name=选择性剪接数据集,
  description={alternative splicing dataset},
  sort={alternative splicing dataset},
}

\newglossaryentry{hamming_distance}
{
  name=汉明距离,
  description={Hamming distance},
  sort={Hamming distance}
}

\newglossaryentry{visible_variable}
{
  name=可见变量,
  description={visible variable},
  sort={visible variable},
}

\newglossaryentry{approxi_inference}
{
  name=近似推断,
  description={approximate inference},
  sort={approximate inference},
}

\newglossaryentry{exact_inference}
{
  name=精确推断,
  description={exact inference},
  sort={exact inference},
}

\newglossaryentry{latent}
{
  name=潜在,
  description={latent},
  sort={latent},
}

\newglossaryentry{latent_layer}
{
  name=潜层,
  description={latent layer},
  sort={latent layer},
}

\newglossaryentry{knowledge_graph}
{
  name=知识图谱,
  description={knowledge graph},
  sort={knowledge graph},
}

\newglossaryentry{factors_of_variation}
{
  name=变差因素,
  description={factors of variation},
  sort={factors of variation},
}

\newglossaryentry{isomap}
{
  name=Isomap,
  description={Isomap},
  sort={isomap},
}


\newcommand{\documentTitle}{
    \texorpdfstring{\kaishu{细胞生物学\\分子生物学\\发育生物学\\生殖生物学\\基础知识}}
    {细胞生物学、分子生物学、发育生物学与生殖生物学基础知识}
}

\hypersetup{
	%hidelinks,
	pdftitle={\documentTitle},
	pdfauthor={张洋},
	pdfsubject={细胞生物学, 分子生物学, 发育生物学, 生殖生物学},
	pdfkeywords={细胞生物学, 分子生物学, 发育生物学, 生殖生物学},
	pdfstartview=FitH,
	pdfproducer = {张洋},
	pdfcreator = {张洋},
	pdfcopyright = {张洋},
	pdflicenseurl = {https://github.com/hiatcg}
}

\begin{document}

\title{\documentTitle}
\author{\kaishu{张洋}}
\date{\kaishu{\today}}
\maketitle

\tableofcontents

\part{细胞生物学}
\chapter{细胞生物学绪论}

\section{细胞生物学研究的内容与现状}

\subsection{细胞生物学是显带生命科学的重要基础学科}

\firstgls{cellbiology}细胞生物学是研究细胞基本生命活动规律的科学,它主要从不同层次(显微、亚显微与分子水平)上研究细胞结构与功能,细胞增殖、分化、衰老与死亡,细胞信号转导,细胞基因的表达与调控,细胞的起源与进化等。分子细胞生物学是当今细胞生物学研究的重点,细胞工程及与之相关的组织工程和修复医学是21世纪生物工程发展的重要组成部分。可以看到,细胞的结构与基本生命活动的研究已经越来越深入,并已经成为21世纪初生命科学研究的重要领域之一。

生命是多层次、非线性、多侧面的复杂结构体系,而细胞是生命体结构与生命活动的基本单位,有了细胞才有完整的生命活动。细胞的研究是生命科学的基础,也是现代生命科学发展的重要支柱。早在1925年,生物学大师Edmund Beecher Wilson就提出:“一切生命的关键问题都要到细胞中去寻找”(The key to every biological problem must finally be sought in the cell)。重温这句名言,至今仍感其内涵深刻。

生物的生殖发育、遗传、神经(脑)活动等重大生命现象的研究都要以细胞为基础。多细胞生物的生长发育是依靠细胞增殖、\firstgls{cell_differentiation} 与\firstgls{cell_apoptosis}来实现的。人脑的活动是靠$ 10^{{12}} $个细胞及其相互协调而进行的。一切疾病发病机制也要以细胞病变的研究为基础。以基因工程和蛋白质工程为核心的显带生物技术主要是通过以细胞操作为基础而进行的。因此,细胞生物学与农业、医学、生物技术的发展有密不可分的关系,它将在解决人类面临的重大问题、促进经济和社会发展中发挥重要的基础作用。

在我国基础科学发展规划中,细胞生物学、分子生物学、神经生物学和生态学被并列为生命科学的四大基础学科,反映了显带生命科学的发展趋势。

由于分子生物学概念、方法和技术的引入,细胞生物学在近些年取得了突破性的进展,产生了许多新的生长点,并逐渐形成新的概念与新的领域。很多科学家认为,在21世纪,细胞生物学将继续徐梦发展,因为它是解释生命奥秘不可缺少的``主角''。细胞生物学自身优势多层次的复杂结构体系,它是物质(结构)、信息与能量相互``辉映''的综合体。它的很多基本生命活动过程是如何有序而被自动调控的还不是很清楚,生物大分子如何逐级有序地组装成行使生命活动的细胞基本结构体系页相当朦胧。因此,细胞的基础研究势必影响21世纪生命科学的整体发展速度。


可以概况地说,细胞生物学是应用显带物理学与化学的技术成就和分子生物学的概念与方法,以细胞作为生命活动的基本单位为出发点,探索生命活动规律的学科,其核心问题是将遗传与发育在细胞水平上结合起来。还应指出,细胞生物学是一门迅速发展中的新兴学科,其研究内容与范畴往往与生命科学的其他学科交错在一起,甚至目前很难为细胞生物学划出一个明确的范围。

\subsection{细胞生物学的主要研究内容}

细胞生物学研究与教学内容一般可分为细胞结构与功能和细胞重要生命活动两大基本部分,但它们又很难割裂开。从20世纪60年代开始,细胞超微结构研究积累的大量资料,大大充实与拓宽了细胞结构与功能的知识范畴,在细胞生物学教科书中,与细胞结构和功能相关的知识所占比例较多。从70年代中期开始,由于分子生物学概念、内容与方法的引入,细胞生物学面貌发生了深刻的变化,不仅使细胞结构和功能的研究更深入,对细胞重大生命活动规律及其调控机制的研究也取得了巨大进展,极大地丰富与改变了细胞生物学的知识结构。因此,现代细胞生物学教科书中细胞重要生命活动的知识所占比例也越来越大。

当前细胞生物学研究内容大致归纳为以下领域:

\subsubsection{细胞核、染色体以及基因表达的研究}

细胞核是遗传物质DNA储存的场所,也是遗传信息进行转录的场所。染色质与染色体是遗传物质的载体,核仁是转录rRNA以及组装核糖体亚单位的具体场所,核孔复合体是物质之间交换与信息交流的结构。细胞核与染色体的研究历来是经典细胞学的重点,也是细胞遗传学的核心部分。而现代细胞生物学的核心课题之一就是研究染色体结构动态变化与基因表达及其调控的关系,它是目前细胞生物学、遗传学与发育生物学在细胞水平与分子水平上相结合的最活跃的热门课题之一,也是后基因组时代生命科学主要的研究内容之一。

\subsubsection{生物膜与细胞器的研究}

生物膜是细胞结构的重要基础,细胞质膜出现之后,才有了真正意义上的生命体。大部分细胞器(包括细胞核)也是以生物膜为基础构建的。近年来,生物膜研究的主要内容是\underline{膜的结构模型}、\underline{物质的跨膜运输}及\underline{信息跨膜传递的机制}。磷脂双分子层与膜蛋白的相互关系是研究生物膜结构与功能的重要内容。几年来,在膜的识别与受体效应、蛋白质分子跨膜运输与定向分选等方面取得了巨大进展。

细胞器的研究历来是认识细胞结构与功能的重要组成部分。线粒体与叶绿体结构与换能机制的研究已很深入,线粒体DNA与叶绿体DNA的发现及其半自主性的研究是人们对这两种细胞器又有了新的认识。近年来对内质网、高尔基体与溶酶体功能的研究也增添了许多新的知识。核糖体RNA能催化肽链的合成的发现,大大加深了人们对核糖体功能的认识,同时也显示了核糖体在生命起源与进化中的重要地位。

\subsubsection{细胞骨架体系的研究}

细胞骨架体系的研究在细胞生物学中是一个重要的研究领域。广义的细胞骨架概念应该包括\underline{细胞质骨架}与\underline{核骨架}两大部分。

细胞骨架体系的研究越来越受到重视,因为细胞骨架在维持细胞形态与保持细胞内部结构的合理布局中起主要作用。细胞骨架与一系列重要生命活动,诸如细胞内大分子的运输与细胞器的运动、细胞信息的传递、基因表达与大分子加工等均有密切关系。今年来,细胞核骨架的研究是颇引人关注的领域之一。核骨架包括\underline{核基质}、\underline{核纤层}和\underline{核孔复合体},不仅参与核染色体的构建,而且与基因表达关系密切。

\subsubsection{细胞增殖及其调控}

一切动植物的生长及发育都是通过细胞的增殖与分化来实现的。研究细胞增殖的基本规律及其调控机制,不仅是控制生物生长及发育的基础,而且是研究癌变发生及逆转的重要途径。目前国际上研究细胞增殖的调控主要从两方面进行:

\begin{enumerate}
	\item 从环境或有机体中寻找控制细胞增殖的因子,以及阐明它们的作用机制;各种生长因子的发现及其作用机制的揭示是这一领域中重要的进展。
	\item 寻找控制细胞增殖的关键性基因,并通过调节基因产物来控制细胞的增殖。
\end{enumerate}

细胞的癌基因与抑癌基因及其表达产物均与细胞增殖有关。

\subsubsection{细胞分化及其调控}

细胞分化是生物发育的基础。近年,细胞分化的研究已越来越显示出其重要性,也是细胞生物学、发育生物学与遗传学的重要汇合点。近年生物科学的发展,尤其是分子生物学技术的建立,已为细胞分化机制的研究奠定了良好的基础,这也是近年发育生物学蓬勃发展的重要原因。

一个受精卵通过分裂与分化如何发育为复杂的有机体,是生命科学中引人入胜的课题之一。利用哺乳动物的体细胞进行动物克隆,进一步揭示了细胞的``全能性'',使人们认识到可以控制细胞的分化,而且可以将已分化的细胞进行``去分化'',并使其分裂与再分化,克隆出新的个体,这为控制生物的生长发育展示了诱人的前景。

目前认为\underline{细胞分化的本质是细胞内基因选择性表达特异功能蛋白质的过程}。近年,细胞分化及其调控机制的研究主要集中在\underline{编码特异蛋白质的基因的选择性表达规律及其调控}方面。围绕干细胞的研究及其再组织工程中的应用前景,极大地促进了细胞分化领域的研究。

\subsubsection{细胞的衰老与凋亡}

细胞衰老的研究是研究人与动植物寿命的基础,细胞总体的衰老导致个体的衰老,但细胞的衰老与有机物的衰老又是不同的概念。不少科学家认为细胞衰老的研究将成为21世纪的热门课题之一。

目前多数科学家是用细胞体外培养的方法来研究细胞衰老的规律。大量实验说明,动物二倍体细胞在体外分裂与传代的次数是有限的,从而推测体内细胞的寿命受分裂次数的限制,细胞的衰老是必然的规律。人们总是希望通过细胞衰老因素与因子的研究来延长细胞的寿命。近年人们力图寻找细胞中的衰老基因及相关信号转导途径。

细胞凋亡的研究是生命科学中发展起来的重要的新兴领域之一。细胞凋亡是由一系列基因控制并受复杂的信号调节的细胞自然死亡现象,细胞凋亡可能是生物正常生理发育与病理过程中的重要平衡因素。

\subsubsection{细胞的起源与进化}

细胞的起源与进化的研究是重要的理论问题,也是难度很大的研究课题,我们应该十分尊重先驱科学家在这一领域所取得的成果。

根据古微生物的证据,原始细胞大约在35亿年,或更早到38亿年前就在地球上出现。生命物质只有当其以细胞形式出现时才能稳定地在地球上存在下去,但是这个进程已成为历史,在现在实验条件下模拟30多亿年前的地球环境,研究细胞的起源过程几乎是不大可能的。目前有关细胞起源学说在很大程度上是推理性的,还需要补充更多的事实证据。

关于真核细胞的起源与进化已取得了较多的成果,近年由于分子生物学方法引入生物进化的研究,为细胞进化与细胞器进化的研究增添了很多新的内容。

\subsubsection{细胞工程}

细胞工程是细胞生物学与遗传学的交叉领域,这种改造细胞的技术是生物工程技术的重要组成部分。它不仅对工农业生成和医学实践有重要意义,而且也是认识细胞生命活动规律的一种重要途径与手段。

细胞工程能用人工方法使不同种细胞的基因或基因组重组到杂交细胞中,或者使基因与基因组由一种细胞转移到另一种细胞中,并使越过种的障碍的基因成为可能。由此人们开始探索人工创造新的遗传性状的细胞。动植物体细胞的杂交试验一直使细胞工程中最活跃的领域。通过动物体细胞杂交而获得单克隆抗体技术的建立,是细胞工程中最富成果性的工作范例。近年,在世界范围内兴起的用哺乳动物体细胞克隆而获得无性繁殖胚胎与个体,是细胞工程最引人注意的进展之一。人胚胎干细胞系的建立,为细胞工程中的一个新的研究领域---干细胞工程的研究与应用打下基础。

还应该说明,当前细胞生物学研究的范畴远不止上述的内容,其他如细胞信号转导已成为细胞生物学最热门的领域之一,一切重大生命活动都涉及到细胞信号转导与调控。此外,细胞外基质、细胞社会学与细胞免疫学等研究,近年也由较快的发展。

\subsection{当前细胞生物学研究的总趋势与重点领域}

细胞生物学与分子生物学相互渗透与交融是总的发展趋势。无论是细胞结构与功能的深入研究,还是对细胞重大生命活动规律的探索,都需要用分子生物学的新概念与新方法,在分子水平上进行研究。换句话说,细胞分子生物学或分子细胞生物学将是今后相当一段时间的主流学科方向。

\subsubsection{当前细胞生物学研究中的几大基本问题}

当前细胞生物学研究的课题归纳起来包括以下几个根本性的问题:

\paragraph{基因组在细胞内是如何在时间与空间上有序表达的?}
如果将一些基因放在试管内,只要条件都满足,它们都可以完成表达。但是在细胞内环境中,其表达程度都将受到严格的调节与控制,它们能否表达?这就是细胞作为结构与生命活动的基本单位的奥妙所在。细胞结构与生命活动的有序性是十分复杂的,是非线性调控过程。这种调控网络可能是迄今任何一台计算机都无法比拟的。

\paragraph{基因表达的产物,主要是结构蛋白与核酸、脂质、多糖及其复合物,如何逐级组装成能行使生命活动的基本结构体系及各种细胞器?}
这种自组装过程的调控程序与调控机制是什么?这又是当前极富挑战性的领域。我们认为它应是新兴的结构生物学的重要组成部分,生命活动很多本质问题都能在这里找到答案。

\paragraph{基因表达的产物,主要是大量活性因子与信号分子,是如何调节诸如细胞的增殖、分化、衰老、与凋亡等细胞最重要的生命活动过程的?}
这些研究工作现在方兴未艾。

\subsubsection{当前细胞基本生命活动研究的若干重大课题}

\paragraph{染色体DNA与蛋白质相互作用关系---主要是非组蛋白对基因组的作用。}
近年来染色体(质)的概念有很大的改变。可以认为染色体是一条巨大的DNA分子与其结合蛋白形成的动态复合结构。其存在状态可以分为细胞间期染色质与分裂期染色体。两者在细胞周期中交替与互相转换,主要是由DNA与蛋白质动态结合关系所决定的。DNA是遗传密码的分子载体,蛋白质是调节DNA复制与转录活性并参与染色体构建的元件。染色体DNA结合蛋白可分为组蛋白与非组蛋白。核小体的发现及其结构的阐明,对5种组蛋白的作用已基本清楚。大量与种类繁多的非组蛋白与DNA的相互关系是研究的主要问题,特别是它们对DNA的复制、基因组有序表达与染色体高层次构建及动态结构变化的作用是核心问题。同时,DNA分子与组蛋白的修饰在基因表达调控中的作用,即表观遗传学的研究也越来越受到人们的关注。

\paragraph{细胞增殖、分化、凋亡(程序性死亡)的相互关系及其调控}
细胞增殖、分化、凋亡与衰老是细胞最重要的生命活动现象,它们既相互联系又相互制约,并收到一系列基因的调控。一切动植物生长发育都是依靠细胞增殖、分化与凋亡来实现的。细胞癌变往往是由于细胞增殖、分化或凋亡失控所致。受精卵如何有条不紊地发育为复杂的成体,一直是人们企图揭开的奥秘。由于近年来遗传与发育在分子水平上的统一并取得了巨大进展,为揭示这一问题提供了光明前景。而细胞分化及其调控机制的研究,将是解决这一重大问题的基础。

细胞凋亡的研究是近年生命科学中发展起来的热点课题。细胞凋亡是受外环境多种信号刺激,由基因调控的``主动自然''死亡,在细胞增殖、分化及多种病理过程中具有重要作用。细胞凋亡可能是发育进程的重要平衡因素。例如,哺乳动物脑的发育过程中有$20 \%\sim 50\% $的脑神经元发生凋亡;泌乳细胞在婴儿断乳后很快凋亡,代之以脂肪细胞。应该凋亡的细胞不凋亡也是致癌的重要因素。

人们已经通过促进细胞凋亡来为癌症治疗提供依据及手段。植物细胞与单细胞生物有无类似动物细胞的凋亡现象也是我们要探索的问题。不论是细胞增殖与分化还是细胞凋亡的研究,核心问题依然是其调控机制的研究。

\paragraph{细胞信号转导的研究}
细胞信号转导的研究与生命科学中的许多重要问题密切相关,它已成为了了解错综复杂的生命现象的主要切入点,并成为生命科学多领域、多层次联系的纽带。细胞信号转导的研究对了解生命本质与细胞基本生命活动有重要理论意义。

当前细胞信号转导的基本内容是:\textbf{细胞间信号传递}:重点是信号分子的结构与功能、信号分子与受体相互作用机制;\textbf{受体与信号跨膜转导}:G蛋白的发现与较深入的研究使受体转换器的概念与结构功能更具体化;\textbf{细胞内信号传递途径}:蛋白磷酸化与去磷酸化可能使信号转导的关键。把信号转导与基因表达调控联系起来的研究可能使最吸引人的课题之一。

应该指出,细胞信号转导使通过复杂的网络系统实现的,呈现高度的非线性、多侧面的特点。细胞信号转导的基础使蛋白质与蛋白质的相互作用,特别使功能蛋白质的相互做哟用。信号转导通路可能不是单一的或简单的信息叠加。因此,研究细胞信号转导通路质检的相互作用将成为该课题的热点。

\paragraph{细胞结构体系的组装}

生物大分子如何逐级组装并最终形成赖以进行生命活动的细胞结构体系?这是当前生命科学面临的最基本问题之一。这一课题不仅是揭示细胞生命本质与奥妙的重要组成部分,而且也是深入研究细胞结构与功能的新的切入点。大分子嗯那个逐级自组装成能行使生命功能的基本细胞体系,本身就是引人入胜的重大问题。

在分子水平上研究与解释生命现象的本质固然十分重要,但细胞结构体系的自组装与去组装动态过程不能简单归属于分子水平的活动,尚需在更高层次上进行探索与研究。生命活动的基础是细胞内高度有序的、动态的复合结构体系,在亚细胞水平我们可以将细胞结构大致归纳为几大基本结构体系:\underline{由蛋白质与核酸构建的遗传信息结构体系}(包括染色体、核仁、核糖体等)、\underline{由蛋白质与脂质构成的膜结构体系}(包括细胞膜、核膜以及其他各种细胞器膜)、\underline{由蛋白质与蛋白质构成的细胞骨架结构体系}。由这些基本结构体系形成的细胞器与各级结构在细胞内进行组装、去组装、重组装与重建(改建)等过程。有不少细胞结构可以在体外进行组装与重组装。体外组装实验模式的建立,为研究细胞结构体系组装机制提供了很好的基础与手段。近年来,在这一领域进展最快的是细胞核体外组装(重建)、细胞骨架组装、人工染色体与人工膜组装等的研究。\\


最后必须指出,上述诸领域与课题不能概括细胞基本生命活动的所有重大问题。还有很多方面,诸如蛋白质合成、分选与跨膜定向运输、真核细胞起源与进化等,也对细胞基本生命活动规律的认识很重要。

美国国立卫生研究院(NIH,National Institutes of Health)在1988年底发表的一份题为《什么是当前科研领域热门话题?》(``What is popular in research today?'')的调查报告中指出,目前全球研究最热门的几种疾病:癌症(Cancer)、心血管病(Cardiovascular Diseases)、艾滋病(AIDS)和肝炎(Hepatitis)等传染病(Infectious Diseases)。研究最热门的方向是:细胞周期调控(Cell Cycle Control)、细胞凋亡(Cell Apoptosis)、细胞衰老(Cell Senescence)、细胞信号转导(Signal Transduction)和DNA的损伤与修复(DNA Damage and Repair)。2004年,NIH提出的生物医学发展规划中,更是把细胞信号转导网络作为今后10年研究的主要内容。

由于细胞是生命活动的基本单位,又是生命的缩影,它不仅体现了生命的多样性和统一性,更体现了生命的复杂性。显然,我们对细胞重大生命活动的认识,必须建立在创新思维的基础上,着力探索它们彼此之间的关系,以及这些重大生命活动与求解人类所面临的诸多问题之间的关系。为此,我们给出一个有关细胞重大生命活动及其相互关系的图示,可能有助于思考和研究这些问题。

\begin{center}
\pgfdeclarelayer{backwrapping}
\pgfdeclarelayer{frontwrapping}
\pgfsetlayers{backwrapping,main,frontwrapping}

\tikzset{
    wrapping/.style={
        draw=black!60, 
        line cap=round, 
        line join=round, 
        ultra thick},
    nucleosome/.style={
        fill=red!40, 
        fill opacity=.9, 
        draw=none},
    top cylinder/.style={
        fill=red!60, 
        fill opacity=.9
    }
}

\begin{tikzpicture}

\foreach \q [remember=\q as \p] in {1, 2, 3, 4}{
    \begin{scope}[shift={(\q*3,0)}, rotate=30]
        \path [nucleosome]
            (0,1)
            arc (90:270:0.375 and 1) -- (1.25,-1)
            arc (270:90:0.375 and 1) -- cycle;
        \path [top cylinder] 
            (1.625, 0) arc (0:360:0.375 and 1) -- cycle;
          
          \begin{scope}[shift={(0.25,0)}]
            \begin{pgfonlayer}{backwrapping}
                \draw [wrapping] 
                    (0.25, -1.125) 
                    \foreach \i in {180,185,...,360}
                        { -- (\i/720+0.375*sin -\i, 1.125*cos \i)};
                    \draw [wrapping] 
                        (0, 1.125)
                        arc(90:0:0.125cm and 0.25cm)
                        arc(0:-90:1cm and 1cm) coordinate (wrapping-start-\q);
            \end{pgfonlayer}
            
            \begin{pgfonlayer}{frontwrapping}
                \draw [wrapping] 
                    (0, 1.125) 
                    \foreach \i in {0,5,...,180}
                        { -- (\i/720+0.375*sin -\i,1.125*cos \i)}
                        [shift={(0.5,0)}]
                        (0, 1.125) 
                    \foreach \i in {0,5,...,150}{ -- (\i/720+0.375*sin -\i,1.125*cos \i)} coordinate (wrapping-end-\q);
            \end{pgfonlayer}
          \end{scope}
          
      \ifnum\q>1
        \draw [wrapping] (wrapping-end-\p) .. controls ++(-60:0.5cm)  and ++(180:0.25cm) .. (wrapping-start-\q);
      \fi
      \ifnum\q=4
        \draw [wrapping] (wrapping-end-\q) arc (210:270:1cm and 0.75cm);
      \fi
      \end{scope}
      }
  \end{tikzpicture}
  
  \tikz \draw[thick,rounded corners=16pt]
  (0,0) -- (0,2) -- (1,3.25) -- (2,2) -- (2,0) -- (0,2) -- (2,2) -- (0,0) -- (2,0);

\end{center}

\section{细胞学与细胞生物学发展简史}

现在许多科学家认为,可以把生物科学的发展划分为3个阶段:19世纪以及更早的时期,是以形态描述为主的生物科学时期;20世纪的前半个世纪,主要是实验的生物学时期;20世纪五六十年代以来,以DNA双螺旋的发现与中心法则的建立,开始了进入精细定位与定量的生物学时期。细胞学的发展史和细胞生物学学科的建立与发展,大致符合这一历史进程。

现代细胞生物学已发展到恨到的水平,已非经典细胞学可以比拟,但遵循科学发展的规律,我们仍有必要简要重温一下细胞学与细胞生物学发展的历史,这将有助于我们加深对细胞生物学学科的形成及其历史条件的认识。

细胞学与细胞生物学发展的历史大致可以划分为以下几个阶段:

\subsection{细胞的发现}


\subsection{细胞学说的建立及其意义}

\subsection{细胞学的经典时期}

\subsection{实验细胞学与细胞学的分支及其发展}

\subsection{细胞生物学学科的形成与发展}

\subsection{细胞生物学的主要学术组织、学术刊物与教科书}










\input{./chapters/CellAndGenomes.tex}
\input{./chapters/CellChemistryAndBioenergetics.tex}
\input{./chapters/Proteins.tex}
\input{./chapters/CellMembrane.tex}
\input{./chapters/CellMembraneTransport.tex}
\input{./chapters/CellSignaling.tex}

\input{./chapters/CellSignaling.tex}
\chapter{细胞骨架}

将微分干涉纤维术和录像增强反差显微术相结合,可以在真核细胞的细胞质内观察到一些纤维样的结构,而且纤维的长度和分布模式总处于动态变化之中;此外,还有一些模型细胞器或颗粒状结构沿着这些纤维移动。用电子显微镜观察经非离子去垢剂处理后的细胞,可以在细胞质内观察到一个复杂的纤维状网架结构体系,这种纤维状网架结构通常称为细胞骨架(Cytoskeleton)。细胞骨架包括\textbf{\underline{微丝(MF,Microfilament)}}、\textbf{\underline{微管(MT,Microtubule)}}和\textbf{\underline{中间丝(IF,Intermediate Filament)}}等。它们都是有相应的蛋白亚基组装而成。

细胞骨架是一种\textbf{\underline{高度动态}}的细胞结构体系。在细胞周期的不同时期,细胞骨架具有完全不同的分布状态;在体内各种不同分化状态的细胞中,不仅细胞骨架分布状态存在很大的差异,甚至连构成细胞骨架的蛋白组分也不尽相同。

用物理或化学的手段破坏细胞骨架的结构,将导致细胞形态发生变化,细胞内部各种细胞器和生物大分子分布异常。可见细胞骨架在细胞内发挥着重要的机械支撑与空间定位作用。同时,细胞骨架还是真核细胞结构与功能的重要组织者。细胞骨架不仅与细胞的形态发生相关,而且还参与所有形式的细胞运动,诸如肌肉的收缩、变形运动、细胞迁移、染色体向极运动、纤毛与鞭毛的运动、细胞器和生物大分子的运输、细胞质内生物大分子的不对称分布等。

\section{微丝与细胞运动}

微丝又称\textbf{肌动蛋白丝(Actin Filament)}或\textbf{纤维状肌动蛋白(Fibrous Actin,F-actin)},这种\textbf{\underline{直径约为7nm}}的细胞骨架纤维存在于所有真核细胞中。无论是处于分裂周期中的细胞,还是终末分化细胞,微丝在细胞生命活动过程中发挥着重要的作用。微丝网络的空间结构与功能取决于所结合的\textbf{\underline{微丝结合蛋白(Microfilament-Associated Proteins)}}的种类。在不同类型的细胞内,甚至是在统一细胞的不同部位与之相结合的不同的微丝结合蛋白赋予了微丝网络不通的结构特征和功能,如小肠上皮细胞微绒毛内部的微丝束及细胞皮层的微丝网络、细胞质中与黏着斑相连的张力纤维、迁移中的成纤维细胞前缘的片状伪足和丝状伪足中临时性的微丝束、动物细胞分裂时的胞质分裂环,还有如存在于肌细胞中的细丝等等。细胞内微丝的组装和去组装的动力学过程与\textbf{细胞突起(微绒毛、伪足)的形成}、\textbf{细胞质分裂}、\textbf{细胞内物质运输}、\textbf{肌肉收缩}、\textbf{吞噬作用}、\textbf{细胞迁移}等多种细胞运动过程相关。

\subsection{微丝的组成及其组装}

\subsubsection{结构与成分}

\textbf{\underline{微丝的主要结构成分是肌动蛋白(actin)}}。在大多数真核细胞中,肌动蛋白是含量最丰富的蛋白质之一。在肌细胞中,肌动蛋白占细胞总蛋白量的10\%左右。即使非肌细胞,肌动蛋白也占细胞总蛋白量的$ 1\% \sim 5\% $。肌动蛋白在细胞内有两种存在形式,即\textbf{\underline{肌动蛋白单体(又称球状肌动蛋白,G-actin)}}和由单体组装而成的\textbf{\underline{纤维状肌动蛋白}}。\underline{肌动蛋白单体是由单个肽链折叠而成},相对分子质量为$ 43\times 10^{3} $,外观呈蝶状结构,中央有一个裂口,裂口内部有\underline{ATP结合位点和\ch{Mg^2+}结合位点}。

肌动蛋白在生物进化过程中是高度保守的。在哺乳动物和鸟类细胞中至少已分离到6种肌动蛋白,4种为$ \alpha- $肌动蛋白,分别为横纹肌、心肌、血管平滑肌和肠道平滑肌所特有,它们均组成细胞的收缩性结构,另2种为$ \beta- $肌动蛋白和$ \gamma- $肌动蛋白,存在与所有肌细胞和非肌细胞中。其中\textbf{\underline{$ \beta- $肌动蛋白通常位于细胞的边缘}},\textbf{\underline{$ \gamma- $肌动蛋白肌动蛋白与张力纤维有关}}。对于一个正在迁移的细胞,$ \beta- $肌动蛋白在细胞的前缘组装成微丝。在不同类型的肌细胞中,$ \alpha- $肌动蛋白的一级结构(约400各氨基酸残基)仅相差$ 4 \sim 6 $个氨基酸残基,$ \beta- $肌动蛋白或$ \gamma- $肌动蛋白与$ \alpha- $肌动蛋白(来自横纹肌)相差约25个氨基酸残基。显然,这些肌动蛋白是从同一个祖先基因演化而来。多数简单的真核生物,如酵母或黏菌,仅含单个肌动蛋白基因。然而许多多细胞真核生物含有多个肌动蛋白基因,如海胆有11个,网柄菌属(Dictyostelium)有17个,在某些种类的植物基因组中编码的肌动蛋白基因数目多达60个。

电子显微镜所观察到的微丝是一条直径约为7nm的扭链。根据对微丝进行X射线衍射分析的结构而建立的结构模型认为:\textbf{\underline{每条微丝是由两股螺旋状的丝相互盘绕而成。每条丝都是由肌动蛋白单体头尾相连呈螺旋状排列而成,螺距为36nm}}。在纤维内部,每个肌动蛋白单体周围都有4个单体,上、下各一个,另外两个位于一侧。肌动蛋白分子上的裂口使得该蛋白本身在结构上具有不对称性,在整根微丝上每一个单体上的裂口都朝向微丝的同一端,从而使\textbf{\underline{微丝在结构上具有极性。具有裂口的一端为负极,而另一端为正极}}。

\subsubsection{微丝的组装及动力学特征}

在体外,微丝的组装/去组装与溶液中所含有的肌动蛋白单体的状态(结合ATP或ADP)、离子的种类及浓度等参数相关。通常,只有结合ATP的肌动蛋白才能参与微丝的组装。当溶液中含有适当浓度的\ch{Ca^2+},而\ch{Na+}、\ch{K+}的浓度很低时,微丝趋向于解聚成G-actin;当溶液中含有ATP、\ch{Mg^2+}以及较高浓度的\ch{Na+}、\ch{K+}时,溶液中的G-actin则趋向于组装成F-actin,即新的G-actin加到微丝末端,使微丝延伸,但通常是微丝正极(+)的组装速度较负极(-)快。当溶液中携带ATP的G-actin处于临界浓度时,微丝的组装和去组装达到平衡状态。

肌动蛋白单体组装成微丝的过程大体上可以分为以下几个阶段:

\paragraph{成核反应}
即形成至少有$ 2 \sim 3 $个肌动蛋白单体组成的寡聚体,然后开始多聚体的组装。微丝在细胞内的成核过程需要肌动蛋白相关蛋白\proteinName{Arp2/3}(Actin-related Protein, Arp)复合物的参与,在该复合物内,\proteinName{Arp2}和\proteinName{Arp3}与其他5种蛋白相互作用,形成微丝组装的起始复合物,使肌动蛋白单体与起始复合物结合,形成一段可供肌动蛋白继续组装的寡聚体。

\paragraph{纤维的延长}

肌动蛋白具有ATP酶活性。肌动蛋白单体在参与微丝的组装前必须先于ATP结合,组装到微丝末端的肌动蛋白发挥ATP酶的活性,将ATP水解成ADP。当微丝的组装速度快于肌动蛋白水解ATP的速度时,在微丝的末端就形成一个肌动蛋白--ATP亚基的帽,这种结构使得微丝比较稳定,可以持续组装。相反,当末端的肌动蛋白亚基所结合的是ADP时,则肌动蛋白单体倾向于从微丝上解聚下来。\textbf{\underline{由于微丝两段在结构上存在差异,新的肌动蛋白亚基通常在正极加入,而很少在负极加入。}}

\paragraph{稳定期}

待微丝组装到一定长度时,肌动蛋白亚基的组装和去组装达到平衡状态,即组装和去组装的肌动蛋白亚基数目相当,微丝的长度几乎保持不变,即所谓的``稳定期''。在体外组装过程种有时可以看到微丝的正极由于肌动蛋白亚基的不断添加而延长,而负极则由于肌动蛋白亚基去组装而缩短,这一现象称为\textbf{\underline{踏车行为(treadmilling)}}。

\subsubsection{影响微丝组装的特异性药物}

一些药物可以影响肌动蛋白的组装或去组装,从而影响细胞内微丝网络的结构。\textbf{\underline{细胞松弛素(Cytochalasin)}}是一组真菌的代谢产物,与微丝结合后可以将微丝切断,并结合在微丝末端阻抑肌动蛋白在该部位的聚合,但对微丝的解聚没有明显影响,因而用细胞松弛素处理细胞可以破环微丝的网络结构,并阻止细胞的运动。\textbf{\underline{鬼笔环肽(Phalloidin)}}是一种由毒蕈(Amanita Phallodies)产生的双环杆肽,与微丝表面有强亲和力,但不与肌动蛋白单体结合,对微丝的解聚有抑制作用,可使肌动蛋白丝保持稳定状态。用荧光标记的鬼笔环肽染色可清晰地显示细胞种微丝的分布。将鬼笔环肽注射到细胞内同样能阻止细胞运动,可见微丝的功能依赖于肌动蛋白的组装和去组装的动态平衡。

\subsection{微丝网络动态结构的调节与细胞运动}

\subsubsection{非肌肉细胞内微丝的结合蛋白}

尽管纯化的肌动蛋白单体可以在合适的体外环境下组装成纤维状肌动蛋白,但其复杂性和有序性都远不能与细胞内的微丝网络相比。细胞内的微丝具有复杂的三维网络结构,有些微丝结构稳定,如肌细胞中的细丝及小肠上皮细胞微绒毛中的轴心微丝束等;另一些微丝结构是暂时性的,如胞质分裂环是由微丝形成的收缩环。血小板激活及无脊椎动物精子细胞顶体反应过程中出现的微丝也是暂时性结构,都是在功能需要时才进行组装的。实际上,在大多数非肌肉细胞中,微丝是一种动态结构,它们持续地进行组装和去组装。微丝的这种动态不稳定性与细胞形态的维持及细胞运动有密切的关系。体内肌动蛋白的组装在两个水平上受到微丝结合蛋白的调节:可溶性肌动蛋白的存在状态、微丝结合蛋白的种类及其存在状态;在不同的细胞,甚至是同一细胞的不同部位,由于微丝结合蛋白的种类及存在状态上的差异而致使微丝网络的结构完全不同。

在细胞内,可溶性的肌动蛋白单体和纤维状肌动蛋白的比例大体是$ 1:s1 $。也就是说,细胞内游离态肌动蛋白的浓度远远高于肌动蛋白在体外组装所需的临界浓度,但由于细胞内游离态肌动蛋白常与另外一些相对分子量较小的蛋白(如胸腺肽和抑制蛋白等)结合在一起,从而是G-actin组装成F-actin的过程受到必要的调控,存储在细胞内的G-actin只有在需要时才加以利用。

细胞内微丝网络的组织形式和功能通常取决于与之结合的微丝结合蛋白,而不是微丝本身。细胞微环境内的各种微丝结合蛋白通过影响微丝的组织与去组装,介导微丝与其他细胞结构之间的相互作用来决定微丝的组织行为。有些微丝结合蛋白如封端蛋白、成束蛋白等与微丝的相互作用,可以使微丝保持相对稳定状态;而另外一些如纤维--切割蛋白与微丝的作用,则通过使微丝网络解聚来调节微丝网络的状态。如在小肠上皮细胞的微绒毛内,毛缘蛋白和绒毛蛋白等结构成分将相邻的微丝交联成平行排列的微丝束。在细胞皮层,交联蛋白和凝溶胶蛋白通过将微丝交联或切断来调节细胞皮层的凝溶胶状态和细胞质膜的形态。此外,微丝还可以通过和肌球蛋白之间的相互作用来运送``货物'',对细胞内生物大分子及细胞器的分布起组织作用,从而调节细胞的行为。人们已经从各种组织细胞中分离到了100多种不同的微丝结合蛋白,根据它们在细胞内功能的不同可以将它们归纳如表 \ref{tableMicrofilamentRelatedProteins} 所示:

\noindent \begin{longtable}{|c|c|p{6em}|p{10em}|}

\hline 
微丝结合蛋白类型 & 微丝结合蛋白名称 & 相对分子质量($10^3$) & 主要功能 \endhead
\hline 
成核蛋白 & Arp2/3复合体 & 由7个亚基聚合而成 & 在微丝开始组装时起成核作用 \\ 
\hline 
单体--隔离蛋白 & 胸腺素(thymosins) & 5 & 与肌动蛋白单体结合,调节肌动蛋白的组装 \\ 
\hline 
单体--聚合蛋白 & 抑制蛋白(profilin) & $ 12 \sim 15 $ & 一种ATP--肌动蛋白结合蛋白,能够在细胞运动过程中促进肌动蛋白的聚合 \\ 
\hline 
\multirow{4}{*}{成束蛋白} & 丝束(毛缘)蛋白(fimbrin) & 68 & 横向连接相邻微丝形成紧密的微丝束 \\ 
\cline{2-4} 
 & 绒毛蛋白(villin) &  95 &  \\ 
\cline{2-4} 
 & 成束蛋白(fascin) & 57 &  \\ 
\cline{2-4} 
 & $\alpha-$辅肌动蛋白($\alpha-$actinin) &  &  \\ 
\hline 
\caption{微丝结合蛋白的主要类型}
\label{tableMicrofilamentRelatedProteins}
\end{longtable} 

\subsubsection{细胞皮层}

免疫荧光染色的结果显示,细胞内大部分微丝都集中在紧贴细胞质膜的细胞质区域,并由微丝结合蛋白交联成凝胶状三维网络结构,该区域通常称为细胞皮层(Cell Cortex)。皮层内一些微丝还与细胞质膜上的蛋白由连接,使膜蛋白的流动性受到一定程度的限制。皮层内密布的微丝网络可以为细胞质膜提供强度和韧性,有助于维持细胞形状。细胞的运动,如胞质环流(cyclosis)、阿米巴运动(amoiboid)、变皱膜运动(ruffled membrane locomotion)、吞噬(phagocytosis)以及膜蛋白的定位等都与皮层肌动蛋白的溶胶态或凝胶态转化相关。

\subsubsection{应力纤维}















\chapter{细胞周期}
\chapter{细胞死亡}
\input{./chapters/CellJunctionsExtracellularMatrix.tex}


\part{生殖生物学}
\chapter{作者简介}

\section{杨增明} 

1962 年 8 月出生于甘肃省兰州市

\lifeSpan{1979}{1986} 年在兰州大学攻读学士和硕士学位;

\lifeSpan{1986}{1989}年在东北农业大学攻读博士学位;

\lifeSpan{1989}{1991}年在中国科学院动物研究所从事博士后研究;

\lifeSpan{1991}{1996}年先后在美国北卡罗来纳州立大学、 贝勒医学院和堪萨斯大学医学中心从事动物生殖方面的研究。 现为东北农业大学教授及博士生导师。

2000 年被聘为教育部“长江学者奖励计划”特聘教授。 \myImportantPoint{研究方向为哺乳动物胚胎发育和胚胎着床}。

1998 年获“国家杰出青年基金”及国家教育部“跨世纪优秀人才培养计划”基金。 

自 1998 年以来, 先后主持国家自然科学基金重点项目 2 项、美国避孕药研究与开发项目 2 项等 12 项课题。 于 1994、 1997 及 1998 年先后获得 3 项美国专利。 共计发表科研论文 110 篇, 其中 35 篇发表在 SCI 源期刊。

现为美国Biologyof Reproduction和Theriogenology等杂志的特邀审稿人,并兼任中国《动物学报)和《动物学杂志)编委。 

1998 年被评为“全国优秀教师”,2002 年获“中国青年科技奖”,2003 年获“国家留学回国人员成就奖”。

现为国务院学位委员会第五届学科评议组成员、 全国生殖生物 学会副理事长、 中国动物学会细胞及分子显微技术学分会副主任委员、 美国生殖生物学会会员。

\section{孙青原} 

1964 年 12 月出生于山东省招远市。 

现任中国科学院动物研究所研究员、 中国科学院研究生院教授、 博士生导师。

1994 年毕业于东北衣业大学, 获理学博士学位。 

1996 年由中国科学院动物研究所博士后出站后留所工作。 

曾先后在美国、日本和以色列进行过四年的博士后培训。\myImportantPoint{研究方向为卵母细胞减数分裂、 受精和早期胚胎发育的基因表达与信号转导和动物克隆机理}。

发表 SCI 收录论文100 篇,其中有关影响因子大于2.0 的46 篇,在生殖生物学领域最具影响的Biology of Reproduction上发表22 篇,影响因子总和190被引用 600 余次。受邀为包括Biology of Reproduction在内的8种SCI期刊撰写综述。 作为副主编编著了由科学出版社出版的《受精生物学》;参编了美国Humana Press出版的《\textit{Methods in Molecular Biology: Germ Cells}》、《动物发育生物学》和《大百科全书》等著作。 是Biol Reprod、Reproduction、DevDynam等20余种国内、 外期刊的审稿人和日本动物繁殖学会会志J Reprod Dev的编委会委员。

1999年获“中国科学院青年科学家奖”,2002年获国家杰出青年科学基金资助,2003 年获“中国科学院十大杰出青年”称号和“国家留学回国人员成就奖”, 2004年获“中国青年科技奖”和“新世纪百千万人才工程国家级人选”称号。

\section{夏国良}

现任中国农业大学生物学院教授、博士生导师。教育部“长江学者奖励计划”特聘教授。“国家杰出青年基金”获得者。

\lifeSpan{1983}{1988}年师从北京农业大学动物生理生化杨传任教授,攻读硕士和博士学位研究生并从事生殖内分泌学的研究。

\lifeSpan{1991}{1994}年,公派到丹麦国家教学研究医院(Rigshospitalet)生殖生物学实验室做博士后研究工作,师从国际著名胚胎学家Anne Grete Byskov教授。\myImportantPoint{从事胚胎卵巢中卵细胞的发育和调节的研究}。

1994年回到母校中国农业大学工作,当年破格提升为教授。在回国人员启动基金和国家自然科学基金的资助下,继续研究\myImportantPoint{促性腺激素诱导卵母细胞体外成熟的作用机制}。已发表论文90余篇,其中在国内外重要的SCI源期刊中发表30余篇。论文被国外引用200余次。

此外,还担任《中国农业大学学报》编辑委员会副主任、主编,《中华中西医》杂志常务编委、《动物学杂志》编委、中国农业大学国家“211工程”项目“畜禽细胞与分子生物学实验室”负责人、中国生殖生物学会副理事长、中国生理学会理事、中国动物学会教学工作委员会委员、中国畜牧兽医学会动物生理生化学会副理事
长。

\chapter{生殖生物学序言}
\newpage
生殖是生物体的基本特征之一,生物通过生殖实现一代到下一代生命的延续。不同生物生殖过程的复杂性差异很大,生殖方式可分为无性生殖和有性生殖两种形式。生殖本身是一个古老的话题,而生殖生物学是用现代生物学手段研究整个生殖过程的一门学科,是在动物胚胎学、繁殖学、妇产科 学、发育生物学及动物生理学等学科的基础上发展起来的一门新兴学科。随着人工授精、体外受精、胚胎移植、细胞核移植,以及很多辅助生殖技术的 广泛应用,这门学科在生物学、医学及农业中的地位越来越重要。而且,随 着细胞生物学、分子生物学、生物化学、生理学等学科的飞速发展,以及各 种现代生物学技术在生殖过程研究中的广泛应用,人们对生殖过程中的各种 现象及其分子机制的了解有了长足的进步。

生殖生物学的研究不仅可使我们了解生命繁衍的奥秘, 也在多方面使人
类自身受益。20世纪中叶, 生殖内分泌基础研究的重大突破, 导致了女用 口服肖体避孕药的出现;精子获能现象的发现, 导致了目前广泛应用于不育 症治疗的试管婴儿技术的产生;20世纪80年代, 对激素受体研究的重大发现产生了第二代肖体避孕药。1997年克隆羊 多莉 的出生, 引发了很 多深层次的理论间题和人们对该技术应用前景的期望。 所有这些突破性研究 成果对生殖生物学的发展起到了巨大的推动作用。 目前, 不孕症的发病率在逐年上升(约为$ 15\% \sim 20\% $), 而且病因也趋向于复杂化, 使得辅助生殖技 术面临极大的挑战。 各种环境因素对生殖过程的不良影响也越来越明显。 人 们对避孕措施的安全性、 可靠性及多样性等方面的要求也越来越高, 发展非 肖体、 无毒无害、 无副作用的避孕措施是社会发展的必然要求。 随着我国畜牧业的迅速发展, 对肉、 蛋和奶的需求逐年增加, 迫切需要增加奶牛、 肉 牛、 肉羊、 猪等的数址,并提高其质量, 亟待加速各种辅助生殖技术在畜牧业中的应用。 这些需求一方面为生殖生物学研究提出了新的挑战, 另一方面 也为生殖生物学的发展提供了一个难得的发展机遇。 随着多聚酶链反应、 基 因敲除、 基因芯片、 蛋白芯片、 基因组学、 蛋白质组学、 细胞分离技术等方 面的飞速发展, 许多以前未知的生殖现象、 生殖过程及生殖功能逐渐被发现 和了解,也为生殖生物学提供了更加宽广的发展空间。

目前,国内尚尤一本系统介绍生殖生物学领域研究成果和发展动态的书,该书是国内从事生殖生物学研究的一批学者,综述当前国内外该领域的最新研究进展,并结合自己的研究成果编写的一部全面反映生殖生物学基本概念、基础理论、研究技术、研究成果和发展动态的综合性教学和科研用书。该书的三位主编均为国家杰出青年科学基金获得者,杨增明博士和夏国良博士现为“教育部长江学者奖励计划”特聘教授,孙青原博士2003年获得“中国科学院十大杰出青年”称号。

我期望这本书的问世将对我国生殖生物学的发展起到积极的推动作用,并对生殖生物学领域的研究、教学和临床实践等有所裨益。

\begin{flushright}
\kaishu 
\begin{tabular}{c}
{\huge 薛社普}\\  
中国医学科学院教授 \\  
中国科学院资深院士 \\  
2004年04月15日 \\  
\end{tabular} 
\end{flushright}
\chapter{生殖与生殖生物学概论}

``天地之大德曰生''。生命永远使宇宙中最宝贵的,生命具有无可争辩的意义,是第一本位的。``种''的繁衍生殖自然就具有无与伦比的重要意义。生命的传承、沿袭是人类赖以永恒存在的源泉。宇宙中的一切事物,因为有了生命的存在才显示了自身的价值和意义。每个有生命的个体总会以某种方式繁衍与自己性状相似的后代以延续生命,这就是生殖(Reproduction)。从生理的角度上看,生殖是一切生物体的基本特征之一,一个个体可以没有生殖而生存,但一个物种的延续则必须依赖于生殖。

生物通过生殖实现亲代与后代个体之间生命的延续。尽管遗传信息决定了后代延承亲代的特征,但遗传是通过生殖而实现的。亲代遗传信息在传递过程中会发生变化,从而使物种在维持稳定的基础上不断进化成为可能。生命的延续本质上是遗传信息的传递。在生物代代繁衍的过程中,遗传和变异与环境的选择相互作用,导致生物的进化。因此,生殖过程本身除了是生物由一代延续到下一代的重要生命现象外,与遗传、进化,甚至生命起源的问题紧密相关。

\section{生殖现象的研究历史}

Macedonian Aristotle(公元前\lifeSpan{384}{322}年)是最早系统从事动物生殖与发育方面研究的学者,首先提出了胚胎是由简单到复杂逐渐形成的观点。1683年Antoni van Leeuwenhoe首次在精液中发现了精子,并提出``精源说'',认为精子存在人的雏形,发育只是这个雏形的放大而已。以后Marcello Malpighi(\lifeSpan{1629}{1694})和Jan Swamerdam(\lifeSpan{1637}{1680})等又提出了``卵源说'',认为卵子中存在一个人的雏形。此后,Charles Bonnet(\lifeSpan{1720}{1793})年在蚜虫中首次发现了孤雌生殖现象。

Lazzaro Spallanzani(\lifeSpan{1729}{1799})首次成功地进行了青蛙的人工授精,并发现在缺乏精子穿入时,则卵子发生退化。在进行狗的实验时,他提出只有当卵子和精液共同存在时,才能产生一个新的个体。Caspar Friedrich Wolff(\lifeSpan{1738}{1794})观察到,从受精卵的卵黄中形成了有形态结构的胚胎。Carl Ernst von Baer(\lifeSpan{1792}{1876})对几种哺乳动物的卵子进行了比较研究。以后,Ernst Haeckel(\lifeSpan{1834}{1919})提出了个体发育是系统发育简要重演的观点。

Oscar Hertwig(\lifeSpan{1849}{1922})和Richard Hertwig(\lifeSpan{1850}{1937})兄弟在Otto Butschli的研究基础上,进一步对受精现象进行了研究,提出受精的本质是雌雄配子细胞核的融合。并且,Oscar Hertwig在海胆的卵子中观察到极体以及极体中的细胞核。

1883年,van Beneden在蛔虫受精卵的第一次有丝分裂纺锤体上看到四条染色体,其中两条来自父方,两条来自母方,提出父母的染色体通过受精卵的融合传递给子代。此后,Theodor Boveri(\lifeSpan{1862}{1915})通过对蛔虫卵的进一步观察,提出了染色体理论,并通过实验证实了染色体对发育的重要作用。20世纪初,美国生物学家McClung(1902)第一次将X染色体和昆虫的性别决定联系起来。Stevens(1905)及Edmund Beecher Wilson(1905)同时将XX性染色体与雌性对应,而XY及XO与雄性相关联,并提出一种特异的核成分在性别表型决定中起作用,即性别有遗传而非环境决定。虽然,自1921年以来,就已知道男性中具有X和Y染色体,而女性中具有两条X染色体,但这些染色体在人性别决定中的作用在1959年以前一直不清楚。Jacobs和Strong(1959)以及Ford等(1959)首次证明,Y染色体在小鼠和人类的性别决定中起关键作用。

\section{生殖过程}

\mykeyword{生殖}是指所有的生物能够产生与它们自己相同或相似的、新的生物个体的能力,也指单细胞或多细胞的动物或植物自我复制的能力。在各种情况下,生殖都包括一个基本的过程,即亲本的原材料或转变为后代,或编程将发育成后代的细胞。生殖过程中也总是发生遗传物质从亲代到子代的传递,从而是子代也能复制它们自己。在不同生物中,尽管生殖过程所采取的方式和复杂性变化很大,但都可分为两种基本的生殖方式,即\mykeyword{无性生殖(asexual reproduction)}和\mykeyword{有性生殖(sexual reproduction)}。在无性生殖中,一个个体可分成两个或两个以上相同或不同的部分,仅有一个亲本的参与,生殖过程中没有配子的形成。而在有性生殖过程中,特化的雄性生殖细胞和雌性生殖细胞发生融合,形成的合子同时携带两个亲本的遗传信息。

\subsection{无性生殖}

无性生殖的优点在于可使有益的性状组合持续存在,不发生改变,并且不需要经过易受环境因素影响的早期胚胎发育的生长期,常见于大多数的植物、细菌、原生生物及低等的无脊椎动物中。单细胞生物常以\mykeyword{分裂(fission)}方式或\mykeyword{有丝分裂(mitosis)}方式,分成两个新的、相同的个体。所形成的细胞可能聚集在一起形成丝状(如真菌),也可能成群生长(如葡萄球菌)。\mykeyword{断裂}或\mykeyword{裂片生殖(fragmentation)}是指在丝状的生物中,身体的一部分断裂后,发育形成一个新的个体。\mykeyword{孢子生殖}或\mykeyword{孢子形成(sporulation)}为原生动物及许多植物中的一种无性生殖方式:一个孢子是一个生殖细胞,不需要受精就能形成一个新的个体。在水螅等一些低等动物和酵母中,\mykeyword{出芽}为一种常见的生殖方式:在身体表面长出一个小突起后,逐渐长大,并于身体分离后形成一个新的个体。在海绵的内部也可长出小芽,称为\mykeyword{芽球(gemmule)}。

\mykeyword{再生}是无性生殖的一种特化形式,海星和蝾螈等动物可通过再生替代受伤或丢失的部分。很大植物通过再生可产生一个完整的个体。分类上越低等的动物,其完全再生能力也越强。到现在为止,还未见到脊椎动物具有再生完整个体的能力。但通过实验的手段,已在鱼类、两栖类和哺乳类等脊椎动物中获得了无性生殖的个体。特别需要提到的是,1997年首次通过体细胞核移植手段,获得了无性生殖的哺乳动物---克隆绵羊。

自然条件下的无性生殖包括\mykeyword{孤雌生殖}和\mykeyword{孤雄生殖}等。\mykeyword{人工辅助无性生殖}是指在物理或化学因子的作用于卵子后的单性生殖,以及利用细胞核移植技术而进行的动物克隆。

\subsection{有性生殖}

在有性生殖的生物体(高等生物)中含有两大类细胞:构成组织和器官并执行各种功能的\mykeyword{体细胞(Somatic Cell)}和携带有特定的遗传信息并具有受精后形成合子能力的\mykeyword{生殖细胞(Germ Cell)}。\myImportantPoint{生殖细胞又包括卵子和精子两类}。有性生殖周期是体细胞与生殖细胞相互转变的过程。在高等生物的机体中,只有一小部分细胞为生殖细胞,然而它们却是正常生命周期中的一个关键环节。

有性生殖发生在很多的单细胞生物和所有的动物和植物中。除在个别动物中可进行孤雌生殖外,有性生殖是高等的无脊椎动物和所有的脊椎动物自然情况下唯一的一种生殖方式。在有性生殖过程中,一种性别的细胞(配子)被另一种性别的细胞(配子)受精后,产生一个新的细胞(合子或受精卵),以后受精卵发育形成一个新的个体。两个结构相同但生理上不同的同形配子(Isogamete)的结合,称为\mykeyword{同配生殖(Isogamy)},仅见于低等的水棉属的绿藻(spirogyra)和一些原生动物中。\mykeyword{异配生殖(heterogamy)}是指两种明显不同的配子的结合,即精子与卵子的结合。许多生物具有特殊的生殖机制来保证受精的进行。在陆生动物中,通过交配进行体内受精,从而提供了一个受精必须的液体环境。

有性生殖的优越性在于来自两个亲本的细胞核融合后,子代可源源不断地继承各种各样的性状组合,从而具有很大的发生变异的空间,对于改进物种本身以及物种的生存具有重要意义。精卵结合形成的子代在遗传学上互不相同,也不同于各自的亲代,从而保证了物种的多样性。有性生殖产生的后代中随机组合的基因对物种可能有利,也可能不利,但至少会增加少数个体在难以预料和不断变化的环境中存活的几率,从而对物种的延续提供了有利的条件。此外,在进行有性生殖的物种中,生命周期中都具有二倍体和单倍体交替的特征。二倍体的物种每一基因都有两份,其中一份在功能上处于备用状体,对各种突变等具有一定的抵御作用。这也可能是高等生物以有性生殖为主的原因。因为即使在细菌等进行无性生殖的生物中,也发生遗传物质的交换。在蚯蚓等雌雄同体(hermaphrodite)的动物中,由于解剖结构的特化或雌雄配子的成熟时间不同,总是避免自体受精的发生。

生殖过程不是一个连续的活动,而是受一些形式和周期的约束。通常,这些形式和周期与环境条件有关,从而使得生殖过程能有限地进行。例如,一些有发情周期的动物仅在一年的一段时间内发情,使得后代能在适宜的环境条件下出生。同样,这些形式和周期也受到激素和季节因素的控制,使得生殖过程中的能量消耗得到很好地控制,从而最大限度地提高了后代的生成能力。

\section{生殖生物学}

生殖是亲代与后代个体之间生命延续的过程。生殖生物学(reproduction biology)是研究整个生殖过程的一门学科,即使发育生物学的一个分支,又是生理学的一个分支,属于一门新的充满活力的、融合了现代生物化学、细胞生物学、内分泌学和分子生物学等学科的交叉学科。过去这些年中,生殖生物学研究领域取得了许多世人瞩目的重大突破。例如,对“下丘脑---垂体---性腺”内分泌轴系这一重大理论问题的揭示,导致了口服避孕药的诞生;从精子获能和卵子体外成熟等基础研究着手而创立的“试管婴儿”技术,使国内外成千上万的不孕夫妇获得了后代;1997年克隆绵羊“多莉”的诞生,也已对人类社会产生了深远的影响。生殖生物学已成为生物学中一个活跃的、充满机遇和挑战的重要研究领域。

\myImportantPoint{生殖生物学主要研究性腺发育、配子发生、受精、胚胎发育及着床、性别决定、妊娠维持、胎盘发育和分娩等过程的调控,以及生殖道的恶性肿瘤、异常妊娠、生殖道感染、环境和职业性危害对生殖的影响等问题。}此外,生殖生物学也研究\myImportantPoint{在青春期、泌乳、衰老和妊娠等过程中与生殖相关的内分泌变化},以及\myImportantPoint{性行为的形成和影响因素}等。

随着人工授精、体外受精、胞浆内单精子注射(ICSI)、胚胎移植、细胞核移植,以及其他辅助生殖技术的广泛应用,这门学科在生物学、医学及农业中的地位将越来越重要。例如,自1978年英国科学家首次成功地获得世界首例试管婴儿至今,全世界已有约100万试管婴儿诞生,体外受精和胞浆内单精子注射技术已成为临床上不育症治疗的最主要手段,为许多不育夫妇带来了福音。

近年来,细胞生物学、分子生物学、生物化学、生理学等学科飞速发展,各种现代生物技术已广泛渗透到生殖生物学的研究过程中,使得人们对生殖过程的各种现象及其分子机制的了解有了长足的进步。

\section{生殖生物学的相关学科}

随着生殖生物学的迅速发展,生殖生物学的研究范围也在主板扩大,与很多学科间的交叉也变得越来越明显,这里仅简单介绍与生殖生物学相关性很强的一些学科。

\subsection{动物胚胎学(Embryology)}

动物胚胎学是研究动物个体发育过程中形态结构及其生理功能变化的一门学科,主要研究受精后到出生前这段时间内动物的发育过程。个体发育包括生殖细胞的起源、发生、成熟、受精、卵裂、胚层分化、器官发生,直至发育为新个体,以及幼体的生长、发育、成熟、衰老和死亡。通常也将个体的发育的整个过程分为胚前发育、胚胎发育和胚后发育。胚前发育主要研究生殖细胞的起源、单倍体的精子和卵子的发生、形成和成熟。胚胎发育是指受精到分娩或孵出前的发育过程。胚后发育包括出生或孵出的幼体的生长发育、性成熟、体成熟,以及以后的衰老和死亡。胚胎学一般只研究胚前发育和胚胎发育。

\subsection{发育生物学(Developmental Biology)}

发育生物学实在动物胚胎学的基础上,结合细胞生物学、遗传学和分子生物学等学科的发展,而逐渐形成的一门学科。它是应用现代生物学技术,来研究生物发育本质的科学,主要研究生殖细胞的发生、受精,胚胎发育、生长、衰老和死亡等过程,分析从受精一直到主要胚胎器官形成时动物发育的基本现象及形式,偏重于研究细胞决定及分化的机制,以及形态发生过程中细胞间相互作用等问题。

\subsection{动物繁殖学(Animal Reproduction or Theriogenology)}

动物繁殖学主要研究家畜和家禽生殖过程中的形态、生理和功能的变化,以及调节和控制生殖过程的相关技术。它是序幕科学的一个重要组成部分,主要包括家畜和家禽的生殖生理、繁殖技术以及家畜繁殖力的评价和家畜生产的影响因素和管理等。

\subsection{生殖医学(Reproduction Medicine)}

生殖医学是一门综合性的学科,涉及生育、不育、节育和出生缺陷等。生殖医学的主要任务是通过常规的诊断和治疗措施,将现在的各种生殖技术应用于不孕不育病人,使其产生后代。自1978年,世界上第一例试管婴儿诞生以来,辅助生殖技术已得到了突飞猛进的发展。在国内外,大量的生殖医学中心或辅助生殖中心相继建立,为越来越多的不孕不育患者解除了痛苦。在提供的服务方面,也由简单的人工授精、体外受精,逐渐向胞浆内单精子注射、卵细胞质互换、着床前胚胎的遗传诊断等多方位发展。

\subsection{产科学(Obsterics)}

产科学主要是研究妊娠、分娩、胎儿出生以及出生后事件的一门临床科学。它的任务是既要保证产生一个健康的后代,又要确保母体的健康不受损害。可通过超声等手段来判断子宫内的情况,对母体子宫的大小、妊娠期的长短、胎儿的大小和位置等进行分析,从而使胎儿顺利产出。在异常情况下,通过剖腹产手术来保证胎儿的分娩。

\subsection{妇科学(Gynecology)}

妇科学是主要研究雌性生殖系统各种失调的一门科学。现代妇科学涉及月经失调、绝经、生殖器官的感染性疾病和异常发育、性激素紊乱、良性和恶性肿瘤,以及各种与避孕相关的问题。由妇科学产生的一门学科为生殖医学。与妇科学相应,也产生了男科学或男性学(Andrology),主要研究男性生殖器官的各种异常或病变,以及男性的不孕症等问题。\\

事实上,与生殖生物学相关的学科还有很多,特别是内分泌学、细胞生物学和分子生物学的进展对于阐明生殖过程的机理起了巨大的推动作用。

\section{生殖生物学的发展前景}

据报道,不孕症在国外的发病率约为$ 15\%\sim20\% $。自从世界上第一例试管婴儿路易斯布朗(Louise Brown ) 1978年7月25日在英国诞生以来,体外受精、促排卵技 术、显微受精、胚胎培养、胚胎冷冻等辅助生殖技术迅速发展,并不断完善,已为很多 不孕患者解除了痛苦。据估计,全世界每年大约有100000例试管婴儿出生。自我国的第一例试骨婴儿1988年出生以来,辅助生殖技术已在全国的绝大多放地区得到推广。

性和生殖健康见人们生活和幸福的核心内容。生殖健康的卞要内容是保证妊娠的正 常及安全进行,使用更安全可节的避孕措施,以及防治生殖道的感染等。由于世界人口及中国人口的猛增,直接危及人的生存环境和生活质量,对自然环境的破坏也在逐年增加。控制人口数量及提高人口质量也是当今世界亟待解决的问题。在生育调节方面,20世界90年代以前的生殖研究主要是围绕“下丘脑---垂体---性腺”所构成的生殖轴系,作为发展避孕药的出发点,即通过干扰激素和生殖轴系之间的相互作用。这些避孕方法的主要缺陷是有不良反应。随着社会的发展和生活水平的提高,人们对避孕措施的安全性、可靠性、多样性的要求也越来越高,而且在达到避孕效果的同时也要兼具预防生殖道感染的功能。人们开始认识到,最理想的抗生育靶点应当是在生殖过程中起直接作用的细胞和因子。因此,加强以生育控制为目的的基础研究,寻找生殖细胞的发生、成熟、受精和胚胎着床等生殖过程中可控制的关键环节,以此发展新一代避孕技术,已成为21世纪生殖生物学研究与发展的主要目标。随着生殖生物学基础研究的深入和对生殖相关的 重要基因和分子的认识,通过干扰基因表达或表达产物的功能,最终可以有望发展出对其他正常生理功能没有影响或影响很小的避孕药或避孕方法。另外,由于艾滋病等传染病的广泛流行,对于预防生殖道感染的要求也越来越高。现在迫切需要普及生育方面的 知识,提供安全及高效的避孕措施,控制生殖相关疾病的传播。

近年来,我国国民经济及畜牧业方面的迅速发展,对提高家畜的繁殖力及加速家畜的品种改良方面提出了更高的要求。在奶牛、肉牛、奶羊和肉羊的繁殖方面,逐渐发展了超数排卵、卵母细胞体外成熟、体外受精、性别鉴定、胚胎分割、克隆、转基因及基 因敲除等方面的一系列技术。随着人们对肉、蛋和奶的需求逐年增加,迫切需要增加奶牛、肉牛、肉羊、猪等的数量,并提高其质量。近年来,各种辅助生殖技术已在畜牧业 中得到广泛应用,人工输精、体外受精、超数排卵、胚胎移植、性别鉴定及转基因等在逐步完善和推广。特别是最近克隆牛、克隆羊、克隆猪的问世,以及基因敲除家畜的获得,极大地促进了对家畜生殖机理的研究,并加速了各种辅助生殖技术在家畜中的应用,将对促进家畜的品种改良及提高繁殖力方面具有重要的意义。

在野生动物的保护方面,提高野生动物的繁殖力也迫在眉睫。目前,需要了解这些动物的基本生殖过程,通过激素处理、人工授精、体外受精、胚胎分割、性别鉴定、克隆等技术手段有望在短时间内增加濒危物种的种群数量。在宠物的饲养方面,了解基本的生殖过程,利用现代生殖技术来提高繁殖力及加速品种改良等方面也具有重要的意义。

随着社会和经济的不断发展,人类对大自然的干预日益加剧。废水、废气、废渣、 农药、化肥等大量的化学物质通过各种途径排入环境,造成了严重的环境污染。噪声、 农药残留、射线等各种环境因素对生殖过程的影响也越来越受到人们的重视。凡是能够影响机体内外环境改变的因素,都将会对生殖健康产生一定程度的影响。目前越来越多的证据表明,许多人工合成的化学物质可干扰人类及野生动物的生长发育,导致人类的不孕不育率、畸胎率和自然流产率上升。

而且,随着社会工业化和现代化的不断发展,人类对野生动物的生存环境不断蚕食,异致大量的野生动物灭绝和濒临灭绝。而由于这些动物的数量有限以及难以接近等原因,对于野生动物基本生殖过程的了解仍很有限。现在也迫切需要利用现有的生殖生物学知识和技术,来促进对野生动物生殖过程的了解,并将现有的辅助生殖技术应用于野生动物,从而延缓或防止野生动物的灭绝,并使一些具有食用和药用价值的动物得到迅速繁殖。

\section{本书的特色}

近年来随着多聚酶链反应、基因敲除、转基因、基因芯片、RNA干涉、 细胞分离技术、免疫测定等方面的飞速发展, 以及随着基因组和后基因组时代的发展和推动, 人们对生殖过程基本调控机制的了解也越来越多。 现代分子生物学和细胞生物学理论与技术的发展, 极大地推动了生殖生物学研究, 使人们对生殖现象的认识深入到细胞和分子水平。 从本质上讲, 生殖过程是个体生命活动的一部分, 与其他生命现象遵循共同的基 本规律,如基因的时空特异性表达调控、细胞的增殖、分化和凋亡、细胞之间的信号转导和细胞外基质的相互作用等。但是, 生殖过程在生命活动中具有特殊使命, 因此也具有生殖细胞发生、性周期、 受精、妊娠和分挽等独特现象。

到目前为止, 国内尚无一本系统介绍现代生殖生物学研究成果的书。 本书将尽最大努力, 综合国内外在生殖生物学领域的研究论文和著作, 介绍生殖细胞的发育及成熟、 受精机理、 胚胎发育、 胚胎着床、 胎盘的形成、 分挽、性别决定、 生殖缺陷、 生殖免 疫、无性生殖等方面的基本知识和国内外的最新研究动态, 并介绍生殖激素、性行为、环境对生殖的影响及现代生殖生物学实验技术。 本书将以小鼠、 人和家畜为主, 既介绍哺乳动物生殖的共性, 又具体阐述各类动物的特性, 力求从广度、深度和新颖性方面比较系统和全面地介绍生殖生物学的内容。



\chapter{生殖器官的结构与功能}

\section{雌性生殖器官}

雌性生殖器官包括卵巢(ovary)、输卵管(oviductor uterine tube)、子宫(uterus)、 阴道(vagina)、前庭(vestibule)、阴门(vulva)和相关腺体。卵子的发生、成熟、运输、受精、妊娠及胎儿的出生等功能均由雌性生殖器官完成。

\subsection{卵巢}

所有哺乳动物的卵巢是成对的, 从性腺原基的形成到发育完成均位于肾脏附近。 卵巢的大小与动物的年龄和生殖状态有关。多数成年动物的卵巢游离面突向体腔, 主要由 外层的皮质(cortex)和内部的髓质(medulla)两部分构成。 卵巢的主要功能包括雌性激素的分泌和卵子的产生。

\subsubsection{形态与结构}

\paragraph{卵巢的形态与基本结构}

多数哺乳动物的卵巢为卵圆形,由皮质和髓质两部分组成。皮质主要由卵泡和黄体组成,并覆盖一层低矮的立方上皮细胞;皮质的基质由疏松结缔组织构成;皮质表面有一层由致密结缔组织构成的臼膜。卵巢的髓质主要由疏松的结缔组织和平滑肌 组成,富含神经、血管和淋巴管,并与卵巢系膜中的平滑肌相连J卵巢网位于骼质部,是由立方上皮细胞或实质细胞束相连而成的不规则的网络管道,这一结构特点在肉食动物和反刍动物中非常明显。

\subparagraph{卵泡(ovarianfollicle)}

根据卵泡的发育时期或生理状态,可将卵泡分为原始卵泡(primordial follicle)、生长卵泡(growing follicle)和成熟卵泡(mature follicle)。其中生长卵泡要经历三个阶段,即初级卵泡(primary follicle)、次级卵泡(secondary follicle)和三级卵泡(tertiaryfollicle)或格拉夫卵泡(Graafian follicle)。

\subparagraph{原始卵泡}
原始卵泡由一个大而圆的初级卵母细胞(primary oocyte)和其周围的扁平上皮细胞(squamous epithelial cell)构成。在肉食动物、羊和猪的原始卵泡中,可能有2-6个 初级卵母细胞,是多卵卵泡(polyovular follicle)。初级卵母细胞大而呈圆形,核内染色质细小而分散,核仁大而明显。原始卵泡要在性成熟时才开始生长发育。

\subparagraph{初级卵泡}

大多数动物的初级卵泡是由一个直径约为20µm的初级卵母细胞和周围的单层立方卵泡细胞组成。胚胎期和出生后,卵巢中大多数是初级卵泡。各种动物在出生时,单个卵巢中的初级卵泡数从数十万到数百万个不等。但在一生中只有几百个卵泡 发育到排卵,大多数都退化。

\subparagraph{次级卵泡}

随着卵泡的进一步发育,在卵子周围形成多层卵泡细胞或颗粒细胞。卵母细胞外有一层$ 3\sim 5\mu m $厚的糖蛋白,称透明带(zona pellucida)。透明带是由紧贴卵母细胞的颗粒细胞和卵母细胞共同分泌的产物。由多层颗粒细胞和由其包围的初级卵母细胞构成次级 卵泡。随着卵泡的继续发育,在颗粒细胞间隙有少量的卵泡液出现。

\subparagraph{三级卵泡}

三级卵泡又叫有腔卵泡(antral follicle)。三级卵泡的特点是在其中央有一空腔,即\myInlineGlossary{卵泡腔}。卵泡腔是由次级卵泡的颗粒细胞间隙增大并融合形成的一个较大腔体,其中充满卵泡液。在卵母细胞中央有一球形的细胞核,染色质稀疏呈网状,核仁特别明显。细胞质中的高尔基复合体浓缩,位于细胞膜附近。随着卵泡腔中液体的增多,卵泡腔继续增大,卵母细胞移位远离卵泡中心,通常靠近卵泡的近卵巢中心部,此时称为\myInlineGlossary{格拉夫卵泡}。在卵母细胞与颗粒细胞层之间形成\myInlineGlossary{卵丘(cumulus cophorus)}。在较大的三级卵泡中,紧裹在卵母细胞周围的颗粒细胞形成的呈放射状排列的结构,称为\myInlineGlossary{放射冠(corona radiata)}。

\subparagraph{成熟卵泡}

当卵泡发育到快排卵时,其中的初级卵母细胞恢复并完成第一次减数分裂,排出第一极体(first polar body),形成次级卵母细胞。 但狗和马在排卵后完成第一次减数分 裂。 第一次减数分裂完成后接着进行第二次减数分裂, 但停滞在分裂的中期, 直到受精时才完成第二次分裂, 卵母细胞释放出第二极体(second polar body)。

\subparagraph{闭锁卵泡 (atretic follicle)}
动物出生后,卵巢中就有数以百万计的原始卵泡。 在个体发育过程中, 卵巢内大多数的卵泡不能发育成熟,在发育的不同阶段逐渐退化。初级卵泡退化时,首先是卵母细胞萎缩,进而卵泡细胞离散,结缔组织在卵泡内形成疤痕。次级卵泡退化时,卵母细胞核偏位、固缩;透明带膨胀、塌陷;颗粒细胞松散并脱落进卵泡腔;卵泡液被吸收,卵泡膜内层细胞增大,呈多角形,被结缔组织分割成团索状,分散在卵巢基质中并形成间\myInlineGlossary{质腺体}。

\subparagraph{黄体}
进入青春期后,卵巢开始排卵。刚排卵后的卵泡腔内由于充满血液和组织液,也称红体(corpus rubrurn)之后卵泡腔中的血凝块及具组织液被重新吸收。与此同时,颗粒细胞和卵泡膜细胞失去原有的形态特征,并取代红体而变为黄体(corpus luteum)。 随后,在黄体中出现结缔组织、脂肪、透明样物质(hyaline-like substance),细胞体积逐渐减小,最终只在卵巢表面形成一个不易观察到的黄体小疤,即由早期的红棕色变为白色或淡褐色,故称为白体(Corpus Albicans)。\myImportantPoint{黄体属于分泌腺,分泌的孕酮(progesterone)能刺激子宫腺体的分泌功能和乳腺发育}。\myImportantPoint{在妊娠期,黄体分泌的孕激素主要是维持动物的妊娠过程}。

\paragraph{不同动物卵巢的形态与结构特征}

人类的卵巢为扁椭圆形。青春期前,卵巢表面光滑;青春期后开始排卵,表面凸凹不平。成年妇女的卵巢大小约为$ 4cm \times 3 cm \times 1 cm $, 重约$ 5 \sim 6 $克,呈灰白色;绝经后卵巢变小变硬。卵巢表面无腹膜,由单层立方上皮疫盖,称生发上皮(germnalepithe­lium), 其内有一层纤维组织,称卵巢白膜(tunicaalbuginea)。白膜内为卵巢实质组织, 分为皮质和随质两部分。皮质在外层,其中有数以万计的原始卵泡及致密结缔组织;筋质在卵巢的中心部分,含有疏松结缔组织和丰富的血管、神经、淋巴管及少扯与卵巢悬韧带相连的平滑肌纤维。髓质内无卵泡。人在出生时,卵巢中有300000 -500 000个原 始卵泡。

马的卵巢呈豆形,平均长约为7.5cm,厚2.5cm,表面光滑,覆盖浆膜,借卵巢系 膜悬于腰下部、肾后方,其游离缘有一凹陷部叫排卵窝。马卵巢的最外层为髓质,内层力皮质,在排卵小凹处出现生殖上皮。

牛和羊的卵巢为稍扁的椭圆形,羊的较圆、较小,约为3.7cmX2.5cmX 1.5cm。般位千骨盆前口的两侧附近。未产母牛卵巢稍向后移,多在骨盆腔内;经产母牛卵巢位于腹腔内。性成熟后,成熟的卵泡与黄体可突出于卵巢表面。卵巢啦宽大。

猪的卵巢较大呈卵圆形,其所处的位趾、形状大小及卵巢系膜的宽度在不同年龄的 个体间有很大的差异。性成熟前小母猪的卵巢较小,约为0.4cmX 0. 5cm, 表面光滑,呈淡红色,位于荐骨呻两侧稍靠后方,由卵巢系膜固定。接近性成熟时,卵巢体积增 大,约为2cmX1.5cm, 系膜增宽,卵巢位宜稍下垂前移。性成熟后和经产母猪的卵巢更大,长约3-Scm,表面卵泡和黄体突出而呈结节状。卵巢系膜宽10-20cm,卵巢位于鹘结节前缘约4cm的横断面上,一般左侧卵巢在正中矢状面上,右侧卵巢在正中矢状面稍偏右。

狗卵巢皮质中有非常明显的皮质小管。皮质小管的管腔狭小,衬以立方上皮细胞; 卵巢的骼质部富含神经、许多大而卷曲的血管和淋巴管;髓质由疏松的结缔组织和平滑 肌组成,并与卵巢系膜中平滑肌相连。卵巢网位于髓质部,是由立方上皮细胞或是实质 细胞束相连而成的不规则的网络管道。


\subsubsection{卵巢的主要功能}

\myImportantPoint{卵巢的功能主要包括卵子的产生和生殖激素的分泌}。

\paragraph{生殖激素的分泌}

卵巢分泌的雌性激素主要有雌激素(estrogen)和孕酮。雌激素主要由存在于颗粒细胞内的芳香化酶作用于卵泡,由卵泡膜细胞产生雄激素,使雄激素芳香化而生成雌激素。孕酮主要由发情后期(metestrus)、间情期(diestrus)、妊娠期的黄体细胞和胎盘产生。雌激素诱异雌性生殖器官的生长、发育以及雌性动物的发情行为。

孕酮可刺激子宫腺的发育及分泌,使子宫内膜处于接受状态;可阻止卵泡的成熟和再次发情,使动物处于妊娠状态。

雌激素和孕激素在促讲乳腺的发育方面具有协同作用。卵泡的生长、成熟和雌激素分泌是在垂体促性腺激素---卵泡刺激素(FSH)和黄体生成素(LH)的调节下完成的。另外,卵巢雌激素的分泌又可以诱导排卵前LH的大量释放,进而引起排卵和黄体的形成。

\subsubsection{卵子的形成及其排卵}

在雌性性腺形成时, 原始生殖细胞已经存在。 随着卵巢的发育,形成初级卵母细胞, 周围的细胞形成单层扁平的卵泡细胞,并与初级卵母细胞共同构成原始卵泡。出生前,初级卵母细胞进入并停留在第一次减数分裂前期。 性成熟接近排卵时,卵母细胞才完成第一次减数分裂,即青春期后卵母细胞才完成第一次减数分裂。 随着卵泡的发育成熟,卵泡逐渐向卵巢表面移行并向外突出。当卵泡接近卵巢表面时,该处表层细胞变薄,最后破裂,出现排卵(ovulation)。排出的卵母细胞外包有放射冠,进入输卵管漏斗(infundibulum)。在大多数动物,卵母细胞进入输卵管与精子相遇时放射冠消散。 在反刍动物,排卵时放射冠就已丢失。\myImportantPoint{卵子保持受精能力的时间大约是一天}。 如果未受精,则被分解吸收。大多数动物是一侧卵巢排卵。 马的左侧卵巢排卵量为60\%, 而牛大约有$ 60\%\sim 65\% $的卵是由右侧卵巢排出的。


\subsection{输卵管}

输卵管是双侧、 弯曲的管道, 起始于卵巢并延伸到子宫角 (uterine horn), 由明显 的三部分组成: 伞部是一个较大的涌斗状结构, 又称输卵管伞;壶腹部(ampulla),其管壁较薄, 是输卵管伞后部的延伸;峡部 (isthmus) 是与子宫相连的一狭窄的管道。 输卵管的管壁由黏膜(mucosa)、肌膜(tunica muscularis)和浆膜(tunica serosa)三层构成,是在雌激素及其他因子的作用下由副中肾管发育而成。


\subsubsection{组织结构}

\paragraph{黏膜}

黏膜上皮属于单层柱状或假复层柱状上皮,由有纤毛柱状细胞和无纤毛柱状细胞构成。无纤毛柱状细胞有分泌功能,其分泌物为卵子提供营养;有纤毛细胞通常分布于输卵管起始端和末尾端。输卵管的固有膜与黏膜下层主要由疏松的结缔组织构成,其中含有许多浆细胞(plasma cell)、肥大细胞(mast cell)和嗜伊红细胞(eosinophil)。壶腹 部的黏膜上皮和黏膜下层高度折叠。如牛的壶腹部大约有40个纵行折叠,且每一个折叠又有二级和三级折叠。在峡部有$ 4\sim 8 $个大的折叠,但没有二级或三级。


\paragraph{肌层}

肌层主要由环行平滑肌组成,但也有少量的纵行肌和斜行肌。在伞部和壶腹部的肌层较薄;但峡部的肌层明显明显增厚,与子宫环行肌无明显的界限。

\paragraph{浆膜}

由疏松结缔组织构成,含丰富的血管和神经。

\subsubsection{功能}

输卵管主要作为卵子、精子(Spermatozoon)和早期胚胎运行的通道,同时也是生殖细胞停留、吸收营养和受精的部位。

漏斗部围绕卵巢,并由卵巢囊包裹(马无卵巢囊)。漏斗部的游离缘有指状突(即伞部)。动物在排卵期,伞部血管充血、肿胀。伞部随平滑肌节律性收缩而在卵巢的表面上部移动。这种结构和功能的变化有利于捕获由卵巢排出的卵母细胞;同时漏斗部上皮细胞的纤毛向子宫方向摆动,能将卵子运送到壶腹部。壶腹部是精子和卵子结合的部位。壶腹部上皮纤毛的运动和平滑肌的收缩共同参与卵子的运动。在峡部肌肉的收缩运动是推动受精卵向子宫方向运动的主要动力。峡部的纤毛运动也有利于受精卵的运动。随着发情周期的变化,峡部肌肉收缩的方向也在发生改变。在卵泡期,逆蠕动收缩将峡部腔内容物送到壶腹部;而在黄体期的分节运动逐渐推动受精卵向子宫方向运动。受精卵通过峡部的时间长短和妊娠时间无关,其通过峡部的时间大约为$ 4\sim 5 $天。

精子在输卵管内除精子自身的运动外,一定程度上也依赖于输卵管管壁肌肉的收缩和黏膜上皮纤毛的运动。

\subsection{子宫}

子宫(uterus)是一个中空的肌质性器官,通常有两个子宫角和一个子宫角;子宫是孕体(conceptus)着床(implantation)的地方。它是在雌激素的作用下由缪氏管发育来的。在发情(estrous)和生殖周期(reproductive cycles)中子宫壁经历一系列特定的变化;大多数动物的子宫由子宫角(cornua uteri)、子宫体(uterine body)、子宫颈(cervix uteri)三部分组成,子宫角与输卵管相连,子宫体与阴道相连的子宫颈连接,整个子宫通过阔韧带附着于盆腔和腹腔壁上,韧带中含有血管和神经,为子宫提供血液和神经支配。

根据子宫形态可将哺乳动物子宫分为:子宫呈一单管状的称单子宫(uterine simplex),如人和灵长类的子宫;在子宫体内腔前部无纵隔的称双角子宫(uterine cervix),如猪和马的子宫;在子宫体内腔前部有纵隔,并将其分开的叫双分子宫(uterus bipartitus),如牛、羊、骆驼等动物的子宫。小鼠和有袋类动物有两个子宫,称为双子宫(didelphia)动物,其中有袋类动物的双子宫有两个子宫颈,两个阴道,在个别的牛、羊、猪可发现有两个子宫颈和两个完全分开的子宫角,这种结构并不影响其生殖。

\subsubsection{子宫的形态结构}

\paragraph{子宫的形态}

子宫是一个肌质性的中腔器官,借助于子宫阔韧带附着于腰下部和骨盆腔的侧壁,大多位于腹腔内,少部分位于骨盆腔内,在直肠与膀胱间。多数家畜的子宫为双子宫,可分为子宫角、子宫体和子宫颈三个部分。子宫的大小、形状、位置和结构因动物的品种、年龄、个体的性周期以及妊娠阶段的不同而有很大的差异。

人的子宫位于骨盆腔中央,呈倒置梨形,前面扁平而后面稍突出。成年子宫重约50g, 长约$ 为7\sim 8 $cm,宽约$ 4\sim 5 $cm,厚$ 2\sim 3 $cm,子宫腔容量约为5ml。子宫上部较宽, 其隆突部为子宫底(fundus uteri) ,子宫底两侧为子宫角。子宫体下方为子宫颈,呈狭窄的圆柱状。

成年牛的子宫角较长,约为40cm,子宫体短,约为4cm;而子宫颈较长,约为 10cm。羊的子宫角长,约为15cm,外形像绵羊角;子宫体长为2cm;子宫颈长为4cm。牛和羊的子宫壁较厚且坚韧。

马的子宫呈“Y”形,子宫角稍曲成弓形,背缘凹,腹缘凸出而游离;子宫体长与子宫角长相当,子宫颈后端突入阴道内,有明显的子宫颈阴道部。

猪的子宫角特别长,经产母猪为$ 12\sim 15 $cm,子宫体短,约为5cm。仔母猪子宫角细而弯曲,子宫颈长13cm左右。子宫颈与阴道无明显的界限, 但其形成两个半圆形的黏膜褶,交错排列,使子宫颈管呈狭窄的螺旋形。

\paragraph{子宫的结构}

子宫由子宫内膜(endometrium)、子宫肌层(myometrium)和子宫外膜 (perimetrium)三层构成。

\subparagraph{子宫内膜}

子宫内膜位于子宫腔面,由上皮和固有层组成。灵长类、哨齿类、马、犬、猫等动物的上皮为单层柱状上皮,而猪和反刍动物的上皮为单层柱状或假复层上皮。固有膜由富含血管的胚性结缔组织组成。 固有膜的浅层有较多的细胞成分,主要为星状细胞。 这些细胞借突起互相连接。此外还有巨噬细胞和肥大细胞。 固有膜的深层, 细胞成分较少, 有子宫腺分布。 大多数动物的子宫腺为弯曲、 分枝管状腺,管状腺的内层是由有绒毛的单层柱状上皮或无绒毛的单层柱状上皮细胞组成。子宫腺的密度因动物的种类、 胎次和发情周期的不同而有差异。

人及灵长类固有层的浅层为功能层,深部较薄的为基底层。在月经周期中,功能层发生部分或全部丢失,而基底层在整个生殖周期都存在。当功能层丢失后,可以从深部的基底层得到恢复。

子宫阜(caruncle)是反刍动物子宫的一种特殊结构,是位于子宫体和子宫角黏膜上特殊的圆形隆起,富含成纤维细胞和血管。反刍动物的子宫阜部位没有腺体。每侧子宫角有四排子宫阜,每排大约15个子宫阜;牛的子宫阜为圆顶状,羊为杯状。子宫阜为胎膜与子宫壁的结合部位。

\subparagraph{子宫肌层}

子宫肌层是子宫最厚的一层,它主要由内层环行平滑肌和外层的纵行平滑肌构成。在两层平滑肌之间及平滑肌深部有大量的动脉血管、静脉血管和淋巴管。

\subparagraph{子宫外膜}
由疏松的结缔组织组成, 其外覆盖腹膜间皮(peritoneal mesot helium)这一层也有平滑肌细胞、血管、淋巴管和神经纤维。

另外,子宫颈是连接阴道和子宫的个通道,主要由平滑肌、弹力纤维、血管等构成;黏膜下层高度折叠,有二级和三级折叠,母牛子宫颈有四个环行折叠和$ 15\sim 25 $个纵行折叠,每一个折叠都有二级和三级折叠存在,这种折叠可能被错认为腺体结构。大多数动物的子宫腺不会延伸到子宫颈,在子宫颈的腺体是黏液腺。

\subsubsection{功能}

肌层是子宫的三层结构中最厚的一层,在这一层中丰富的血管为子宫提供营养。在妊娠期,子宫肌层中平滑肌细胞的长度逐渐增长,对胎儿的发育和产出都有重要作用。

子宫内膜中有子宫腺,能分泌多种物质,除对胎儿具有重要的营养作用外,对于胚胎着床、妊娠识别和胎儿的存活与发育起着重要的调节作用。在食肉类动物,子宫内膜分泌活性的改变将导致胚胎着床的延迟。在啃齿类,由子宫分泌白血病抑制因子、降钙素等因子,对子宫接受性的建立和胚胎的着床起作用。在猪、牛、马和羊等家畜,子宫内膜的分泌物影响到胎儿存活和胚胎的发育。利用孕酮抑制母羊子宫内膜腺的分化,或由其他疾病引起的子宫内膜的纤维化,均可导致不孕、早期胎儿的死亡或早期流产等。


\subsection{阴道}

\subsubsection{阴道形态与大小}

阴道从子宫颈延伸到阴道前庭,是雌性动物的交配器官和胎儿产出的通道。

人的阴道位于骨盆下部的中央,上端包围子宫颈,下端开口于阴道前庭后部,上端较下端宽;前壁与膀胱和尿道相接,后壁与直肠紧贴,前壁长约$ 7\sim 9 $cm,后壁长约$ 10\sim 12 $cm,平时阴道前后壁相互贴近。阴道黏膜淡红色,并且受性激素的影响而有周期性的变化。牛的阴道长$ 20\sim 25 $cm,妊娠母牛阴道可增至30cm以上。阴道壁较厚, 阴道穹隆呈半环状,仅见于阴道前端的背侧和两侧。牛的阴瓣较不明显,在尿道外口的腹侧有一尿道下憩室。马的阴道长约$ 15\sim20 $cm,阴道穹隆呈环状;母马驹的阴瓣发达,经产的老龄母马的阴瓣常不明显。猪的阴道长约$ 10 \sim 12 $cm,肌层较厚,直径小;黏膜一
有皱褶,不形成阴道穹隆。阴瓣为环形皱褶,阴蒂细长,突出千阴道窝的表面;尿生殖前庭腹侧壁的黏膜形成两对纵皱,前庭许多开口位于纵褶之间。

\subsubsection{组织结构}

阴道壁由黏膜层、肌层和外膜(tunica adventitia)或浆膜组成。

\paragraph{黏膜}

阴道黏膜上皮为一复层扁平上皮,在发悄前期和发情期增厚。固有层和黏膜下层(propria scubmucosa)由疏松结缔组织或致密的不规则的结缔组织组成;阴道后部的固有层有少量的淋巴结。阴道黏膜受性激素影响而有周期性的变化。绝经后的妇女的阴道粘膜上皮甚簿,皱襞少,伸展性小,易创伤而感染。

\subsubsection{肌层}

由厚的内环肌层和较薄的外纵行平滑肌组成,内环肌由结缔组织分成束状。

\subsubsection{浆膜}

浆膜层为疏松的结缔组织, 有大植的血管、 神经和神经节组成。外部纵行肌可看作是浆膜的 一部分。

\subsection{前庭、 阴蒂、 阴门}

\paragraph{前庭}

前庭与阴道以处女膜(hymen)为分界;前庭壁上有尿道开口(orifices of the ure­thra)和大小前庭腺。除黏膜深部有较多的淋巴小结外, 前庭壁类似于阴道后部, 特别是在阴蒂, 有更多的上皮下淋巴结存在。发生炎症时可能影响生殖。

前庭大腺体与雄性动物的尿道球腺(bulbourethral gland)同源。前庭大腺体是一种复合的管泡黏液腺, 位千黏膜层深部, 末端的分泌泡含有大的黏液细胞,与腺泡相连的 小管内衬柱状黏液细胞和杯状细胞;大管直通前庭, 内衬一层厚的复层鳞状上皮细胞, 分散或是聚集的淋巴结位于大管周围。在交配时, 腺体受压而释放黏液以润滑前庭。前庭小腺体比较小, 在大多数动物的前庭黏膜中有散在分布的管状分枝黏液腺,衬以复层 扁平上皮细胞。

\paragraph{阴蒂}
阴蒂位于前庭尾区远端, 由阴蒂海绵体、 阴蒂头(glans clitoridis)和阴蒂包皮(preputium clitoridis)组成。阴蒂海绵体与阴茎海绵体的结构相似。阴蒂头与阴茎头 (glans penis) 同源。阴蒂包皮是前庭黏膜的延续, 有侧壁和内脏层, 内脏层有大拭的神 经末梢。

\paragraph{阴门}

阴门由大小唇构成,并有皮肤覆盖。皮肤上有汗毛(finehair)。真皮下有使阴门发生收缩的横纹肌纤维(striatedmuscle fiber),阴门部有丰富的小血管和淋巴管。动物在发悄期阴门充血肿胀, 局部温度升高。


\section{雄性生殖器官}

雄性动物的生殖器官包括成对睾丸(testis)、附睾(epididymis)、输出管(efferent duct)、尿道(urethra)、阴茎(penis)、包皮(prepuce)及附性腺(accessory gland)。附性腺包括精报腺(seminalvesicle)、前列腺(prostategland)和尿道球腺(bul­bourethral gland)。雄性动物的生殖系统参与完成精子的发生和成熟,并将精子释放到雌性动物生殖道中。

\subsection{睾丸}

睾丸是雄性动物最为重要的生殖器官,主要由曲细精管和间质构成。曲细精管由支持细胞(sertoli cell)和各级生精细胞组成,前者对生殖细胞具有支持、保护和营养作用,后者包括精原细胞、初级精母细胞、次级精母细胞、精子细胞和精子,它们分别处于不同的发育阶段。构成睾丸的间质包含有动、静脉血管和睾丸间质细胞(leydig cell),血管主要为睾丸提供营养、调节温度和排除代谢产物;睾丸间质细胞分泌雄激素,为精子的发生提供一个合适的激素环境。各种动物睾丸结构模式如图 \ref{figure_testis_of_several_animals} 所示。

\begin{figure}
\centering
\myFigurePlaceholder
\caption{几种动物睾丸和附睾的结构模式图}
\label{figure_testis_of_several_animals}
\end{figure}

\subsubsection{睾丸的发育}

在脊椎动物,卵巢和睾丸都是由相同的原基组织---生殖嵴发育和分化而来。在胚胎发育早期,未分化性腺在形态方面无明显的性别间差异,并具有分化成睾丸或是卵巢的潜能。尽管哺乳动物间睾丸发育的过程各有不同变化,但整个过程可分为四个阶段:胚胎期睾丸的分化发育、发育的睾丸下降到阴囊、胎儿期睾丸的生长发育及青春期前后睾丸的成熟。 下面主要介绍睾丸下降与隐睾及出生后睾丸的发育。

\paragraph{睾丸下降与隐睾}
睾丸下降(descent of the testis)是指睾丸从其分化形成的部位, 即第 $ 16 \sim 24 $体节处经腹腔迁移到阴囊的过程。人睾丸的下降有两个明显阶段:第一阶段是在妊娠第$ 8 \sim 15 $周,发生穿过腹部的相对移位;第二阶段是腹股沟向阴囊的迁移。 下丘脑---垂体---性腺轴 (hypothalamo-pituitary-gonadal axis) 的正常发育是睾丸正常下降所必需的。已证实,睾丸的下降与\myHumanGene{Insl3} 基因(insulin-like 3 gene) 、 雄激素(androgen)、 G-蛋白耦联受体(G-protein-coupled receptor )、 CGRP (calcitonin gene - related peptide)因子、同源框基因 (homeobox gene)和雌激素等有关。

各种家畜睾丸下降到阴囊的时间分别为: 马在怀孕$ 9 \sim 11 $个月, 牛是在怀孕$ 3.4 \sim 4 $个月,羊为怀孕80天,猪为怀孕90天,骆驼睾丸是在出生时下降,而狗和猫分别是在出生后5天和$ 2 \sim 5 $天。 猪、 马、 狗发生隐睾比较普遍, 而在牛、 羊和鹿比较少见。

 人的睾丸最早都在腹腔内, 胎儿发育到第9个月时睾丸逐渐从腹腔下降到阴囊里。 一般出生时, 90\%左右的男孩睾丸已下降到阴囊; 出生后有些男孩的睾丸继续下降, 1周岁时有$ 95\% \sim 97\% $男孩的睾丸到达阴囊;只有少数男孩到青春期时, 睾丸才下降到阴囊。
 
在人和多数高等哺乳动物中, 睾丸的正常功能, 特别是产生精子的功能, 与温度有密切关系。 睾丸产生精子要求一个温度低于正常体温的环境。 睾丸的正常位置是在腹腔外的阴囊。 如果一侧或是双侧睾丸未下降到阴囊而停留在腹腔则称隐睾 (cryp­torchidism)。到目前为止,隐睾发生的机理还不十分清楚。 尽管隐睾仍产生雄激素, 但不能产生正常的精子。双侧隐睾的动物无生殖能力。

\paragraph{出生后睾丸的发育}

从出生到青春期,睾丸都处于连续的发育过程, 其中包括睾丸体积逐渐增大和功能的完善。

在灵长类, 幼儿期的睾丸有一个相对较长而变化不明显的时期, 传统认为这是睾丸的静止期。实际上,从出生后到幼儿早期, 睾丸的体积在短期内快速增大,这主要是由于支持细胞的快速增殖和生精索的增长。与此同时, 睾丸间质细胞在胎儿出生后又开始增殖、睾酮分泌增多。睾丸的这种活动在幼儿后期逐渐消退,这与下丘脑---垂体---睾丸轴的调节功能的变化相一致。

青春期开始后, 睾丸在形态和功能方面发生显著的变化:一是睾丸间质细胞的分泌功能加强, 二是生殖细胞的分化、 发育或退化连续发生。这两方面的变化与成熟精子的生成有关。在这个阶段睾丸的快速生长主要表现在曲细精管直径的增大;支持细胞有丝分裂停止, 但它继续分化形成具有完整结构的睾丸屏障。因此, 睾丸在青春期前的正常发育与成年动物的生育能力有密切关系。

\subsubsection{睾丸的结构}

睾丸的组织结构如图 \ref{figure_structure_of_human_testis_and_epididymis} 所示。

\begin{figure}
\centering
\myFigurePlaceholder
\caption{人睾丸与附睾结构}
\label{figure_structure_of_human_testis_and_epididymis}
\end{figure}

\paragraph{鞘膜(tunicavaginalis)}

鞘膜分为壁层(parietal layer)和脏层(visceral layer)。壁层与阴囊紧密相贴,而脏层包裹睾丸表面,由间皮(mesotheliurn)和结缔组织构成,不易与白膜(tunicaalbug­inea)分离。

\paragraph{白膜}

白膜是一层致密结缔组织囊,主要由胶原纤维(collagenfiber)和少许弹性纤维(elastic fiber)构成。在成年公马、公猪和绵羊还可见平滑肌细胞。大量分支的动脉血管和静脉血管构成血管层(vascular layer)。羊和狗的血管层位于白膜的表层,而猪和马的血管层位于白膜的深层。

\paragraph{睾丸隔膜}

白膜的结缔组织深入睾丸实质后,将睾丸分成多个小叶(lobule),称为睾丸隔膜,其组成成分和白膜相同。睾丸隔膜的结缔组织进入睾丸小叶内称睾丸小梁,与睾丸隔膜相连。猪和狗的睾丸小梁较厚且是完整的隔板,反刍动物和猫较簿而且不完整。每一小叶内有$ 1\sim 4 $个曲细精管。曲细精管外有基膜。

\paragraph{睾丸纵隔(mediastinumtestis)}

在睾丸的中央,隔膜与睾丸的疏松结缔组织相连而形成。在大多数家畜,睾丸纵隔占据睾丸中央位置。

\paragraph{间质细胞(interstitial cell或Leydig cell)}

在睾丸的曲细精管间有血管、淋巴管、成纤维细胞、游离单核细胞和间质内分泌细胞,即睾丸间质细胞。睾丸间质细胞产生睾丸雄激素,但公猪和公马同时也产生大量的雌激素。间质细胞因动物的种类、年龄的不同会有很大的变化。成年公牛的间质细胞约占睾丸体积的70\%,而公猪的间质细胞较多,占整个睾丸体积的$ 20\%\sim30\% $,公马的间质细胞也较多。

睾丸间质细胞成束状或簇状存在,相邻的睾丸间质细胞由微管和间隙连接相连。在睾丸组织和淋巴中有高浓度的类固醇。睾丸间质细胞的形状不规则,呈多面形、细胞核为椭圆形。所有动物的睾丸间质细胞都含有大量的脂类,滑面内质网上有类固醇脱氢酶(steroid dehyrogcnase),线粒体数量也增多,内有许多管状嵴,它们参与睾酮的合成。睾酮在细胞内的存储及释放并不引起该细胞形态上的明显变化。

\paragraph{曲细精管(convolutedseminiferous tubule)}

曲细精管为卷曲的、直径为$ 200 \sim 400\mu m $的小管,曲细精管上皮主要是由支持细胞和生精细胞所构成。

\subparagraph{支持细胞}
该细胞来源于青春期前未分化的性腺支持细胞。这种支持细胞有丝分裂活动强, 含有大量的粗面内质网,合成的抗缪勒氏管激素可抑制输卵管、子宫、阴道在雄性个体中的发育。在青春期支持细胞分化并伴有形态改变,最后失去有丝分裂能力。在成年动物中,支持细胞不规则地排列在生精小管基膜上,横切面由$ 25\sim 30 $个支持细胞组成。支持细胞含有脂类物质和糖元,大量的微丝、微管和滑面内质网,细胞顶端表面呈锯齿状。另外,位于基膜上细胞的紧密连接为精原细胞的减数分裂和精子发生提供了相对稳定的内环境,睾丸的这一特殊结构称之为血---睾屏障(blood-testisbarrier)。

支持细胞的功能包括: 
\begin{enumerate}
\item 营养、 支持、 保护生精细胞及精子;
\item 吞食退化精子和精子脱落的残体;
\item 参与FSH对生殖细胞的调节作用,产生雄激素结合蛋白;
\item 分泌含有钾、肌醇(inositol)、 谷氨酸铁转运蛋白(glutamate transferrin)等管腔液成分;
\item 分泌抑制素 (inhibin) 
\end{enumerate}s等。

\subparagraph{生精细胞(spermatogeniccell)}

生精细胞是指发育过程中处于不同阶段的精细胞,这些细胞位于支持细胞间和该细胞之上。从精原细胞发育到精子的一系列过程称为精子发生(spermatogenesis), 一般分为三个阶段:精母细胞的发生(spermatocytogencsis),即精原细胞(spermatogonium)发育为精母细胞(spermatocyte)的过程;精母细胞成熟分裂形成单倍体精子细胞 (spermatid);精子形成(spermiogenesis) , 即精子细胞发生形态变化的过程。

\paragraph{直细精管(Straight testicular tubule)}

直细精管较短,可以认为是曲细精管的延续,与睾丸网相连,在马和猪,有些曲细精管终止于睾丸周边,并通过长距离的直细精管与睾丸网相连。在曲细精管的终末段内衬变形的支持细胞,所有的精子必须通过这些变形细胞间的狭缝进入直细精管。曲细精管的终末段的这种结构起到类似阀门的作用,防止睾丸网中的液体逆流到曲细精管而影响精子的发生。直细精管内衬单层扁平或柱状上皮细胞。另外,有大量的巨噬细胞和淋巴细胞,能够吞噬精子。

\paragraph{睾丸网(retetestis)}

睾丸网为直细小管进入睾丸纵隔内分支吻合成网状的管道,管腔内衬单层扁平或柱状上皮细胞,弹性纤维和收缩细胞位于上皮下。大部分睾丸液是由睾丸网产生的。

\subsubsection{睾丸的功能}

\myImportantPoint{睾丸的功能主要包括精子的生成和雄性激素的分泌}。人的精子从精原细胞开始到成熟大约需要$ 65\sim 70 $天。在绝大多数动物,一般睾丸的生精功能从青春期开始后,可以持续终生。一个初级精母细胞经过减数分裂后产生四个单倍体精子。睾丸产生的雄激素主要促使雄性动物第二性征的出现和生殖功能的维持。

\subsection{附睾}

哺乳动物的附睾是由$ 8\sim 25 $根输出小管和一条长而卷曲的附睾管组成。附睾分为头、体、尾三部分,其外有致密结缔组织所构成的白膜和鞘膜血管层。

在公马的白膜层有少量平滑肌散布于整个致密结缔组织中。

附睾不仅仅是一个精子运输通道,还是精子浓缩、获得运动能力和受精能力以及储存精子的部位。

\subsubsection{附睾的发育}

\myImportantPoint{附睾是由中肾管发育而来的}。在中肾退化进程中,除了输出管外的其他中肾小管完全退化,留下一光滑管通过腹膜与睾丸疏松地连接;中度卷曲的输出管逐渐变得高度卷曲,并穿过薄薄的睾丸系膜进入附睾头侧。60日龄牛胚胎的附睾为一直管,胚胎发育到80天时附睾管开始卷曲成环状,这种环状结构再次生长并卷曲,最终形成高度卷曲的附睾管。所有动物的附睾都可分为三个部分,位于睾丸前端的是附睾头,附睾体沿睾丸侧缘发育,在睾丸后端是附睾尾。附睾的这三个部分没有明显的分界线。

\subsubsection{结构与功能}

\paragraph{附睾的结构}

主要由输出小管(ductuli efferentes)和附睾管(ductus epididymidis)组成(图 \ref{figure_structure_of_human_testis_and_epididymis} )。

\subparagraph{输出小管}

由$ 8\sim 25 $根输出小管将睾丸网与附睾管连在一起。这些小管聚集在一起并形成具有明显组织界限的小叶。输出小管主要由有纤毛或无纤毛柱状细胞构成,另有淋巴细胞散布于上皮基部。无纤毛细胞上还有微绒毛(microvilli)和发育完全的内吞细胞器.它由被膜小泡(coatedvesicle)、微泡内陷膜、微小管(microcanuliculi)和吸收空泡(resorptive vacuole)组成。

\subparagraph{附睾管}

以组织学、组织化学和超微结构为标准,可将附睾管分为六段,但因动物种类不同而有不同的特点。附睾管极度的弯曲和盘绕,其长度因动物的不同而不等,公牛和公猪的附睾管长约40m,而公马约为70m。另外,精子完全通过附睾管的时间也因动物的不同而有不同,大多数哺乳动物大约需要$ 10\sim 15 $天的时间。附睾管衬以假复层上皮细胞,外被少量疏松结缔组织和环平滑肌纤维,在接近附睾尾时,平滑肌数量明显增多。在上皮中有柱状主细胞(principal cell)、小的多角基底细胞、顶细胞(apical cell)和透明细胞(clear cell)。在大多数动物中,还有巨噬细胞(macrophage)和淋巴细胞(lymphocyte)。在附睾头部的主细胞数量常常较其他部位多。柱状细胞的端面有一长的,且有时是分支的微绒毛,在靠近尾部时微绒毛逐渐变短。

\subsubsection{附睾的主要功能}

\paragraph{输出小管}

构成输出小管的两种细胞的结构特点决定了它的功能主要是运送精子和对小管液重吸收。输出小管的无纤毛细胞上有微绒毛和发育完好的内吞细胞器,有利于对小管液重吸收,从而使管腔中的内环境处于相对的稳定状态,以利于精子的成熟。另外,有纤毛细胞中纤毛的运动可推动精子向附睾管运动,利于精子的成熟与排出。

\paragraph{附睾管}

从睾丸产生的液体大约有90\%被输出管和附睾管吸收。睾丸支持细胞产生的雄激素结合蛋白也是在附睾管的起始段被吸收。附睾管的另一功能是精子在附睾中逗留期,使精子发生下列形态和功能上的变化:
\begin{itemize}
\item 完善其运动性;
\item 代谢改变;
\item 胞质膜改变(膜上有在受精时发挥识别能力的分子);
\item 巯基团的结合增加了膜的稳定性;
\item 失去胞质中的多余小滴。
\end{itemize}


\subsection{输精管}

输精管是附睾管的延续。附睾管在附睾尾部突然弯曲,然后逐渐变直。输精管的初始段位于精索中,腹腔段位于腹膜折叠中。马和反刍动物的输精管与精囊的排出管相连而构成短的射精管,开口于精阜(colliculu seminalts)进入尿道;猪的输精管和精囊排出管分别开口于尿道;\myImportantPoint{肉食动物没有精囊腺},输精管直接与尿道相接。

输精管黏膜上皮为假复层柱状上皮,在该管尾部则变为单一的柱状上皮。固有膜及黏膜下层为疏松结缔组织以及大量的弹性纤维。输精管终末段的固有膜---黏膜层都包含管泡状腺体。肌层包括内层的内环行肌、外层的纵行肌和少量的斜肌肉。

输精管是附睾管的延续,主要是将精子从附睾中输出。另外,输精管线体段的分泌物有利于精子的运动和生存。

\subsection{副性腺}

大多数动物的副性腺包括精囊腺、前列腺、尿道球腺等。肉食动物缺少精囊腺精囊腺。


\subsubsection{精囊腺}

成对的精囊腺属于复合管状腺或管泡状腺。腺上皮是假复层柱状上皮,由高柱状细胞和小而圆的基地细胞组成。叶内小管主要是立方上皮或柱状上皮。黏膜下层富含血管以及精囊腺小叶。

精囊腺的分泌物为白色或黄色胶状液体,一般占射精量的$ 25\%\sim 30\% $,富含果糖,具有\myImportantPoint{为精子提供能量和稀释精子}的功能。

\subsubsection{前列腺}

前列腺的数量不定,位于盆腔部的尿道上皮,属于单管腺泡。根据其局部解剖可将前列腺分为致密部(或叫外部)和扩散部(或叫内部)两部分。整个前列腺由富有平滑肌纤维的结缔组织包裹,平滑肌在致密部较多。被膜的结缔组织伸入腺体内形成的肌质小梁,将腺体分隔成多个单独的小叶。

分泌管、分泌泡以及腺体内小管由单一的立方或是柱状上皮细胞组成,有时可见基地细胞。大部分细胞都含有蛋白质分泌颗粒。高柱状细胞具有微绒毛,有时还可见水泡样顶部凸起。

前列腺的分泌物为黏稠的蛋白样分泌物,偏碱性。\myImportantPoint{前列腺分泌物的作用主要是中和精液和刺激精子的运动}。

\subsubsection{尿道球腺}

成对的尿道球腺位于尿道球状部的背部两侧。尿道球腺外面包有致密的结缔组织,其中含有横纹肌纤维。被膜伸入腺体的实质,将腺体分隔成多个单独的小叶。叶间结缔组织亦含有横纹肌和平滑肌纤维。尿道球腺的分泌部由高柱状上皮细胞构成,有时可见基底细胞。腺管直接开口于集合管或由单一立方上皮组成小管与集合管相连。集合管由单一立方或柱状上皮组成,多个集合管组成较大的腺内导管。

尿道球腺的黏液和蛋白样分泌物在射精时先流出,具有\myImportantPoint{中和尿道内环境、润滑尿道和阴道}的作用。

\subsection{尿道}

雄性动物的尿道分为三段,第一段从膀胱起到前列腺的后缘,第二段从前列腺后缘到阴茎的球状部,第三段是阴茎的海绵体部分,即从阴茎球状部到尿道外开口。整个尿道黏膜呈现纵向折叠,但在阴茎勃起或排尿时纵向折叠消失。尿道上皮主要是由单层柱状上皮、复层柱状上皮或立方上皮组成的移行上皮。固有膜和黏膜下层由致密结缔组织、弹性纤维、平滑肌和弥散的淋巴细胞组成。尿道的第一段和第二段衬以血管层,所以尿道亦与动物生殖器的勃起(erection)功能有关。

\subsection{阴茎}

阴茎由阴茎海绵体(corpora cavernosa penis)和龟头(glans penis)组成。

成对的阴茎海绵体在坐骨结节部融合成阴茎体,外被一层致密的结缔组织白膜。该结缔组织膜含有弹性纤维和平滑肌细胞。阴茎内由完整的结缔组织隔膜分隔的阴茎海绵体。在白膜与小梁间充满海绵状的勃起组织,海绵腔内衬内皮组织,有大量的血管和神经分布。这种结构主要与阴茎的勃起有关。

供应阴茎血液的血管主要是螺旋动脉(helicine artery)。螺旋动脉管壁的平滑肌细胞的舒张可导致大量血液进入海绵体空腔中。海绵体内血液的增多压迫静脉管壁,使海绵体中流出的血液减少,最终血液充满海绵体导致阴茎的勃起。当螺旋动脉中平滑肌细胞收缩时,动脉血流减少,充血的静脉逐渐恢复到原有的状态。
\chapter{精子发生与成熟}

在有性生殖的生物体中,细胞可以分为两大类:体细胞(Somatic Cell)和生殖细胞(Germ Cell)。有性生殖周期是体细胞与生殖细胞相互转变的过程。在胚胎发育的早期,部分体细胞可分化为生殖细胞。生殖细胞在分化过程中发生减数分裂,由二倍体的细胞产生单倍体细胞:精子和卵子。精、卵通过受精重新形成二倍体的细胞,开始下一轮的生命周期。

在高等生物的机体内,只有一小部分细胞为生殖细胞,然而它们是正常生命周期中的一个关键环节。二倍体细胞在形成单倍体细胞的过程中,发生了简述分裂。期间,同源染色体DNA发生重组,产生的每个单倍体生殖细胞都含有不完全相同的基因组合。因此,精卵结合形成的子代在遗传学上互不相同,并且也不同于他们的亲代,这一生殖模式最大的优点是保持着物种的多样性。

在多数物种中,只有两类生殖细胞:卵子和精子。这两种细胞具有很大的区别,卵子是机体中最大的细胞,而精子通常最小。这种结构最适合于所带基因的扩增。卵子是不运动的,但通过提供大量生长大于所需要的原料来帮助保存母本基因;与此相反,精子通常具有较强的运动能力,为流线形,以适应有效的受精,它通过利用母本资源来扩增父本基因。

精子起源于原始生殖细胞。在胚胎发育的早期,少数细胞形成配子(gamete)的前体,称之为\mykeyword{原始生殖细胞(PGC,primordial germ cell)}。其后,原始生殖细胞迁移到早期的性腺---\mykeyword{生殖嵴(Genital ridge)},在那里进行一段时间的有丝分裂增殖,然后部分细胞进入减数分裂,并进一步分化成成熟的配子:卵子或精子。睾丸是精子发生的场所,在这里PGC发育成\mykeyword{原始精原细胞(primitive spermatogonium)},进入精子发生过程,经过简述分裂及一系列形态变化,最后形成特殊形态的完整精子。在曲细精管中形成的精子并没有完全成熟,需要进入附睾,在附睾管运行过程中,吸收多种物质,发生一些列形态、生理和生化方面的变化,完成成熟过程,形成具有一定活力的精子。

\section{精子发生}

精子发生(spermatogenesis)是指精原细胞(spermatogonium)经过一些列的分裂增殖、分化、变形,最终形成完整精子(spermatozoon)的过程。这一过程是在雄性生殖腺(睾丸)的曲细精管(seminiferous tubule)中进行的。精子发生可分为三个时期:有丝分裂期、减数分裂期和精子形成期(spermiogenesis)。精子发生是一个特殊的细胞分化过程,在这一过程中发生了许多特殊的时间,如减数分裂,形态变化等。

\subsection{精子发生的场所}

\subsubsection{睾丸}

在雄性个体中,生殖嵴发育为雄性生殖腺---睾丸(图 \ref{figure_structure_of_testis})。睾丸是雄性生殖器官,椭圆形,其背面表层与附睾相连。睾丸是精子发生的场所,附睾为精子成熟的器官。睾丸外由一层囊膜包裹,囊膜为致密坚硬的结缔组织,囊膜向内延伸把睾丸分割为 名个分隔间,分隔间充满了弯曲的上皮性管道,称为曲细精管。附睾附着于睾丸的背面,其间有输出小管(ductuli efferentes)相连,在睾丸中产生的完整精子通过输出小管进入附睾, 在附睾中成熟, 变为成熟的精子排出。

\begin{figure}
\centering
\myFigurePlaceholder
\caption{睾丸结构}
\label{figure_structure_of_testis}
\end{figure}

在胚胎期发育、 分化阶段及出生后, 睾丸行使两种功能:分泌激素(睾酮及其他类固醇激素, 在胚胎期也分泌)以及在成年期产生精子。经过胚胎期及出生后期的一个准备阶段后, 精子发生开始于青春期。

\subsubsection{曲细精管}

睾丸的实质部分为弯曲的曲细精管。曲细精管周边为结缔组织薄层,由弹性纤维及一些平滑肌细胞组成。管壁由两类细胞组成:\myGlossaryEntry{sertoli_cell}及各期的生精细胞(spermatogeniccell)。生精细胞根据它们的发育阶段有规律地排列成多层,这一结构称之为生精上皮(spermatogenicepithelium或serminiferousepithelium)。生精细胞包括精原细胞、初级精母细胞、次级精母细胞、圆形精子细胞及长形精子细胞。它们由曲细精管的基底部向管腔排列(图 \ref{figure_structure_of_serminiferous_epithelium})。

\begin{figure}
\centering
\myFigurePlaceholder
\caption{曲细精管结构示意图}
\label{figure_structure_of_serminiferous_epithelium}
\end{figure}

\subsubsection{生精上皮}

生精上皮由支持细胞及不同阶段的生精细胞高度有序排列组成, 其组织的复杂性是上皮中独有的,包括非增殖状态的支持细胞及各期生精细胞, 从位于基底的精原细胞到管腔部分的精子细胞。支持细胞占成年生精上皮的25\%,生精上皮的基底膜由扁平的肌样细胞、 成纤维细胞及胶原纤维组成(图 \ref{figure_cell_alignment_of_serminiferous_epithelium})。

\begin{figure}
\centering
\myFigurePlaceholder
\caption{曲细精管结构示意图}
\label{figure_cell_alignment_of_serminiferous_epithelium}
\end{figure}

\subsubsection{支持细胞}

除了生殖细胞,生精上皮还包含一群支持细胞,它们比生殖细胞大得多,而且形态复杂。它们不再分裂,支持着整个上皮,附着于生精上皮的基底层,并穿过生殖细胞间伸向官腔,为精子发生提供了一个合适的环境。在老年的睾丸中存在着多核支持细胞,表明它们可能在一定情形下恢复了分裂能力,发生了没有进行胞质分裂的有丝分裂,导致形成多核的支持细胞。支持细胞的分布是随机的,数量是恒定的。


如图 \ref{figure_structure_of_sertoli_cell} 所示,支持细胞有一个多形核,有丰富的细胞质和细胞器,包括丰富细长 的线粒体、一个很大的高尔基体、丰富的内质网,一些溶酶体、微丝和微管。在大鼠生精上皮周期的$ 12\sim 14 $期,支持细胞中线粒体的体积增大,同时伴随着大量脂质形成,以及内吞活性显增多,反映了生精上皮周期中支持细胞对能量需求的变化。已证明在大鼠中支持明细胞的内吞活性具有周期性变化的特征。支持细胞的骨架发达,在同一支持细胞中的不同区域,构成细胞骨架的胞质微丝及微管的数量及分布变化很大。微丝主要分布于细胞核周围及细胞基底部。

\begin{figure}
\centering
\myFigurePlaceholder
\caption{曲细精管结构示意图}
\label{figure_structure_of_sertoli_cell}
\end{figure}

\subsubsection{支持细胞}

支持细胞间存在着多种形式的连接, 其中紧密或闭缩连接 (occluding junction) 是指相邻的支持细胞膜互相融合形成一种特化的连接复合体 (junctional complex)。它是 血---睾屏障 (blood-testis barrier) 的结构成分, 这一结构把生精上皮分为两个分隔间: 基底间 (basal compartment),为精原细胞和前细线期的精母细胞的场所;血---睾屏障的管腔一侧为中央间 (adluminal compartment),包含处于减数分裂的精母细胞及精子细 胞。 血---睾屏障解释了在管液和血浆中化学物质的不同,也是在两个隔间中存在不同基质的结构基础, 它在精子发生的特定阶段中起关键作用。 它们的重要性还有待研究,然而一股认为, 基底间可直接接受血液中的激素, 而中央间则通过支持细胞接受激素和营养物质, 解释了基底间中的细胞更容易受到激素水平的影响。 通过闭缩连接进行的物质 运转依赖于物质分子大小和物理性质, 对物质的转移具有筛选作用。

\subsection{精子发生过程}

精子发生是一个复杂而有规律的细胞分化过程。从精原细胞的分裂增殖、精母细胞的减数分裂到精子细胞变态分化和运行至附睾的成熟过程中,都受到众多基因和激素的协同调控。精子发生过程可分为三个主要的阶段(图 \ref{figure_spermatogenesis})。

\begin{figure}
\centering
\myFigurePlaceholder
\caption{精子发生示意图}
\label{figure_spermatogenesis}
\end{figure}

\subsubsection{精原细胞的有丝分裂期}

精原细胞由原始生殖细胞分化而来, 其增殖能力增强, 为进入减数分裂做准备。它通过有丝分裂产生两类细胞, 一类不进入精子发生周期,继续保持有丝分裂的能力, 在下一个周期前一直处于静止状态, 称之为 “储存的生殖干细胞” ;另一类进入精子发生,周期通过分化途径形成精子, 称之为 ”更新的生殖干细胞” 。

\subsubsection{精母细胞的减数分裂}

进入分化途径的精原细胞发育为初级精母细胞, 进行最后一次染色体的复制,为成熟分裂做准备。 根据其生长发育顺序及细胞、 染色质形态可将初级精母细胞分为前细线期精母细胞、细线期精母细胞、偶线期精母细胞、 粗线期精母细胞及双线期精母细胞等几个时相。 一个双线期精母细胞发生第一次减数分裂, 产生两个次级精母细胞。次级精 母细胞的间期很短, 不发生染色体复制, 很快进行第二次减数分裂, 产生单倍体的圆形 精子细胞, 完成减数分裂。

\subsubsection{精子形成期}

精子形成期是精子细胞的分化变态过程,这是精子分化的重要环节。圆形的精子细 胞要经过伸长变态的复杂过程,包括细胞核的浓缩变长,顶体的生成,核蛋白的转型, 染色质的浓缩包装,核骨架及细胞骨架---中心体(粒)体系的演变,鞭毛、轴丝的发 生及尾的成形分化,精子特异性乳酸脱氢酶LDH-X的出现等。在此变态过程中,糖 原、脂质、蛋白质等代谢产物大批随细胞质排弃,代之以出现的LDH-X及六碳糖激酶来适应能量需要。在变态后期核蛋白质出现不断的磷酸化和脱磷酸化,蛋白质SH--基向SS--键转变,以精氨酸为主的鱼精蛋白(protamine)替换以组氨酸为主的核组蛋白(histone),使核蛋白和DNA紧密结合,以保证精子基因处于浓缩包装和不活跃状态。

原始生殖细胞经历精原细胞、精母细胞、精子细胞和精子,其间发生了减数分裂、组蛋白/鱼精蛋白替换、精子变态等特异细胞活动,许多特异性的基因对精子发生过程进行严密的调控。

精子发生过程中,生精细胞可分为一下几个阶段:\myGlossaryEntry{pri_type_a_spermatogonium}、\myGlossaryEntry{a_spermatogonium}、\myGlossaryEntry{b_spermatogonium}、\myGlossaryEntry{primary_spermatocyte}、\myGlossaryEntry{secondary_spermatocyte}、\myGlossaryEntry{round_spermatid}、\myGlossaryEntry{condensing_spermatid}、\myGlossaryEntry{spermatozoon}、\myGlossaryEntry{preleptotene_spermatocyte}、\myGlossaryEntry{leptotene_spermatocyte}、\myGlossaryEntry{zygotene_spermatocyte}、\myGlossaryEntry{pachytene_spermatocyte}。精子发生过程及各期生精细胞特征参见图 \ref{figure_spermatogenesis_process}。

\begin{figure}
\centering
\myFigurePlaceholder
\caption{精子发生示意图}
\label{figure_spermatogenesis_process}
\end{figure}

\subsubsection{有丝分裂期}

\paragraph{\myGlossaryEntry{pri_type_a_spermatogonium}}

精子在睾丸中的发生起源于\myGlossaryEntry{pri_type_a_spermatogonium},也被称为\myGlossaryEntry{spermatogonial_stem_cell}。这类细胞通过有丝分裂进行增殖,所产生的子代细胞可以分为两类:一类仍保持精原干细胞的特征进行有丝分裂,成为长期精子发生的”源泉”;另一 类子代细胞则进入分化途径。

\paragraph{\myGlossaryEntry{a_spermatogonium}}

一部分\myGlossaryEntry{pri_type_a_spermatogonium}的子代细胞进入分化过程首先形成\myGlossaryEntry{a_spermatogonium}。\myGlossaryEntry{a_spermatogonium}的分化也是一个复杂的过程。目前认为至少要经过以下几个阶段,通过$ A_1 $型、$ A_2 $型、$ A_3 $型和$ A_4 $型精原细胞形成\myGlossaryEntry{intermediate_spermatogonium}。

\paragraph{\myGlossaryEntry{b_spermatogonium}}

这是精原细胞的最后阶段。在此之前,精原细胞都是通过有丝分裂进行增殖。\myGlossaryEntry{intermediate_spermatogonium}进行最后的有丝分裂,形成\myGlossaryEntry{b_spermatogonium},随后停止有丝分裂,由它们发育形成初级精母细胞,进入减数分裂期。

\paragraph{精子发生的同步化现象}

精子发生过程中的一个特点是许多生精细胞进行同步化分裂,并且细胞的
\myGlossaryEntry{cytokinesis}不完全,导致子细胞由细胞间桥相连(图 \ref{figure_spermatogenesis_synchronization}),这可能是它们同步分行的基础。

\begin{figure}
\centering
\myFigurePlaceholder
\caption{精子发生的同步化现象示意图}
\label{figure_spermatogenesis_synchronization}
\end{figure}

\subsubsection{减数分裂期}

配子是单倍体的,这种单倍体的细胞必须通过一种特殊的细胞分裂形式---
\myGlossaryEntry{meiosis}产生。简述分裂仅发生于有性生殖细胞发生过程中的某个阶段,其特点是细胞进行连续两次分裂而DNA只复制一次,结果产生了只含有单倍体遗传物质的细胞。含有单倍体遗传物质的两性生殖细胞通过受精形成合子,染色体又恢复到体细胞的数目,从而可以维持物种的正常繁衍。因此,\myGlossaryEntry{meiosis}是生物有性生殖的基础。

在减数分裂过程中同源染色体间DNA发生了重组,这就增加了子代发生遗传变异的机会,确保了生物的多样性以更加适应环境的变化,所以减数分裂也是生物进化及生物多样性的基础保证。这些特殊的现象必定受特殊机理的调节。近年来,在这一领域的研究集中于寻找减数分裂的特异蛋白,揭示这一过程中染色体出现的一些特殊事件的机理。

B型精原细胞有丝分裂停止后,发育为初级精母细胞,并进入减数分裂期。在这期间,细胞进行了两次减数分裂。初级精母细胞经过第一次减数分裂形成次级精母细胞,次级精母细胞经过第二次减数分裂形成单倍体的精子细胞。两次减数分裂之间的间期或长或短,但无DNA的合成。第一次减数分裂可分为前期I、中期I、后期I及末期I。第二次减数分裂可分为前期II、中期II、后期II及末期II。减数分裂的过程见图 \ref{图-减数分裂示意图}。

\begin{figure}[H]
\centering
\myFigurePlaceholder
\caption{减数分裂示意图}
\label{图-减数分裂示意图}
\end{figure}

\paragraph{初级精母细胞}

处于第一次减数分裂期的细胞为初级精母细胞。初级精母细胞随着第一次减数分裂过程中染色质的变化,又可分为前细线期(Preleptotene)、细线期(leptotene)、偶线期(zygotene)及粗线期(pachytene)精母细胞。初级精母细胞的体积不断增大,粗线期精母细胞的体积可达到前细线期的两倍以上。第一次减数分裂的时间较长,在人类可长达22天,分裂后形成次级精母细胞。

\paragraph{次级精母细胞}

由初级精母细胞经第一次减数分裂而来,细胞的体积比初级精母细胞小。细胞及细胞核均为圆形,核内染色质呈细网状,着色较浅,细胞质较少。次级精母细胞存在时间较短,很快完成第二次减数分裂,形成两个精子细胞,所以在切片中很少看到次级精母细胞。

\paragraph{圆形精子细胞}

次级精母细胞经过第二次减数分裂后,首先形成圆形精子细胞。由于在第二次减数分裂前没有进行DNA的复制,所以圆形精子细胞中染色体数目减少一半,成为单倍体细胞。此时细胞核圆形,着色较深。细胞质少,内含丰富的线粒体核高尔基复合体。精子细胞不在分裂,而是进入一个复杂有序的形态演变过程,形成头、颈、尾结构的精子,该过程为精子形成期。

\subsubsection{精子形成期(Spermiogenesis)}

精母细胞进过两次减数分裂,而DNA只复制一次,形成了单倍的圆形精子细胞,此后细胞不再分裂,而是进入一系列的形态变化,最后形成具有头、颈、尾结构的完整精子(图 \ref{图-人类精子分化示意图})。这一过程中细胞形态的变化主要表现再以下几个方面。

\begin{figure}[H]
\centering
\myFigurePlaceholder
\caption{人类精子分化示意图}
\label{图-人类精子分化示意图}
\end{figure}

\paragraph{细胞核的变化}

细胞核内染色质浓缩、体积变小、偏向细胞的一极,形成精子的头部。染色质的浓缩主要是由于染色质中的组蛋白(histone)被富含精氨酸的鱼精蛋白(protamine)取代。这一碱性蛋白含有的大量的正电荷,吸引着带负电的DNA发生集聚。然而这一蛋白质替换受哪些因素的调节,以及调节细胞核变化的其他因素有待于进一步的了解。

\paragraph{顶体的形成}

精子顶体是由高尔基复合体形成的。精子细胞的高尔基复合体首先产生许多小液泡。小液泡融合变大,形成一个大液泡叫顶体囊,内含一个大的颗粒,称为顶体颗粒(acrosomal granule)。然后由于液泡失去液体,以致液泡壁靠近位于核的前半部,组成了一双层膜,称为顶体帽,内有一个顶体颗粒。此后顶体颗粒物质分散于整个顶体帽中,这就是顶体,内含多种水解酶。

\paragraph{线粒体鞘的形成}

在精子形成过程中,精子细胞的线粒体体积变小伸长,并精确地迁移到中段,围绕着中央轴丝而形成螺旋状排列的线粒体鞘。物种不同,线粒体鞘总体构型也有很大差别。淡水无脊椎动物和海洋动物比较简单,只是由几个长的线粒体组合起来形成的线粒体鞘。而在哺乳类动物中则形成了典型的螺旋状排列的线粒体鞘。

\paragraph{中心粒的迁移}

精子形成早期,两个中心粒移值核的后方,当核的后端表面形成一个凹时,一个中心粒恰好位于凹之中,称为近端中心粒,它与精子长轴呈横向排列;另一个称为远端中心粒,位于近端中心粒的后方,它的长轴平行于精子长轴,由它产生精子尾部的中央轴丝。哺乳动物的远端中心粒,在颈段发育完成后,最终消失,有些哺乳动物的近端中心粒在精子成形后也会消失。

\paragraph{精子尾部的形成}

中央轴丝是构成鞭毛型精子尾部的基本细胞器,其外周还有致密的纤维和纤维鞘,这种粗大的纤维从精子中段起始,并不完全达到尾的末端。

\paragraph{细胞质的变化}

精子细胞的大部分细胞质,在精子形成过程中成为残体(residual body)而被抛弃。当细胞核前端形成顶体时,细胞质向后方移动,仅留下一薄层细胞质与质膜覆盖在细胞核上。当尾部的后端生长时,细胞质的大部分附着在精子中段,当线粒体围绕着轴丝时,该处的细胞质核高尔基体成为残余细胞质被抛弃,仅剩下一薄层细胞质包围着中段的线粒体。

\subsection{精子发生的特点}

\subsubsection{生精上皮周期}

一代(generation)生殖细胞是指这样一群细胞,它们在大概相同的时间形成。然后同步通过生精过程。在一定的曲细精管区域、两次分化程度相同的细胞群相继出现之间,有一个明显的时间间隔,称为生精上皮周期(cycle of seminiferous epithelium)。

不同物种的这一周期长短不同,人为16天、 公牛和大鼠为13天、羊和兔为10天、小鼠为8.6天。

不同代的生精细胞组成固定的细胞群体组合, 由于精子形成期精密而有序的时间步骤, 某一期的精子细胞总是与一定分化期的精母细胞和精原细胞相关。 对这些细胞群体进一步的观察表明:在曲细精管的任何一个区域, 它们以一定的顺序先后出现。 这些细胞群体以规律性的间隔重复出现,代表一个生精上皮周期的多个阶段。 由于这些期表示一个
连续过程的任意亚群, 因此一期的结束与下一期开始间的分界常常是不精确的。对一定的哺乳动物种属, 细胞群体的数目或周期的阶段依据采取的鉴定标准而不同。

有两种方法常用来鉴定生精小管周期的期相。第一种方法是利用精子细胞核的形态以及在生精上皮的位置。更加成熟的生精细胞成簇排列深深嵌在上皮内,对着支持细胞的核。 其后进一步分化的细胞向管腔移动,最终释放到管腔中。 在这一分类方法中,多种染色方法被应用:最常用的是苏木素---伊红(haematoxylin---eosin)染色法, 这一分类方法可把大鼠、公羊、公牛、兔、猪、水貂和猴的生精上皮分为8个期。 在一些研究中这8 个期又可以进一步细分。

第二种方法用过碘酸---希夫(PAS)---苏木素, 对生精管进行组织切片染色。 可把发育中的顶体系统染成深紫色,利用顶体来区分精子细胞。利用PAS---苏木素技术,可将大鼠的生精上皮周期分为14个期,仓鼠与豚鼠分为13期,小鼠与猴子分为12期。对在人类可以分为6期,然而这期间的界限并不十分清楚,常缺少一代或多代生精细胞,或在组织处理过程中细胞易位。 此外,一些研究者否认人类存在精子发生波,与猩猩中的现象相似。但是导致这种人类和猩猩精子发生无规律模式的因素还不清楚。

把细胞分为一个周期中的可区分的几个阶段有利于研究生殖细胞的结构与细胞化学 变化,研究一些因素(物理或化学因素)对有丝分裂 、减数分裂及细胞分化的影响。还 可以帮助我们对一些物质在生殖细胞中进行精确定位。

对生精上皮周期不同阶段的确定,是认识精子发生过程中染色体活动及变化的基础。已经观察到在小鼠中,生精上皮周期中DNA的合成发生在7个阶段,即8、10、 12、2、3、5、7 $ \sim $ 8期,主要涉及$ A_1 $、$ A_2 $、$ A_3 $、$ A_4 $、中间型及B型精原细胞。在精子发生过程中,DNA合成有两个高峰期,第一个高峰期在4 $ \sim $ 6期,对应于中间型及B型精原细胞的有丝分裂,第二个高峰位于8 $ \sim $  9期,反映前细线期精母细胞DNA的复制。而主要的RNA合成发生在精原细胞和粗线期精母细胞 。

\subsubsection{生精上皮波}

每一个生精细胞群,除了环形对称以外,还表现出沿生精管长轴有序组织排列,在那里产生精子发生\doubleQuote{波},称为生精上皮波(wave of seminiferous epithelium)。生精上皮周期与生精上皮波是不同的,周期是生精上皮的一个区域,在一定时间内发生的动态的组织学现象;\doubleQuote{波}是指一定时间内细胞群沿生精管有序的分布。换句话说,\doubleQuote{波}是空间,而\doubleQuote{周期}是时间。

已经报道在多种哺孔动物中存在着生精上皮波。但是在人类难以证明一点。因为人类睾丸中的细胞组合不规则。在狒狒睾丸中,可把生精小管分为三类。第一类包括只有单一细胞组合的生精小管段,第二类包括两种或多种细胞群有序的排列,而第三类小管只在人的生精小管中观察到。有关这些在拂拂、猩猩和人类不规则的原因所知甚少。有人认为这种无序可能反映了这些物种生殖能力的退化。但有研究表明,人类生精阶段的排列并不是随机无序的,而是有序地沿管长轴螺旋式排列。

\subsubsection{生精细胞的同步发育}

生精上皮的一个特征就是生精细胞的同步发育。人们用电子显微镜观察到发育中的雄性细 胞群,如精原细胞、精母细胞、精子细胞,通过细胞间桥连接。这种间桥是由不完整的胞质分裂所致,这种合胞体构成了它们同步发育的基础。尽管在多种哺乳类中发现这种细胞间桥的现象,然而有关这种间桥相连的确切的细胞数量存在着争论。有研究表明, 一组由细胞间桥连接的精子细胞可能有几百个,然而这比理论上的数目要少,可能由于精原细胞和精母细胞的退化所致。合胞体可能作为同步分化的装置起作用。也有人认为除了细胞间桥外,还存在着别的因素参与生殖细胞的同步分化。

\subsection{生精细胞的结构}

\subsubsection{精原细胞}

精原细胞是精子发生的起点,它紧靠在曲细精管的基底膜上,属于生精干细胞。细胞呈椭圆形或圆形, 直径$ 12\mu m $,胞核圆形,染色质着色较深,有1$ \sim $ 2个核仁。精原细胞经有丝分裂不断增殖,一部分作为贮备干细胞,另一部分进入生长期、发育成初级精母细胞。精原细胞的染色体组型,人为$ 46XY $、马为$ 66XY $、猪为$ 40XY $、狗为$ 78XY $。

\subsubsection{初级精母细胞}

初级精母细胞,是由精原细胞发育而来,体积增大,最终达到精原细胞体积的二倍,细胞器完备、数目增多,细胞核呈圆形,分裂期时间在人中可长达22天左右。分裂后形成次级精母细胞。

\subsubsection{次级精母细胞}

初级精母细胞经第一次减数分裂形成次级精母细胞,位于初级精母细胞近管腔侧,细胞体积比初级精母细胞小。细胞和细胞核均为圆形,核内染色质呈细网状,着色 较浅,细胞质较少。次级精母细胞存在时间较短,很快进入第二次减数分裂,形精子细胞,染色体数目减少一半,成为单倍体细胞。由于次级精母细胞存在时间短,所以在切片中较少见。

\subsubsection{精子细胞}
次级精母细胞分裂后产生精子细胞,它们靠近管腔。细胞核圆形,着色较深。细胞 质少,内含中心粒、线粒体和高尔基复合体等细胞器。精子细胞不再分裂,经过复杂的 形态演变后形成精子,该过程为精子形成期。

\subsubsection{精子}

精子发生最终形成精子。在睾丸中发育成完整的精子后,还需在附睾中经过一段成熟发育,才达到具有运动能力真正成熟的精子。成熟精子由两个主要部分构成,即头部和尾部。尾部又分成颈段(necksegment)、中段(middle piece)、主段(principal piece)和末段(end piece)。人类成熟精子的结构见图 \ref{图-人类精子示意图}。

\begin{figure}[H]
\centering
\myFigurePlaceholder
\caption{人类精子示意图}
\label{图-人类精子示意图}
\end{figure}

\paragraph{头部}

头部的绝大部分, 被染色质高度集聚的细胞核所占据, 具有几乎不见核孔的双层核被膜。在浓缩的细胞核前端, 盖着帽形囊状顶体。顶体后缘的后方, 称为头部顶体后区(图 \ref{图-人类精子头部超微结构示意图})。

\begin{figure}[H]
\centering
\myFigurePlaceholder
\caption{人类精子头部超微结构示意图}
\label{图-人类精子头部超微结构示意图}
\end{figure}

\subparagraph{精子核}

精子核的主要特征是其高度浓缩的核物质,由DNA和蛋白质组成。精子核体积远小于体细胞的细胞核,一般情况下核的形状与头部的性状一致。核的染色均匀,但在人和猩猩的精子核中出现一些空泡。核是父本遗传物质的携带者,含有单倍体的常染色体和一个性染色体($ X $或$ Y $)。

\subparagraph{顶体}

顶体为一层单位膜包裹的囊凹状结构,靠近核膜的单位膜称为顶体内膜,靠近细胞质膜一侧的单位膜叫顶体外膜,顶体内、外膜平行排列,并在顶体后缘彼此相连。顶体腔中具有不定形基质,其中含有透明质酸酶(hyaluronidase)、神经氨酸酶(neuraminidase)、酸性磷酸酶(acid phosphatase)、􀀌$ \beta $-N-乙酰葡糖胺糖昔酶($ \beta $-N-acetylglu­cosaminidase)、芳基硫酸酯酶(arylsulfatase)和顶体蛋白(acrosin)等水解酶类。当受精时,精子发生顶体反应,质膜和顶体外膜发生融合,形成囊泡,从而使顶体内酶类放出来,有利于精子通过卵外的各层结构。

顶体后部的狭窄区,称为赤道段(equatorial segment)。受精时,赤道段基本完好无损,而顶体的其他部位,均在顶体反应中丢失。顶体后区的质膜,是受精时精子首先与卵表面接触和发生融合的位置,所以该部位在功能上非常重要。顶体尾侧细胞质浓缩,特化为一薄层环状致密带,紧贴在细胞质膜的内表面,称为顶体后环(postacrosomal ring)。顶体后环紧贴着质膜,在该环尾缘,核膜和质膜紧密相贴,形成一环状黏合线,称为核后环(postnuclear ring)。核后环尾侧,核膜与质膜分开,核膜向下,形成一皱褶,延伸入颈段。在核的后端,有一浅窝,称为植入窝(implantation fossa),恰与尾部颈段凸出的小头相嵌。

用电子显微镜观察精子头部时发现,头部质膜外有呈细丝状和颗粒状的糖萼(glycocalyx)。糖萼较厚,尤其在顶体前缘对应处的质膜外表面最集中。凝集素(lectin)可特异性地与糖萼上的多糖链结合。当用外源凝集素处理精子后,便可阻止受精,说明受精时特异性识别卵母细胞,与糖萼的糖链直接相关。

顶体和细胞核决定了精子头部的形状。不同动物的精子,头部形态差异很大。例如,猪、羊、牛精子头部为扁卵圆形,马的精子头部为完整的椭圆形。人和狗的精子头部为梨形,小鼠精子头部为镰刀型。

\paragraph{尾部}

\section{精子发生的调控机制}


\section{精子的成熟与获能}
\chapter{卵泡发育、卵子发生和排卵}

\section{卵泡发育与卵子发生}

\section{卵子成熟}

\section{排卵}

\section{黄体的形成和退化}
\chapter{受精}

\section{受精研究历史的简单回顾}

\section{受精概述}

\section{精子与卵透明带识别和结合的分子基础}

\section{精子的顶替反应}

\section{精子与卵质膜结合和融合的分子基础}

\section{卵子激活}

\section{精子核去浓缩及雄原核的形成}

\section{卵子皮质反应及多精受精的阻止}

\section{受精过程的中心体遗传}


\chapter{胚胎发育}

\section{卵裂}

\section{早期胚胎发育过程中基因的表达}

\section{基因印记}

\section{细胞分化与胚层形成}

\section{子宫胎盘循环的建立}

\section{胚胎干细胞}
\chapter{胚胎着床与妊娠识别}

\section{胚胎着床}

\section{胚胎着床的分子调控}

\section{延迟着床}

\section{蜕膜化}

\section{妊娠识别}

\section{宫外孕和葡萄胎}
\chapter{胎盘和胎盘的结构与功能}

\section{胎膜}

\section{胎盘的类型与结构}

\section{胎盘的生理功能}
\chapter{妊娠的维持和妊娠期的生理变化}

\section{胎儿的组织性营养与血液性营养}

\section{妊娠的维持}

\section{妊娠期的生理变化}

\section{妊娠诊断}
\chapter{分娩与泌乳}

\section{分娩发动和分娩预兆}

\section{分娩}

\section{乳腺的结构与发育}

\section{泌乳的发动和维持}

\section{乳汁的排出(排乳)}
\chapter{性别决定与性腺发育}

\section{性别决定}

\section{性腺的发育}

\section{原始生殖细胞的迁移}

\section{不同性别胚胎的发育}
\chapter{生殖激素}

在生理学中,把直接作用于动物的生殖活动,并以调节生殖过程为主要生理功能的激素叫生殖激素。生殖激素的种类很多,按其来源、分泌器官以及转运机制的不同,可分为 \ref{生殖激素分类列表} 大类:\begin{inparaenum}[\itshape 1:\upshape]\item \myImportantPoint{脑部激素},由脑部各区神经细胞或核团如松果体、下丘脑和垂体等分泌,主要调节脑内和脑外生殖激素的分泌活动;\item \myImportantPoint{性腺激素},由睾丸或卵巢分泌,接受脑部激素和其他因素的调节或影响,对生殖细胞的发生和发育、排卵、受精、妊娠和分娩,以及雌性和雄性动物生殖器官的发育等生殖活动都有直接或间接作用;\item \myImportantPoint{胎盘激素},由雌性动物胎盘产生,对于维持妊娠和启动分娩等有直接作用;\item \myImportantPoint{其他组织器官分泌的激素},指产生于生殖系统以外的内分泌组织或器官,对卵巢发育、黄体消退等具有直接作用,主要为前列腺素;\item \myImportantPoint{外激素},由外分泌腺体(有管腺)所分泌,主要借助空气和水传播而作用于靶器官,主要影响动物的性行为\label{生殖激素分类列表}\end{inparaenum}。

在内分泌系统中,下丘脑、垂体和性腺分泌的激素在功能和调节上相互作用,构成了一个完整的神经内分泌生殖调节体系,即下丘脑---垂体---性腺轴(HPG,hypothalamus---pituitary--gonadal axis),它在动物生殖活动的调节中起着核心作用。

\section{脑部生殖激素}

脑是神经内分泌激素的主要分泌器官。调节生殖活动的神经内分泌激素主要由下丘脑机器周边组织、间脑、垂体及松果体等神经组织的细胞合成和分泌。神经内分泌学是了解神经调节垂体和其他内分泌组织和器官的机制,以及类固醇或其他内分泌激素作用于神经系统的机制,它是内分泌和神经科学的融合。

内分泌的主要概念是内分泌组织分泌的激素进入血液循环调节靶器官或细胞的功能,而神经科学的概念是单个神经元之间的相互联系形成的神经系统的活动。神经元能够通过它们的生物电特征快速有效地将信息传递到远方,此外,神经元间的联系依赖于信息从一个神经元传递到其他神经细胞、腺体或肌细胞,即通过突触间化学递质的释放。

这两个系统曾经是两个独立的体系。自从认识到下丘脑内特殊的神经元调节着垂体前叶的内分泌细胞,并且通过这种下丘脑---垂体的相互作用,最终调节着整个内分泌系统的功能,从此这两个系统才融合到一起。这种科学上的进步最早源于对鱼神经分泌的研究,发现其下丘脑内的神经元通过释放激素进入血液而发挥内分泌作用。而后通过Ernest Scharrer和他的合作者的研究,最终认识到垂体后叶不是真正的腺体,而是由下丘脑内某些神经元的突触组成的,它们释放激素产物直接进入血液循环。

该系统的进一步扩增是认识到神经元调节垂体前叶也是通过释放递质进入血液完成的。不同的是这些递质并不进入体循环,而是严格限制在与前叶细胞密切联系的特殊化的循环系统中。这一系统的最终形成,是由于发现了这些神经元释放的递质,在化学组成和作用机理上,与内分泌系统分泌的激素有着很高的同源性。

这样人们才对神经内分泌系统参与机体对情绪变化产生的影响有了认识,同时对内分泌功能对于精神活动的影响也有了理论依据。本节只重点介绍与生殖内分泌有关的神经内分泌知识。

\subsection{下丘脑与垂体的结构和功能}

\subsubsection{下丘脑}

下丘脑位于丘脑的腹侧、间脑的基底部,是下丘脑沟以下的神经组织,被第三脑室下部均分为左右对称的两半,构成第三脑室下部的侧面和底部,其体积约为整个大脑的$ \frac{1}{300} $。下丘脑内包含许多左右成对的核团,其中与内分泌有关的核团大多在下丘脑的内侧部,可分为 \ref{下丘脑解剖结构列表} 个区:\begin{inparaenum}[\itshape 1:\upshape]\item 前区(视上区),包括视交叉上核、视上核、视旁核等。\item 中区(结节区),包括正中隆起、弓状核、腹内侧核等。\item 后区(乳头区),包括背内侧核、乳头体等\label{下丘脑解剖结构列表}\end{inparaenum}。构成这些核团的、具有内分泌功能的神经细胞有两种:神经内分泌大细胞和神经内分泌小细胞,神经内分泌大细胞主要分布在视上核和视旁核,主要分泌催产素和加压素;神经内分泌小细胞较集中地分布在正中隆起、弓状核、视交叉上核、腹内侧核等,分泌各种释放激素,调节腺垂体的功能,因此将这一部分又统称为下丘脑促垂体区。通常,下丘脑前区、视前核和视交叉上核等神经核团分泌的释放激素可调节垂体在排卵前促卵泡激素(FSH)和促黄体激素(LH)的分泌活动,故又将这些区域称为周期分泌中枢。腹中核、弓状核和正中隆起等神经核团分泌的释放激素可调节垂体FSH和LH的基础性或持续性波动分泌,因此这些区域又称为持续分泌中枢。

\subsubsection{垂体}

垂体位于下丘脑之下的蝶骨垂体凹内,通过狭窄的垂体柄与下丘脑相连。垂体由两个主要部分组成,即腺垂体(adenohypohysis)和神经垂体(neurohypophysis)。也有人将垂体分为前叶、中间部和后叶三个部分。前叶与腺垂体含义相近,而后叶与神经垂体含意相近,但并不完全等同,中间部则是腺垂体的一部分。腺垂体由远侧部、结节部和中间部组成,神经垂体由神经部和漏斗部组成。漏斗部包括漏斗柄和灰结节的正中隆起,远侧部和结节部合称垂体前叶(anterior pituitary),神经部和中间部合称为垂体后叶(posterior pituitary)。腺垂体和神经垂体是由不同的胚层发育而来,其中远侧部来自于内胚层(源自于神经颊囊),而中间部和神经部来自于神经外胚层。

\subsubsection{下丘脑与垂体的联系}

下丘脑虽然没有神经纤维直接进入垂体前叶,但它们之间通过血管(垂体门脉管)建立了功能上的联系。下丘脑神经元的纤维末梢在正中隆起处大量分布,因此这些神经元合成的激素通过神经纤维末梢分泌到正中隆起,进入下丘脑---垂体门脉血管的毛细血管丛,再经门脉血液循环到达垂体前叶。例如,促性腺激素释放激素(GnRH)是在正中视前核产生的,多巴胺是在弓状核产生的。这两种物质从下丘脑通过轴突运输到正中隆起,在那里被释放进入门脉系统。垂体后叶可以说是下丘脑的延续部分,它是下丘脑垂体束的无髓鞘神经末梢与神经胶质细胞分化的神经纤维组成,没有腺体细胞,不能产生激素,只能储存和释放视上核、视旁核分泌的神经激素。

\subsection{下丘脑激素}

由下丘脑神经细胞合成并分泌的激素均为多肽类激素,其中对垂体激素的分泌和释放活动具有促进作用的称为释放激素或释放因子,而另一些对垂体激素的分泌和释放活动具有抑制作用的称为抑制激素或抑制因子。与生殖相关的下丘脑激素主要有促性腺激素释放激素、催产素、促乳素释放因子和促乳素释放抑制因子等四种。

\subsubsection{促性腺激素释放激素(GnRH,gonadotropin releasing hormone)}

促性腺激素释放激素最早于1971年由Guillemin和Schally研究组各自独立地从绵羊及猪的下丘脑中分离出来,随后其机构得到进一步研究,现已可人工合成。至今,人们从各种脊椎动物和原索动物中已鉴别出15中GnRH的分子结构。GnRH已被证明是下丘脑---垂体---性腺轴的关键信号分子。

\paragraph{GnRH的结构}

哺乳类(猪和羊的下丘脑以及人的胎盘)GnRH(mGnRH,mammalian GnRH)具有相同的化学结构,是由9种不同氨基酸残基组成的十肽($ PGlu -His-Try-Ser-Tyr-Gly-Leu-Arg-Pro-Gly-NH_2 $),而禽类、两栖类及鱼类的GnRH则具有不同的结构。GnRH基因内有3个内含子和4个外显子,由第2、3外显子和第4外显子的一部分共同编码GnRH前体,该前体包含一段$ 21 \sim 23 $个氨基酸的信号肽、10个氨基酸的GnRH、一个断裂位点(Gly-Lys-Arg)和$ 40 \sim 60 $个氨基酸的相关肽(GAP,GnRH associated peptide)。



\section{性腺激素}

\section{胎盘激素}

\section{前列腺素}
\chapter{性行为与激素调节}

\section{控制性行为的神经---内分泌系统}

\section{雄性性成熟和性行为}

\section{雌性性成熟和性行为}

\section{性周期及其影响因素}
\chapter{生殖缺陷与辅助生殖}

\section{生殖缺陷}

\section{辅助生殖技术}

\chapter{生殖免疫}

\section{配子免疫}

\section{母胎免疫}

\section{妊娠期免疫}

\section{病理妊娠与免疫}
\chapter{动物的无性生殖}

\section{自然条件下动物的无性生殖}

\section{人工辅助条件下的动物的无性生殖}
\chapter{环境与生殖健康}

\section{物理环境对生殖健康的影响}

\section{化学环境对生殖健康的影响}

\section{生物环境对生殖健康的影响}

\section{个人行为对生殖健康的影响}
\chapter{现代生殖生物学研究方法与技术简介}

\section{细胞立体培养和操作}

\section{基因的克隆、表达检测及功能分析}

\section{微观形态学技术}

\section{模式生物}

\section{研究模型的选择}

\part{附录}


\printnoidxglossary
\end{document}
