\chapter{生殖器官的结构与功能}

\section{雌性生殖器官}

雌性生殖器官包括卵巢(ovary)、输卵管(oviductor uterine tube)、子宫(uterus)、 阴道(vagina)、前庭(vestibule)、阴门(vulva)和相关腺体。卵子的发生、成熟、运输、受精、妊娠及胎儿的出生等功能均由雌性生殖器官完成。

\subsection{卵巢}

所有哺乳动物的卵巢是成对的, 从性腺原基的形成到发育完成均位于肾脏附近。 卵巢的大小与动物的年龄和生殖状态有关。多数成年动物的卵巢游离面突向体腔, 主要由 外层的皮质(cortex)和内部的髓质(medulla)两部分构成。 卵巢的主要功能包括雌性激素的分泌和卵子的产生。

\subsubsection{形态与结构}

\paragraph{卵巢的形态与基本结构}

多数哺乳动物的卵巢为卵圆形,由皮质和髓质两部分组成。皮质主要由卵泡和黄体组成,并覆盖一层低矮的立方上皮细胞;皮质的基质由疏松结缔组织构成;皮质表面有一层由致密结缔组织构成的臼膜。卵巢的髓质主要由疏松的结缔组织和平滑肌 组成,富含神经、血管和淋巴管,并与卵巢系膜中的平滑肌相连J卵巢网位于骼质部,是由立方上皮细胞或实质细胞束相连而成的不规则的网络管道,这一结构特点在肉食动物和反刍动物中非常明显。

\subparagraph{卵泡(ovarianfollicle)}

根据卵泡的发育时期或生理状态,可将卵泡分为原始卵泡(primordial follicle)、生长卵泡(growing follicle)和成熟卵泡(mature follicle)。其中生长卵泡要经历三个阶段,即初级卵泡(primary follicle)、次级卵泡(secondary follicle)和三级卵泡(tertiaryfollicle)或格拉夫卵泡(Graafian follicle)。

\subparagraph{原始卵泡}
原始卵泡由一个大而圆的初级卵母细胞(primary oocyte)和其周围的扁平上皮细胞(squamous epithelial cell)构成。在肉食动物、羊和猪的原始卵泡中,可能有2-6个 初级卵母细胞,是多卵卵泡(polyovular follicle)。初级卵母细胞大而呈圆形,核内染色质细小而分散,核仁大而明显。原始卵泡要在性成熟时才开始生长发育。

\subparagraph{初级卵泡}

大多数动物的初级卵泡是由一个直径约为20µm的初级卵母细胞和周围的单层立方卵泡细胞组成。胚胎期和出生后,卵巢中大多数是初级卵泡。各种动物在出生时,单个卵巢中的初级卵泡数从数十万到数百万个不等。但在一生中只有几百个卵泡 发育到排卵,大多数都退化。

\subparagraph{次级卵泡}

随着卵泡的进一步发育,在卵子周围形成多层卵泡细胞或颗粒细胞。卵母细胞外有一层$ 3\sim 5\mu m $厚的糖蛋白,称透明带(zona pellucida)。透明带是由紧贴卵母细胞的颗粒细胞和卵母细胞共同分泌的产物。由多层颗粒细胞和由其包围的初级卵母细胞构成次级 卵泡。随着卵泡的继续发育,在颗粒细胞间隙有少量的卵泡液出现。

\subparagraph{三级卵泡}

三级卵泡又叫有腔卵泡(antral follicle)。三级卵泡的特点是在其中央有一空腔,即\myInlineGlossary{卵泡腔}。卵泡腔是由次级卵泡的颗粒细胞间隙增大并融合形成的一个较大腔体,其中充满卵泡液。在卵母细胞中央有一球形的细胞核,染色质稀疏呈网状,核仁特别明显。细胞质中的高尔基复合体浓缩,位于细胞膜附近。随着卵泡腔中液体的增多,卵泡腔继续增大,卵母细胞移位远离卵泡中心,通常靠近卵泡的近卵巢中心部,此时称为\myInlineGlossary{格拉夫卵泡}。在卵母细胞与颗粒细胞层之间形成\myInlineGlossary{卵丘(cumulus cophorus)}。在较大的三级卵泡中,紧裹在卵母细胞周围的颗粒细胞形成的呈放射状排列的结构,称为\myInlineGlossary{放射冠(corona radiata)}。

\subparagraph{成熟卵泡}

当卵泡发育到快排卵时,其中的初级卵母细胞恢复并完成第一次减数分裂,排出第一极体(first polar body),形成次级卵母细胞。 但狗和马在排卵后完成第一次减数分 裂。 第一次减数分裂完成后接着进行第二次减数分裂, 但停滞在分裂的中期, 直到受精时才完成第二次分裂, 卵母细胞释放出第二极体(second polar body)。

\subparagraph{闭锁卵泡 (atretic follicle)}
动物出生后,卵巢中就有数以百万计的原始卵泡。 在个体发育过程中, 卵巢内大多数的卵泡不能发育成熟,在发育的不同阶段逐渐退化。初级卵泡退化时,首先是卵母细胞萎缩,进而卵泡细胞离散,结缔组织在卵泡内形成疤痕。次级卵泡退化时,卵母细胞核偏位、固缩;透明带膨胀、塌陷;颗粒细胞松散并脱落进卵泡腔;卵泡液被吸收,卵泡膜内层细胞增大,呈多角形,被结缔组织分割成团索状,分散在卵巢基质中并形成间\myInlineGlossary{质腺体}。

\subparagraph{黄体}
进入青春期后,卵巢开始排卵。刚排卵后的卵泡腔内由于充满血液和组织液,也称红体(corpus rubrurn)之后卵泡腔中的血凝块及具组织液被重新吸收。与此同时,颗粒细胞和卵泡膜细胞失去原有的形态特征,并取代红体而变为黄体(corpus luteum)。 随后,在黄体中出现结缔组织、脂肪、透明样物质(hyaline-like substance),细胞体积逐渐减小,最终只在卵巢表面形成一个不易观察到的黄体小疤,即由早期的红棕色变为白色或淡褐色,故称为白体(Corpus Albicans)。\myImportantPoint{黄体属于分泌腺,分泌的孕酮(progesterone)能刺激子宫腺体的分泌功能和乳腺发育}。\myImportantPoint{在妊娠期,黄体分泌的孕激素主要是维持动物的妊娠过程}。

\paragraph{不同动物卵巢的形态与结构特征}

人类的卵巢为扁椭圆形。青春期前,卵巢表面光滑;青春期后开始排卵,表面凸凹不平。成年妇女的卵巢大小约为$ 4cm \times 3 cm \times 1 cm $, 重约$ 5 \sim 6 $克,呈灰白色;绝经后卵巢变小变硬。卵巢表面无腹膜,由单层立方上皮疫盖,称生发上皮(germnalepithe­lium), 其内有一层纤维组织,称卵巢白膜(tunicaalbuginea)。白膜内为卵巢实质组织, 分为皮质和随质两部分。皮质在外层,其中有数以万计的原始卵泡及致密结缔组织;筋质在卵巢的中心部分,含有疏松结缔组织和丰富的血管、神经、淋巴管及少扯与卵巢悬韧带相连的平滑肌纤维。髓质内无卵泡。人在出生时,卵巢中有300000 -500 000个原 始卵泡。

马的卵巢呈豆形,平均长约为7.5cm,厚2.5cm,表面光滑,覆盖浆膜,借卵巢系 膜悬于腰下部、肾后方,其游离缘有一凹陷部叫排卵窝。马卵巢的最外层为髓质,内层力皮质,在排卵小凹处出现生殖上皮。

牛和羊的卵巢为稍扁的椭圆形,羊的较圆、较小,约为3.7cmX2.5cmX 1.5cm。般位千骨盆前口的两侧附近。未产母牛卵巢稍向后移,多在骨盆腔内;经产母牛卵巢位于腹腔内。性成熟后,成熟的卵泡与黄体可突出于卵巢表面。卵巢啦宽大。

猪的卵巢较大呈卵圆形,其所处的位趾、形状大小及卵巢系膜的宽度在不同年龄的 个体间有很大的差异。性成熟前小母猪的卵巢较小,约为0.4cmX 0. 5cm, 表面光滑,呈淡红色,位于荐骨呻两侧稍靠后方,由卵巢系膜固定。接近性成熟时,卵巢体积增 大,约为2cmX1.5cm, 系膜增宽,卵巢位宜稍下垂前移。性成熟后和经产母猪的卵巢更大,长约3-Scm,表面卵泡和黄体突出而呈结节状。卵巢系膜宽10-20cm,卵巢位于鹘结节前缘约4cm的横断面上,一般左侧卵巢在正中矢状面上,右侧卵巢在正中矢状面稍偏右。

狗卵巢皮质中有非常明显的皮质小管。皮质小管的管腔狭小,衬以立方上皮细胞; 卵巢的骼质部富含神经、许多大而卷曲的血管和淋巴管;髓质由疏松的结缔组织和平滑 肌组成,并与卵巢系膜中平滑肌相连。卵巢网位于髓质部,是由立方上皮细胞或是实质 细胞束相连而成的不规则的网络管道。


\subsubsection{卵巢的主要功能}

\myImportantPoint{卵巢的功能主要包括卵子的产生和生殖激素的分泌}。

\paragraph{生殖激素的分泌}

卵巢分泌的雌性激素主要有雌激素(estrogen)和孕酮。雌激素主要由存在于颗粒细胞内的芳香化酶作用于卵泡,由卵泡膜细胞产生雄激素,使雄激素芳香化而生成雌激素。孕酮主要由发情后期(metestrus)、间情期(diestrus)、妊娠期的黄体细胞和胎盘产生。雌激素诱异雌性生殖器官的生长、发育以及雌性动物的发情行为。

孕酮可刺激子宫腺的发育及分泌,使子宫内膜处于接受状态;可阻止卵泡的成熟和再次发情,使动物处于妊娠状态。

雌激素和孕激素在促讲乳腺的发育方面具有协同作用。卵泡的生长、成熟和雌激素分泌是在垂体促性腺激素---卵泡刺激素(FSH)和黄体生成素(LH)的调节下完成的。另外,卵巢雌激素的分泌又可以诱导排卵前LH的大量释放,进而引起排卵和黄体的形成。

\subsubsection{卵子的形成及其排卵}

在雌性性腺形成时, 原始生殖细胞已经存在。 随着卵巢的发育,形成初级卵母细胞, 周围的细胞形成单层扁平的卵泡细胞,并与初级卵母细胞共同构成原始卵泡。出生前,初级卵母细胞进入并停留在第一次减数分裂前期。 性成熟接近排卵时,卵母细胞才完成第一次减数分裂,即青春期后卵母细胞才完成第一次减数分裂。 随着卵泡的发育成熟,卵泡逐渐向卵巢表面移行并向外突出。当卵泡接近卵巢表面时,该处表层细胞变薄,最后破裂,出现排卵(ovulation)。排出的卵母细胞外包有放射冠,进入输卵管漏斗(infundibulum)。在大多数动物,卵母细胞进入输卵管与精子相遇时放射冠消散。 在反刍动物,排卵时放射冠就已丢失。\myImportantPoint{卵子保持受精能力的时间大约是一天}。 如果未受精,则被分解吸收。大多数动物是一侧卵巢排卵。 马的左侧卵巢排卵量为60\%, 而牛大约有$ 60\%\sim 65\% $的卵是由右侧卵巢排出的。


\subsection{输卵管}

输卵管是双侧、 弯曲的管道, 起始于卵巢并延伸到子宫角 (uterine horn), 由明显 的三部分组成: 伞部是一个较大的涌斗状结构, 又称输卵管伞;壶腹部(ampulla),其管壁较薄, 是输卵管伞后部的延伸;峡部 (isthmus) 是与子宫相连的一狭窄的管道。 输卵管的管壁由黏膜(mucosa)、肌膜(tunica muscularis)和浆膜(tunica serosa)三层构成,是在雌激素及其他因子的作用下由副中肾管发育而成。


\subsubsection{组织结构}

\paragraph{黏膜}

黏膜上皮属于单层柱状或假复层柱状上皮,由有纤毛柱状细胞和无纤毛柱状细胞构成。无纤毛柱状细胞有分泌功能,其分泌物为卵子提供营养;有纤毛细胞通常分布于输卵管起始端和末尾端。输卵管的固有膜与黏膜下层主要由疏松的结缔组织构成,其中含有许多浆细胞(plasma cell)、肥大细胞(mast cell)和嗜伊红细胞(eosinophil)。壶腹 部的黏膜上皮和黏膜下层高度折叠。如牛的壶腹部大约有40个纵行折叠,且每一个折叠又有二级和三级折叠。在峡部有$ 4\sim 8 $个大的折叠,但没有二级或三级。


\paragraph{肌层}

肌层主要由环行平滑肌组成,但也有少量的纵行肌和斜行肌。在伞部和壶腹部的肌层较薄;但峡部的肌层明显明显增厚,与子宫环行肌无明显的界限。

\paragraph{浆膜}

由疏松结缔组织构成,含丰富的血管和神经。

\subsubsection{功能}

输卵管主要作为卵子、精子(Spermatozoon)和早期胚胎运行的通道,同时也是生殖细胞停留、吸收营养和受精的部位。

漏斗部围绕卵巢,并由卵巢囊包裹(马无卵巢囊)。漏斗部的游离缘有指状突(即伞部)。动物在排卵期,伞部血管充血、肿胀。伞部随平滑肌节律性收缩而在卵巢的表面上部移动。这种结构和功能的变化有利于捕获由卵巢排出的卵母细胞;同时漏斗部上皮细胞的纤毛向子宫方向摆动,能将卵子运送到壶腹部。壶腹部是精子和卵子结合的部位。壶腹部上皮纤毛的运动和平滑肌的收缩共同参与卵子的运动。在峡部肌肉的收缩运动是推动受精卵向子宫方向运动的主要动力。峡部的纤毛运动也有利于受精卵的运动。随着发情周期的变化,峡部肌肉收缩的方向也在发生改变。在卵泡期,逆蠕动收缩将峡部腔内容物送到壶腹部;而在黄体期的分节运动逐渐推动受精卵向子宫方向运动。受精卵通过峡部的时间长短和妊娠时间无关,其通过峡部的时间大约为$ 4\sim 5 $天。

精子在输卵管内除精子自身的运动外,一定程度上也依赖于输卵管管壁肌肉的收缩和黏膜上皮纤毛的运动。

\subsection{子宫}

子宫(uterus)是一个中空的肌质性器官,通常有两个子宫角和一个子宫角;子宫是孕体(conceptus)着床(implantation)的地方。它是在雌激素的作用下由缪氏管发育来的。在发情(estrous)和生殖周期(reproductive cycles)中子宫壁经历一系列特定的变化;大多数动物的子宫由子宫角(cornua uteri)、子宫体(uterine body)、子宫颈(cervix uteri)三部分组成,子宫角与输卵管相连,子宫体与阴道相连的子宫颈连接,整个子宫通过阔韧带附着于盆腔和腹腔壁上,韧带中含有血管和神经,为子宫提供血液和神经支配。

根据子宫形态可将哺乳动物子宫分为:子宫呈一单管状的称单子宫(uterine simplex),如人和灵长类的子宫;在子宫体内腔前部无纵隔的称双角子宫(uterine cervix),如猪和马的子宫;在子宫体内腔前部有纵隔,并将其分开的叫双分子宫(uterus bipartitus),如牛、羊、骆驼等动物的子宫。小鼠和有袋类动物有两个子宫,称为双子宫(didelphia)动物,其中有袋类动物的双子宫有两个子宫颈,两个阴道,在个别的牛、羊、猪可发现有两个子宫颈和两个完全分开的子宫角,这种结构并不影响其生殖。

\subsubsection{子宫的形态结构}

\paragraph{子宫的形态}

子宫是一个肌质性的中腔器官,借助于子宫阔韧带附着于腰下部和骨盆腔的侧壁,大多位于腹腔内,少部分位于骨盆腔内,在直肠与膀胱间。多数家畜的子宫为双子宫,可分为子宫角、子宫体和子宫颈三个部分。子宫的大小、形状、位置和结构因动物的品种、年龄、个体的性周期以及妊娠阶段的不同而有很大的差异。

人的子宫位于骨盆腔中央,呈倒置梨形,前面扁平而后面稍突出。成年子宫重约50g, 长约$ 为7\sim 8 $cm,宽约$ 4\sim 5 $cm,厚$ 2\sim 3 $cm,子宫腔容量约为5ml。子宫上部较宽, 其隆突部为子宫底(fundus uteri) ,子宫底两侧为子宫角。子宫体下方为子宫颈,呈狭窄的圆柱状。

成年牛的子宫角较长,约为40cm,子宫体短,约为4cm;而子宫颈较长,约为 10cm。羊的子宫角长,约为15cm,外形像绵羊角;子宫体长为2cm;子宫颈长为4cm。牛和羊的子宫壁较厚且坚韧。

马的子宫呈“Y”形,子宫角稍曲成弓形,背缘凹,腹缘凸出而游离;子宫体长与子宫角长相当,子宫颈后端突入阴道内,有明显的子宫颈阴道部。

猪的子宫角特别长,经产母猪为$ 12\sim 15 $cm,子宫体短,约为5cm。仔母猪子宫角细而弯曲,子宫颈长13cm左右。子宫颈与阴道无明显的界限, 但其形成两个半圆形的黏膜褶,交错排列,使子宫颈管呈狭窄的螺旋形。

\paragraph{子宫的结构}

子宫由子宫内膜(endometrium)、子宫肌层(myometrium)和子宫外膜 (perimetrium)三层构成。

\subparagraph{子宫内膜}

子宫内膜位于子宫腔面,由上皮和固有层组成。灵长类、哨齿类、马、犬、猫等动物的上皮为单层柱状上皮,而猪和反刍动物的上皮为单层柱状或假复层上皮。固有膜由富含血管的胚性结缔组织组成。 固有膜的浅层有较多的细胞成分,主要为星状细胞。 这些细胞借突起互相连接。此外还有巨噬细胞和肥大细胞。 固有膜的深层, 细胞成分较少, 有子宫腺分布。 大多数动物的子宫腺为弯曲、 分枝管状腺,管状腺的内层是由有绒毛的单层柱状上皮或无绒毛的单层柱状上皮细胞组成。子宫腺的密度因动物的种类、 胎次和发情周期的不同而有差异。

人及灵长类固有层的浅层为功能层,深部较薄的为基底层。在月经周期中,功能层发生部分或全部丢失,而基底层在整个生殖周期都存在。当功能层丢失后,可以从深部的基底层得到恢复。

子宫阜(caruncle)是反刍动物子宫的一种特殊结构,是位于子宫体和子宫角黏膜上特殊的圆形隆起,富含成纤维细胞和血管。反刍动物的子宫阜部位没有腺体。每侧子宫角有四排子宫阜,每排大约15个子宫阜;牛的子宫阜为圆顶状,羊为杯状。子宫阜为胎膜与子宫壁的结合部位。

\subparagraph{子宫肌层}

子宫肌层是子宫最厚的一层,它主要由内层环行平滑肌和外层的纵行平滑肌构成。在两层平滑肌之间及平滑肌深部有大量的动脉血管、静脉血管和淋巴管。

\subparagraph{子宫外膜}
由疏松的结缔组织组成, 其外覆盖腹膜间皮(peritoneal mesot helium)这一层也有平滑肌细胞、血管、淋巴管和神经纤维。

另外,子宫颈是连接阴道和子宫的个通道,主要由平滑肌、弹力纤维、血管等构成;黏膜下层高度折叠,有二级和三级折叠,母牛子宫颈有四个环行折叠和$ 15\sim 25 $个纵行折叠,每一个折叠都有二级和三级折叠存在,这种折叠可能被错认为腺体结构。大多数动物的子宫腺不会延伸到子宫颈,在子宫颈的腺体是黏液腺。

\subsubsection{功能}

肌层是子宫的三层结构中最厚的一层,在这一层中丰富的血管为子宫提供营养。在妊娠期,子宫肌层中平滑肌细胞的长度逐渐增长,对胎儿的发育和产出都有重要作用。

子宫内膜中有子宫腺,能分泌多种物质,除对胎儿具有重要的营养作用外,对于胚胎着床、妊娠识别和胎儿的存活与发育起着重要的调节作用。在食肉类动物,子宫内膜分泌活性的改变将导致胚胎着床的延迟。在啃齿类,由子宫分泌白血病抑制因子、降钙素等因子,对子宫接受性的建立和胚胎的着床起作用。在猪、牛、马和羊等家畜,子宫内膜的分泌物影响到胎儿存活和胚胎的发育。利用孕酮抑制母羊子宫内膜腺的分化,或由其他疾病引起的子宫内膜的纤维化,均可导致不孕、早期胎儿的死亡或早期流产等。


\subsection{阴道}

\subsubsection{阴道形态与大小}

阴道从子宫颈延伸到阴道前庭,是雌性动物的交配器官和胎儿产出的通道。

人的阴道位于骨盆下部的中央,上端包围子宫颈,下端开口于阴道前庭后部,上端较下端宽;前壁与膀胱和尿道相接,后壁与直肠紧贴,前壁长约$ 7\sim 9 $cm,后壁长约$ 10\sim 12 $cm,平时阴道前后壁相互贴近。阴道黏膜淡红色,并且受性激素的影响而有周期性的变化。牛的阴道长$ 20\sim 25 $cm,妊娠母牛阴道可增至30cm以上。阴道壁较厚, 阴道穹隆呈半环状,仅见于阴道前端的背侧和两侧。牛的阴瓣较不明显,在尿道外口的腹侧有一尿道下憩室。马的阴道长约$ 15\sim20 $cm,阴道穹隆呈环状;母马驹的阴瓣发达,经产的老龄母马的阴瓣常不明显。猪的阴道长约$ 10 \sim 12 $cm,肌层较厚,直径小;黏膜一
有皱褶,不形成阴道穹隆。阴瓣为环形皱褶,阴蒂细长,突出千阴道窝的表面;尿生殖前庭腹侧壁的黏膜形成两对纵皱,前庭许多开口位于纵褶之间。

\subsubsection{组织结构}

阴道壁由黏膜层、肌层和外膜(tunica adventitia)或浆膜组成。

\paragraph{黏膜}

阴道黏膜上皮为一复层扁平上皮,在发悄前期和发情期增厚。固有层和黏膜下层(propria scubmucosa)由疏松结缔组织或致密的不规则的结缔组织组成;阴道后部的固有层有少量的淋巴结。阴道黏膜受性激素影响而有周期性的变化。绝经后的妇女的阴道粘膜上皮甚簿,皱襞少,伸展性小,易创伤而感染。

\subsubsection{肌层}

由厚的内环肌层和较薄的外纵行平滑肌组成,内环肌由结缔组织分成束状。

\subsubsection{浆膜}

浆膜层为疏松的结缔组织, 有大植的血管、 神经和神经节组成。外部纵行肌可看作是浆膜的 一部分。

\subsection{前庭、 阴蒂、 阴门}

\paragraph{前庭}

前庭与阴道以处女膜(hymen)为分界;前庭壁上有尿道开口(orifices of the ure­thra)和大小前庭腺。除黏膜深部有较多的淋巴小结外, 前庭壁类似于阴道后部, 特别是在阴蒂, 有更多的上皮下淋巴结存在。发生炎症时可能影响生殖。

前庭大腺体与雄性动物的尿道球腺(bulbourethral gland)同源。前庭大腺体是一种复合的管泡黏液腺, 位千黏膜层深部, 末端的分泌泡含有大的黏液细胞,与腺泡相连的 小管内衬柱状黏液细胞和杯状细胞;大管直通前庭, 内衬一层厚的复层鳞状上皮细胞, 分散或是聚集的淋巴结位于大管周围。在交配时, 腺体受压而释放黏液以润滑前庭。前庭小腺体比较小, 在大多数动物的前庭黏膜中有散在分布的管状分枝黏液腺,衬以复层 扁平上皮细胞。

\paragraph{阴蒂}
阴蒂位于前庭尾区远端, 由阴蒂海绵体、 阴蒂头(glans clitoridis)和阴蒂包皮(preputium clitoridis)组成。阴蒂海绵体与阴茎海绵体的结构相似。阴蒂头与阴茎头 (glans penis) 同源。阴蒂包皮是前庭黏膜的延续, 有侧壁和内脏层, 内脏层有大拭的神 经末梢。

\paragraph{阴门}

阴门由大小唇构成,并有皮肤覆盖。皮肤上有汗毛(finehair)。真皮下有使阴门发生收缩的横纹肌纤维(striatedmuscle fiber),阴门部有丰富的小血管和淋巴管。动物在发悄期阴门充血肿胀, 局部温度升高。


\section{雄性生殖器官}

雄性动物的生殖器官包括成对睾丸(testis)、附睾(epididymis)、输出管(efferent duct)、尿道(urethra)、阴茎(penis)、包皮(prepuce)及附性腺(accessory gland)。附性腺包括精报腺(seminalvesicle)、前列腺(prostategland)和尿道球腺(bul­bourethral gland)。雄性动物的生殖系统参与完成精子的发生和成熟,并将精子释放到雌性动物生殖道中。

\subsection{睾丸}

睾丸是雄性动物最为重要的生殖器官,主要由曲细精管和间质构成。曲细精管由支持细胞(sertoli cell)和各级生精细胞组成,前者对生殖细胞具有支持、保护和营养作用,后者包括精原细胞、初级精母细胞、次级精母细胞、精子细胞和精子,它们分别处于不同的发育阶段。构成睾丸的间质包含有动、静脉血管和睾丸间质细胞(leydig cell),血管主要为睾丸提供营养、调节温度和排除代谢产物;睾丸间质细胞分泌雄激素,为精子的发生提供一个合适的激素环境。各种动物睾丸结构模式如图 \ref{figure_testis_of_several_animals} 所示。

\begin{figure}
\centering
\myFigurePlaceholder
\caption{几种动物睾丸和附睾的结构模式图}
\label{figure_testis_of_several_animals}
\end{figure}

\subsubsection{睾丸的发育}

在脊椎动物,卵巢和睾丸都是由相同的原基组织---生殖嵴发育和分化而来。在胚胎发育早期,未分化性腺在形态方面无明显的性别间差异,并具有分化成睾丸或是卵巢的潜能。尽管哺乳动物间睾丸发育的过程各有不同变化,但整个过程可分为四个阶段:胚胎期睾丸的分化发育、发育的睾丸下降到阴囊、胎儿期睾丸的生长发育及青春期前后睾丸的成熟。 下面主要介绍睾丸下降与隐睾及出生后睾丸的发育。

\paragraph{睾丸下降与隐睾}
睾丸下降(descent of the testis)是指睾丸从其分化形成的部位, 即第 $ 16 \sim 24 $体节处经腹腔迁移到阴囊的过程。人睾丸的下降有两个明显阶段:第一阶段是在妊娠第$ 8 \sim 15 $周,发生穿过腹部的相对移位;第二阶段是腹股沟向阴囊的迁移。 下丘脑---垂体---性腺轴 (hypothalamo-pituitary-gonadal axis) 的正常发育是睾丸正常下降所必需的。已证实,睾丸的下降与\myHumanGene{Insl3} 基因(insulin-like 3 gene) 、 雄激素(androgen)、 G-蛋白耦联受体(G-protein-coupled receptor )、 CGRP (calcitonin gene - related peptide)因子、同源框基因 (homeobox gene)和雌激素等有关。

各种家畜睾丸下降到阴囊的时间分别为: 马在怀孕$ 9 \sim 11 $个月, 牛是在怀孕$ 3.4 \sim 4 $个月,羊为怀孕80天,猪为怀孕90天,骆驼睾丸是在出生时下降,而狗和猫分别是在出生后5天和$ 2 \sim 5 $天。 猪、 马、 狗发生隐睾比较普遍, 而在牛、 羊和鹿比较少见。

 人的睾丸最早都在腹腔内, 胎儿发育到第9个月时睾丸逐渐从腹腔下降到阴囊里。 一般出生时, 90\%左右的男孩睾丸已下降到阴囊; 出生后有些男孩的睾丸继续下降, 1周岁时有$ 95\% \sim 97\% $男孩的睾丸到达阴囊;只有少数男孩到青春期时, 睾丸才下降到阴囊。
 
在人和多数高等哺乳动物中, 睾丸的正常功能, 特别是产生精子的功能, 与温度有密切关系。 睾丸产生精子要求一个温度低于正常体温的环境。 睾丸的正常位置是在腹腔外的阴囊。 如果一侧或是双侧睾丸未下降到阴囊而停留在腹腔则称隐睾 (cryp­torchidism)。到目前为止,隐睾发生的机理还不十分清楚。 尽管隐睾仍产生雄激素, 但不能产生正常的精子。双侧隐睾的动物无生殖能力。

\paragraph{出生后睾丸的发育}

从出生到青春期,睾丸都处于连续的发育过程, 其中包括睾丸体积逐渐增大和功能的完善。

在灵长类, 幼儿期的睾丸有一个相对较长而变化不明显的时期, 传统认为这是睾丸的静止期。实际上,从出生后到幼儿早期, 睾丸的体积在短期内快速增大,这主要是由于支持细胞的快速增殖和生精索的增长。与此同时, 睾丸间质细胞在胎儿出生后又开始增殖、睾酮分泌增多。睾丸的这种活动在幼儿后期逐渐消退,这与下丘脑---垂体---睾丸轴的调节功能的变化相一致。

青春期开始后, 睾丸在形态和功能方面发生显著的变化:一是睾丸间质细胞的分泌功能加强, 二是生殖细胞的分化、 发育或退化连续发生。这两方面的变化与成熟精子的生成有关。在这个阶段睾丸的快速生长主要表现在曲细精管直径的增大;支持细胞有丝分裂停止, 但它继续分化形成具有完整结构的睾丸屏障。因此, 睾丸在青春期前的正常发育与成年动物的生育能力有密切关系。

\subsubsection{睾丸的结构}

睾丸的组织结构如图 \ref{figure_structure_of_human_testis_and_epididymis} 所示。

\begin{figure}
\centering
\myFigurePlaceholder
\caption{人睾丸与附睾结构}
\label{figure_structure_of_human_testis_and_epididymis}
\end{figure}

\paragraph{鞘膜(tunicavaginalis)}

鞘膜分为壁层(parietal layer)和脏层(visceral layer)。壁层与阴囊紧密相贴,而脏层包裹睾丸表面,由间皮(mesotheliurn)和结缔组织构成,不易与白膜(tunicaalbug­inea)分离。

\paragraph{白膜}

白膜是一层致密结缔组织囊,主要由胶原纤维(collagenfiber)和少许弹性纤维(elastic fiber)构成。在成年公马、公猪和绵羊还可见平滑肌细胞。大量分支的动脉血管和静脉血管构成血管层(vascular layer)。羊和狗的血管层位于白膜的表层,而猪和马的血管层位于白膜的深层。

\paragraph{睾丸隔膜}

白膜的结缔组织深入睾丸实质后,将睾丸分成多个小叶(lobule),称为睾丸隔膜,其组成成分和白膜相同。睾丸隔膜的结缔组织进入睾丸小叶内称睾丸小梁,与睾丸隔膜相连。猪和狗的睾丸小梁较厚且是完整的隔板,反刍动物和猫较簿而且不完整。每一小叶内有$ 1\sim 4 $个曲细精管。曲细精管外有基膜。

\paragraph{睾丸纵隔(mediastinumtestis)}

在睾丸的中央,隔膜与睾丸的疏松结缔组织相连而形成。在大多数家畜,睾丸纵隔占据睾丸中央位置。

\paragraph{间质细胞(interstitial cell或Leydig cell)}

在睾丸的曲细精管间有血管、淋巴管、成纤维细胞、游离单核细胞和间质内分泌细胞,即睾丸间质细胞。睾丸间质细胞产生睾丸雄激素,但公猪和公马同时也产生大量的雌激素。间质细胞因动物的种类、年龄的不同会有很大的变化。成年公牛的间质细胞约占睾丸体积的70\%,而公猪的间质细胞较多,占整个睾丸体积的$ 20\%\sim30\% $,公马的间质细胞也较多。

睾丸间质细胞成束状或簇状存在,相邻的睾丸间质细胞由微管和间隙连接相连。在睾丸组织和淋巴中有高浓度的类固醇。睾丸间质细胞的形状不规则,呈多面形、细胞核为椭圆形。所有动物的睾丸间质细胞都含有大量的脂类,滑面内质网上有类固醇脱氢酶(steroid dehyrogcnase),线粒体数量也增多,内有许多管状嵴,它们参与睾酮的合成。睾酮在细胞内的存储及释放并不引起该细胞形态上的明显变化。

\paragraph{曲细精管(convolutedseminiferous tubule)}

曲细精管为卷曲的、直径为$ 200 \sim 400\mu m $的小管,曲细精管上皮主要是由支持细胞和生精细胞所构成。

\subparagraph{支持细胞}
该细胞来源于青春期前未分化的性腺支持细胞。这种支持细胞有丝分裂活动强, 含有大量的粗面内质网,合成的抗缪勒氏管激素可抑制输卵管、子宫、阴道在雄性个体中的发育。在青春期支持细胞分化并伴有形态改变,最后失去有丝分裂能力。在成年动物中,支持细胞不规则地排列在生精小管基膜上,横切面由$ 25\sim 30 $个支持细胞组成。支持细胞含有脂类物质和糖元,大量的微丝、微管和滑面内质网,细胞顶端表面呈锯齿状。另外,位于基膜上细胞的紧密连接为精原细胞的减数分裂和精子发生提供了相对稳定的内环境,睾丸的这一特殊结构称之为血---睾屏障(blood-testisbarrier)。

支持细胞的功能包括: 
\begin{enumerate}
\item 营养、 支持、 保护生精细胞及精子;
\item 吞食退化精子和精子脱落的残体;
\item 参与FSH对生殖细胞的调节作用,产生雄激素结合蛋白;
\item 分泌含有钾、肌醇(inositol)、 谷氨酸铁转运蛋白(glutamate transferrin)等管腔液成分;
\item 分泌抑制素 (inhibin) 
\end{enumerate}s等。

\subparagraph{生精细胞(spermatogeniccell)}

生精细胞是指发育过程中处于不同阶段的精细胞,这些细胞位于支持细胞间和该细胞之上。从精原细胞发育到精子的一系列过程称为精子发生(spermatogenesis), 一般分为三个阶段:精母细胞的发生(spermatocytogencsis),即精原细胞(spermatogonium)发育为精母细胞(spermatocyte)的过程;精母细胞成熟分裂形成单倍体精子细胞 (spermatid);精子形成(spermiogenesis) , 即精子细胞发生形态变化的过程。

\paragraph{直细精管(Straight testicular tubule)}

直细精管较短,可以认为是曲细精管的延续,与睾丸网相连,在马和猪,有些曲细精管终止于睾丸周边,并通过长距离的直细精管与睾丸网相连。在曲细精管的终末段内衬变形的支持细胞,所有的精子必须通过这些变形细胞间的狭缝进入直细精管。曲细精管的终末段的这种结构起到类似阀门的作用,防止睾丸网中的液体逆流到曲细精管而影响精子的发生。直细精管内衬单层扁平或柱状上皮细胞。另外,有大量的巨噬细胞和淋巴细胞,能够吞噬精子。

\paragraph{睾丸网(retetestis)}

睾丸网为直细小管进入睾丸纵隔内分支吻合成网状的管道,管腔内衬单层扁平或柱状上皮细胞,弹性纤维和收缩细胞位于上皮下。大部分睾丸液是由睾丸网产生的。

\subsubsection{睾丸的功能}

\myImportantPoint{睾丸的功能主要包括精子的生成和雄性激素的分泌}。人的精子从精原细胞开始到成熟大约需要$ 65\sim 70 $天。在绝大多数动物,一般睾丸的生精功能从青春期开始后,可以持续终生。一个初级精母细胞经过减数分裂后产生四个单倍体精子。睾丸产生的雄激素主要促使雄性动物第二性征的出现和生殖功能的维持。

\subsection{附睾}

哺乳动物的附睾是由$ 8\sim 25 $根输出小管和一条长而卷曲的附睾管组成。附睾分为头、体、尾三部分,其外有致密结缔组织所构成的白膜和鞘膜血管层。

在公马的白膜层有少量平滑肌散布于整个致密结缔组织中。

附睾不仅仅是一个精子运输通道,还是精子浓缩、获得运动能力和受精能力以及储存精子的部位。

\subsubsection{附睾的发育}

\myImportantPoint{附睾是由中肾管发育而来的}。在中肾退化进程中,除了输出管外的其他中肾小管完全退化,留下一光滑管通过腹膜与睾丸疏松地连接;中度卷曲的输出管逐渐变得高度卷曲,并穿过薄薄的睾丸系膜进入附睾头侧。60日龄牛胚胎的附睾为一直管,胚胎发育到80天时附睾管开始卷曲成环状,这种环状结构再次生长并卷曲,最终形成高度卷曲的附睾管。所有动物的附睾都可分为三个部分,位于睾丸前端的是附睾头,附睾体沿睾丸侧缘发育,在睾丸后端是附睾尾。附睾的这三个部分没有明显的分界线。

\subsubsection{结构与功能}

\paragraph{附睾的结构}

主要由输出小管(ductuli efferentes)和附睾管(ductus epididymidis)组成(图 \ref{figure_structure_of_human_testis_and_epididymis} )。

\subparagraph{输出小管}

由$ 8\sim 25 $根输出小管将睾丸网与附睾管连在一起。这些小管聚集在一起并形成具有明显组织界限的小叶。输出小管主要由有纤毛或无纤毛柱状细胞构成,另有淋巴细胞散布于上皮基部。无纤毛细胞上还有微绒毛(microvilli)和发育完全的内吞细胞器.它由被膜小泡(coatedvesicle)、微泡内陷膜、微小管(microcanuliculi)和吸收空泡(resorptive vacuole)组成。

\subparagraph{附睾管}

以组织学、组织化学和超微结构为标准,可将附睾管分为六段,但因动物种类不同而有不同的特点。附睾管极度的弯曲和盘绕,其长度因动物的不同而不等,公牛和公猪的附睾管长约40m,而公马约为70m。另外,精子完全通过附睾管的时间也因动物的不同而有不同,大多数哺乳动物大约需要$ 10\sim 15 $天的时间。附睾管衬以假复层上皮细胞,外被少量疏松结缔组织和环平滑肌纤维,在接近附睾尾时,平滑肌数量明显增多。在上皮中有柱状主细胞(principal cell)、小的多角基底细胞、顶细胞(apical cell)和透明细胞(clear cell)。在大多数动物中,还有巨噬细胞(macrophage)和淋巴细胞(lymphocyte)。在附睾头部的主细胞数量常常较其他部位多。柱状细胞的端面有一长的,且有时是分支的微绒毛,在靠近尾部时微绒毛逐渐变短。

\subsubsection{附睾的主要功能}

\paragraph{输出小管}

构成输出小管的两种细胞的结构特点决定了它的功能主要是运送精子和对小管液重吸收。输出小管的无纤毛细胞上有微绒毛和发育完好的内吞细胞器,有利于对小管液重吸收,从而使管腔中的内环境处于相对的稳定状态,以利于精子的成熟。另外,有纤毛细胞中纤毛的运动可推动精子向附睾管运动,利于精子的成熟与排出。

\paragraph{附睾管}

从睾丸产生的液体大约有90\%被输出管和附睾管吸收。睾丸支持细胞产生的雄激素结合蛋白也是在附睾管的起始段被吸收。附睾管的另一功能是精子在附睾中逗留期,使精子发生下列形态和功能上的变化:
\begin{itemize}
\item 完善其运动性;
\item 代谢改变;
\item 胞质膜改变(膜上有在受精时发挥识别能力的分子);
\item 巯基团的结合增加了膜的稳定性;
\item 失去胞质中的多余小滴。
\end{itemize}


\subsection{输精管}

输精管是附睾管的延续。附睾管在附睾尾部突然弯曲,然后逐渐变直。输精管的初始段位于精索中,腹腔段位于腹膜折叠中。马和反刍动物的输精管与精囊的排出管相连而构成短的射精管,开口于精阜(colliculu seminalts)进入尿道;猪的输精管和精囊排出管分别开口于尿道;\myImportantPoint{肉食动物没有精囊腺},输精管直接与尿道相接。

输精管黏膜上皮为假复层柱状上皮,在该管尾部则变为单一的柱状上皮。固有膜及黏膜下层为疏松结缔组织以及大量的弹性纤维。输精管终末段的固有膜---黏膜层都包含管泡状腺体。肌层包括内层的内环行肌、外层的纵行肌和少量的斜肌肉。

输精管是附睾管的延续,主要是将精子从附睾中输出。另外,输精管线体段的分泌物有利于精子的运动和生存。

\subsection{副性腺}

大多数动物的副性腺包括精囊腺、前列腺、尿道球腺等。肉食动物缺少精囊腺精囊腺。


\subsubsection{精囊腺}

成对的精囊腺属于复合管状腺或管泡状腺。腺上皮是假复层柱状上皮,由高柱状细胞和小而圆的基地细胞组成。叶内小管主要是立方上皮或柱状上皮。黏膜下层富含血管以及精囊腺小叶。

精囊腺的分泌物为白色或黄色胶状液体,一般占射精量的$ 25\%\sim 30\% $,富含果糖,具有\myImportantPoint{为精子提供能量和稀释精子}的功能。

\subsubsection{前列腺}

前列腺的数量不定,位于盆腔部的尿道上皮,属于单管腺泡。根据其局部解剖可将前列腺分为致密部(或叫外部)和扩散部(或叫内部)两部分。整个前列腺由富有平滑肌纤维的结缔组织包裹,平滑肌在致密部较多。被膜的结缔组织伸入腺体内形成的肌质小梁,将腺体分隔成多个单独的小叶。

分泌管、分泌泡以及腺体内小管由单一的立方或是柱状上皮细胞组成,有时可见基地细胞。大部分细胞都含有蛋白质分泌颗粒。高柱状细胞具有微绒毛,有时还可见水泡样顶部凸起。

前列腺的分泌物为黏稠的蛋白样分泌物,偏碱性。\myImportantPoint{前列腺分泌物的作用主要是中和精液和刺激精子的运动}。

\subsubsection{尿道球腺}

成对的尿道球腺位于尿道球状部的背部两侧。尿道球腺外面包有致密的结缔组织,其中含有横纹肌纤维。被膜伸入腺体的实质,将腺体分隔成多个单独的小叶。叶间结缔组织亦含有横纹肌和平滑肌纤维。尿道球腺的分泌部由高柱状上皮细胞构成,有时可见基底细胞。腺管直接开口于集合管或由单一立方上皮组成小管与集合管相连。集合管由单一立方或柱状上皮组成,多个集合管组成较大的腺内导管。

尿道球腺的黏液和蛋白样分泌物在射精时先流出,具有\myImportantPoint{中和尿道内环境、润滑尿道和阴道}的作用。

\subsection{尿道}

雄性动物的尿道分为三段,第一段从膀胱起到前列腺的后缘,第二段从前列腺后缘到阴茎的球状部,第三段是阴茎的海绵体部分,即从阴茎球状部到尿道外开口。整个尿道黏膜呈现纵向折叠,但在阴茎勃起或排尿时纵向折叠消失。尿道上皮主要是由单层柱状上皮、复层柱状上皮或立方上皮组成的移行上皮。固有膜和黏膜下层由致密结缔组织、弹性纤维、平滑肌和弥散的淋巴细胞组成。尿道的第一段和第二段衬以血管层,所以尿道亦与动物生殖器的勃起(erection)功能有关。

\subsection{阴茎}

阴茎由阴茎海绵体(corpora cavernosa penis)和龟头(glans penis)组成。

成对的阴茎海绵体在坐骨结节部融合成阴茎体,外被一层致密的结缔组织白膜。该结缔组织膜含有弹性纤维和平滑肌细胞。阴茎内由完整的结缔组织隔膜分隔的阴茎海绵体。在白膜与小梁间充满海绵状的勃起组织,海绵腔内衬内皮组织,有大量的血管和神经分布。这种结构主要与阴茎的勃起有关。

供应阴茎血液的血管主要是螺旋动脉(helicine artery)。螺旋动脉管壁的平滑肌细胞的舒张可导致大量血液进入海绵体空腔中。海绵体内血液的增多压迫静脉管壁,使海绵体中流出的血液减少,最终血液充满海绵体导致阴茎的勃起。当螺旋动脉中平滑肌细胞收缩时,动脉血流减少,充血的静脉逐渐恢复到原有的状态。