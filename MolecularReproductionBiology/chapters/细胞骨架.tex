\chapter{细胞骨架}

将微分干涉纤维术和录像增强反差显微术相结合,可以在真核细胞的细胞质内观察到一些纤维样的结构,而且纤维的长度和分布模式总处于动态变化之中;此外,还有一些模型细胞器或颗粒状结构沿着这些纤维移动。用电子显微镜观察经非离子去垢剂处理后的细胞,可以在细胞质内观察到一个复杂的纤维状网架结构体系,这种纤维状网架结构通常称为细胞骨架(Cytoskeleton)。细胞骨架包括\textbf{\underline{微丝(MF,Microfilament)}}、\textbf{\underline{微管(MT,Microtubule)}}和\textbf{\underline{中间丝(IF,Intermediate Filament)}}等。它们都是有相应的蛋白亚基组装而成。

细胞骨架是一种\textbf{\underline{高度动态}}的细胞结构体系。在细胞周期的不同时期,细胞骨架具有完全不同的分布状态;在体内各种不同分化状态的细胞中,不仅细胞骨架分布状态存在很大的差异,甚至连构成细胞骨架的蛋白组分也不尽相同。

用物理或化学的手段破坏细胞骨架的结构,将导致细胞形态发生变化,细胞内部各种细胞器和生物大分子分布异常。可见细胞骨架在细胞内发挥着重要的机械支撑与空间定位作用。同时,细胞骨架还是真核细胞结构与功能的重要组织者。细胞骨架不仅与细胞的形态发生相关,而且还参与所有形式的细胞运动,诸如肌肉的收缩、变形运动、细胞迁移、染色体向极运动、纤毛与鞭毛的运动、细胞器和生物大分子的运输、细胞质内生物大分子的不对称分布等。

\section{微丝与细胞运动}

微丝又称\textbf{肌动蛋白丝(Actin Filament)}或\textbf{纤维状肌动蛋白(Fibrous Actin,F-actin)},这种\textbf{\underline{直径约为7nm}}的细胞骨架纤维存在于所有真核细胞中。无论是处于分裂周期中的细胞,还是终末分化细胞,微丝在细胞生命活动过程中发挥着重要的作用。微丝网络的空间结构与功能取决于所结合的\textbf{\underline{微丝结合蛋白(Microfilament-Associated Proteins)}}的种类。在不同类型的细胞内,甚至是在统一细胞的不同部位与之相结合的不同的微丝结合蛋白赋予了微丝网络不通的结构特征和功能,如小肠上皮细胞微绒毛内部的微丝束及细胞皮层的微丝网络、细胞质中与黏着斑相连的张力纤维、迁移中的成纤维细胞前缘的片状伪足和丝状伪足中临时性的微丝束、动物细胞分裂时的胞质分裂环,还有如存在于肌细胞中的细丝等等。细胞内微丝的组装和去组装的动力学过程与\textbf{细胞突起(微绒毛、伪足)的形成}、\textbf{细胞质分裂}、\textbf{细胞内物质运输}、\textbf{肌肉收缩}、\textbf{吞噬作用}、\textbf{细胞迁移}等多种细胞运动过程相关。

\subsection{微丝的组成及其组装}

\subsubsection{结构与成分}

\textbf{\underline{微丝的主要结构成分是肌动蛋白(actin)}}。在大多数真核细胞中,肌动蛋白是含量最丰富的蛋白质之一。在肌细胞中,肌动蛋白占细胞总蛋白量的10\%左右。即使非肌细胞,肌动蛋白也占细胞总蛋白量的$ 1\% \sim 5\% $。肌动蛋白在细胞内有两种存在形式,即\textbf{\underline{肌动蛋白单体(又称球状肌动蛋白,G-actin)}}和由单体组装而成的\textbf{\underline{纤维状肌动蛋白}}。\underline{肌动蛋白单体是由单个肽链折叠而成},相对分子质量为$ 43\times 10^{3} $,外观呈蝶状结构,中央有一个裂口,裂口内部有\underline{ATP结合位点和\ch{Mg^2+}结合位点}。

肌动蛋白在生物进化过程中是高度保守的。在哺乳动物和鸟类细胞中至少已分离到6种肌动蛋白,4种为$ \alpha- $肌动蛋白,分别为横纹肌、心肌、血管平滑肌和肠道平滑肌所特有,它们均组成细胞的收缩性结构,另2种为$ \beta- $肌动蛋白和$ \gamma- $肌动蛋白,存在与所有肌细胞和非肌细胞中。其中\textbf{\underline{$ \beta- $肌动蛋白通常位于细胞的边缘}},\textbf{\underline{$ \gamma- $肌动蛋白肌动蛋白与张力纤维有关}}。对于一个正在迁移的细胞,$ \beta- $肌动蛋白在细胞的前缘组装成微丝。在不同类型的肌细胞中,$ \alpha- $肌动蛋白的一级结构(约400各氨基酸残基)仅相差$ 4 \sim 6 $个氨基酸残基,$ \beta- $肌动蛋白或$ \gamma- $肌动蛋白与$ \alpha- $肌动蛋白(来自横纹肌)相差约25个氨基酸残基。显然,这些肌动蛋白是从同一个祖先基因演化而来。多数简单的真核生物,如酵母或黏菌,仅含单个肌动蛋白基因。然而许多多细胞真核生物含有多个肌动蛋白基因,如海胆有11个,网柄菌属(Dictyostelium)有17个,在某些种类的植物基因组中编码的肌动蛋白基因数目多达60个。

电子显微镜所观察到的微丝是一条直径约为7nm的扭链。根据对微丝进行X射线衍射分析的结构而建立的结构模型认为:\textbf{\underline{每条微丝是由两股螺旋状的丝相互盘绕而成。每条丝都是由肌动蛋白单体头尾相连呈螺旋状排列而成,螺距为36nm}}。在纤维内部,每个肌动蛋白单体周围都有4个单体,上、下各一个,另外两个位于一侧。肌动蛋白分子上的裂口使得该蛋白本身在结构上具有不对称性,在整根微丝上每一个单体上的裂口都朝向微丝的同一端,从而使\textbf{\underline{微丝在结构上具有极性。具有裂口的一端为负极,而另一端为正极}}。

\subsubsection{微丝的组装及动力学特征}

在体外,微丝的组装/去组装与溶液中所含有的肌动蛋白单体的状态(结合ATP或ADP)、离子的种类及浓度等参数相关。通常,只有结合ATP的肌动蛋白才能参与微丝的组装。当溶液中含有适当浓度的\ch{Ca^2+},而\ch{Na+}、\ch{K+}的浓度很低时,微丝趋向于解聚成G-actin;当溶液中含有ATP、\ch{Mg^2+}以及较高浓度的\ch{Na+}、\ch{K+}时,溶液中的G-actin则趋向于组装成F-actin,即新的G-actin加到微丝末端,使微丝延伸,但通常是微丝正极(+)的组装速度较负极(-)快。当溶液中携带ATP的G-actin处于临界浓度时,微丝的组装和去组装达到平衡状态。

肌动蛋白单体组装成微丝的过程大体上可以分为以下几个阶段:

\paragraph{成核反应}
即形成至少有$ 2 \sim 3 $个肌动蛋白单体组成的寡聚体,然后开始多聚体的组装。微丝在细胞内的成核过程需要肌动蛋白相关蛋白\proteinName{Arp2/3}(Actin-related Protein, Arp)复合物的参与,在该复合物内,\proteinName{Arp2}和\proteinName{Arp3}与其他5种蛋白相互作用,形成微丝组装的起始复合物,使肌动蛋白单体与起始复合物结合,形成一段可供肌动蛋白继续组装的寡聚体。

\paragraph{纤维的延长}

肌动蛋白具有ATP酶活性。肌动蛋白单体在参与微丝的组装前必须先于ATP结合,组装到微丝末端的肌动蛋白发挥ATP酶的活性,将ATP水解成ADP。当微丝的组装速度快于肌动蛋白水解ATP的速度时,在微丝的末端就形成一个肌动蛋白--ATP亚基的帽,这种结构使得微丝比较稳定,可以持续组装。相反,当末端的肌动蛋白亚基所结合的是ADP时,则肌动蛋白单体倾向于从微丝上解聚下来。\textbf{\underline{由于微丝两段在结构上存在差异,新的肌动蛋白亚基通常在正极加入,而很少在负极加入。}}

\paragraph{稳定期}

待微丝组装到一定长度时,肌动蛋白亚基的组装和去组装达到平衡状态,即组装和去组装的肌动蛋白亚基数目相当,微丝的长度几乎保持不变,即所谓的``稳定期''。在体外组装过程种有时可以看到微丝的正极由于肌动蛋白亚基的不断添加而延长,而负极则由于肌动蛋白亚基去组装而缩短,这一现象称为\textbf{\underline{踏车行为(treadmilling)}}。

\subsubsection{影响微丝组装的特异性药物}

一些药物可以影响肌动蛋白的组装或去组装,从而影响细胞内微丝网络的结构。\textbf{\underline{细胞松弛素(Cytochalasin)}}是一组真菌的代谢产物,与微丝结合后可以将微丝切断,并结合在微丝末端阻抑肌动蛋白在该部位的聚合,但对微丝的解聚没有明显影响,因而用细胞松弛素处理细胞可以破环微丝的网络结构,并阻止细胞的运动。\textbf{\underline{鬼笔环肽(Phalloidin)}}是一种由毒蕈(Amanita Phallodies)产生的双环杆肽,与微丝表面有强亲和力,但不与肌动蛋白单体结合,对微丝的解聚有抑制作用,可使肌动蛋白丝保持稳定状态。用荧光标记的鬼笔环肽染色可清晰地显示细胞种微丝的分布。将鬼笔环肽注射到细胞内同样能阻止细胞运动,可见微丝的功能依赖于肌动蛋白的组装和去组装的动态平衡。

\subsection{微丝网络动态结构的调节与细胞运动}

\subsubsection{非肌肉细胞内微丝的结合蛋白}

尽管纯化的肌动蛋白单体可以在合适的体外环境下组装成纤维状肌动蛋白,但其复杂性和有序性都远不能与细胞内的微丝网络相比。细胞内的微丝具有复杂的三维网络结构,有些微丝结构稳定,如肌细胞中的细丝及小肠上皮细胞微绒毛中的轴心微丝束等;另一些微丝结构是暂时性的,如胞质分裂环是由微丝形成的收缩环。血小板激活及无脊椎动物精子细胞顶体反应过程中出现的微丝也是暂时性结构,都是在功能需要时才进行组装的。实际上,在大多数非肌肉细胞中,微丝是一种动态结构,它们持续地进行组装和去组装。微丝的这种动态不稳定性与细胞形态的维持及细胞运动有密切的关系。体内肌动蛋白的组装在两个水平上受到微丝结合蛋白的调节:可溶性肌动蛋白的存在状态、微丝结合蛋白的种类及其存在状态;在不同的细胞,甚至是同一细胞的不同部位,由于微丝结合蛋白的种类及存在状态上的差异而致使微丝网络的结构完全不同。

在细胞内,可溶性的肌动蛋白单体和纤维状肌动蛋白的比例大体是$ 1:s1 $。也就是说,细胞内游离态肌动蛋白的浓度远远高于肌动蛋白在体外组装所需的临界浓度,但由于细胞内游离态肌动蛋白常与另外一些相对分子量较小的蛋白(如胸腺肽和抑制蛋白等)结合在一起,从而是G-actin组装成F-actin的过程受到必要的调控,存储在细胞内的G-actin只有在需要时才加以利用。

细胞内微丝网络的组织形式和功能通常取决于与之结合的微丝结合蛋白,而不是微丝本身。细胞微环境内的各种微丝结合蛋白通过影响微丝的组织与去组装,介导微丝与其他细胞结构之间的相互作用来决定微丝的组织行为。有些微丝结合蛋白如封端蛋白、成束蛋白等与微丝的相互作用,可以使微丝保持相对稳定状态;而另外一些如纤维--切割蛋白与微丝的作用,则通过使微丝网络解聚来调节微丝网络的状态。如在小肠上皮细胞的微绒毛内,毛缘蛋白和绒毛蛋白等结构成分将相邻的微丝交联成平行排列的微丝束。在细胞皮层,交联蛋白和凝溶胶蛋白通过将微丝交联或切断来调节细胞皮层的凝溶胶状态和细胞质膜的形态。此外,微丝还可以通过和肌球蛋白之间的相互作用来运送``货物'',对细胞内生物大分子及细胞器的分布起组织作用,从而调节细胞的行为。人们已经从各种组织细胞中分离到了100多种不同的微丝结合蛋白,根据它们在细胞内功能的不同可以将它们归纳如表 \ref{tableMicrofilamentRelatedProteins} 所示:

\noindent \begin{longtable}{|c|c|p{6em}|p{10em}|}

\hline 
微丝结合蛋白类型 & 微丝结合蛋白名称 & 相对分子质量($10^3$) & 主要功能 \endhead
\hline 
成核蛋白 & Arp2/3复合体 & 由7个亚基聚合而成 & 在微丝开始组装时起成核作用 \\ 
\hline 
单体--隔离蛋白 & 胸腺素(thymosins) & 5 & 与肌动蛋白单体结合,调节肌动蛋白的组装 \\ 
\hline 
单体--聚合蛋白 & 抑制蛋白(profilin) & $ 12 \sim 15 $ & 一种ATP--肌动蛋白结合蛋白,能够在细胞运动过程中促进肌动蛋白的聚合 \\ 
\hline 
\multirow{4}{*}{成束蛋白} & 丝束(毛缘)蛋白(fimbrin) & 68 & 横向连接相邻微丝形成紧密的微丝束 \\ 
\cline{2-4} 
 & 绒毛蛋白(villin) &  95 &  \\ 
\cline{2-4} 
 & 成束蛋白(fascin) & 57 &  \\ 
\cline{2-4} 
 & $\alpha-$辅肌动蛋白($\alpha-$actinin) &  &  \\ 
\hline 
\caption{微丝结合蛋白的主要类型}
\label{tableMicrofilamentRelatedProteins}
\end{longtable} 

\subsubsection{细胞皮层}

免疫荧光染色的结果显示,细胞内大部分微丝都集中在紧贴细胞质膜的细胞质区域,并由微丝结合蛋白交联成凝胶状三维网络结构,该区域通常称为细胞皮层(Cell Cortex)。皮层内一些微丝还与细胞质膜上的蛋白由连接,使膜蛋白的流动性受到一定程度的限制。皮层内密布的微丝网络可以为细胞质膜提供强度和韧性,有助于维持细胞形状。细胞的运动,如胞质环流(cyclosis)、阿米巴运动(amoiboid)、变皱膜运动(ruffled membrane locomotion)、吞噬(phagocytosis)以及膜蛋白的定位等都与皮层肌动蛋白的溶胶态或凝胶态转化相关。

\subsubsection{应力纤维}














