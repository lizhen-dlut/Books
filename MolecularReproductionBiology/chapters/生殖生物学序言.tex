\chapter{生殖生物学序言}
\newpage
生殖是生物体的基本特征之一,生物通过生殖实现一代到下一代生命的延续。不同生物生殖过程的复杂性差异很大,生殖方式可分为无性生殖和有性生殖两种形式。生殖本身是一个古老的话题,而生殖生物学是用现代生物学手段研究整个生殖过程的一门学科,是在动物胚胎学、繁殖学、妇产科 学、发育生物学及动物生理学等学科的基础上发展起来的一门新兴学科。随着人工授精、体外受精、胚胎移植、细胞核移植,以及很多辅助生殖技术的 广泛应用,这门学科在生物学、医学及农业中的地位越来越重要。而且,随 着细胞生物学、分子生物学、生物化学、生理学等学科的飞速发展,以及各 种现代生物学技术在生殖过程研究中的广泛应用,人们对生殖过程中的各种 现象及其分子机制的了解有了长足的进步。

生殖生物学的研究不仅可使我们了解生命繁衍的奥秘, 也在多方面使人
类自身受益。20世纪中叶, 生殖内分泌基础研究的重大突破, 导致了女用 口服肖体避孕药的出现;精子获能现象的发现, 导致了目前广泛应用于不育 症治疗的试管婴儿技术的产生;20世纪80年代, 对激素受体研究的重大发现产生了第二代肖体避孕药。1997年克隆羊 多莉 的出生, 引发了很 多深层次的理论间题和人们对该技术应用前景的期望。 所有这些突破性研究 成果对生殖生物学的发展起到了巨大的推动作用。 目前, 不孕症的发病率在逐年上升(约为$ 15\% \sim 20\% $), 而且病因也趋向于复杂化, 使得辅助生殖技 术面临极大的挑战。 各种环境因素对生殖过程的不良影响也越来越明显。 人 们对避孕措施的安全性、 可靠性及多样性等方面的要求也越来越高, 发展非 肖体、 无毒无害、 无副作用的避孕措施是社会发展的必然要求。 随着我国畜牧业的迅速发展, 对肉、 蛋和奶的需求逐年增加, 迫切需要增加奶牛、 肉 牛、 肉羊、 猪等的数址,并提高其质量, 亟待加速各种辅助生殖技术在畜牧业中的应用。 这些需求一方面为生殖生物学研究提出了新的挑战, 另一方面 也为生殖生物学的发展提供了一个难得的发展机遇。 随着多聚酶链反应、 基 因敲除、 基因芯片、 蛋白芯片、 基因组学、 蛋白质组学、 细胞分离技术等方 面的飞速发展, 许多以前未知的生殖现象、 生殖过程及生殖功能逐渐被发现 和了解,也为生殖生物学提供了更加宽广的发展空间。

目前,国内尚尤一本系统介绍生殖生物学领域研究成果和发展动态的书,该书是国内从事生殖生物学研究的一批学者,综述当前国内外该领域的最新研究进展,并结合自己的研究成果编写的一部全面反映生殖生物学基本概念、基础理论、研究技术、研究成果和发展动态的综合性教学和科研用书。该书的三位主编均为国家杰出青年科学基金获得者,杨增明博士和夏国良博士现为“教育部长江学者奖励计划”特聘教授,孙青原博士2003年获得“中国科学院十大杰出青年”称号。

我期望这本书的问世将对我国生殖生物学的发展起到积极的推动作用,并对生殖生物学领域的研究、教学和临床实践等有所裨益。

\begin{flushright}
\kaishu 
\begin{tabular}{c}
{\huge 薛社普}\\  
中国医学科学院教授 \\  
中国科学院资深院士 \\  
2004年04月15日 \\  
\end{tabular} 
\end{flushright}