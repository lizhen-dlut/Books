\chapter{生殖激素}

在生理学中,把直接作用于动物的生殖活动,并以调节生殖过程为主要生理功能的激素叫生殖激素。生殖激素的种类很多,按其来源、分泌器官以及转运机制的不同,可分为 \ref{生殖激素分类列表} 大类:\begin{inparaenum}[\itshape 1:\upshape]\item \myImportantPoint{脑部激素},由脑部各区神经细胞或核团如松果体、下丘脑和垂体等分泌,主要调节脑内和脑外生殖激素的分泌活动;\item \myImportantPoint{性腺激素},由睾丸或卵巢分泌,接受脑部激素和其他因素的调节或影响,对生殖细胞的发生和发育、排卵、受精、妊娠和分娩,以及雌性和雄性动物生殖器官的发育等生殖活动都有直接或间接作用;\item \myImportantPoint{胎盘激素},由雌性动物胎盘产生,对于维持妊娠和启动分娩等有直接作用;\item \myImportantPoint{其他组织器官分泌的激素},指产生于生殖系统以外的内分泌组织或器官,对卵巢发育、黄体消退等具有直接作用,主要为前列腺素;\item \myImportantPoint{外激素},由外分泌腺体(有管腺)所分泌,主要借助空气和水传播而作用于靶器官,主要影响动物的性行为\label{生殖激素分类列表}\end{inparaenum}。

在内分泌系统中,下丘脑、垂体和性腺分泌的激素在功能和调节上相互作用,构成了一个完整的神经内分泌生殖调节体系,即下丘脑---垂体---性腺轴(HPG,hypothalamus---pituitary--gonadal axis),它在动物生殖活动的调节中起着核心作用。

\section{脑部生殖激素}

脑是神经内分泌激素的主要分泌器官。调节生殖活动的神经内分泌激素主要由下丘脑机器周边组织、间脑、垂体及松果体等神经组织的细胞合成和分泌。神经内分泌学是了解神经调节垂体和其他内分泌组织和器官的机制,以及类固醇或其他内分泌激素作用于神经系统的机制,它是内分泌和神经科学的融合。

内分泌的主要概念是内分泌组织分泌的激素进入血液循环调节靶器官或细胞的功能,而神经科学的概念是单个神经元之间的相互联系形成的神经系统的活动。神经元能够通过它们的生物电特征快速有效地将信息传递到远方,此外,神经元间的联系依赖于信息从一个神经元传递到其他神经细胞、腺体或肌细胞,即通过突触间化学递质的释放。

这两个系统曾经是两个独立的体系。自从认识到下丘脑内特殊的神经元调节着垂体前叶的内分泌细胞,并且通过这种下丘脑---垂体的相互作用,最终调节着整个内分泌系统的功能,从此这两个系统才融合到一起。这种科学上的进步最早源于对鱼神经分泌的研究,发现其下丘脑内的神经元通过释放激素进入血液而发挥内分泌作用。而后通过Ernest Scharrer和他的合作者的研究,最终认识到垂体后叶不是真正的腺体,而是由下丘脑内某些神经元的突触组成的,它们释放激素产物直接进入血液循环。

该系统的进一步扩增是认识到神经元调节垂体前叶也是通过释放递质进入血液完成的。不同的是这些递质并不进入体循环,而是严格限制在与前叶细胞密切联系的特殊化的循环系统中。这一系统的最终形成,是由于发现了这些神经元释放的递质,在化学组成和作用机理上,与内分泌系统分泌的激素有着很高的同源性。

这样人们才对神经内分泌系统参与机体对情绪变化产生的影响有了认识,同时对内分泌功能对于精神活动的影响也有了理论依据。本节只重点介绍与生殖内分泌有关的神经内分泌知识。

\subsection{下丘脑与垂体的结构和功能}

\subsubsection{下丘脑}

下丘脑位于丘脑的腹侧、间脑的基底部,是下丘脑沟以下的神经组织,被第三脑室下部均分为左右对称的两半,构成第三脑室下部的侧面和底部,其体积约为整个大脑的$ \frac{1}{300} $。下丘脑内包含许多左右成对的核团,其中与内分泌有关的核团大多在下丘脑的内侧部,可分为 \ref{下丘脑解剖结构列表} 个区:\begin{inparaenum}[\itshape 1:\upshape]\item 前区(视上区),包括视交叉上核、视上核、视旁核等。\item 中区(结节区),包括正中隆起、弓状核、腹内侧核等。\item 后区(乳头区),包括背内侧核、乳头体等\label{下丘脑解剖结构列表}\end{inparaenum}。构成这些核团的、具有内分泌功能的神经细胞有两种:神经内分泌大细胞和神经内分泌小细胞,神经内分泌大细胞主要分布在视上核和视旁核,主要分泌催产素和加压素;神经内分泌小细胞较集中地分布在正中隆起、弓状核、视交叉上核、腹内侧核等,分泌各种释放激素,调节腺垂体的功能,因此将这一部分又统称为下丘脑促垂体区。通常,下丘脑前区、视前核和视交叉上核等神经核团分泌的释放激素可调节垂体在排卵前促卵泡激素(FSH)和促黄体激素(LH)的分泌活动,故又将这些区域称为周期分泌中枢。腹中核、弓状核和正中隆起等神经核团分泌的释放激素可调节垂体FSH和LH的基础性或持续性波动分泌,因此这些区域又称为持续分泌中枢。

\subsubsection{垂体}

垂体位于下丘脑之下的蝶骨垂体凹内,通过狭窄的垂体柄与下丘脑相连。垂体由两个主要部分组成,即腺垂体(adenohypohysis)和神经垂体(neurohypophysis)。也有人将垂体分为前叶、中间部和后叶三个部分。前叶与腺垂体含义相近,而后叶与神经垂体含意相近,但并不完全等同,中间部则是腺垂体的一部分。腺垂体由远侧部、结节部和中间部组成,神经垂体由神经部和漏斗部组成。漏斗部包括漏斗柄和灰结节的正中隆起,远侧部和结节部合称垂体前叶(anterior pituitary),神经部和中间部合称为垂体后叶(posterior pituitary)。腺垂体和神经垂体是由不同的胚层发育而来,其中远侧部来自于内胚层(源自于神经颊囊),而中间部和神经部来自于神经外胚层。

\subsubsection{下丘脑与垂体的联系}

下丘脑虽然没有神经纤维直接进入垂体前叶,但它们之间通过血管(垂体门脉管)建立了功能上的联系。下丘脑神经元的纤维末梢在正中隆起处大量分布,因此这些神经元合成的激素通过神经纤维末梢分泌到正中隆起,进入下丘脑---垂体门脉血管的毛细血管丛,再经门脉血液循环到达垂体前叶。例如,促性腺激素释放激素(GnRH)是在正中视前核产生的,多巴胺是在弓状核产生的。这两种物质从下丘脑通过轴突运输到正中隆起,在那里被释放进入门脉系统。垂体后叶可以说是下丘脑的延续部分,它是下丘脑垂体束的无髓鞘神经末梢与神经胶质细胞分化的神经纤维组成,没有腺体细胞,不能产生激素,只能储存和释放视上核、视旁核分泌的神经激素。

\subsection{下丘脑激素}

由下丘脑神经细胞合成并分泌的激素均为多肽类激素,其中对垂体激素的分泌和释放活动具有促进作用的称为释放激素或释放因子,而另一些对垂体激素的分泌和释放活动具有抑制作用的称为抑制激素或抑制因子。与生殖相关的下丘脑激素主要有促性腺激素释放激素、催产素、促乳素释放因子和促乳素释放抑制因子等四种。

\subsubsection{促性腺激素释放激素(GnRH,gonadotropin releasing hormone)}

促性腺激素释放激素最早于1971年由Guillemin和Schally研究组各自独立地从绵羊及猪的下丘脑中分离出来,随后其机构得到进一步研究,现已可人工合成。至今,人们从各种脊椎动物和原索动物中已鉴别出15中GnRH的分子结构。GnRH已被证明是下丘脑---垂体---性腺轴的关键信号分子。

\paragraph{GnRH的结构}

哺乳类(猪和羊的下丘脑以及人的胎盘)GnRH(mGnRH,mammalian GnRH)具有相同的化学结构,是由9种不同氨基酸残基组成的十肽($ PGlu -His-Try-Ser-Tyr-Gly-Leu-Arg-Pro-Gly-NH_2 $),而禽类、两栖类及鱼类的GnRH则具有不同的结构。GnRH基因内有3个内含子和4个外显子,由第2、3外显子和第4外显子的一部分共同编码GnRH前体,该前体包含一段$ 21 \sim 23 $个氨基酸的信号肽、10个氨基酸的GnRH、一个断裂位点(Gly-Lys-Arg)和$ 40 \sim 60 $个氨基酸的相关肽(GAP,GnRH associated peptide)。

\paragraph{合成和运输}

利用放射免疫和免疫酶标记定位技术,现已基本确定GnRH主要由下丘脑视前区、内侧视交叉前区、弓状核等区域或核团中的肽能神经元合成。另外,在松果体、脊髓液和脑外组织,包括肠、胃、胰脏、输卵管、子宫内膜、胎盘及交感神经节等器官和组织中,也发现有GnRH类似物存在。

GnRH合成后以颗粒或囊泡的形式储存于细胞中。当神经元受到某种刺激时,这些GnRH或沿轴突输送至正中隆起处释放出来,通过垂体门脉血液循环进入垂体前叶,或进入第三脑室的闹脊髓液中,由正中隆起处的多突室管膜细胞转运至垂体门脉。

\paragraph{下丘脑GnRH脉冲的产生}

有关这方面的研究较为复杂,这里只介绍一些下丘脑GnRH脉冲产生,以及控制GnRH脉冲性释放的功能性和解剖性证据。

测定恒河猴垂体门静脉血中GnRH浓度的实验表明,它具有独立的脉冲释放类型。GnRH脉波峰的峰顶与峰谷分别在$ 400 pg/ml $到$ 0 pg/ml $之间变化。在人第三脑室闹脊髓液中也有类似的GnRH脉冲类型,但幅度较小。GnRH分泌到第三脑室表明其他下丘脑释放因子有进入第三脑室的可能,如CRF。这可能是由于室管周轴突网络与第三脑室室管膜联系密切的结果。有关GnRH在脑脊髓液中的生理意义还不清楚。

一些在恒河猴中进行的实验表明,GnRH分泌的快速短节律性的释放月为1小时一次,来源于下丘脑内侧基底部(MBH,MedioBasal Hypothalamus)的弓状核附近。可以观察到垂体门静脉中GnRH的脉冲,电生理多单位活性(MUA,Multi-Unit activity)和外周LH脉冲之间有明显的同步化。可见,MBH内的昼夜节律变化是控制神经轴突末端GnRH在正中隆起处的脉冲释放,并最终控制垂体促性腺激素脉冲式释放的关键控制器。

在人类,下丘脑的GnRH脉冲释放以及GnRH脉冲释放的产生器也已经被间接地证明。在非人灵长类动物,构成GnRH脉冲式释放的产生器位于MBH,能够起功能性起搏器的作用(不依赖于来自脑部其他部位的神经分布)。MBH的这种自发性的脉冲已得到体外实验的证实。用体外灌流实验曾经观察到,人胎儿MBH(妊娠20到30周的MBH)和成年人MBH有单独的GnRH脉冲。GnRH的脉冲周期约为$ 60min $(胎儿MBH)和$ 60\sim 100min $(成年人MBH)。这些结果确定了人类和猴一样,下丘脑GnRH脉冲产生系统都位于MBH。

在摘除卵巢的猴子中,GnRH脉冲产生器的功能表达可明显增强。尽管摘除卵巢的猴子的脉冲产生器的频率与处于卵泡期早期的猴子中的频率是相似的,但集团脉冲排放的持续时间在前者中明显降低(从$ 11min $到$ 2min $)。这种在摘除卵巢的猴子中看到的集团冲动持续期降低的现象可用注射雌激素和鸦片类物质来恢复。可见雌激素和鸦片类物质对于MBH的自发冲动产生具有调节作用。

在月经周期的不同时期中,GnRH脉冲产生器活动的相对变化与在月经周期中观察到的LH脉冲重卵泡期到黄体期的变化一样,即从卵泡期的高频低幅变成黄体期的低频高幅。GnRH脉冲产生的频率在周期的卵泡期的夜晚明显降低,这支持了妇女卵泡期早期在睡眠时LH脉冲变慢的事实。当雌激素和孕酮在黄体期晚期开始下降时,GnRH脉冲产生器的频率变得明显,并持续增加知道卵泡期的头几天,此时达到频率高峰。

GnRH脉冲产生器是维持生殖整合的基础。然而,有关GnRH脉冲产生的细胞学基础还不十分清楚。

\paragraph{GnRH的生物学功能}

GnRH对生殖过程的神经内分泌调控起着中心作用,是性行为的重要介导者,它对下丘脑及其以外的一些神经元有兴奋和抑制两种效应。GnRH在哺乳动物中含量极低,且不同组织中的GnRH具有不同的生物学功能。下丘脑中GnRH可调控促性腺激素的释放;胎盘中的GnRH可调控人绒毛膜促性腺激素(hCG,human chorionic gonadotropin)的分泌;肿瘤中的GnRH可一直癌细胞的增值;消化系统中的GnRH的功能目前尚不明确。一般认为,GnRH通过旁分泌/自分泌机制,局部调节血浆促性腺激素以及性类固醇激素的水平,从而改变动物的性行为。因此,GnRH可在垂体、性腺等多个水平上影响生殖活动过程。

\subparagraph{在垂体水平的功能}

在正常生理状态下,GnRH呈脉冲式释放,故垂体促性腺细胞也呈脉冲式分泌LH与FSH。GnRH的脉冲式释放可调节LH/FSH的比值。当GnRH的脉冲频率减慢时,卵巢摘除和下丘脑功能丧失的猴血中FSH水平升高,LH水平下降,这祥LH/FSH值下降,在人类中也有相似的研究结果。这可能是因为LH的半衰期较短($ 47min $),很快被消除,而FSH的半衰期较长($ 240min $),从而在血中积累的结果。反之,GnRH的脉冲频率增加,使LH/FSH比值上升。此外,LH和FSH对GnRH的促分泌反应有所不同。例如,给大鼠、免、绵羊等动物快速静脉注射GnRH时,主要引起血浆中LH水平明显升高,可是当用一定剂量的GnRH缓慢注射时,不但使LH的水平升高,而且
FSH水平也明显升高。给人注射人工合成的GnRH类似物时发现,注射后$ 5min $血浆中LH水平就升高,$ 25\sim 30 min $达到峰值,而FSH的反应较缓慢,在$ 45min $才达到峰值。

一些体外实验亦证实,垂体细胞受GnRH刺激时,LH分泌量的变化幅度比FSH的大。另有研究表明,GnRH主动免疫引起动物血浆中LH水平急剧降低,而FSH只有轻度降低,从而进一步证实了上述差异的存在,即在GnRH刺激下,垂体细胞LH分泌相应地出现明显脉冲,而FSH分泌则缓慢而持久。

造成这种差异的原因可能有以下两方面:一是下丘脑GnRH神经元受到神经中枢其他部位传入的信息的影响及性腺激素反馈作用,以不同的脉冲频率释放GnRH后作用于垂体,引起垂体细胞分泌反应的改变。因为一些研究发现,GnRH以较高脉冲频率(3次每小时)释放且作用时间较短时,主要促进LH分泌;而以较低频率(1次每小时)且持续释放时,则主要引起FSH分泌。所以,在GnRH刺激下,垂体细胞LH分泌相应地出现明显的脉冲,而FSH分泌则缓慢而持久。二是性腺分泌的抑制素(inhibin)对垂体FSH特异性的抑制作用,可能是引起FSH对GnRH的促分泌反应不如LH明显的一个因素。除了调节促性腺激素的分泌以外,GnRH对促性腺激素亚单位的mRNA表达也有调节作用。在卵巢切除造成的绵羊下丘脑---垂体功能性联系切断(HPD)的实验中,当HPD一周后,LH两个亚单位LH-$ \alpha $和LH-$ \beta $的mRNA表达降低到不能检测的水平,而当GnRH每小时注射一次的方法处理时又能恢复到高水平。在大鼠模型中,较高的GnRH频率有道LH和FSH的$ \alpha $--亚单位mRNA表达,但对$ \beta $--亚单位mRNA影响极小。生理频率(大鼠约为每小时2次)的GnRH增加可使所有促性腺激素亚单位的mRNA表达,而较低频率(每两小时1次)只选择性增加FSH $ \beta $--mRNA的表达。






\section{性腺激素}

\section{胎盘激素}

\section{前列腺素}