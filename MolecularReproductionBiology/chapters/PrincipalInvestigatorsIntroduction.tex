\chapter{作者简介}

\section{杨增明} 

1962 年 8 月出生于甘肃省兰州市

\lifeSpan{1979}{1986} 年在兰州大学攻读学士和硕士学位;

\lifeSpan{1986}{1989}年在东北农业大学攻读博士学位;

\lifeSpan{1989}{1991}年在中国科学院动物研究所从事博士后研究;

\lifeSpan{1991}{1996}年先后在美国北卡罗来纳州立大学、 贝勒医学院和堪萨斯大学医学中心从事动物生殖方面的研究。 现为东北农业大学教授及博士生导师。

2000 年被聘为教育部“长江学者奖励计划”特聘教授。 \myImportantPoint{研究方向为哺乳动物胚胎发育和胚胎着床}。

1998 年获“国家杰出青年基金”及国家教育部“跨世纪优秀人才培养计划”基金。 

自 1998 年以来, 先后主持国家自然科学基金重点项目 2 项、美国避孕药研究与开发项目 2 项等 12 项课题。 于 1994、 1997 及 1998 年先后获得 3 项美国专利。 共计发表科研论文 110 篇, 其中 35 篇发表在 SCI 源期刊。

现为美国Biologyof Reproduction和Theriogenology等杂志的特邀审稿人,并兼任中国《动物学报)和《动物学杂志)编委。 

1998 年被评为“全国优秀教师”,2002 年获“中国青年科技奖”,2003 年获“国家留学回国人员成就奖”。

现为国务院学位委员会第五届学科评议组成员、 全国生殖生物 学会副理事长、 中国动物学会细胞及分子显微技术学分会副主任委员、 美国生殖生物学会会员。

\section{孙青原} 

1964 年 12 月出生于山东省招远市。 

现任中国科学院动物研究所研究员、 中国科学院研究生院教授、 博士生导师。

1994 年毕业于东北衣业大学, 获理学博士学位。 

1996 年由中国科学院动物研究所博士后出站后留所工作。 

曾先后在美国、日本和以色列进行过四年的博士后培训。\myImportantPoint{研究方向为卵母细胞减数分裂、 受精和早期胚胎发育的基因表达与信号转导和动物克隆机理}。

发表 SCI 收录论文100 篇,其中有关影响因子大于2.0 的46 篇,在生殖生物学领域最具影响的Biology of Reproduction上发表22 篇,影响因子总和190被引用 600 余次。受邀为包括Biology of Reproduction在内的8种SCI期刊撰写综述。 作为副主编编著了由科学出版社出版的《受精生物学》;参编了美国Humana Press出版的《\textit{Methods in Molecular Biology: Germ Cells}》、《动物发育生物学》和《大百科全书》等著作。 是Biol Reprod、Reproduction、DevDynam等20余种国内、 外期刊的审稿人和日本动物繁殖学会会志J Reprod Dev的编委会委员。

1999年获“中国科学院青年科学家奖”,2002年获国家杰出青年科学基金资助,2003 年获“中国科学院十大杰出青年”称号和“国家留学回国人员成就奖”, 2004年获“中国青年科技奖”和“新世纪百千万人才工程国家级人选”称号。

\section{夏国良}

现任中国农业大学生物学院教授、博士生导师。教育部“长江学者奖励计划”特聘教授。“国家杰出青年基金”获得者。

\lifeSpan{1983}{1988}年师从北京农业大学动物生理生化杨传任教授,攻读硕士和博士学位研究生并从事生殖内分泌学的研究。

\lifeSpan{1991}{1994}年,公派到丹麦国家教学研究医院(Rigshospitalet)生殖生物学实验室做博士后研究工作,师从国际著名胚胎学家Anne Grete Byskov教授。\myImportantPoint{从事胚胎卵巢中卵细胞的发育和调节的研究}。

1994年回到母校中国农业大学工作,当年破格提升为教授。在回国人员启动基金和国家自然科学基金的资助下,继续研究\myImportantPoint{促性腺激素诱导卵母细胞体外成熟的作用机制}。已发表论文90余篇,其中在国内外重要的SCI源期刊中发表30余篇。论文被国外引用200余次。

此外,还担任《中国农业大学学报》编辑委员会副主任、主编,《中华中西医》杂志常务编委、《动物学杂志》编委、中国农业大学国家“211工程”项目“畜禽细胞与分子生物学实验室”负责人、中国生殖生物学会副理事长、中国生理学会理事、中国动物学会教学工作委员会委员、中国畜牧兽医学会动物生理生化学会副理事
长。
