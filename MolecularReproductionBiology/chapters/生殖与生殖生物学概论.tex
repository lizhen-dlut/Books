\chapter{生殖与生殖生物学概论}

``天地之大德曰生''。生命永远使宇宙中最宝贵的,生命具有无可争辩的意义,是第一本位的。``种''的繁衍生殖自然就具有无与伦比的重要意义。生命的传承、沿袭是人类赖以永恒存在的源泉。宇宙中的一切事物,因为有了生命的存在才显示了自身的价值和意义。每个有生命的个体总会以某种方式繁衍与自己性状相似的后代以延续生命,这就是生殖(Reproduction)。从生理的角度上看,生殖是一切生物体的基本特征之一,一个个体可以没有生殖而生存,但一个物种的延续则必须依赖于生殖。

生物通过生殖实现亲代与后代个体之间生命的延续。尽管遗传信息决定了后代延承亲代的特征,但遗传是通过生殖而实现的。亲代遗传信息在传递过程中会发生变化,从而使物种在维持稳定的基础上不断进化成为可能。生命的延续本质上是遗传信息的传递。在生物代代繁衍的过程中,遗传和变异与环境的选择相互作用,导致生物的进化。因此,生殖过程本身除了是生物由一代延续到下一代的重要生命现象外,与遗传、进化,甚至生命起源的问题紧密相关。

\section{生殖现象的研究历史}

Macedonian Aristotle(公元前\lifeSpan{384}{322}年)是最早系统从事动物生殖与发育方面研究的学者,首先提出了胚胎是由简单到复杂逐渐形成的观点。1683年Antoni van Leeuwenhoe首次在精液中发现了精子,并提出``精源说'',认为精子存在人的雏形,发育只是这个雏形的放大而已。以后Marcello Malpighi(\lifeSpan{1629}{1694})和Jan Swamerdam(\lifeSpan{1637}{1680})等又提出了``卵源说'',认为卵子中存在一个人的雏形。此后,Charles Bonnet(\lifeSpan{1720}{1793})年在蚜虫中首次发现了孤雌生殖现象。

Lazzaro Spallanzani(\lifeSpan{1729}{1799})首次成功地进行了青蛙的人工授精,并发现在缺乏精子穿入时,则卵子发生退化。在进行狗的实验时,他提出只有当卵子和精液共同存在时,才能产生一个新的个体。Caspar Friedrich Wolff(\lifeSpan{1738}{1794})观察到,从受精卵的卵黄中形成了有形态结构的胚胎。Carl Ernst von Baer(\lifeSpan{1792}{1876})对几种哺乳动物的卵子进行了比较研究。以后,Ernst Haeckel(\lifeSpan{1834}{1919})提出了个体发育是系统发育简要重演的观点。

Oscar Hertwig(\lifeSpan{1849}{1922})和Richard Hertwig(\lifeSpan{1850}{1937})兄弟在Otto Butschli的研究基础上,进一步对受精现象进行了研究,提出受精的本质是雌雄配子细胞核的融合。并且,Oscar Hertwig在海胆的卵子中观察到极体以及极体中的细胞核。

1883年,van Beneden在蛔虫受精卵的第一次有丝分裂纺锤体上看到四条染色体,其中两条来自父方,两条来自母方,提出父母的染色体通过受精卵的融合传递给子代。此后,Theodor Boveri(\lifeSpan{1862}{1915})通过对蛔虫卵的进一步观察,提出了染色体理论,并通过实验证实了染色体对发育的重要作用。20世纪初,美国生物学家McClung(1902)第一次将X染色体和昆虫的性别决定联系起来。Stevens(1905)及Edmund Beecher Wilson(1905)同时将XX性染色体与雌性对应,而XY及XO与雄性相关联,并提出一种特异的核成分在性别表型决定中起作用,即性别有遗传而非环境决定。虽然,自1921年以来,就已知道男性中具有X和Y染色体,而女性中具有两条X染色体,但这些染色体在人性别决定中的作用在1959年以前一直不清楚。Jacobs和Strong(1959)以及Ford等(1959)首次证明,Y染色体在小鼠和人类的性别决定中起关键作用。

\section{生殖过程}

\mykeyword{生殖}是指所有的生物能够产生与它们自己相同或相似的、新的生物个体的能力,也指单细胞或多细胞的动物或植物自我复制的能力。在各种情况下,生殖都包括一个基本的过程,即亲本的原材料或转变为后代,或编程将发育成后代的细胞。生殖过程中也总是发生遗传物质从亲代到子代的传递,从而是子代也能复制它们自己。在不同生物中,尽管生殖过程所采取的方式和复杂性变化很大,但都可分为两种基本的生殖方式,即\mykeyword{无性生殖(asexual reproduction)}和\mykeyword{有性生殖(sexual reproduction)}。在无性生殖中,一个个体可分成两个或两个以上相同或不同的部分,仅有一个亲本的参与,生殖过程中没有配子的形成。而在有性生殖过程中,特化的雄性生殖细胞和雌性生殖细胞发生融合,形成的合子同时携带两个亲本的遗传信息。

\subsection{无性生殖}

无性生殖的优点在于可使有益的性状组合持续存在,不发生改变,并且不需要经过易受环境因素影响的早期胚胎发育的生长期,常见于大多数的植物、细菌、原生生物及低等的无脊椎动物中。单细胞生物常以\mykeyword{分裂(fission)}方式或\mykeyword{有丝分裂(mitosis)}方式,分成两个新的、相同的个体。所形成的细胞可能聚集在一起形成丝状(如真菌),也可能成群生长(如葡萄球菌)。\mykeyword{断裂}或\mykeyword{裂片生殖(fragmentation)}是指在丝状的生物中,身体的一部分断裂后,发育形成一个新的个体。\mykeyword{孢子生殖}或\mykeyword{孢子形成(sporulation)}为原生动物及许多植物中的一种无性生殖方式:一个孢子是一个生殖细胞,不需要受精就能形成一个新的个体。在水螅等一些低等动物和酵母中,\mykeyword{出芽}为一种常见的生殖方式:在身体表面长出一个小突起后,逐渐长大,并于身体分离后形成一个新的个体。在海绵的内部也可长出小芽,称为\mykeyword{芽球(gemmule)}。

\mykeyword{再生}是无性生殖的一种特化形式,海星和蝾螈等动物可通过再生替代受伤或丢失的部分。很大植物通过再生可产生一个完整的个体。分类上越低等的动物,其完全再生能力也越强。到现在为止,还未见到脊椎动物具有再生完整个体的能力。但通过实验的手段,已在鱼类、两栖类和哺乳类等脊椎动物中获得了无性生殖的个体。特别需要提到的是,1997年首次通过体细胞核移植手段,获得了无性生殖的哺乳动物---克隆绵羊。

自然条件下的无性生殖包括\mykeyword{孤雌生殖}和\mykeyword{孤雄生殖}等。\mykeyword{人工辅助无性生殖}是指在物理或化学因子的作用于卵子后的单性生殖,以及利用细胞核移植技术而进行的动物克隆。

\subsection{有性生殖}

在有性生殖的生物体(高等生物)中含有两大类细胞:构成组织和器官并执行各种功能的\mykeyword{体细胞(Somatic Cell)}和携带有特定的遗传信息并具有受精后形成合子能力的\mykeyword{生殖细胞(Germ Cell)}。\myImportantPoint{生殖细胞又包括卵子和精子两类}。有性生殖周期是体细胞与生殖细胞相互转变的过程。在高等生物的机体中,只有一小部分细胞为生殖细胞,然而它们却是正常生命周期中的一个关键环节。

有性生殖发生在很多的单细胞生物和所有的动物和植物中。除在个别动物中可进行孤雌生殖外,有性生殖是高等的无脊椎动物和所有的脊椎动物自然情况下唯一的一种生殖方式。在有性生殖过程中,一种性别的细胞(配子)被另一种性别的细胞(配子)受精后,产生一个新的细胞(合子或受精卵),以后受精卵发育形成一个新的个体。两个结构相同但生理上不同的同形配子(Isogamete)的结合,称为\mykeyword{同配生殖(Isogamy)},仅见于低等的水棉属的绿藻(spirogyra)和一些原生动物中。\mykeyword{异配生殖(heterogamy)}是指两种明显不同的配子的结合,即精子与卵子的结合。许多生物具有特殊的生殖机制来保证受精的进行。在陆生动物中,通过交配进行体内受精,从而提供了一个受精必须的液体环境。

有性生殖的优越性在于来自两个亲本的细胞核融合后,子代可源源不断地继承各种各样的性状组合,从而具有很大的发生变异的空间,对于改进物种本身以及物种的生存具有重要意义。精卵结合形成的子代在遗传学上互不相同,也不同于各自的亲代,从而保证了物种的多样性。有性生殖产生的后代中随机组合的基因对物种可能有利,也可能不利,但至少会增加少数个体在难以预料和不断变化的环境中存活的几率,从而对物种的延续提供了有利的条件。此外,在进行有性生殖的物种中,生命周期中都具有二倍体和单倍体交替的特征。二倍体的物种每一基因都有两份,其中一份在功能上处于备用状体,对各种突变等具有一定的抵御作用。这也可能是高等生物以有性生殖为主的原因。因为即使在细菌等进行无性生殖的生物中,也发生遗传物质的交换。在蚯蚓等雌雄同体(hermaphrodite)的动物中,由于解剖结构的特化或雌雄配子的成熟时间不同,总是避免自体受精的发生。

生殖过程不是一个连续的活动,而是受一些形式和周期的约束。通常,这些形式和周期与环境条件有关,从而使得生殖过程能有限地进行。例如,一些有发情周期的动物仅在一年的一段时间内发情,使得后代能在适宜的环境条件下出生。同样,这些形式和周期也受到激素和季节因素的控制,使得生殖过程中的能量消耗得到很好地控制,从而最大限度地提高了后代的生成能力。

\section{生殖生物学}

生殖是亲代与后代个体之间生命延续的过程。生殖生物学(reproduction biology)是研究整个生殖过程的一门学科,即使发育生物学的一个分支,又是生理学的一个分支,属于一门新的充满活力的、融合了现代生物化学、细胞生物学、内分泌学和分子生物学等学科的交叉学科。过去这些年中,生殖生物学研究领域取得了许多世人瞩目的重大突破。例如,对“下丘脑---垂体---性腺”内分泌轴系这一重大理论问题的揭示,导致了口服避孕药的诞生;从精子获能和卵子体外成熟等基础研究着手而创立的“试管婴儿”技术,使国内外成千上万的不孕夫妇获得了后代;1997年克隆绵羊“多莉”的诞生,也已对人类社会产生了深远的影响。生殖生物学已成为生物学中一个活跃的、充满机遇和挑战的重要研究领域。

\myImportantPoint{生殖生物学主要研究性腺发育、配子发生、受精、胚胎发育及着床、性别决定、妊娠维持、胎盘发育和分娩等过程的调控,以及生殖道的恶性肿瘤、异常妊娠、生殖道感染、环境和职业性危害对生殖的影响等问题。}此外,生殖生物学也研究\myImportantPoint{在青春期、泌乳、衰老和妊娠等过程中与生殖相关的内分泌变化},以及\myImportantPoint{性行为的形成和影响因素}等。

随着人工授精、体外受精、胞浆内单精子注射(ICSI)、胚胎移植、细胞核移植,以及其他辅助生殖技术的广泛应用,这门学科在生物学、医学及农业中的地位将越来越重要。例如,自1978年英国科学家首次成功地获得世界首例试管婴儿至今,全世界已有约100万试管婴儿诞生,体外受精和胞浆内单精子注射技术已成为临床上不育症治疗的最主要手段,为许多不育夫妇带来了福音。

近年来,细胞生物学、分子生物学、生物化学、生理学等学科飞速发展,各种现代生物技术已广泛渗透到生殖生物学的研究过程中,使得人们对生殖过程的各种现象及其分子机制的了解有了长足的进步。

\section{生殖生物学的相关学科}

随着生殖生物学的迅速发展,生殖生物学的研究范围也在主板扩大,与很多学科间的交叉也变得越来越明显,这里仅简单介绍与生殖生物学相关性很强的一些学科。

\subsection{动物胚胎学(Embryology)}

动物胚胎学是研究动物个体发育过程中形态结构及其生理功能变化的一门学科,主要研究受精后到出生前这段时间内动物的发育过程。个体发育包括生殖细胞的起源、发生、成熟、受精、卵裂、胚层分化、器官发生,直至发育为新个体,以及幼体的生长、发育、成熟、衰老和死亡。通常也将个体的发育的整个过程分为胚前发育、胚胎发育和胚后发育。胚前发育主要研究生殖细胞的起源、单倍体的精子和卵子的发生、形成和成熟。胚胎发育是指受精到分娩或孵出前的发育过程。胚后发育包括出生或孵出的幼体的生长发育、性成熟、体成熟,以及以后的衰老和死亡。胚胎学一般只研究胚前发育和胚胎发育。

\subsection{发育生物学(Developmental Biology)}

发育生物学实在动物胚胎学的基础上,结合细胞生物学、遗传学和分子生物学等学科的发展,而逐渐形成的一门学科。它是应用现代生物学技术,来研究生物发育本质的科学,主要研究生殖细胞的发生、受精,胚胎发育、生长、衰老和死亡等过程,分析从受精一直到主要胚胎器官形成时动物发育的基本现象及形式,偏重于研究细胞决定及分化的机制,以及形态发生过程中细胞间相互作用等问题。

\subsection{动物繁殖学(Animal Reproduction or Theriogenology)}

动物繁殖学主要研究家畜和家禽生殖过程中的形态、生理和功能的变化,以及调节和控制生殖过程的相关技术。它是序幕科学的一个重要组成部分,主要包括家畜和家禽的生殖生理、繁殖技术以及家畜繁殖力的评价和家畜生产的影响因素和管理等。

\subsection{生殖医学(Reproduction Medicine)}

生殖医学是一门综合性的学科,涉及生育、不育、节育和出生缺陷等。生殖医学的主要任务是通过常规的诊断和治疗措施,将现在的各种生殖技术应用于不孕不育病人,使其产生后代。自1978年,世界上第一例试管婴儿诞生以来,辅助生殖技术已得到了突飞猛进的发展。在国内外,大量的生殖医学中心或辅助生殖中心相继建立,为越来越多的不孕不育患者解除了痛苦。在提供的服务方面,也由简单的人工授精、体外受精,逐渐向胞浆内单精子注射、卵细胞质互换、着床前胚胎的遗传诊断等多方位发展。

\subsection{产科学(Obsterics)}

产科学主要是研究妊娠、分娩、胎儿出生以及出生后事件的一门临床科学。它的任务是既要保证产生一个健康的后代,又要确保母体的健康不受损害。可通过超声等手段来判断子宫内的情况,对母体子宫的大小、妊娠期的长短、胎儿的大小和位置等进行分析,从而使胎儿顺利产出。在异常情况下,通过剖腹产手术来保证胎儿的分娩。

\subsection{妇科学(Gynecology)}

妇科学是主要研究雌性生殖系统各种失调的一门科学。现代妇科学涉及月经失调、绝经、生殖器官的感染性疾病和异常发育、性激素紊乱、良性和恶性肿瘤,以及各种与避孕相关的问题。由妇科学产生的一门学科为生殖医学。与妇科学相应,也产生了男科学或男性学(Andrology),主要研究男性生殖器官的各种异常或病变,以及男性的不孕症等问题。\\

事实上,与生殖生物学相关的学科还有很多,特别是内分泌学、细胞生物学和分子生物学的进展对于阐明生殖过程的机理起了巨大的推动作用。

\section{生殖生物学的发展前景}

据报道,不孕症在国外的发病率约为$ 15\%\sim20\% $。自从世界上第一例试管婴儿路易斯布朗(Louise Brown ) 1978年7月25日在英国诞生以来,体外受精、促排卵技 术、显微受精、胚胎培养、胚胎冷冻等辅助生殖技术迅速发展,并不断完善,已为很多 不孕患者解除了痛苦。据估计,全世界每年大约有100000例试管婴儿出生。自我国的第一例试骨婴儿1988年出生以来,辅助生殖技术已在全国的绝大多放地区得到推广。

性和生殖健康见人们生活和幸福的核心内容。生殖健康的卞要内容是保证妊娠的正 常及安全进行,使用更安全可节的避孕措施,以及防治生殖道的感染等。由于世界人口及中国人口的猛增,直接危及人的生存环境和生活质量,对自然环境的破坏也在逐年增加。控制人口数量及提高人口质量也是当今世界亟待解决的问题。在生育调节方面,20世界90年代以前的生殖研究主要是围绕“下丘脑---垂体---性腺”所构成的生殖轴系,作为发展避孕药的出发点,即通过干扰激素和生殖轴系之间的相互作用。这些避孕方法的主要缺陷是有不良反应。随着社会的发展和生活水平的提高,人们对避孕措施的安全性、可靠性、多样性的要求也越来越高,而且在达到避孕效果的同时也要兼具预防生殖道感染的功能。人们开始认识到,最理想的抗生育靶点应当是在生殖过程中起直接作用的细胞和因子。因此,加强以生育控制为目的的基础研究,寻找生殖细胞的发生、成熟、受精和胚胎着床等生殖过程中可控制的关键环节,以此发展新一代避孕技术,已成为21世纪生殖生物学研究与发展的主要目标。随着生殖生物学基础研究的深入和对生殖相关的 重要基因和分子的认识,通过干扰基因表达或表达产物的功能,最终可以有望发展出对其他正常生理功能没有影响或影响很小的避孕药或避孕方法。另外,由于艾滋病等传染病的广泛流行,对于预防生殖道感染的要求也越来越高。现在迫切需要普及生育方面的 知识,提供安全及高效的避孕措施,控制生殖相关疾病的传播。

近年来,我国国民经济及畜牧业方面的迅速发展,对提高家畜的繁殖力及加速家畜的品种改良方面提出了更高的要求。在奶牛、肉牛、奶羊和肉羊的繁殖方面,逐渐发展了超数排卵、卵母细胞体外成熟、体外受精、性别鉴定、胚胎分割、克隆、转基因及基 因敲除等方面的一系列技术。随着人们对肉、蛋和奶的需求逐年增加,迫切需要增加奶牛、肉牛、肉羊、猪等的数量,并提高其质量。近年来,各种辅助生殖技术已在畜牧业 中得到广泛应用,人工输精、体外受精、超数排卵、胚胎移植、性别鉴定及转基因等在逐步完善和推广。特别是最近克隆牛、克隆羊、克隆猪的问世,以及基因敲除家畜的获得,极大地促进了对家畜生殖机理的研究,并加速了各种辅助生殖技术在家畜中的应用,将对促进家畜的品种改良及提高繁殖力方面具有重要的意义。

在野生动物的保护方面,提高野生动物的繁殖力也迫在眉睫。目前,需要了解这些动物的基本生殖过程,通过激素处理、人工授精、体外受精、胚胎分割、性别鉴定、克隆等技术手段有望在短时间内增加濒危物种的种群数量。在宠物的饲养方面,了解基本的生殖过程,利用现代生殖技术来提高繁殖力及加速品种改良等方面也具有重要的意义。

随着社会和经济的不断发展,人类对大自然的干预日益加剧。废水、废气、废渣、 农药、化肥等大量的化学物质通过各种途径排入环境,造成了严重的环境污染。噪声、 农药残留、射线等各种环境因素对生殖过程的影响也越来越受到人们的重视。凡是能够影响机体内外环境改变的因素,都将会对生殖健康产生一定程度的影响。目前越来越多的证据表明,许多人工合成的化学物质可干扰人类及野生动物的生长发育,导致人类的不孕不育率、畸胎率和自然流产率上升。

而且,随着社会工业化和现代化的不断发展,人类对野生动物的生存环境不断蚕食,异致大量的野生动物灭绝和濒临灭绝。而由于这些动物的数量有限以及难以接近等原因,对于野生动物基本生殖过程的了解仍很有限。现在也迫切需要利用现有的生殖生物学知识和技术,来促进对野生动物生殖过程的了解,并将现有的辅助生殖技术应用于野生动物,从而延缓或防止野生动物的灭绝,并使一些具有食用和药用价值的动物得到迅速繁殖。

\section{本书的特色}

近年来随着多聚酶链反应、基因敲除、转基因、基因芯片、RNA干涉、 细胞分离技术、免疫测定等方面的飞速发展, 以及随着基因组和后基因组时代的发展和推动, 人们对生殖过程基本调控机制的了解也越来越多。 现代分子生物学和细胞生物学理论与技术的发展, 极大地推动了生殖生物学研究, 使人们对生殖现象的认识深入到细胞和分子水平。 从本质上讲, 生殖过程是个体生命活动的一部分, 与其他生命现象遵循共同的基 本规律,如基因的时空特异性表达调控、细胞的增殖、分化和凋亡、细胞之间的信号转导和细胞外基质的相互作用等。但是, 生殖过程在生命活动中具有特殊使命, 因此也具有生殖细胞发生、性周期、 受精、妊娠和分挽等独特现象。

到目前为止, 国内尚无一本系统介绍现代生殖生物学研究成果的书。 本书将尽最大努力, 综合国内外在生殖生物学领域的研究论文和著作, 介绍生殖细胞的发育及成熟、 受精机理、 胚胎发育、 胚胎着床、 胎盘的形成、 分挽、性别决定、 生殖缺陷、 生殖免 疫、无性生殖等方面的基本知识和国内外的最新研究动态, 并介绍生殖激素、性行为、环境对生殖的影响及现代生殖生物学实验技术。 本书将以小鼠、 人和家畜为主, 既介绍哺乳动物生殖的共性, 又具体阐述各类动物的特性, 力求从广度、深度和新颖性方面比较系统和全面地介绍生殖生物学的内容。


