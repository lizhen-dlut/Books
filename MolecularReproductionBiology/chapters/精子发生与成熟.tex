\chapter{精子发生与成熟}

在有性生殖的生物体中,细胞可以分为两大类:体细胞(Somatic Cell)和生殖细胞(Germ Cell)。有性生殖周期是体细胞与生殖细胞相互转变的过程。在胚胎发育的早期,部分体细胞可分化为生殖细胞。生殖细胞在分化过程中发生减数分裂,由二倍体的细胞产生单倍体细胞:精子和卵子。精、卵通过受精重新形成二倍体的细胞,开始下一轮的生命周期。

在高等生物的机体内,只有一小部分细胞为生殖细胞,然而它们是正常生命周期中的一个关键环节。二倍体细胞在形成单倍体细胞的过程中,发生了简述分裂。期间,同源染色体DNA发生重组,产生的每个单倍体生殖细胞都含有不完全相同的基因组合。因此,精卵结合形成的子代在遗传学上互不相同,并且也不同于他们的亲代,这一生殖模式最大的优点是保持着物种的多样性。

在多数物种中,只有两类生殖细胞:卵子和精子。这两种细胞具有很大的区别,卵子是机体中最大的细胞,而精子通常最小。这种结构最适合于所带基因的扩增。卵子是不运动的,但通过提供大量生长大于所需要的原料来帮助保存母本基因;与此相反,精子通常具有较强的运动能力,为流线形,以适应有效的受精,它通过利用母本资源来扩增父本基因。

精子起源于原始生殖细胞。在胚胎发育的早期,少数细胞形成配子(gamete)的前体,称之为\mykeyword{原始生殖细胞(PGC,primordial germ cell)}。其后,原始生殖细胞迁移到早期的性腺---\mykeyword{生殖嵴(Genital ridge)},在那里进行一段时间的有丝分裂增殖,然后部分细胞进入减数分裂,并进一步分化成成熟的配子:卵子或精子。睾丸是精子发生的场所,在这里PGC发育成\mykeyword{原始精原细胞(primitive spermatogonium)},进入精子发生过程,经过简述分裂及一系列形态变化,最后形成特殊形态的完整精子。在曲细精管中形成的精子并没有完全成熟,需要进入附睾,在附睾管运行过程中,吸收多种物质,发生一些列形态、生理和生化方面的变化,完成成熟过程,形成具有一定活力的精子。

\section{精子发生}

精子发生(spermatogenesis)是指精原细胞(spermatogonium)经过一些列的分裂增殖、分化、变形,最终形成完整精子(spermatozoon)的过程。这一过程是在雄性生殖腺(睾丸)的曲细精管(seminiferous tubule)中进行的。精子发生可分为三个时期:有丝分裂期、减数分裂期和精子形成期(spermiogenesis)。精子发生是一个特殊的细胞分化过程,在这一过程中发生了许多特殊的时间,如减数分裂,形态变化等。

\subsection{精子发生的场所}

\subsubsection{睾丸}

在雄性个体中,生殖嵴发育为雄性生殖腺---睾丸(图 \ref{figure_structure_of_testis})。睾丸是雄性生殖器官,椭圆形,其背面表层与附睾相连。睾丸是精子发生的场所,附睾为精子成熟的器官。睾丸外由一层囊膜包裹,囊膜为致密坚硬的结缔组织,囊膜向内延伸把睾丸分割为 名个分隔间,分隔间充满了弯曲的上皮性管道,称为曲细精管。附睾附着于睾丸的背面,其间有输出小管(ductuli efferentes)相连,在睾丸中产生的完整精子通过输出小管进入附睾, 在附睾中成熟, 变为成熟的精子排出。

\begin{figure}
\centering
\myFigurePlaceholder
\caption{睾丸结构}
\label{figure_structure_of_testis}
\end{figure}

在胚胎期发育、 分化阶段及出生后, 睾丸行使两种功能:分泌激素(睾酮及其他类固醇激素, 在胚胎期也分泌)以及在成年期产生精子。经过胚胎期及出生后期的一个准备阶段后, 精子发生开始于青春期。

\subsubsection{曲细精管}

睾丸的实质部分为弯曲的曲细精管。曲细精管周边为结缔组织薄层,由弹性纤维及一些平滑肌细胞组成。管壁由两类细胞组成:\myGlossaryEntry{sertoli_cell}及各期的生精细胞(spermatogeniccell)。生精细胞根据它们的发育阶段有规律地排列成多层,这一结构称之为生精上皮(spermatogenicepithelium或serminiferousepithelium)。生精细胞包括精原细胞、初级精母细胞、次级精母细胞、圆形精子细胞及长形精子细胞。它们由曲细精管的基底部向管腔排列(图 \ref{figure_structure_of_serminiferous_epithelium})。

\begin{figure}
\centering
\myFigurePlaceholder
\caption{曲细精管结构示意图}
\label{figure_structure_of_serminiferous_epithelium}
\end{figure}

\subsubsection{生精上皮}

生精上皮由支持细胞及不同阶段的生精细胞高度有序排列组成, 其组织的复杂性是上皮中独有的,包括非增殖状态的支持细胞及各期生精细胞, 从位于基底的精原细胞到管腔部分的精子细胞。支持细胞占成年生精上皮的25\%,生精上皮的基底膜由扁平的肌样细胞、 成纤维细胞及胶原纤维组成(图 \ref{figure_cell_alignment_of_serminiferous_epithelium})。

\begin{figure}
\centering
\myFigurePlaceholder
\caption{曲细精管结构示意图}
\label{figure_cell_alignment_of_serminiferous_epithelium}
\end{figure}

\subsubsection{支持细胞}

除了生殖细胞,生精上皮还包含一群支持细胞,它们比生殖细胞大得多,而且形态复杂。它们不再分裂,支持着整个上皮,附着于生精上皮的基底层,并穿过生殖细胞间伸向官腔,为精子发生提供了一个合适的环境。在老年的睾丸中存在着多核支持细胞,表明它们可能在一定情形下恢复了分裂能力,发生了没有进行胞质分裂的有丝分裂,导致形成多核的支持细胞。支持细胞的分布是随机的,数量是恒定的。


如图 \ref{figure_structure_of_sertoli_cell} 所示,支持细胞有一个多形核,有丰富的细胞质和细胞器,包括丰富细长 的线粒体、一个很大的高尔基体、丰富的内质网,一些溶酶体、微丝和微管。在大鼠生精上皮周期的$ 12\sim 14 $期,支持细胞中线粒体的体积增大,同时伴随着大量脂质形成,以及内吞活性显增多,反映了生精上皮周期中支持细胞对能量需求的变化。已证明在大鼠中支持明细胞的内吞活性具有周期性变化的特征。支持细胞的骨架发达,在同一支持细胞中的不同区域,构成细胞骨架的胞质微丝及微管的数量及分布变化很大。微丝主要分布于细胞核周围及细胞基底部。

\begin{figure}
\centering
\myFigurePlaceholder
\caption{曲细精管结构示意图}
\label{figure_structure_of_sertoli_cell}
\end{figure}

\subsubsection{支持细胞}

支持细胞间存在着多种形式的连接, 其中紧密或闭缩连接 (occluding junction) 是指相邻的支持细胞膜互相融合形成一种特化的连接复合体 (junctional complex)。它是 血---睾屏障 (blood-testis barrier) 的结构成分, 这一结构把生精上皮分为两个分隔间: 基底间 (basal compartment),为精原细胞和前细线期的精母细胞的场所;血---睾屏障的管腔一侧为中央间 (adluminal compartment),包含处于减数分裂的精母细胞及精子细 胞。 血---睾屏障解释了在管液和血浆中化学物质的不同,也是在两个隔间中存在不同基质的结构基础, 它在精子发生的特定阶段中起关键作用。 它们的重要性还有待研究,然而一股认为, 基底间可直接接受血液中的激素, 而中央间则通过支持细胞接受激素和营养物质, 解释了基底间中的细胞更容易受到激素水平的影响。 通过闭缩连接进行的物质 运转依赖于物质分子大小和物理性质, 对物质的转移具有筛选作用。

\subsection{精子发生过程}

精子发生是一个复杂而有规律的细胞分化过程。从精原细胞的分裂增殖、精母细胞的减数分裂到精子细胞变态分化和运行至附睾的成熟过程中,都受到众多基因和激素的协同调控。精子发生过程可分为三个主要的阶段(图 \ref{figure_spermatogenesis})。

\begin{figure}
\centering
\myFigurePlaceholder
\caption{精子发生示意图}
\label{figure_spermatogenesis}
\end{figure}

\subsubsection{精原细胞的有丝分裂期}

精原细胞由原始生殖细胞分化而来, 其增殖能力增强, 为进入减数分裂做准备。它通过有丝分裂产生两类细胞, 一类不进入精子发生周期,继续保持有丝分裂的能力, 在下一个周期前一直处于静止状态, 称之为 “储存的生殖干细胞” ;另一类进入精子发生,周期通过分化途径形成精子, 称之为 ”更新的生殖干细胞” 。

\subsubsection{精母细胞的减数分裂}

进入分化途径的精原细胞发育为初级精母细胞, 进行最后一次染色体的复制,为成熟分裂做准备。 根据其生长发育顺序及细胞、 染色质形态可将初级精母细胞分为前细线期精母细胞、细线期精母细胞、偶线期精母细胞、 粗线期精母细胞及双线期精母细胞等几个时相。 一个双线期精母细胞发生第一次减数分裂, 产生两个次级精母细胞。次级精 母细胞的间期很短, 不发生染色体复制, 很快进行第二次减数分裂, 产生单倍体的圆形 精子细胞, 完成减数分裂。

\subsubsection{精子形成期}

精子形成期是精子细胞的分化变态过程,这是精子分化的重要环节。圆形的精子细 胞要经过伸长变态的复杂过程,包括细胞核的浓缩变长,顶体的生成,核蛋白的转型, 染色质的浓缩包装,核骨架及细胞骨架---中心体(粒)体系的演变,鞭毛、轴丝的发 生及尾的成形分化,精子特异性乳酸脱氢酶LDH-X的出现等。在此变态过程中,糖 原、脂质、蛋白质等代谢产物大批随细胞质排弃,代之以出现的LDH-X及六碳糖激酶来适应能量需要。在变态后期核蛋白质出现不断的磷酸化和脱磷酸化,蛋白质SH--基向SS--键转变,以精氨酸为主的鱼精蛋白(protamine)替换以组氨酸为主的核组蛋白(histone),使核蛋白和DNA紧密结合,以保证精子基因处于浓缩包装和不活跃状态。

原始生殖细胞经历精原细胞、精母细胞、精子细胞和精子,其间发生了减数分裂、组蛋白/鱼精蛋白替换、精子变态等特异细胞活动,许多特异性的基因对精子发生过程进行严密的调控。

精子发生过程中,生精细胞可分为一下几个阶段:\myGlossaryEntry{pri_type_a_spermatogonium}、\myGlossaryEntry{a_spermatogonium}、\myGlossaryEntry{b_spermatogonium}、\myGlossaryEntry{primary_spermatocyte}、\myGlossaryEntry{secondary_spermatocyte}、\myGlossaryEntry{round_spermatid}、\myGlossaryEntry{condensing_spermatid}、\myGlossaryEntry{spermatozoon}、\myGlossaryEntry{preleptotene_spermatocyte}、\myGlossaryEntry{leptotene_spermatocyte}、\myGlossaryEntry{zygotene_spermatocyte}、\myGlossaryEntry{pachytene_spermatocyte}。精子发生过程及各期生精细胞特征参见图 \ref{figure_spermatogenesis_process}。

\begin{figure}
\centering
\myFigurePlaceholder
\caption{精子发生示意图}
\label{figure_spermatogenesis_process}
\end{figure}

\subsection{有丝分裂期}

\subsubsection{\myGlossaryEntry{pri_type_a_spermatogonium}}

精子在睾丸中的发生起源于\myGlossaryEntry{pri_type_a_spermatogonium},也被称为\myGlossaryEntry{spermatogonial_stem_cell}。这类细胞通过有丝分裂进行增殖,所产生的子代细胞可以分为两类:一类仍保持精原干细胞的特征进行有丝分裂,成为长期精子发生的”源泉”;另一 类子代细胞则进入分化途径。

\subsubsection{\myGlossaryEntry{a_spermatogonium}}

一部分\myGlossaryEntry{pri_type_a_spermatogonium}的子代细胞进入分化过程首先形成\myGlossaryEntry{a_spermatogonium}。\myGlossaryEntry{a_spermatogonium}的分化也是一个复杂的过程。目前认为至少要经过以下几个阶段,通过$ A_1 $型、$ A_2 $型、$ A_3 $型和$ A_4 $型精原细胞形成\myGlossaryEntry{intermediate_spermatogonium}。

\subsubsection{\myGlossaryEntry{b_spermatogonium}}

这是精原细胞的最后阶段。在此之前,精原细胞都是通过有丝分裂进行增殖。\myGlossaryEntry{intermediate_spermatogonium}进行最后的有丝分裂,形成\myGlossaryEntry{b_spermatogonium},随后停止有丝分裂,由它们发育形成初级精母细胞,进入减数分裂期。

\subsubsection{精子发生的同步化现象}

精子发生过程中的一个特点是许多生精细胞进行同步化分裂,并且细胞的
\myGlossaryEntry{cytokinesis}不完全,导致子细胞由细胞间桥相连(图 \ref{figure_spermatogenesis_synchronization}),这可能是它们同步分行的基础。

\begin{figure}
\centering
\myFigurePlaceholder
\caption{精子发生的同步化现象示意图}
\label{figure_spermatogenesis_synchronization}
\end{figure}

\subsection{减数分裂期}

配子是单倍体的,这种单倍体的细胞必须通过一种特殊的细胞分裂形式---
\myGlossaryEntry{meiosis}产生。简述分裂仅发生于有性生殖细胞发生过程中的某个阶段,其特点是细胞进行连续两次分裂而DNA只复制一次,结果产生了只含有单倍体遗传物质的细胞。含有单倍体遗传物质的两性生殖细胞通过受精形成合子,染色体又恢复到体细胞的数目,从而可以维持物种的正常繁衍。因此,\myGlossaryEntry{meiosis}是生物有性生殖的基础。


\section{精子发生的调控机制}


\section{精子的成熟与获能}