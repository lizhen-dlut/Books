\chapter{精子发生与成熟}

在有性生殖的生物体中,细胞可以分为两大类:体细胞(Somatic Cell)和生殖细胞(Germ Cell)。有性生殖周期是体细胞与生殖细胞相互转变的过程。在胚胎发育的早期,部分体细胞可分化为生殖细胞。生殖细胞在分化过程中发生减数分裂,由二倍体的细胞产生单倍体细胞:精子和卵子。精、卵通过受精重新形成二倍体的细胞,开始下一轮的生命周期。

在高等生物的机体内,只有一小部分细胞为生殖细胞,然而它们是正常生命周期中的一个关键环节。二倍体细胞在形成单倍体细胞的过程中,发生了简述分裂。期间,同源染色体DNA发生重组,产生的每个单倍体生殖细胞都含有不完全相同的基因组合。因此,精卵结合形成的子代在遗传学上互不相同,并且也不同于他们的亲代,这一生殖模式最大的优点是保持着物种的多样性。

在多数物种中,只有两类生殖细胞:卵子和精子。这两种细胞具有很大的区别,卵子是机体中最大的细胞,而精子通常最小。这种结构最适合于所带基因的扩增。卵子是不运动的,但通过提供大量生长大于所需要的原料来帮助保存母本基因;与此相反,精子通常具有较强的运动能力,为流线形,以适应有效的受精,它通过利用母本资源来扩增父本基因。

精子起源于原始生殖细胞。在胚胎发育的早期,少数细胞形成配子(gamete)的前体,称之为\mykeyword{原始生殖细胞(PGC,primordial germ cell)}。其后,原始生殖细胞迁移到早期的性腺---\mykeyword{生殖嵴(Genital ridge)},在那里进行一段时间的有丝分裂增殖,然后部分细胞进入减数分裂,并进一步分化成成熟的配子:卵子或精子。睾丸是精子发生的场所,在这里PGC发育成\mykeyword{原始精原细胞(primitive spermatogonium)},进入精子发生过程,经过简述分裂及一系列形态变化,最后形成特殊形态的完整精子。在曲细精管中形成的精子并没有完全成熟,需要进入附睾,在附睾管运行过程中,吸收多种物质,发生一些列形态、生理和生化方面的变化,完成成熟过程,形成具有一定活力的精子。

\section{精子发生}

精子发生(spermatogenesis)是指精原细胞(spermatogonium)经过一些列的分裂增殖、分化、变形,最终形成完整精子(spermatozoon)的过程。这一过程是在雄性生殖腺(睾丸)的曲细精管(seminiferous tubule)中进行的。精子发生可分为三个时期:有丝分裂期、减数分裂期和精子形成期(spermiogenesis)。精子发生是一个特殊的细胞分化过程,在这一过程中发生了许多特殊的时间,如减数分裂,形态变化等。

\subsection{精子发生的场所}

\subsubsection{睾丸}

在雄性个体中,生殖嵴发育为雄性生殖腺---睾丸(图 \ref{figure_structure_of_testis})。睾丸是雄性生殖器官,椭圆形,其背面表层与附睾相连。睾丸是精子发生的场所,附睾为精子成熟的器官。睾丸外由一层囊膜包裹,囊膜为致密坚硬的结缔组织,囊膜向内延伸把睾丸分割为 名个分隔间,分隔间充满了弯曲的上皮性管道,称为曲细精管。附睾附着于睾丸的背面,其间有输出小管(ductuli efferentes)相连,在睾丸中产生的完整精子通过输出小管进入附睾, 在附睾中成熟, 变为成熟的精子排出。

\begin{figure}
\centering
\myFigurePlaceholder
\caption{睾丸结构}
\label{figure_structure_of_testis}
\end{figure}

在胚胎期发育、 分化阶段及出生后, 睾丸行使两种功能:分泌激素(睾酮及其他类固醇激素, 在胚胎期也分泌)以及在成年期产生精子。经过胚胎期及出生后期的一个准备阶段后, 精子发生开始于青春期。

\subsubsection{曲细精管}

睾丸的实质部分为弯曲的曲细精管。曲细精管周边为结缔组织薄层,由弹性纤维及一些平滑肌细胞组成。管壁由两类细胞组成:\myGlossaryEntry{sertoli_cell}及各期的生精细胞(spermatogeniccell)。生精细胞根据它们的发育阶段有规律地排列成多层,这一结构称之为生精上皮(spermatogenicepithelium或serminiferousepithelium)。生精细胞包括精原细胞、初级精母细胞、次级精母细胞、圆形精子细胞及长形精子细胞。它们由曲细精管的基底部向管腔排列(图 \ref{figure_structure_of_serminiferous_epithelium})。

\begin{figure}
\centering
\myFigurePlaceholder
\caption{曲细精管结构示意图}
\label{figure_structure_of_serminiferous_epithelium}
\end{figure}

\subsubsection{生精上皮}

生精上皮由支持细胞及不同阶段的生精细胞高度有序排列组成, 其组织的复杂性是上皮中独有的,包括非增殖状态的支持细胞及各期生精细胞, 从位于基底的精原细胞到管腔部分的精子细胞。支持细胞占成年生精上皮的25\%,生精上皮的基底膜由扁平的肌样细胞、 成纤维细胞及胶原纤维组成(图 \ref{figure_cell_alignment_of_serminiferous_epithelium})。

\begin{figure}
\centering
\myFigurePlaceholder
\caption{曲细精管结构示意图}
\label{figure_cell_alignment_of_serminiferous_epithelium}
\end{figure}

\subsubsection{支持细胞}

除了生殖细胞,生精上皮还包含一群支持细胞,它们比生殖细胞大得多,而且形态复杂。它们不再分裂,支持着整个上皮,附着于生精上皮的基底层,并穿过生殖细胞间伸向官腔,为精子发生提供了一个合适的环境。在老年的睾丸中存在着多核支持细胞,表明它们可能在一定情形下恢复了分裂能力,发生了没有进行胞质分裂的有丝分裂,导致形成多核的支持细胞。支持细胞的分布是随机的,数量是恒定的。


如图 \ref{figure_structure_of_sertoli_cell} 所示,支持细胞有一个多形核,有丰富的细胞质和细胞器,包括丰富细长 的线粒体、一个很大的高尔基体、丰富的内质网,一些溶酶体、微丝和微管。在大鼠生精上皮周期的$ 12\sim 14 $期,支持细胞中线粒体的体积增大,同时伴随着大量脂质形成,以及内吞活性显增多,反映了生精上皮周期中支持细胞对能量需求的变化。已证明在大鼠中支持明细胞的内吞活性具有周期性变化的特征。支持细胞的骨架发达,在同一支持细胞中的不同区域,构成细胞骨架的胞质微丝及微管的数量及分布变化很大。微丝主要分布于细胞核周围及细胞基底部。

\begin{figure}
\centering
\myFigurePlaceholder
\caption{曲细精管结构示意图}
\label{figure_structure_of_sertoli_cell}
\end{figure}

\subsubsection{支持细胞}

支持细胞间存在着多种形式的连接, 其中紧密或闭缩连接 (occluding junction) 是指相邻的支持细胞膜互相融合形成一种特化的连接复合体 (junctional complex)。它是 血---睾屏障 (blood-testis barrier) 的结构成分, 这一结构把生精上皮分为两个分隔间: 基底间 (basal compartment),为精原细胞和前细线期的精母细胞的场所;血---睾屏障的管腔一侧为中央间 (adluminal compartment),包含处于减数分裂的精母细胞及精子细 胞。 血---睾屏障解释了在管液和血浆中化学物质的不同,也是在两个隔间中存在不同基质的结构基础, 它在精子发生的特定阶段中起关键作用。 它们的重要性还有待研究,然而一股认为, 基底间可直接接受血液中的激素, 而中央间则通过支持细胞接受激素和营养物质, 解释了基底间中的细胞更容易受到激素水平的影响。 通过闭缩连接进行的物质 运转依赖于物质分子大小和物理性质, 对物质的转移具有筛选作用。

\subsection{精子发生过程}

精子发生是一个复杂而有规律的细胞分化过程。从精原细胞的分裂增殖、精母细胞的减数分裂到精子细胞变态分化和运行至附睾的成熟过程中,都受到众多基因和激素的协同调控。精子发生过程可分为三个主要的阶段(图 \ref{figure_spermatogenesis})。

\begin{figure}
\centering
\myFigurePlaceholder
\caption{精子发生示意图}
\label{figure_spermatogenesis}
\end{figure}

\subsubsection{精原细胞的有丝分裂期}

精原细胞由原始生殖细胞分化而来, 其增殖能力增强, 为进入减数分裂做准备。它通过有丝分裂产生两类细胞, 一类不进入精子发生周期,继续保持有丝分裂的能力, 在下一个周期前一直处于静止状态, 称之为 “储存的生殖干细胞” ;另一类进入精子发生,周期通过分化途径形成精子, 称之为 ”更新的生殖干细胞” 。

\subsubsection{精母细胞的减数分裂}

进入分化途径的精原细胞发育为初级精母细胞, 进行最后一次染色体的复制,为成熟分裂做准备。 根据其生长发育顺序及细胞、 染色质形态可将初级精母细胞分为前细线期精母细胞、细线期精母细胞、偶线期精母细胞、 粗线期精母细胞及双线期精母细胞等几个时相。 一个双线期精母细胞发生第一次减数分裂, 产生两个次级精母细胞。次级精 母细胞的间期很短, 不发生染色体复制, 很快进行第二次减数分裂, 产生单倍体的圆形 精子细胞, 完成减数分裂。

\subsubsection{精子形成期}

精子形成期是精子细胞的分化变态过程,这是精子分化的重要环节。圆形的精子细 胞要经过伸长变态的复杂过程,包括细胞核的浓缩变长,顶体的生成,核蛋白的转型, 染色质的浓缩包装,核骨架及细胞骨架---中心体(粒)体系的演变,鞭毛、轴丝的发 生及尾的成形分化,精子特异性乳酸脱氢酶LDH-X的出现等。在此变态过程中,糖 原、脂质、蛋白质等代谢产物大批随细胞质排弃,代之以出现的LDH-X及六碳糖激酶来适应能量需要。在变态后期核蛋白质出现不断的磷酸化和脱磷酸化,蛋白质SH--基向SS--键转变,以精氨酸为主的鱼精蛋白(protamine)替换以组氨酸为主的核组蛋白(histone),使核蛋白和DNA紧密结合,以保证精子基因处于浓缩包装和不活跃状态。

原始生殖细胞经历精原细胞、精母细胞、精子细胞和精子,其间发生了减数分裂、组蛋白/鱼精蛋白替换、精子变态等特异细胞活动,许多特异性的基因对精子发生过程进行严密的调控。

精子发生过程中,生精细胞可分为一下几个阶段:\myGlossaryEntry{pri_type_a_spermatogonium}、\myGlossaryEntry{a_spermatogonium}、\myGlossaryEntry{b_spermatogonium}、\myGlossaryEntry{primary_spermatocyte}、\myGlossaryEntry{secondary_spermatocyte}、\myGlossaryEntry{round_spermatid}、\myGlossaryEntry{condensing_spermatid}、\myGlossaryEntry{spermatozoon}、\myGlossaryEntry{preleptotene_spermatocyte}、\myGlossaryEntry{leptotene_spermatocyte}、\myGlossaryEntry{zygotene_spermatocyte}、\myGlossaryEntry{pachytene_spermatocyte}。精子发生过程及各期生精细胞特征参见图 \ref{figure_spermatogenesis_process}。

\begin{figure}
\centering
\myFigurePlaceholder
\caption{精子发生示意图}
\label{figure_spermatogenesis_process}
\end{figure}

\subsubsection{有丝分裂期}

\paragraph{\myGlossaryEntry{pri_type_a_spermatogonium}}

精子在睾丸中的发生起源于\myGlossaryEntry{pri_type_a_spermatogonium},也被称为\myGlossaryEntry{spermatogonial_stem_cell}。这类细胞通过有丝分裂进行增殖,所产生的子代细胞可以分为两类:一类仍保持精原干细胞的特征进行有丝分裂,成为长期精子发生的”源泉”;另一 类子代细胞则进入分化途径。

\paragraph{\myGlossaryEntry{a_spermatogonium}}

一部分\myGlossaryEntry{pri_type_a_spermatogonium}的子代细胞进入分化过程首先形成\myGlossaryEntry{a_spermatogonium}。\myGlossaryEntry{a_spermatogonium}的分化也是一个复杂的过程。目前认为至少要经过以下几个阶段,通过$ A_1 $型、$ A_2 $型、$ A_3 $型和$ A_4 $型精原细胞形成\myGlossaryEntry{intermediate_spermatogonium}。

\paragraph{\myGlossaryEntry{b_spermatogonium}}

这是精原细胞的最后阶段。在此之前,精原细胞都是通过有丝分裂进行增殖。\myGlossaryEntry{intermediate_spermatogonium}进行最后的有丝分裂,形成\myGlossaryEntry{b_spermatogonium},随后停止有丝分裂,由它们发育形成初级精母细胞,进入减数分裂期。

\paragraph{精子发生的同步化现象}

精子发生过程中的一个特点是许多生精细胞进行同步化分裂,并且细胞的
\myGlossaryEntry{cytokinesis}不完全,导致子细胞由细胞间桥相连(图 \ref{figure_spermatogenesis_synchronization}),这可能是它们同步分行的基础。

\begin{figure}
\centering
\myFigurePlaceholder
\caption{精子发生的同步化现象示意图}
\label{figure_spermatogenesis_synchronization}
\end{figure}

\subsubsection{减数分裂期}

配子是单倍体的,这种单倍体的细胞必须通过一种特殊的细胞分裂形式---
\myGlossaryEntry{meiosis}产生。简述分裂仅发生于有性生殖细胞发生过程中的某个阶段,其特点是细胞进行连续两次分裂而DNA只复制一次,结果产生了只含有单倍体遗传物质的细胞。含有单倍体遗传物质的两性生殖细胞通过受精形成合子,染色体又恢复到体细胞的数目,从而可以维持物种的正常繁衍。因此,\myGlossaryEntry{meiosis}是生物有性生殖的基础。

在减数分裂过程中同源染色体间DNA发生了重组,这就增加了子代发生遗传变异的机会,确保了生物的多样性以更加适应环境的变化,所以减数分裂也是生物进化及生物多样性的基础保证。这些特殊的现象必定受特殊机理的调节。近年来,在这一领域的研究集中于寻找减数分裂的特异蛋白,揭示这一过程中染色体出现的一些特殊事件的机理。

B型精原细胞有丝分裂停止后,发育为初级精母细胞,并进入减数分裂期。在这期间,细胞进行了两次减数分裂。初级精母细胞经过第一次减数分裂形成次级精母细胞,次级精母细胞经过第二次减数分裂形成单倍体的精子细胞。两次减数分裂之间的间期或长或短,但无DNA的合成。第一次减数分裂可分为前期I、中期I、后期I及末期I。第二次减数分裂可分为前期II、中期II、后期II及末期II。减数分裂的过程见图 \ref{图-减数分裂示意图}。

\begin{figure}[H]
\centering
\myFigurePlaceholder
\caption{减数分裂示意图}
\label{图-减数分裂示意图}
\end{figure}

\paragraph{初级精母细胞}

处于第一次减数分裂期的细胞为初级精母细胞。初级精母细胞随着第一次减数分裂过程中染色质的变化,又可分为前细线期(Preleptotene)、细线期(leptotene)、偶线期(zygotene)及粗线期(pachytene)精母细胞。初级精母细胞的体积不断增大,粗线期精母细胞的体积可达到前细线期的两倍以上。第一次减数分裂的时间较长,在人类可长达22天,分裂后形成次级精母细胞。

\paragraph{次级精母细胞}

由初级精母细胞经第一次减数分裂而来,细胞的体积比初级精母细胞小。细胞及细胞核均为圆形,核内染色质呈细网状,着色较浅,细胞质较少。次级精母细胞存在时间较短,很快完成第二次减数分裂,形成两个精子细胞,所以在切片中很少看到次级精母细胞。

\paragraph{圆形精子细胞}

次级精母细胞经过第二次减数分裂后,首先形成圆形精子细胞。由于在第二次减数分裂前没有进行DNA的复制,所以圆形精子细胞中染色体数目减少一半,成为单倍体细胞。此时细胞核圆形,着色较深。细胞质少,内含丰富的线粒体核高尔基复合体。精子细胞不在分裂,而是进入一个复杂有序的形态演变过程,形成头、颈、尾结构的精子,该过程为精子形成期。

\subsubsection{精子形成期(Spermiogenesis)}

精母细胞进过两次减数分裂,而DNA只复制一次,形成了单倍的圆形精子细胞,此后细胞不再分裂,而是进入一系列的形态变化,最后形成具有头、颈、尾结构的完整精子(图 \ref{图-人类精子分化示意图})。这一过程中细胞形态的变化主要表现再以下几个方面。

\begin{figure}[H]
\centering
\myFigurePlaceholder
\caption{人类精子分化示意图}
\label{图-人类精子分化示意图}
\end{figure}

\paragraph{细胞核的变化}

细胞核内染色质浓缩、体积变小、偏向细胞的一极,形成精子的头部。染色质的浓缩主要是由于染色质中的组蛋白(histone)被富含精氨酸的鱼精蛋白(protamine)取代。这一碱性蛋白含有的大量的正电荷,吸引着带负电的DNA发生集聚。然而这一蛋白质替换受哪些因素的调节,以及调节细胞核变化的其他因素有待于进一步的了解。

\paragraph{顶体的形成}

精子顶体是由高尔基复合体形成的。精子细胞的高尔基复合体首先产生许多小液泡。小液泡融合变大,形成一个大液泡叫顶体囊,内含一个大的颗粒,称为顶体颗粒(acrosomal granule)。然后由于液泡失去液体,以致液泡壁靠近位于核的前半部,组成了一双层膜,称为顶体帽,内有一个顶体颗粒。此后顶体颗粒物质分散于整个顶体帽中,这就是顶体,内含多种水解酶。

\paragraph{线粒体鞘的形成}

在精子形成过程中,精子细胞的线粒体体积变小伸长,并精确地迁移到中段,围绕着中央轴丝而形成螺旋状排列的线粒体鞘。物种不同,线粒体鞘总体构型也有很大差别。淡水无脊椎动物和海洋动物比较简单,只是由几个长的线粒体组合起来形成的线粒体鞘。而在哺乳类动物中则形成了典型的螺旋状排列的线粒体鞘。

\paragraph{中心粒的迁移}

精子形成早期,两个中心粒移值核的后方,当核的后端表面形成一个凹时,一个中心粒恰好位于凹之中,称为近端中心粒,它与精子长轴呈横向排列;另一个称为远端中心粒,位于近端中心粒的后方,它的长轴平行于精子长轴,由它产生精子尾部的中央轴丝。哺乳动物的远端中心粒,在颈段发育完成后,最终消失,有些哺乳动物的近端中心粒在精子成形后也会消失。

\paragraph{精子尾部的形成}

中央轴丝是构成鞭毛型精子尾部的基本细胞器,其外周还有致密的纤维和纤维鞘,这种粗大的纤维从精子中段起始,并不完全达到尾的末端。

\paragraph{细胞质的变化}

精子细胞的大部分细胞质,在精子形成过程中成为残体(residual body)而被抛弃。当细胞核前端形成顶体时,细胞质向后方移动,仅留下一薄层细胞质与质膜覆盖在细胞核上。当尾部的后端生长时,细胞质的大部分附着在精子中段,当线粒体围绕着轴丝时,该处的细胞质核高尔基体成为残余细胞质被抛弃,仅剩下一薄层细胞质包围着中段的线粒体。

\subsection{精子发生的特点}

\subsubsection{生精上皮周期}

一代(generation)生殖细胞是指这样一群细胞,它们在大概相同的时间形成。然后同步通过生精过程。在一定的曲细精管区域、两次分化程度相同的细胞群相继出现之间,有一个明显的时间间隔,称为生精上皮周期(cycle of seminiferous epithelium)。

不同物种的这一周期长短不同,人为16天、 公牛和大鼠为13天、羊和兔为10天、小鼠为8.6天。

不同代的生精细胞组成固定的细胞群体组合, 由于精子形成期精密而有序的时间步骤, 某一期的精子细胞总是与一定分化期的精母细胞和精原细胞相关。 对这些细胞群体进一步的观察表明:在曲细精管的任何一个区域, 它们以一定的顺序先后出现。 这些细胞群体以规律性的间隔重复出现,代表一个生精上皮周期的多个阶段。 由于这些期表示一个
连续过程的任意亚群, 因此一期的结束与下一期开始间的分界常常是不精确的。对一定的哺乳动物种属, 细胞群体的数目或周期的阶段依据采取的鉴定标准而不同。

有两种方法常用来鉴定生精小管周期的期相。第一种方法是利用精子细胞核的形态以及在生精上皮的位置。更加成熟的生精细胞成簇排列深深嵌在上皮内,对着支持细胞的核。 其后进一步分化的细胞向管腔移动,最终释放到管腔中。 在这一分类方法中,多种染色方法被应用:最常用的是苏木素---伊红(haematoxylin---eosin)染色法, 这一分类方法可把大鼠、公羊、公牛、兔、猪、水貂和猴的生精上皮分为8个期。 在一些研究中这8 个期又可以进一步细分。

第二种方法用过碘酸---希夫(PAS)---苏木素, 对生精管进行组织切片染色。 可把发育中的顶体系统染成深紫色,利用顶体来区分精子细胞。利用PAS---苏木素技术,可将大鼠的生精上皮周期分为14个期,仓鼠与豚鼠分为13期,小鼠与猴子分为12期。对在人类可以分为6期,然而这期间的界限并不十分清楚,常缺少一代或多代生精细胞,或在组织处理过程中细胞易位。 此外,一些研究者否认人类存在精子发生波,与猩猩中的现象相似。但是导致这种人类和猩猩精子发生无规律模式的因素还不清楚。

把细胞分为一个周期中的可区分的几个阶段有利于研究生殖细胞的结构与细胞化学 变化,研究一些因素(物理或化学因素)对有丝分裂 、减数分裂及细胞分化的影响。还 可以帮助我们对一些物质在生殖细胞中进行精确定位。

对生精上皮周期不同阶段的确定,是认识精子发生过程中染色体活动及变化的基础。已经观察到在小鼠中,生精上皮周期中DNA的合成发生在7个阶段,即8、10、 12、2、3、5、7 $ \sim $ 8期,主要涉及$ A_1 $、$ A_2 $、$ A_3 $、$ A_4 $、中间型及B型精原细胞。在精子发生过程中,DNA合成有两个高峰期,第一个高峰期在4 $ \sim $ 6期,对应于中间型及B型精原细胞的有丝分裂,第二个高峰位于8 $ \sim $  9期,反映前细线期精母细胞DNA的复制。而主要的RNA合成发生在精原细胞和粗线期精母细胞 。

\subsubsection{生精上皮波}

每一个生精细胞群,除了环形对称以外,还表现出沿生精管长轴有序组织排列,在那里产生精子发生\doubleQuote{波},称为生精上皮波(wave of seminiferous epithelium)。生精上皮周期与生精上皮波是不同的,周期是生精上皮的一个区域,在一定时间内发生的动态的组织学现象;\doubleQuote{波}是指一定时间内细胞群沿生精管有序的分布。换句话说,\doubleQuote{波}是空间,而\doubleQuote{周期}是时间。

已经报道在多种哺孔动物中存在着生精上皮波。但是在人类难以证明一点。因为人类睾丸中的细胞组合不规则。在狒狒睾丸中,可把生精小管分为三类。第一类包括只有单一细胞组合的生精小管段,第二类包括两种或多种细胞群有序的排列,而第三类小管只在人的生精小管中观察到。有关这些在拂拂、猩猩和人类不规则的原因所知甚少。有人认为这种无序可能反映了这些物种生殖能力的退化。但有研究表明,人类生精阶段的排列并不是随机无序的,而是有序地沿管长轴螺旋式排列。

\subsubsection{生精细胞的同步发育}

生精上皮的一个特征就是生精细胞的同步发育。人们用电子显微镜观察到发育中的雄性细 胞群,如精原细胞、精母细胞、精子细胞,通过细胞间桥连接。这种间桥是由不完整的胞质分裂所致,这种合胞体构成了它们同步发育的基础。尽管在多种哺乳类中发现这种细胞间桥的现象,然而有关这种间桥相连的确切的细胞数量存在着争论。有研究表明, 一组由细胞间桥连接的精子细胞可能有几百个,然而这比理论上的数目要少,可能由于精原细胞和精母细胞的退化所致。合胞体可能作为同步分化的装置起作用。也有人认为除了细胞间桥外,还存在着别的因素参与生殖细胞的同步分化。

\subsection{生精细胞的结构}

\subsubsection{精原细胞}

精原细胞是精子发生的起点,它紧靠在曲细精管的基底膜上,属于生精干细胞。细胞呈椭圆形或圆形, 直径$ 12\mu m $,胞核圆形,染色质着色较深,有1$ \sim $ 2个核仁。精原细胞经有丝分裂不断增殖,一部分作为贮备干细胞,另一部分进入生长期、发育成初级精母细胞。精原细胞的染色体组型,人为$ 46XY $、马为$ 66XY $、猪为$ 40XY $、狗为$ 78XY $。

\subsubsection{初级精母细胞}

初级精母细胞,是由精原细胞发育而来,体积增大,最终达到精原细胞体积的二倍,细胞器完备、数目增多,细胞核呈圆形,分裂期时间在人中可长达22天左右。分裂后形成次级精母细胞。

\subsubsection{次级精母细胞}

初级精母细胞经第一次减数分裂形成次级精母细胞,位于初级精母细胞近管腔侧,细胞体积比初级精母细胞小。细胞和细胞核均为圆形,核内染色质呈细网状,着色 较浅,细胞质较少。次级精母细胞存在时间较短,很快进入第二次减数分裂,形精子细胞,染色体数目减少一半,成为单倍体细胞。由于次级精母细胞存在时间短,所以在切片中较少见。

\subsubsection{精子细胞}
次级精母细胞分裂后产生精子细胞,它们靠近管腔。细胞核圆形,着色较深。细胞 质少,内含中心粒、线粒体和高尔基复合体等细胞器。精子细胞不再分裂,经过复杂的 形态演变后形成精子,该过程为精子形成期。

\subsubsection{精子}

精子发生最终形成精子。在睾丸中发育成完整的精子后,还需在附睾中经过一段成熟发育,才达到具有运动能力真正成熟的精子。成熟精子由两个主要部分构成,即头部和尾部。尾部又分成颈段(necksegment)、中段(middle piece)、主段(principal piece)和末段(end piece)。人类成熟精子的结构见图 \ref{图-人类精子示意图}。

\begin{figure}[H]
\centering
\myFigurePlaceholder
\caption{人类精子示意图}
\label{图-人类精子示意图}
\end{figure}

\paragraph{头部}

头部的绝大部分, 被染色质高度集聚的细胞核所占据, 具有几乎不见核孔的双层核被膜。在浓缩的细胞核前端, 盖着帽形囊状顶体。顶体后缘的后方, 称为头部顶体后区(图 \ref{图-人类精子头部超微结构示意图})。

\begin{figure}[H]
\centering
\myFigurePlaceholder
\caption{人类精子头部超微结构示意图}
\label{图-人类精子头部超微结构示意图}
\end{figure}

\subparagraph{精子核}

精子核的主要特征是其高度浓缩的核物质,由DNA和蛋白质组成。精子核体积远小于体细胞的细胞核,一般情况下核的形状与头部的性状一致。核的染色均匀,但在人和猩猩的精子核中出现一些空泡。核是父本遗传物质的携带者,含有单倍体的常染色体和一个性染色体($ X $或$ Y $)。

\subparagraph{顶体}

顶体为一层单位膜包裹的囊凹状结构,靠近核膜的单位膜称为顶体内膜,靠近细胞质膜一侧的单位膜叫顶体外膜,顶体内、外膜平行排列,并在顶体后缘彼此相连。顶体腔中具有不定形基质,其中含有透明质酸酶(hyaluronidase)、神经氨酸酶(neuraminidase)、酸性磷酸酶(acid phosphatase)、􀀌$ \beta $-N-乙酰葡糖胺糖昔酶($ \beta $-N-acetylglu­cosaminidase)、芳基硫酸酯酶(arylsulfatase)和顶体蛋白(acrosin)等水解酶类。当受精时,精子发生顶体反应,质膜和顶体外膜发生融合,形成囊泡,从而使顶体内酶类放出来,有利于精子通过卵外的各层结构。

顶体后部的狭窄区,称为赤道段(equatorial segment)。受精时,赤道段基本完好无损,而顶体的其他部位,均在顶体反应中丢失。顶体后区的质膜,是受精时精子首先与卵表面接触和发生融合的位置,所以该部位在功能上非常重要。顶体尾侧细胞质浓缩,特化为一薄层环状致密带,紧贴在细胞质膜的内表面,称为顶体后环(postacrosomal ring)。顶体后环紧贴着质膜,在该环尾缘,核膜和质膜紧密相贴,形成一环状黏合线,称为核后环(postnuclear ring)。核后环尾侧,核膜与质膜分开,核膜向下,形成一皱褶,延伸入颈段。在核的后端,有一浅窝,称为植入窝(implantation fossa),恰与尾部颈段凸出的小头相嵌。

用电子显微镜观察精子头部时发现,头部质膜外有呈细丝状和颗粒状的糖萼(glycocalyx)。糖萼较厚,尤其在顶体前缘对应处的质膜外表面最集中。凝集素(lectin)可特异性地与糖萼上的多糖链结合。当用外源凝集素处理精子后,便可阻止受精,说明受精时特异性识别卵母细胞,与糖萼的糖链直接相关。

顶体和细胞核决定了精子头部的形状。不同动物的精子,头部形态差异很大。例如,猪、羊、牛精子头部为扁卵圆形,马的精子头部为完整的椭圆形。人和狗的精子头部为梨形,小鼠精子头部为镰刀型。

\paragraph{尾部}

\section{精子发生的调控机制}


\section{精子的成熟与获能}