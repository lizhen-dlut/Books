\section{R绘制中国地图}

\href{http://bbs.pinggu.org/thread-4182165-1-1.html}{http://bbs.pinggu.org/thread-4182165-1-1.html}

最近关注R绘制地图的方法,跟大家分享一下。总体来说,有很多种绘制地图的方法,常用的方法主要是基于以下三种方法来绘制地图:(1)ggplot2;(2)maps;(3)googleVis;还有一个程序包值得推荐:REmap。

当然还有很多其他的方法可以绘制地图,详见:\href{https://rstudio-pubs-static.s3.amazonaws.com/79029_b56eaffe36ef44f29b8efc0a07d67208.html}{Create Maps in R using Base Plotting, Lattice, ggplot2, GoogleVis and rChart}。

简单总结几种常用程序包绘制地图的优缺点:

(1)ggplot2:优点,可灵活调整图形的任意组成成分,同时可在图形上添加2个或多个维度的数据(如在地图上同时显示总人口数和每千人卫生人口数,详见下面的示例2),其他程序包通常只能绘制一个维度的数据(总人口数或每千人卫生人口数,详见下面的示例1);缺点,参数较多,较难短时掌握。同时绘制地区前需先下载shp文件,如果想获得省市级地区的最新二级地图的shp文件,通常很难。

强烈推荐ggplot2官方学习网址:\href{http://docs.ggplot2.org/current/}{http://docs.ggplot2.org/current/}。

另外推荐余光创的两篇博文:http://guangchuangyu.github.io/2014/05/use-ggplot2/ 及\href{http://guangchuangyu.github.io/2011/08/ggplot2-version-of-figures-in-25-recipes-for-getting-started-with-r/}{http://guangchuangyu.github.io/2011/08/ggplot2-version-of-figures-in-25-recipes-for-getting-started-with-r/}

(2)maps:优点,相对灵活,不加赘述;缺点,中国的基础地图中,没有将四川和重庆区分开,这是被无数maps的中国地图使用者最为诟病的地方。其他地区的基础地图是否有类似问题,不得而知。

(3)googleVis:优点,功能由起初的主要绘制地图的功能,逐步扩展,已经演变成非常强大的可视化工具,推荐学习网址:\href{http://cran.r-project.org/web/packages/googleVis/vignettes/googleVis\_examples.html}{http://cran.r-project.org/web/packages/googleVis/vignettes/googleVis\_examples.html}。缺点:绘制基础地图方法,仍然只能绘制一维的数据。同时绘制的地图依赖google地图,所以如果不能显示google地图,也就不能绘制地图。

(4)REmap:国人开发的基于百度地图Echart。优点,绘制地图方便快捷,省市级地区的二级地图非常精准,并可绘制炫酷的迁徙图和热图,推荐学习网址:\href{http://lchiffon.github.io/REmap/}{http://lchiffon.github.io/REmap/} ;缺点,同googleVis一样,只能绘制一维的数据,同时地图上只能显示中文地名,所以想让它来发英文文章,估计就不行了。目前主要用它来获取精细的经纬度信息,在获取经纬度信息上,中文地名能很好识别,部分英文省市级名称也能识别,但有限。

回到正题,以下分别介绍ggplot2和REmap绘制中国地图的方法代码:

在绘制地图前准备以下数据: