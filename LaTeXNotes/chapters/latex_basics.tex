\chapter{\LaTeX 基础知识}

\section{\LaTeX 源文件的结构}
所有的完整的\LaTeX 源文件都分为导言和正文两大部分。出现在“\verb|\begin{document}|”之前的部分,称为“导言”部分。出现在“\verb|\begin{document}|”和“\verb|\end{document}|”的内容称为“正文”部分。在完整的\LaTeX 源文件的“\verb|\end{document}|”之后的任何字符都会被忽略。

\section{\LaTeX 提供的文档类型}

文档类型简称为“文类”,是有\LaTeX 命令编写的、用于规范某种文档格式排版的程序性文件,其扩展名为“.cls”。完整的\LaTeX 源文件的第一条命令,就需要指定该文档的类型。比如“\verb|\documentclass[11pt,a4paper,oneside]{book}|”。

\section{\LaTeX 命令简介}

\LaTeX 命令以反斜杠“\textbackslash” 作为开头,其后跟命令名\footnote{这意味着在\LaTeX 源文件中需要通过特殊方法输入“\textbackslash”,其输入方法为通过命令“\textbackslash textbackslash”。}。\LaTeX 命令可以分为以下两种类型:

\begin{enumerate}
	\item \label{latexcmd_type_one} 反斜杠“\textbackslash”后跟多个英文字母组成的命令,命令区分大小写。这类命令以空格、数字或非字母符号作为结束标志。这类命令包括\latexbuiltincmd{\textbackslash newline}、\latexbuiltincmd{\textbackslash textbackslash}、\latexbuiltincmd{\textbackslash today}等。
	\item 反斜杠“\textbackslash”后跟一个非字母符号组成的命令名, 如“\latexbuiltincmd{\textbackslash !}”、“\latexbuiltincmd{\textbackslash~\^}”等。
\end{enumerate}

在编译源文件时,紧跟在类型\ref{latexcmd_type_one}之后的所有空格,都将被系统视为命令结束符而被忽略。如果需要在这类命令后保留一个空格,可以采用以下几种方法:

\begin{enumerate}
	\item 可在命令之后紧跟一对大括号{}和一个空格。比如命令“\textbackslash textbackslash\{\} ”
	\item 紧跟一个反斜杠和一个空格。比如命令“\textbackslash textbackslash\textbackslash\ ”
	\item 在命令后紧跟一个“\~”,比如“\textbackslash textbackslash\~”
\end{enumerate}

\subsection{\LaTeX 命令的参数}

\LaTeX 命令可以附带多个参数,这些参数可以分为必要参数和可选参数:

\begin{enumerate}
	\item 必要参数:这类参数需要置于一对大括号中,且各参数之间的顺序不能调换。
	\item 可选参数:这类参数置于一对中括号中,各参数之间以英文逗号“,”分割。
\end{enumerate}


\section{\LaTeX 计数器学习}
\LaTeX 系统内置了23个计数器,其中17个为序号计数器,6个为控制计数器。

序号计数器包括:

\begin{center}
	\begin{longtable}{|c|c|l|}
		\hline
		计数器名称 & 计数器类别 & 用途简要说明 \endhead \hline
		part & 序号计数器 & \\ \hline
		chapter & 序号计数器 & \\ \hline
		section & 序号计数器 & \\ \hline
		subsection & 序号计数器 & \\ \hline
		subsubsection & 序号计数器 & \\ \hline
		paragraph & 序号计数器 & \\ \hline
		subparagraph & 序号计数器 & \\ \hline
		figure & 序号计数器 & \\ \hline
		table & 序号计数器 & \\ \hline
		equation & 序号计数器 & \\ \hline
		footnote & 序号计数器 & \\ \hline
		mpfootnote & 序号计数器 & \\ \hline
		page & 序号计数器 & \\ \hline
		enumi & 序号计数器 & \\ \hline
		enumii & 序号计数器 & \\ \hline
		enumiiI & 序号计数器 & \\ \hline
		enumiv & 序号计数器 & \\ \hline
		
		\caption{\LaTeX 内置序号计数器}
	\end{longtable}
\end{center}
控制计数器包括:
\begin{center}
	\begin{longtable}{|c|c|l|}
		\hline
		计数器名称 & 计数器类别 & 用途简要说明 \endhead \hline
		bottomnumber & 控制计数器 & \\ \hline
		dbltopnumber & 控制计数器 & \\ \hline
		secnumdepth & 控制计数器 & \\ \hline
		tocdepth & 控制计数器 & \\ \hline
		topnumber & 控制计数器 & \\ \hline
		totalnumber & 控制计数器 & \\ \hline
		
		\caption{\LaTeX 内置控制计数器}
	\end{longtable}
\end{center}


\subsection{定义和修改计数器}
定义一个计数器需要用到的命令是:

\begin{minipage}{\textwidth}
	\begin{verbatim}
	\newcounter{⟨counter name⟩}
	\newcounter{⟨counter name⟩}[⟨parent counter name⟩]
	\end{verbatim}
\end{minipage}


其中⟨counter name⟩为计数器的名称,可选参数⟨parent counter name⟩ 定义为⟨counter name⟩的上级计数器。