\section{通过\mintinline{latex}{hyperref} 包生成PDF表单(PDF Form)}
	\LaTeX 在生成PDF的同时,也可以生成PDF表单。生成PDF表达需要依赖 \mintinline{latex}{hyperref} 包。同时,PDF表达需要包含在 \mintinline{latex}{Form} 环境中。一个基本的包含PDF表单的\LaTeX 文档如下:
\begin{minted}{latex}
\documentclass{ctexbook}
\usepackage{hyperref}
\begin{document}
    \begin{Form}[action=mailto:hiseq@outlook.com,encoding=html]
        姓名:\TextField[name=name,width=5em]{} \\
        性别:\ChoiceMenu[radio,default=f,name=sex, align=1]{}
            {男=m,女=f} \\
        城市:\ChoiceMenu[combo,name=城市,width=6cm]{}
            {北京,上海,武汉,广州,深圳} \\
        爱好:\CheckBox[name=movie]{电影}
            \CheckBox[name=music]{音乐}
            \CheckBox[name=painting]{美术} \\
        \Submit{提交} \qquad \Reset{重置}
    \end{Form}
\end{document}
\end{minted}

这段代码生成的表单如下:

\noindent \fbox{\begin{minipage}{\textwidth}
	姓名:\TextField[name=name,width=5em]{} \\
	性别:\ChoiceMenu[radio,default=f,name=sex, align=1]{}
	{男=m,女=f} \\
	城市:\ChoiceMenu[combo,name=城市,width=4cm]{}
	{北京,上海,武汉,广州,深圳} \\
	爱好:\CheckBox[name=movie]{电影}
	\CheckBox[name=music]{音乐}
	\CheckBox[name=painting]{美术}\\
	\Submit{提交} \qquad	\Reset{重置}
\end{minipage}}\\

\subsection{\mintinline{latex}{Form} 环境}
\mintinline{latex}{hyperref} 包提供了\mintinline{latex}{Form} 环境。每一个通过\mintinline{latex}{hyperref} 包实现的PDF表单的\LaTeX 文档都必须包含其只包含一个\mintinline{latex}{Form} 环境。

\begin{minted}{latex}
\begin{Form}[action=URL,encoding=name,method=name]

\end{Form}
\end{minted}

\mintinline{latex}{Form} 环境包含三个可选参数:action、encoding和method。

\begin{tabular}{|c|c|l|}
	\hline
	action & URL & 接受表单数据的URL(可以是邮箱)。 \\ \hline
	encoding & name & 表单数据提交的格式,仅可选html。  \\ \hline
	method & name & 仅针对HTML有用,可选post或get。\\
	\hline
\end{tabular}

\subsection{\mintinline{latex}{Form} 控件}
\mintinline{latex}{hyperref} 包支持的表达控件有六种:

\begin{tabular}{|c|l|}
	\hline
	类型 & 使用方法 \\ \hline
	文本框 & \mintinline{latex}{\TextField[parameters]{label}} \\ \hline
	复选框 & \mintinline{latex}{\CheckBox[parameters]{label}}   \\ \hline
	单选框 & \mintinline{latex}{\ChoiceMenu[parameters]{label}{choices}}   \\ \hline
	PushButton & \mintinline{latex}{\PushButton[parameters]{label}}  \\ \hline
	提交按钮 & \mintinline{latex}{\Submit[parameters]{label}}  \\ \hline
	重置按钮 & \mintinline{latex}{\Reset[parameters]{label}}  \\ \hline
\end{tabular}

\subsubsection{\mintinline{latex}{Form} 控件的可选参数}
在Form控件的使用方法中,都有可选的parameters,这些可选参数包括:

\begin{longtable}{|c|c|c|l|}
	\hline
	参数名 & 值类型 & 默认值 & 说明 \endhead \hline 
	align & number & 0 & 设置排列方式,0:左对齐;1:居中;2:右对齐。 \\ \hline
	name & name & & 控件名称,强烈建议设置。\\ \hline
	charsize &  & & 控件的字体大小。\\ \hline
	width & 长度 & & 控件宽度。 \\ \hline
	height & 高度 & & 控件高度。\\ \hline
	value & & & 控件初始值。 \\ \hline
	multiline & 布尔值 & false & 控件是否支持多行 \\ \hline
	\multicolumn{4}{|c|}{颜色相关} \\ \hline
	backgroundcolor &   & & 背景颜色 \\ \hline
	bordercolor &  &   & 边框颜色 \\ \hline
\end{longtable}




\subsection{设置表单字体大小为:auto}
对于这段代码,生成的表单,在填写时,我们会发现,可能无法正确的处理汉子,比如汉子无法完全显示等。

\begin{minted}{latex}
\documentclass{ctexbook}
\usepackage{hyperref}
\begin{document}
    \begin{Form}[action=mailto:hiseq@outlook.com,encoding=html]
        姓名:\TextField[name=name,charsize=0pt,width=5em]{} \\
        性别:\ChoiceMenu[radio,default=f,charsize=0pt,name=sex, align=1]{}
            {男=m,女=f} \\
        城市:\ChoiceMenu[combo,charsize=0pt,name=city,width=6cm]{}
            {北京,上海,武汉,广州,深圳} \\
        爱好:\CheckBox[name=movie]{电影}
            \CheckBox[name=music]{音乐}
            \CheckBox[name=painting]{美术} \\
        \Submit{提交} \qquad \Reset{重置}
    \end{Form}
\end{document}
\end{minted}

这段代码生成的表单如下:

\noindent \fbox{\begin{minipage}{\textwidth}
		姓名:\TextField[name=name,charsize=0pt,width=5em]{} \\
		性别:\ChoiceMenu[radio,default=f,charsize=0pt,name=sex, align=1]{}
		{男=m,女=f} \\
		城市:\ChoiceMenu[combo,charsize=0pt,name=city,width=6cm]{}
		{北京,上海,武汉,广州,深圳} \\
		爱好:\CheckBox[name=movie]{电影}
		\CheckBox[name=music]{音乐}
		\CheckBox[name=painting]{美术} \\
		\Submit{提交} \qquad \Reset{重置}
\end{minipage}}\\

\subsection{表单数据验证}
PDF表单也可以对所输入的数据进行验证,在\mintinline{latex}{hyperref} 包中是通过JavaScript来实现的。

\begin{tabular}{|c|}
	\hline
	validate\\ \hline
	keystroke\\ \hline
	format\\ \hline
	onblur\\ \hline
	onchange\\ \hline
	onclick\\ \hline
	ondblclick\\ \hline
	onfocus\\ \hline
	onkeydown\\ \hline
	onkeypress\\ \hline
	onkeyup\\ \hline
	onmousedown\\ \hline
	onmousemove\\ \hline
	onmouseout\\ \hline
	onmouseover\\ \hline
	onmouseup\\ \hline
	onselect\\ \hline	
\end{tabular}\\


具体怎么操作,我也不会。