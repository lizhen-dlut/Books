\section{定义参数变长的命令}

在 C++ 中,我们可以为同一个函数赋予不同的执行内容,这种行为称之为「函数重载」。具体重载的函数,共享同一个函数名,但是接收的函数参数在数量、类型上不同。LaTeX 是宏语言,没有一般意义上的参数类型的说法。但是,有没有办法在 LaTeX 中「重载」一个宏,根据输入的参数数量不同,而产生不同的效果呢?
本文给出解决方案。

在 TeX 和 LaTeX2e 中定义新命令
TeX 中,定义新命令的标准方法是使用 TeX 原语 \def。它有几个变种,记录如下。

\def:局部定义,定义时不展开;
\edef:局部定义,定义时完全展开;
\gdef:相当于 \global\def;
\xdef:相当于 \global\edef。

建立在 TeX 之上的各种格式,其提供的定义新命令的方案,都是通过这四个 \def 来实现的。LaTeX2e 中定义新命令的标准方法是使用 \newcommand。它也有几个变种,记录如下。
\newcommand:新定义一个命令,如果该命令已有定义,则报错;
\renewcommand:重定义一个命令,如果该命令未定义,则报错;
\providecommand:如果该命令未定义,则定义一个新的命令;否则,啥也不干。
当然,在 LaTeX2e 中,也有 \DeclareRobustCommand 一系列命令,可以用来定义新的命令。这一系列命令,是 LaTeX2e 针对「脆弱命令」问题,提供的一些保护机制。此处不表。
在标准的方法中,不论是 TeX 还是 LaTeX2e,都没有提供「参数变长」的实现方法。也就是说,如果不引入奇怪的技巧,我们在普通的 LaTeX 文稿中,是无法重载命令的。