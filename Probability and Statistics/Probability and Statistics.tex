\documentclass{ctexbook}
\usepackage{ulem}


\usepackage{hyperref}
\hypersetup{
	pdftitle = {概率论和数理统计},
	pdfauthor = {Roger Young},
	pdfkeywords = {概率论,数理统计},
	pdfstartview = {FitH},
	%hidelinks,
	bookmarksnumbered={true}
}

\newcommand{\newconcept}[1]{\textbf{\uline{#1}}}

\begin{document}
	
\chapter{概率论的基本概念}
	
	自然界和社会上发生的想想是多种多样的。有一类现象,在一定条件下必然发生,例如,向上抛一石子必然下落,同性电荷必相互排斥,等等。这类详细称为\newconcept{确定性现象}\label{concept:Certanty}。在自然界和社会上存在着另一类现象,例如,在相同条件下抛同一枚硬币,其结果可能是正面朝上,也可能是反面朝上,并且每次抛掷前无法肯定抛掷的结果是什么;用同一门炮向同一目标射击,各次弹着点不尽相同,在一次射击之前无法预测弹着点的确切位置。这类现象,在一定的条件下,可能出现这样的结果,也可能出现那样的结果,而在试验或观察之前不能预知确切的结果。但人们经过长期实践并深入研究之后,发现这类现象在大量重复试验或观察下,它的结果呈现某种规律性。例如,多次重复抛一枚硬币得到正面朝上大致有一半,同一门炮射击同一目标的弹着点按照一定规律分布,等等。这种在大量重复试验或观察中呈现出的固有规律性,就是我们以后所说的\newconcept{统计规律性}\label{concept:StatisticalRegularity}。
	
	这种在个别试验中其结果呈现出不确定性,在大量重复试验中其结果又具有\hyperref[concept:StatisticalRegularity]{统计规律性}的现象,我们称之为\newconcept{随机现象}。概率论和数理统计是研究和揭示随机现象\hyperref[concept:StatisticalRegularity]{统计规律性}的一门数学学科。
	
\section{随机试验}
	
	我们遇到过各种试验。在这里,我们把试验作为一个含义广泛的术语。它包括各种各样的科学实验,甚至对某一事物的某一特征的观察也认为是一种试验。下面举一些试验的例子:\label{example:RandomExperiment}
	\begin{itemize}
		\item $ E_1 $:抛一枚硬币,观察正面$ H $、反面$ T $出现的情况。
		\item $ E_2 $:将一枚硬币抛掷三次,观察正面$ H $、反面$ T $出现的情况。
		\item $ E_3 $:将一枚硬币抛掷三次,观察正面出现的次数。
		\item $ E_4 $:抛一颗骰子,观察出现的点数。
		\item $ E_5 $:记录某城市120急救电话一昼夜接到的呼叫次数。
		\item $ E_6 $:在一批灯泡中任意抽取一只,测试它的寿命。
		\item $ E_7 $:记录某地一昼夜的最高温度和最低文档。
	\end{itemize}
	
	上面举出了七个试验的例子,它们有着共同的特点。例如,试验$ E_1 $有两种可能出现的结果,正面$ H $和反面$ T $,但在抛掷之前不能确定出现$ H $还是出现$ T $,这个试验可以在相同的条件下重复地进行。又如试验$ E_6 $,我们知道灯泡的寿命(以小时计)$ t \ge 0 $,但在测试之前不能确定它的寿命有多长。这一试验也可以在相同田间下重复地进行。概括起来,这些试验具有一下特点:
	\begin{enumerate}
		\item 可以在相同的条件下重复地进行;
		\item 每次试验的可能结果不止一个,并且能事先明确试验的所有可能结果;
		\item 进行一次试验之前不能确定哪一个结果会出现。
	\end{enumerate}

在概率论中,我们将具有上述三个特点的试验称为\newconcept{随机试验\label{concept:RandomExperiment}}。
我们通过研究随机试验来研究随机现象。

\section{样本空间、随机事件}

\subsection{样本空间}
对于\hyperref[concept:RandomExperiment]{随机试验},尽管在每次试验之前不能预知试验的结果,但试验的所有可能结果组成的集合是已知的。我们将随机试验$ E $的所有可能结果组成的集合称为$ E $的\newconcept{样本空间\label{concept:SampleSpace}},记为$ S $。样本空间的元素,即E的每个结果,称为\newconcept{样本点}。

下面写出\ref{example:RandomExperiment}中试验$ E_k(k=1,2,\dots,7) $的样本空间$ S_k $:


\end{document}