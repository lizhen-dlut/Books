\documentclass{book}
\usepackage{ctex}
\usepackage[left=1in, right=1in]{geometry}

\usepackage{amsmath}
\usepackage{amssymb}
\usepackage{amsthm}
\usepackage{diagbox}
\usepackage{ulem}

\usepackage{paralist}%段内列表

\usepackage{hyperref}
\usepackage{hyperxmp}
%%% END Article customizations

\DeclareMathOperator\dif{d\!}
\newcommand{\dx}{\ensuremath{\dif x}}
\newcommand{\dd}[1]{\ensuremath{\dif#1}}

\newcommand{\mydoublequote}[1]{``#1''}
\newcommand{\myconcepts}[1]{\textbf{\uline{{#1}}}}
\newcommand{\myimportantpoint}[1]{\textbf{\uline{#1}}}
\newcommand{\myparenthese}[1]{(#1)}

\newtheorem{axiom}{公理}[chapter]
\newtheorem{claim}{断言}[chapter]
\newtheorem{conjecture}{猜想}[chapter]
\newtheorem{corollary}{推论}[chapter]
\newtheorem{definition}{定义}[chapter]
\newtheorem{identity}{恒等式}[chapter]
\newtheorem{lemma}{引理}[chapter]
\newtheorem{note}{注}[chapter]
\newtheorem{paradox}{悖论}[chapter]
\newtheorem{proposition}{命题}[chapter]
\newtheorem{theorem}{定理}[chapter]

%%% The "real" document content comes below...
\newcommand{\documentTitle}{\texorpdfstring{数学学习笔记}{数学学习笔记}}
\newcommand{\documentAuthor}{ZHANG Yang}
\newcommand{\documentKeywords}{{高等数学,线性代数,概率论,数理统计}}
\newcommand{\documentSubject}{\documentKeywords}

\hypersetup{
	pdftitle=\documentTitle,
	pdfauthor=\documentAuthor,
	pdfsubject= \documentSubject,
	pdfkeywords= \documentKeywords,
	pdfstartview=FitH,
	pdfproducer = \documentAuthor,
	pdfcreator = \documentAuthor, 
	pdfcopyright = \documentAuthor, 
	pdflicenseurl = {https://github.com/hiatcg},
	colorlinks=true,       % false: boxed links; true: colored links
	linkcolor=red,          % color of internal links
	citecolor=red,        % color of links to bibliography
	filecolor=red,      % color of file links
	urlcolor=red           % color of external links
}

\title{\documentTitle}
\author{\documentAuthor}
\date{\today}

\begin{document}
	\maketitle
	\setcounter{secnumdepth}{4}
	\tableofcontents

	\part{高等数学}

	\chapter{函数与极限} %函数与极限
	\chapter{导数与微分} %导数与微分
	\chapter{微分中值定理与导数的应用}%微分中值定理与导数的应用
	\input{./chapters/indefinite_integral} %不定积分
	\input{./chapters/definite_integral} %定积分
	\input{./chapters/applications_of_integration} %定积分的应用
	\input{./chapters/differential_equations}%微分方程
	\input{./chapters/vector_algebra_and_spatial_analytic_geometry} %向量代数与空间解析几何
	\input{./chapters/differentiation_of_functions_of_several_variables}%多元函数微分法及其应用
	\input{./chapters/multiple_integrals} %重积分
	\input{./chapters/line_integral_and_surface_integral} %曲线积分与曲面积分
	\input{./chapters/infinite_sequences_and_series} %无穷级数
	\chapter{积分表}

\newcommand{\dx}{\ensuremath{\hspace{2pt}\dif x}}
\newcommand{\dd}[1]{\ensuremath{\hspace{2pt}\dif#1}}

\section{基本积分表}

\begin{equation}
\int x^n \dx = \frac{1}{n+1}x^{n+1}, \hspace{1ex} n\neq-1
\end{equation}

\begin{equation}
\int \frac{1}{x}\dx = \ln |x|
\end{equation}

\begin{equation}
\int u \hspace{2pt} \dd{v} = uv - \int v du
\end{equation}



\begin{equation}
\int e^x \dx = e^x 
\end{equation}

\begin{equation}
\int a^x \dx = \frac{1}{\ln a} a^x
\end{equation}

\begin{equation}
\int \ln x \dx = x \ln x - x
\end{equation}


\begin{equation}
\int \sin x \dx = -\cos x
\end{equation}

\begin{equation}
\int \cos x \dx = \sin x
\end{equation}

\begin{equation}
\int \tan x \dx = \ln |\sec x| 
\end{equation}

\begin{equation}
\int \sec x \dx = \ln |\sec x + \tan x|
\end{equation}

\begin{equation}
\int \sec^2 x \dx = \tan x
\end{equation}

\begin{equation}
\int \sec x \tan x \dx = \sec x
\end{equation}

\begin{equation}
\int \frac{a}{a^2+x^2}\dx = \tan^{-1}\frac{x}{a}
\end{equation}

\begin{equation}
\int \frac{a}{a^2-x^2}\dx = \frac{1}{2}\ln\left|\frac{x+a}{x-a}\right|
\end{equation}

\begin{equation}
\int \frac{1}{\sqrt{a^2-x^2}} \dx = \sin^{-1} \frac{x}{a}
\end{equation}

\begin{equation}
\int \frac{a}{x \sqrt{x^2-a^2}} \dx = \sec^{-1} \frac{x}{a}
\end{equation}

\begin{align}
\int \frac{1}{\sqrt{x^2-a^2}} \dx &= \cosh^{-1} \frac{x}{a} \\&= \nonumber \ln (x+\sqrt{x^2-a^2})
\end{align}

\begin{align}
\int \frac{1}{\sqrt{x^2+a^2}} \dx &= \sinh^{-1} \frac{x}{a} \\&=\nonumber \ln (x+\sqrt{x^2+a^2})
\end{align}

\section{含有$ ax+b $的积分}

\section{含有$ \sqrt{ax+b} $的积分} %积分表
	
	\part{线性代数}
%	\chapter{行列式}

行列式是线性代数中常用的工具。本章主要介绍$ n $阶行列式的定义、性质及其计算方法。

\section{二阶与三阶行列式}

用\terminologyItem{消元法}解二元线性方程组

\begin{equation} \label{equationBinaryLinearEquationsGeneralForm}
	\left\{
	\begin{array}{r}
		a_{11}x_1 + a_{12}x_2 = b_1, \\
		a_{21}x_1 + a_{22}x_2 = b_2. \\
	\end{array} \right.
\end{equation}

为消去未知数$ x_2 $,以$ a_{22} $和$ a_{12} $分别乘上列方程的两端,然乎两个方程相减,得到

\begin{equation}
	(a_{11}a_{22}-a_{12}a_{21})x_{1} = b_{1}a_{22}-a_{12}b_{2};
\end{equation}

类似地,消去$ x_1 $,得
\begin{equation}
	(a_{11}a_{22}-a_{12}a_{21})x_{2} = a_{11}b_{2}-b_{1}a_{21};
\end{equation}

当$ a_{11}a_{22}-a_{12}a_{21} \ne 0 $时,求得方程组 \ref{equationBinaryLinearEquationsGeneralForm} 的解为:

\begin{equation} \label{equationBinaryLinearEquationsGeneralFormResult}
	\left\{
	\begin{array}{r}
	x_{1} = \dfrac{b_{1}a_{22}-a_{12}b_{2}}{a_{11}a_{22}-a_{12}a_{21}} \\
	\\
	x_{2} = \dfrac{a_{11}b_{2}-b_{1}a_{21}}{a_{11}a_{22}-a_{12}a_{21}}
	
	\end{array} \right.
\end{equation}

\ref{equationBinaryLinearEquationsGeneralFormResult} 式中的分子、分母都是四个数分两对相乘、再相减而得,其中分母$ (a_{11}a_{22}-a_{12}a_{21}) $是由方程组 \ref{equationBinaryLinearEquationsGeneralForm} 的四个系数确定的,把这四个数按它们在方程组 \ref{equationBinaryLinearEquationsGeneralForm} 中的位置,排列成两行两列(横排称\terminologyItem{行}、竖排成\terminologyItem{列})的数表

\begin{equation} \label{equationBinaryLinearEquationsGeneralFormNumberTable}
\begin{array}{cc}
a_{11} & a_{12}\\
a_{21} & a_{22}.
\end{array}
\end{equation}

表达式$ (a_{11}a_{22}-a_{12}a_{21}) $称为数表 \ref{equationBinaryLinearEquationsGeneralFormNumberTable} 所确定的二阶行列式,并记作

\begin{equation} 
\left | \begin{array}{cc}
a_{11} & a_{12}\\
a_{21} & a_{22}
\end{array} \right |
\end{equation}


%	\chapter{矩阵及其运算}

\section{线性方程组和矩阵}

\subsection{线性方程组}
%	\input{./chapters/矩阵的初等变换与线性方程组}
%	\input{./chapters/向量组的线性相关性}
%	\input{./chapters/相似矩阵及二次型}
%	\input{./chapters/线性空间与线性变换}
	
	\part{概率论与数理统计}
	\chapter{概率论的基本概念}

自然界和社会上发生的现象是多种多样的。有一类现象,在一定条件下必然发生,例如,向上抛一石子必然下落,同性电荷必相互排斥,等等。这类现象称为\myconcepts{确定性现象}。在自然界和社会上存在着另一类现象,例如,在相同条件下抛同一枚硬币,其结果可能是正面朝上,也可能是反面朝上,并且在每次抛掷之前无法肯定抛掷的结果是什么;这类现象,在一定的条件下,可能出现这样的结果,也可能出现那样的结果,而在试验或观察之前不能预知确切的结果。但人们经过长期实践并深入研究之后,发现这类现象在大量重复试验或观察下,它的结果却呈现出某种规律性。这种在大量试验或观察中所呈现出的固有规律性,就是我们以后所说的统计规律性。

这种在个别试验中其结果呈现出不确定性,在大量重复实验中其结果又具有统计规律性的现象,我们称之为随机现象。概率论与数理统计是研究和揭示随机现象统计规律性的一门数学学科。

\section{随机试验}

我们遇到过各种试验。在这里,我们把试验作为一个含义广泛的术语。它包括各种各样的科学实验,甚至对某一事物的某一特征的观察也认为是一种试验。下面举一些试验的例子:
\begin{itemize}
	\item[$ E_1 $] :抛一枚硬币,观察正面$ H $、反面$ T $出现的情况;
	\item[$ E_2 $] :将一枚硬币抛掷三次,观察正面H、反面T出现的情况;
	\item[$ E_3 $] :将一枚硬币抛掷三次,观察出现正面的次数;
	\item[$ E_4 $] :抛一颗骰子,观察出现的点数;
	\item[$ E_5 $] :记录某城市120急救电话台一昼夜接到的呼唤次数;
	\item[$ E_6 $] :在一批灯泡中任意抽取一只,测试它的寿命;
	\item[$ E_7 $] :记录某地一昼夜的最高温度和最低温度。
\end{itemize}

上面举出了七个试验的例子,它们有着共同的特点。例如,试验
 %概率论的基本概念
	\chapter{随机变量及其分布}

\section{随机变量}

现在来讨论如何引入一个法则,将随机实验的每一个结果,即将$ S $的每个元素$ e $与实数$ x $对应起来,从而引入了随机变量的概念。

\begin{definition}
	设随机试验的样本空间为$ S = \left\lbrace e \right\rbrace  $,$ X = X(e) $是定义在样本空间$ S $上的实值单值函数,称$ X = X(e) $为\myconcepts{随机变量} \footnote{严格地说,\mydoublequote{对于任意实数$ x $,集合$ \left\lbrace e|X(e) \le x \right\rbrace  $(即:使得$ X(e) \le x $的所有样本点$ e $所组成的集合)有确定的概率}这一要求应包括在随机变量的定义之中,一般来说,不满足这一条件的情况,在实际应用中是很少遇到的。因此,我们在定义中未提及这一要求。}。
\end{definition}

我们一般以大写的字母如$ X $、$ Y $、$ Z $、$ W $、$ \cdots $ 表示随机变量,而以小写字母$ x $、$ y $、$ z $、$ w $、$ \cdots $ 表示实数。

随机变量的取值随实验的结果而定,而试验的各个结果出现有一定的概率,因而随机变量的取值有一定的概率。

随机变量的取值随试验的结果而定,在试验之前不能预知它取什么值,且它的取值有一定的概率。这些性质显示了随机变量与普通函数有着本质的差异。

随机变量的引入,使我们能用随机变量来描述各种随机现象,并能利用数学分析的方法对随机试验的结果进行深入广泛的研究和讨论。

\section{离散型随机变量及其分布律}

有些随机变量,它全部可能取到的值是有限个或可列无限多个,这种随机变量称为\myconcepts{离散型随机变量}。

容易知道,要掌握一个离散型随机变量$ X $的统计规律,必须且只需知道$ X $的所有可能取值以及取每一个可能值的概率。

设离散型随机变量X所有可能取的值为$ x_k(k=1,2,\cdots) $,X取各个可能值的概率,即事件$ \left\lbrace X = x_k \right\rbrace  $,为
\begin{equation}
	P\left\lbrace X = x_{k} \right\rbrace = p_{k} \text{,} k = 1, 2, \cdots \text{。}
\end{equation}
由概率的定义可知,$ p_{k} $满足如下两个条件:
\begin{enumerate}
	\item $ p_{k} \ge 0\text{,} k = 1, 2, \cdots $;
	\item $ \sum\limits_{k=1}^{\infty}p_{k} = 1 $。
\end{enumerate}

下面介绍三种重要的离散型随机变量。

\subsection{$ (0-1) $分布}

\subsection{伯努利试验、二项分布}

\begin{definition}
	设试验$ E $只有两个可能结果:$ A $及$ \bar{A} $,则$ E $为\myconcepts{伯努利(Bernoulli)试验}。设$ P(A) = p (0<p<1) $,此时$ P(\bar{A}) = 1 - p $。将$ E $独立重复地进行$ n $次,则称这一串重复的独立试验为\myconcepts{n重伯努利试验}。
\end{definition}

这里\mydoublequote{重复}是指在每次试验中$ P(A) = p $保持不变;\mydoublequote{独立}是指各次试验的结果互不影响。

以$ X $表示$ n $重伯努利试验中事件$ A $发生的次数,$ X $是一个随机变量,我们来求它的分布律。X所有可能取的值为$ 0 $,$ 1 $,$ 2 $,$ \cdots $,$ n $。由于各次试验是相互独立的,因此事件$ A $在指定的$ k(0 \le k \le n) $次试验中发生,在其他$ n-k $次试验中$ A $不发生(例如在前$ k $次试验中$ A $发生,而后$ n-k $次试验中$ A $不发生)的概率为
\begin{equation}
	\underbrace{p \cdot p \cdot \dots \cdot p}_{k\text{个}} \cdot
	\underbrace{(1 - p) \cdot (1 - p) \cdot \dots \cdot (1 - p)}_{n - k\text{个}}
	= p^{k}(1-p)^{n-k} \text{。}
\end{equation}
这种指定的方式共有$ \binom{n}{k} $种。它们是两两互不相容的,故在$ n $次试验中$ A $发生$ k $次的概率为$ \binom{n}{k}p^{k}(1-p)^{n-k} $,记$ q = 1 - p $,即有
\begin{equation}\label{equation:binomial_distribution}
	P\left\lbrace X = k \right\rbrace = \binom{n}{k}p^{k}q^{n-k} \text{,} k = 0, 1, 2, \cdots , n \text{。}
\end{equation}
显然
\begin{equation}
	P\left\lbrace X = k \right\rbrace \ge 0 \text{,} k = 0, 1, 2, \cdots , n \text{;}
\end{equation}

\begin{equation}
	\sum\limits_{k=0}^{n}P\left\lbrace X = k \right\rbrace = \sum\limits_{k=0}^{n}\binom{n}{k}p^{k}q^{n-k} = (p+q)^{n} = 1\text{。}
\end{equation}
注意到$ \binom{n}{k}p^{k}q^{n-k} $刚好是二项式$ (p+q)^{n} $的展开式中出现$ p^k $的那一项,我们称随机变量$ X $服从参数$ n $,$ p $的二项分布,并记为$ X \sim b(n,p) $。

特别地,当$ n=1 $时二项分布(\ref{equation:binomial_distribution})化为
\begin{equation}
	P\left\lbrace X = k \right\rbrace = p^{k}q^{1-k} \text{,} k = 0, 1 \text{。}
\end{equation}
这就是$ (0-1) $分布。

\subsection{泊松分布}

\begin{definition}
	设随机变量$ X $所有可能取的值为$ 0 $,$ 1 $,$ 2 $,$ \cdots $,而取各个值的概率为
	\begin{equation}
		P\left\lbrace X = k \right\rbrace = \frac{\lambda^{k}e^{-\lambda}}{k!}\text{,} k = 0, 1, 2, \cdots \text{,}
	\end{equation}
	其中,$ \lambda > 0 $是常数,则称$ X $服从参数为$ \lambda $的泊松分布,记为$ X \sim \pi(\lambda) $。
\end{definition}

\section{随机变量的分布函数}

\section{连续型随机变量及其概率密度}

如果对于随机变量$ X $的分布函数$ F(x) $,存在非负可积函数$ f(x) $,使对于任意实数$ x $有
\begin{equation}
	F(x) = \int_{-\infty}^{x}f(t)\dd{t}\text{,}
\end{equation}
则称$ X $为\myconcepts{连续型随机变量},$ f(x) $称为$ X $的\myconcepts{概率密度函数},简称\myconcepts{概率密度}。

\subsection{均匀分布}

\subsection{指数分布}

\subsection{正态分布}

\begin{definition}
	若连续型随机变量$ X $的概率密度为
	\begin{equation}
		f(x) = \frac{1}{\sqrt{2\pi}\sigma}e^{-\frac{(x-\mu)^2}{2\sigma^{2}}}\text{,} -\infty<x<\infty\text{,}
	\end{equation}
	其中$ \mu $,$ \sigma (\sigma > 0)$为常数,则称X服从参数为$ \mu $,$ \sigma $的\myconcepts{正态分布}或\myconcepts{高斯\myparenthese{Gauss}分布},记为$ X \sim N(\mu, \sigma^2) $。
\end{definition}

显然$ f(x) \geqslant 0 $,下面来证明$ \displaystyle\int_{-\infty}^{\infty}f(x)\dd{x} = 1 $。令$ \dfrac{x - \mu}{\sigma} = t $,得到
\begin{equation}
	\int\limits_{-\infty}^{\infty}f(x)\dd{x} = \int_{-\infty}^{\infty}\frac{1}{\sqrt{2\pi}\sigma}e^{-\frac{(x-\mu)^2}{2\sigma^2}}\dd{x} = \frac{1}{\sqrt{2\pi}} \int_{-\infty}^{\infty}e^{-\frac{t^2}{2}}\dd{t} \notag
\end{equation}
记$ \displaystyle I = \int_{-\infty}^{\infty}e^{-\frac{t^2}{2}}\dd{t} $,则有$ \displaystyle I^2 = \int_{-\infty}^{\infty}\int_{-\infty}^{\infty}e^{-\frac{t^2 + u^2}{2}}\dd{t}\dd{u} $,利用极坐标将它化成累次积分,得到
\begin{equation}
	I^2 = \int_{0}^{2\pi}\int_{0}^{\infty}re^{-\frac{r^2}{2}}\dd{r}\dd{\theta}\text{。} \notag
\end{equation}
而$ I > 0 $,故有$ I = \sqrt{2\pi} $,即有
\begin{equation}
	\int_{-\infty}^{\infty}e^{-\frac{t^2}{2}}\dd{t} = \sqrt{2\pi}
\end{equation}
于是$ \displaystyle\int_{-\infty}^{\infty}f(x)\dd{x} = \int_{-\infty}^{\infty}\frac{1}{\sqrt{2\pi}\sigma}e^{-\frac{(x-\mu)^2}{2\sigma^2}}\dd{x} = \frac{1}{\sqrt{2\pi}} \int_{-\infty}^{\infty}e^{-\frac{t^2}{2}}\dd{t} = 1 $。

$ f(x) $具有以下性质:
\begin{enumerate}
	\item 曲线关于$ x = \mu $对称。这表明对于任意$ h > 0 $有
		\begin{equation}
			P\left\lbrace \mu - h < X \leqslant \mu \right\rbrace = P\left\lbrace \mu < X \leqslant \mu + h \right\rbrace \notag
		\end{equation}
	\item 当$ x = \mu  $时取得最大值
		\begin{equation}
			f(\mu) = \frac{1}{\sqrt{2\pi}\sigma} \notag
		\end{equation}
\end{enumerate}
$ x $离$ \mu $越远,$ f(x) $的值越小。这表明对于同样长度的区间,当区间离$ \mu $越远,$ X $落在这个区间上的概率越小。

在$ x=\mu\pm\sigma $处曲线有拐点,曲线以$ Ox $轴为渐进线。

特别,当$ \mu = 0, \sigma = 1 $时称随机变量$ X $服从\myconcepts{标准正态分布}。其概率密度和分布函数分别用$ \varphi(x) $,$ \varPhi(x) $表示,即有
\begin{equation}
	\varphi(x) = \frac{1}{\sqrt{2\pi}}e^{-\frac{x^{2}}{2}}
\end{equation}
\begin{equation}
	\varPhi(x) = \frac{1}{\sqrt{2\pi}}\int_{-\infty}^{x}e^{-\frac{x^{2}}{2}}\dd{t}
\end{equation}
易知
\begin{equation}
	\varPhi(-x) = 1 - \varPhi(x)
\end{equation}

\begin{lemma}
	若$ X\sim N(\mu,\sigma^{2}) $,则$ Z = \dfrac{X-\mu}{\sigma}\sim N(0,1) $。
\end{lemma}

\begin{proof}
	$ Z = \dfrac{X-\mu}{\sigma} $的分布函数为
	\begin{align}
	P\left\lbrace Z\le x \right\rbrace & = P\left\lbrace Z\le \frac{X-\mu}{\sigma} \right\rbrace \notag \\
	& = P\left\lbrace X \le \mu + \sigma x \right\rbrace \notag\\
	& = \frac{1}{\sqrt{2\pi}}\int_{-\infty}^{\mu+\sigma x}e^{-\frac{(t-\mu)^2}{2\sigma^2}}\dd{t} \text{,} \notag
	\end{align}
	令$ \dfrac{t-\mu}{\sigma} = u $,得
	\begin{equation}
		P\left\lbrace Z \le x \right\rbrace = \frac{1}{\sqrt{2\pi}}\int_{-\infty}^{x}e^{-\frac{u^2}{2}}\dd{u} = \varPhi(x) \text{,} \notag
	\end{equation}
	由此知$ Z = \frac{X-u}{\sigma} \sim N(0,1) $。
\end{proof}

于是,若$ X\sim N(\mu,\sigma^{2}) $,则它的分布函数F(x)可写成
\begin{equation}
	F(x) = P\left\lbrace X\leqslant x \right\rbrace = P\left\lbrace \frac{X-\mu}{\sigma} \leqslant \frac{x-\mu}{\sigma} \right\rbrace = \varPhi(\frac{x-\mu}{\sigma})
\end{equation}
对于任意区间$ \left( x_1, x_2 \right]  $,有
\begin{align}
	P\left\lbrace x_1 < X \leqslant x_2 \right\rbrace  & = P\left\lbrace \frac{x_1 - \mu}{\sigma} < \frac{X - \mu}{\sigma} \leqslant \frac{x_2 - \mu}{\sigma} \right\rbrace \\
	& = \varPhi(\frac{x_2 - \mu}{\sigma}) - \varPhi(\frac{x_1 - \mu}{\sigma})
\end{align}

尽管正态变量的取值范围是$ \left( -\infty, \infty \right)  $,但它的值落在$ \left( \mu - 3\sigma, \mu + 3\sigma \right)  $内几乎是肯定的事,这就是人们所说的\mydoublequote{$ 3\sigma $}法则。

为了便于今后在数理统计中的应用,对于标准正态随机变量,我们引入了上$ \alpha $分位点的定义。

\begin{definition}
	设$ X \sim N(0,1) $,若$ z_{a} $满足条件
	\begin{equation}
		P\left\lbrace X > z_{a} \right\rbrace = a\text{,}0 < a < 1\text{,}
	\end{equation}
	则称点$ z_{a} $为标准正态分布的\myconcepts{上$ \alpha $分位点}。
\end{definition}

\section{随机变量的函数的分布}

\begin{theorem}
	设随机变量$ X $具有密度函数$ f_X(x),-\infty < x < \infty $,又设函数$ g(x) $处处可导且恒有$ g'(x) > 0 $\myparenthese{或恒有$ g'(x) < 0 $},则$ Y = g(X) $是连续型随机变量,其概率密度为
	\begin{equation}
		f_Y(y) = \left\lbrace
			\begin{array}{ll}
				f_X[h(y)]\left| h'(y)\right| , & \alpha < y < \beta\\
				0, & \text{其他}
			\end{array}
		\right. 
	\end{equation}
	其中$ \alpha = \min\left\lbrace g(-\infty), g(\infty) \right\rbrace $,$ \beta = \max\left\lbrace g(-\infty), g(\infty) \right\rbrace $,$ h(y) $是$ g(x) $的反函数。
\end{theorem} %随机变量及其分布
	\chapter{多维随机变量及其分布} %多维随机变量及其分布
%	\chapter{随机变量的数字特征}

前面我们介绍了随机变量的分布函数、概率函数和分布律,它们都能完整地描述随机变量,但在某些实际或理论问题中,人们感兴趣于某些能描述随机变量某一种特征的常数。这种由随机变量的分布所确定的、能刻画随机变量某一方面的特征的常数统称为\myconcepts{数字特征},它在理论和实际应用中都很重要。

\section{数学期望}

\begin{definition}
	设离散型随机变量X的分布律为
	\begin{equation}
		P\left\lbrace X = x_k \right\rbrace = p_k\text{,} k = 1, 2, \dots. \notag
	\end{equation}
	若级数
	\begin{equation}
		\sum_{k =1}^{\infty}x_{k}p_{k} \notag
	\end{equation}
	绝对收敛,则称级数$ \displaystyle \sum_{k=1}^{\infty}x_{k}p_{k} $的和为随机变量$ X $的\myconcepts{数学期望},记为$ E(X) $,即
	\begin{equation}
		E(X) = \sum_{k =1}^{\infty}x_{k}p_{k} \notag
	\end{equation}
	
	设连续型随机变量X的概率密度为f(x),若积分
	\begin{equation}
		\int_{-\infty}^{\infty}xf(x)\dx \notag
	\end{equation}
	绝对收敛,则称积分$ \int_{-\infty}^{\infty}xf(x)\dx $的值为随机变量$ X $的\myconcepts{数学期望},记为$ E(X) $,即
	\begin{equation}
		E(X) = \int_{-\infty}^{\infty}xf(x)\dx
	\end{equation}
\end{definition}

数学期望简称\myconcepts{期望},又称为\myconcepts{均值}。

数学期望$ E(X) $完全由随机变量$ X $的概率分布所确定,若$ X $服从某一分布,也称$ E(X) $是这一分布的数学期望。

\begin{theorem}
	设$ Y $是随机变量$ X $的函数:$ Y=g(X) $\myparenthese{g是连续函数}
	\begin{enumerate}
		\item 如果$ X $是离散型随机变量,它的分布律为$ P\left\lbrace X=x_k\right\rbrace = p_k, k = 1, 2, \dots $,若$ \displaystyle \sum_{k=1}^{\infty}g(x_k)p_k $绝对收敛,则有
		\begin{equation}
			E(Y) = E[g(X)] = \sum_{k=1}^{\infty}g(x_k)p_K \text{。}
		\end{equation}
		\item 如果$ X $是连续型随机变量,它的概率密度为$ f(x) $,若$ \displaystyle \int_{\infty}^{\infty}g(x)f(x)\dx $绝对收敛,则有
		\begin{equation}
			E(Y) = E[g(X)] = \int_{\infty}^{\infty}g(x)f(x)\dx \text{。}
		\end{equation}
	\end{enumerate}
\end{theorem}

定理的重要意义在于当我们求$ E(Y) $时,不必算出$ Y $的分布律或概率密度,而只需要利用$ X $的分布律或概率密度就可以了。

\section{方差}

\begin{definition}
	设$ X $是一个随机变量,若$ E\left\lbrace [X-E(X)]^2 \right\rbrace $存在,则称$ E\left\lbrace [X-E(X)]^2 \right\rbrace $为$ X $的方差,记为$ D(X) $或$ Var(X) $,即
	\begin{equation}
		D(X) = Var(X) = E\left\lbrace [X-E(X)]^2 \right\rbrace \text{。}
	\end{equation}
	在应用上,还引入量$ \sqrt{D(X)} $,记为$ \sigma(X) $,称为标准差或均方差。
\end{definition}

\begin{theorem}
	设随机变量$ X $具有数学期望$ E(X)=\mu $,方差$ D(X)=\sigma^2 $,则对于任意整数$ \epsilon $,不等式
	\begin{equation}
		P\left\lbrace |X-\mu|\geqslant\epsilon \right\rbrace \leqslant \frac{\sigma^2}{\epsilon^2}
	\end{equation}
	成立。
	
	这一不等式称为\myconcepts{切比雪夫\myparenthese{Chebyshev}不等式}}。
\end{theorem}

\section{协方差及相关系数}

\begin{definition}
	量$ E\left\lbrace [X-E(X)][Y-E(Y)] \right\rbrace $称为随机变量$ X $与$ Y $的协方差,记为$ Cov(X, Y) $,即
	\begin{equation}
		Cov(X,Y)=E\left\lbrace [X-E(X)][Y-E(Y)] \right\rbrace \text{。}
	\end{equation}
	而
	\begin{equation}
		\rho_{XY} = \frac{Cov(X, Y)}{\sqrt{D(X)}\sqrt{D(Y)}}
	\end{equation}
	称为随机变量$ X $与$ Y $的相关系数。
\end{definition}

\section{矩、协方差矩阵}
	\chapter{大数定律及中心极限定理}

极限定理是概率论的基本理论,在理论研究和应用中起着重要的作用,其中最重要的是称为\mydoublequote{大叔定律}与\mydoublequote{中心极限定理}的一些定理。\myimportantpoint{大数定律是叙述随机变量序列的前一些项的算术平均值在某种条件下收敛到这些项的均值的算术平均值};\myimportantpoint{中心极限定理则是确定在什么条件下,大量随机变量之和的分布逼近于正态分布}。

\section{大数定律}

大量试验证实,随机事件$ A $的频率$ f_n(A) $当重复试验的次数$ n $增大时总呈现出稳定性,稳定在某一个常数的附件。频率的稳定性是概率定义的客观基础。

\begin{theorem}[弱大数定理,辛钦大数定理]
	设$ X_1 $,$ X_2 $,$ \cdots $ 是相互独立,服从同一分布的随机变量序列,且具有数学期望$ E(X_k)=\mu \  (k=1,2,\cdots) $。作前$ n $个变量的算术平均$ \displaystyle \frac{1}{n}\sum_{k=1}^{n}X_{k} $,则对于任意$ \epsilon > 0 $,有
	\begin{equation}
		\lim_{n\to\infty}P\left\lbrace \left| \frac{1}{n}\sum_{k=1}^{n}X_{k} - \mu\rvert < \epsilon \right| \right\rbrace  = 1
	\end{equation}
\end{theorem}

\begin{theorem}[伯努利大数定理]
	设$ f_{A} $是$ n $次独立重复试验中事件$ A $发生的次数,$ p $是事件$ A $在每次试验中发生的概率,则对于任意正数$ \epsilon > 0 $,有
	\begin{equation}
		\lim_{n\to\infty}P\left\lbrace \left|\frac{f_A}{n}-p \right| < \epsilon \right\rbrace  = 1
	\end{equation}
	或
	\begin{equation}
		\lim_{n\to\infty}P\left\lbrace \left|\frac{f_A}{n}-p \right|\geqslant\epsilon \right\rbrace  = 0
	\end{equation}
\end{theorem}

\section{中心极限定理}

\begin{theorem}[独立同分布的中心极限定理]
	设随机变量
\end{theorem}

\begin{theorem}[李雅普诺夫\myparenthese{Lyapunov}定理]
	设随机变量
\end{theorem}

\begin{theorem}[棣莫弗---拉普拉斯\myparenthese{De Moivre---Laplace}定理]
	设随机变量
\end{theorem} %大数定律及中心极限定理
%	\input{./chapters/样本及抽样分布}
%	\input{./chapters/参数估计}
%	\input{./chapters/假设检验}
	\chapter{方差分析及回归分析}

\section{单因素试验的方差分析}

\subsection{单因素试验}

在科学试验和生产实践中,影响一事物的因素往往是很多的。例如,在化工生产中,有原料成分、原料剂量、催化剂、反应温度、压力、溶液浓度、反应时间、机器设备及操作人员的水平等因素。每一因素的改变都有可能影响产品的数量和质量。有些因素影响较大,有些较小。为了使生产过程得以稳定,保证优质、高产,就有必要找出对产品质量有显著影响的那些因素。为此,我们需进行试验。方差分析就是根据试验的结果进行分析,鉴别各个有关因素对试验结果影响的有效方法。

在试验中,我们将要考察的指标称为\myconcepts{试验指标}。影响试验指标的条件称为\myconcepts{因素}。因素可分为两类,一类是人们可以控制的\myparenthese{可控因素};一类是人们不能控制的。例如,反应温度、原料剂量、溶液浓度等是可以控制的,而测量误差、气象条件等一般是难以控制的。以下我们所说的因素都是指可控因素。因素所处的状态,称为该因素的水平。如果一项试验的过程中只有一个因素在改变,则称为\myconcepts{单因素试验};如果多于一个因素在改变,则称\myconcepts{多因素试验}。

现在开始讨论单因素试验的方差分析。设因素$ A $有$ s $个水平$ A_1 $,$ A_2 $,$ \cdots $,$ A_s $,在水平$ A_j\ (j=1,2,\cdots,s) $下,进行$ n_j\;\left( n_j \geqslant 2 \right)  $次独立试验,得到如表 \ref{table:result_of_single_factor_analysis_of_variance} 的结果。

\begin{table}[ht]
	\centering
	\caption{单因素试验结果}
	\label{table:result_of_single_factor_analysis_of_variance}
	\begin{tabular}{|c|c|c|c|c|}
		\hline 
		\diagbox{观察结果}{水平} & $ A_{1} $ & $ A_{2} $ & $ \cdots $ & $ A_{s} $ \\ 
		\hline 
		& $ X_{11} $ & $ X_{12} $ & $ \cdots $ & $ X_{1s} $ \\ 
		\cline{2-5} 
		& $ X_{21} $ & $ X_{22} $ & $ \cdots $ & $ X_{2s} $ \\ 
		\cline{2-5} 
		& $ \vdots $ & $ \vdots $ & $ \ddots $  & $ \vdots $ \\ 
		\cline{2-5}  
		& $ X_{n_{1}1} $ & $ X_{n_{2}2} $ & $ \cdots $ & $ X_{n_{s}s} $ \\ 
		\hline 
		样本总和 & $ T_{\cdot 1} $ & $ T_{\cdot 2} $ & $ \cdots $ & $ T_{\cdot s} $ \\ 
		\hline 
		样本均值 & $ \overline{X}_{\cdot 1} $ & $ \overline{X}_{\cdot 2} $ & $ \cdots $ & $ \overline{X}_{\cdot s} $ \\ 
		\hline 
		总体均值 & $ \mu_{1} $ & $ \mu_{2} $ & $ \cdots $ & $ \mu_{s} $ \\ 
		\hline 
	\end{tabular} 
\end{table}

我们假定:各个水平$ A_j\ \left(j=1,2,\cdots,s\right) $下的样本$ X_{1j} $,$ X_{2j} $,$ \cdots $,$ X_{n_{j}j} $来自具有相同方差$ \sigma^2 $,均值分布为$ \mu_j\ (j=1,2,\cdots,s) $的正态总体$ N(\mu_j, \sigma^2) $,$ \mu_j $与$ \sigma^2 $未知。且设不同水平$ A_j $下的样本之间相互独立。

由于$ X_{ij} \sim N\left( \mu_j, \sigma^2\right)  $,即有$ X_{ij} - \mu_j \sim N\left( 0, \sigma^2\right)  $,故$ X_{ij} - \mu_j $可看成是随机误差。记$ X_{ij} - \mu_j = \epsilon_{ij} $,则$ X_{ij} $可写出
\begin{equation}\label{equation:model_of_single_factor_analysis_of_variance}
	\left. 
	\begin{array}{l}
		X_{ij} = \mu_j + \epsilon_{ij} \text{,} \\
		\epsilon_{ij} \sim N\left( 0, \sigma^2\right)  \text{,各} \epsilon_{ij} \text{独立,} \\
		i = 1, 2, \cdots, n_j \text{,} j = 1,2,\cdots, s\text{。}
	\end{array}
	\right\rbrace 
\end{equation}

其中$ \mu_j $ 与 $ \sigma^2 $均为未知参数。式 \ref{equation:model_of_single_factor_analysis_of_variance} 称单因素试验方差分析的数学模型。这是本节的研究对象。

方差分析的任务是对于模型 \ref{equation:model_of_single_factor_analysis_of_variance} ,
\begin{enumerate}
	\item 检验$ s $个总体$ N\left( \mu_1, \sigma^2\right)  $,$ \cdots $,$ N\left( \mu_s, \sigma^2\right) $的均值是否相等,即检验假设
		\begin{equation}\label{equation:hypothesis_of_single_factor_analysis_of_variance}
			\begin{array}{rcl}
				H_{0} & : & \mu_1 = \mu_2 = \cdots = \mu_s \text{,}\\
				H_{1} & : & \mu_1 \text{,} \mu_2 \text{,} \cdots \text{,} \mu_s \text{不全相等} \text{。}\\
			\end{array}
		\end{equation}
	\item 作出未知参数$ \mu_1 $,$ \mu_2 $,$ \cdots $,$ \mu_s $,$ \sigma^2 $,的估计。
\end{enumerate}

为了将问题 \ref{equation:hypothesis_of_single_factor_analysis_of_variance} 写成便于讨论的形式,我们将$ \mu_1 $,$ \mu_2 $,$ \cdots $,$ \mu_s $的加权平均值$ \displaystyle \frac{1}{n}\sum_{j=1}^{s}n_{j}\mu_{j} $记为$ \mu $,即
\begin{equation}\label{eq:mu_of_single_factor_analysis_of_variance}
	\mu = \frac{1}{n}\sum_{j=1}^{s}n_{j}\mu_{j} \text{,}
\end{equation}
其中$ \displaystyle n = \sum_{j=1}^{s}n_j $,$ \mu $称为总平均。再引入
\begin{equation}
	\delta_j = \mu_j - \mu \text{,} j = 1, 2, \cdots, s \text{。}
\end{equation}
此时有$ n_1\delta_1 + n_2\delta_2 + \cdots + n_s\delta_s = 0 $,$ \delta^j $表示水平$ A_j $下的总体平均数与总平均的差异,习惯上将$ \delta_j $ 称为水平$ A_j $的效应。

利用这些记号,模型 \ref{equation:model_of_single_factor_analysis_of_variance} 可改写成
\begin{equation}\label{equation:updated_model_of_single_factor_analysis_of_variance}
	\left. 
	\begin{array}{l}
		X_{ij} = \mu_j + \epsilon_{ij} \text{,} \\
		\epsilon_{ij} \sim N\left( 0, \sigma^2\right) \text{,各} \epsilon_{ij} \text{独立,} \\
		i = 1, 2, \cdots, n_j \text{,} j = 1,2,\cdots, s \text{,} \\
		\displaystyle \sum_{j=1}^{s}n_{j}\delta_{j} = 0 \text{。}
	\end{array}
	\right\rbrace 
\end{equation}
而假设 \ref{equation:hypothesis_of_single_factor_analysis_of_variance} 等价于假设
\begin{equation}\label{equation:updated_hypothesis_of_single_factor_analysis_of_variance}
	\begin{array}{rcl}
		H_{0} & : & \delta_1 = \delta_2 = \cdots = \delta_s = 0 \text{,}\\
		H_{1} & : & \delta_1 \text{,} \delta_2 \text{,} \cdots \text{,} \delta_s \text{不全相等} \text{。}\\
	\end{array}
\end{equation}
这是因为当且仅当$ \mu_1 = \mu_2 = \cdots = \mu_s $时$ \mu_j = \mu $,即$ \delta_j = 0 $,$ j=1,2,\cdots, s $。

\subsection{平方和的分解}

下面我们从平方和的分解着手,导出假设检验问题 \ref{equation:updated_hypothesis_of_single_factor_analysis_of_variance} 的检验统计量。

引入总偏差平方和
\begin{equation}
	S_T = \sum_{j=1}^{s}\sum_{i=1}^{n_j}\left( X_{ij} - \overline{X}\right) ^2 \text{,}
\end{equation}
其中
\begin{equation}\label{eq:average_of_single_factor_analysis_of_variance}
	\overline{X} = \frac{1}{n}\sum_{j=1}^{s}\sum_{i=1}^{n_j}X_{ij}
\end{equation}
是数据的总平均。$ S_T $能反映全部试验数据之间的差异,因此$ S_T $又称为总变差。

又记水平$ A_j $下的样本平均值为$ \overline{X}_{\cdot j} $,即
\begin{equation}
	\overline{X}_{\cdot j} = \frac{1}{n}\sum_{i=1}^{n_j}X_{ij} \text{。}
\end{equation}

我们将$ S_T $写出
\begin{equation}
	\begin{aligned}
		S_T &= \sum_{j=1}^{s}\sum_{i=1}^{n_j}\left[ (X_{ij} - \overline{X}_{\cdot j}) + (\overline{X}_{\cdot j} - \overline{X}) \right]^{2} \\
		&= \sum_{j=1}^{s}\sum_{i=1}^{n_j}(X_{ij} - \overline{X}_{\cdot j})^2 + \sum_{j=1}^{s}\sum_{i=1}^{n_j}(\overline{X}_{\cdot j} - \overline{X})^2 + 2 \sum_{j=1}^{s}\sum_{i=1}^{n_j}(X_{ij} - \overline{X}_{\cdot j})(\overline{X}_{\cdot j} - \overline{X})
	\end{aligned}
\end{equation}
注意到上式第三项\myparenthese{即交叉项}
\begin{equation}
	\begin{aligned}
		2 \sum_{j=1}^{s}\sum_{i=1}^{n_j}(X_{ij} - \overline{X}_{\cdot j})(\overline{X}_{\cdot j} - \overline{X}) &= 2 \sum_{j=1}^{s}(\overline{X}_{\cdot j} - \overline{X})\left[ \sum_{i=1}^{n_j}(X_{ij} - \overline{X}_{\cdot j}) \right] \\
		&= 2 \sum_{j=1}^{s}(\overline{X}_{\cdot j} - \overline{X})(\sum_{j=1}^{n_j}X_{ij} - n_{j}\overline{X}_{\cdot j}) \\
		&= 0
	\end{aligned}
\end{equation}
于是,我们就将$ S_T $分解成我
\begin{equation} \label{equation:decomposition_of_sum_of_squares}
	S_T = S_E + S_A \text{,}
\end{equation}
其中
\begin{equation}
	S_E = \sum_{j=1}^{s}\sum_{i=1}^{n_j}\left( X_{ij} - \overline{X}_{\cdot j}\right)^2  \text{,}
\end{equation}
\begin{equation}
	S_A = \sum_{j=1}^{s}\sum_{i=1}^{n_j}\left( \overline{X}_{\cdot j} - \overline{X}\right)^2 = \sum_{j=1}^{s}n_{j}\left( \overline{X}_{\cdot j} - \overline{X}\right) ^2 = \sum_{j=1}^{s}n_{j}\overline{X}_{\cdot j}^{2} - n\overline{X}^2 \text{,}
\end{equation}
上述$ S_E $的各项$ (X_{ij} - \overline{X}_{\cdot j})^{2} $表示在水平$ A_j $下,样本观测值与样本均值的差异,这是由随机误差所引起的。$ S_E $叫做误差平方和。$ S_A $的各项$ n_j(\overline{X}_{\cdot j} - \overline{X})^2 $表示$ A_j $水平下的样本平均值与数据总平均的差异,这是由水平$ A_j $的效应的差异以及随机误差引起的。$ S_A $叫做因素A的效应平方和。式 \ref{equation:decomposition_of_sum_of_squares}  就是我们所需要的平方和和分解式。

\subsection{$ S_E $,$ S_A $的统计特征}

为了引出检验问题 \ref{equation:updated_hypothesis_of_single_factor_analysis_of_variance} 的检验统计量,我们依次来讨论$ S_E $,$ S_A $的一些统计特征。先将$ S_E $写成
\begin{equation}\label{equation:se}
	S_E = \sum_{i=1}^{n_1}\left( X_{i1} - \overline{X}_{\cdot 1}\right)^2 + \cdots + \sum_{i=1}^{n_s}\left( X_{is} - \overline{X}_{\cdot s}\right)^2
\end{equation}
注意到$ \displaystyle \sum_{i=1}^{n_j}\left( X_{ij} - \overline{X}_{\cdot j}\right) ^2 $是总体$ N\left( \mu_j, \sigma^2\right)  $的样本方差的$ n_j - 1 $倍,于是有
\begin{equation}
	\frac{\displaystyle \sum_{i=1}^{n_j}\left( X_{ij} - \overline{X}_{\cdot j}\right) ^2}{\sigma^2} \sim \chi^2\left( n_j -1\right) 
\end{equation}
因各$ X_{ij} $相互独立,故式 \ref{equation:se} 中各平方和相互独立。由$ \chi^2 $分布的可加性知
\begin{equation}
	\frac{S_E}{\sigma^2} \sim \chi^2\left( \sum_{j=1}^{s}\left( n_j - 1\right) \right) 
\end{equation}
即
\begin{equation}
	\frac{S_E}{\sigma^2} \sim \chi^2(n - s)
\end{equation}
这里$ \displaystyle n = \sum_{j=1}^{s}n_{j} $,由式 还可知,$ S_E $的自由度为$ n-s $,且有
\begin{equation}
	E(S_E) = (n - s)\sigma^2
\end{equation}

下面讨论$ S_A $的统计特性,我们看到$ S_A $是$ s $个变量$ \sqrt{n_j}(\overline{X}_{\cdot j} - \overline{X}) \ (j=1,2,\cdots,s) $的平方和,它们之间仅有一个线性约束条件
\begin{equation}
	\sum_{j=1}^{s}\sqrt{n_j}\left[\sqrt{n_j}(\overline{X}_{\cdot j} - \overline{X})\right] = \sum_{j=1}^{s}n_{j}(\overline{X}_{\cdot j} - \overline{X}) = \sum_{j=1}^{s}\sum_{i=1}^{n_j}X_{ij} - n\overline{X} = 0
\end{equation}
故知,$ S_A $的自由度是$ s-1 $。

再由式 \ref{eq:mu_of_single_factor_analysis_of_variance}、式 \ref{eq:average_of_single_factor_analysis_of_variance} 及$ X_{ij} $的独立性,知
\begin{equation}
	\overline{X} \sim N\left( \mu, \frac{\sigma^2}{n}\right) 
\end{equation}
即得
\begin{equation}
	\begin{aligned}
		E\left( S_A\right) &= E\left[ \sum_{j=1}^{s}n_j\overline{X}_{\cdot j}^{2} - n\overline{X}^2 \right] \\
		&= \sum_{j=1}^{s}n_jE\left( \overline{X}_{\cdot j}^2\right)  - nE\left( \overline{X}^2\right) \\
		&= \sum_{j=1}^{s}n_j\left[ \frac{\sigma^2}{n_j} + \left( \mu + \delta_{j} \right)^2  \right] - n\left( \frac{\sigma^2}{n} + \mu^2 \right) \\
		&= \left( s-1 \right) \sigma^2 + 2\mu\sum_{j=1}^{s}n_{j}\delta_{j} + n\mu^2 + \sum_{j=1}^{s}n_{j}\delta_{j}^{2} - n\mu^2
	\end{aligned}
\end{equation}
由式 \ref{equation:updated_hypothesis_of_single_factor_analysis_of_variance} 知,$ \sum_{j=1}^{s}n_{j}\delta^{j} = 0 $,故有
\begin{equation}
	E\left( S_A \right) = \left( s-1 \right) \sigma^2 + \sum_{j=1}^{s}n_{j}\delta_{j}^{2}
\end{equation}
进一步还可以证明$ S_A $与$ S_E $独立,且当$ H_0 $为真时
\begin{equation}
	\frac{S_A}{\sigma^2} \sim \chi^2\left( s-1 \right) 
\end{equation}

\subsection{假设检验问题的拒绝域}

\begin{equation}
	\cfrac{\quad\cfrac{S_A}{s-1}\quad}{\quad\cfrac{S_E}{n-s}\quad} = 
	\cfrac{\quad \cfrac{\ \cfrac{S_A}{\sigma^2}\ }{\ s-1\ }\quad}{\quad \cfrac{\ \cfrac{S_E}{\sigma^2}\ }{\ n-s\ }\quad} \sim F(s-1, n-s)
\end{equation}

\subsection{未知参数的估计}


 %方差分析及回归分析
%	\input{./chapters/bootstrap方法}
%	\input{./chapters/随机过程及其统计描述}
%	\input{./chapters/马尔可夫链}
\end{document}