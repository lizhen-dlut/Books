\chapter{大数定律及中心极限定理}

极限定理是概率论的基本理论,在理论研究和应用中起着重要的作用,其中最重要的是称为\mydoublequote{大叔定律}与\mydoublequote{中心极限定理}的一些定理。\myimportantpoint{大数定律是叙述随机变量序列的前一些项的算术平均值在某种条件下收敛到这些项的均值的算术平均值};\myimportantpoint{中心极限定理则是确定在什么条件下,大量随机变量之和的分布逼近于正态分布}。

\section{大数定律}

大量试验证实,随机事件$ A $的频率$ f_n(A) $当重复试验的次数$ n $增大时总呈现出稳定性,稳定在某一个常数的附件。频率的稳定性是概率定义的客观基础。

\begin{theorem}[弱大数定理,辛钦大数定理]
	设$ X_1 $,$ X_2 $,$ \cdots $ 是相互独立,服从同一分布的随机变量序列,且具有数学期望$ E(X_k)=\mu \  (k=1,2,\cdots) $。作前$ n $个变量的算术平均$ \displaystyle \frac{1}{n}\sum_{k=1}^{n}X_{k} $,则对于任意$ \epsilon > 0 $,有
	\begin{equation}
		\lim_{n\to\infty}P\left\lbrace \left| \frac{1}{n}\sum_{k=1}^{n}X_{k} - \mu\rvert < \epsilon \right| \right\rbrace  = 1
	\end{equation}
\end{theorem}

\begin{theorem}[伯努利大数定理]
	设$ f_{A} $是$ n $次独立重复试验中事件$ A $发生的次数,$ p $是事件$ A $在每次试验中发生的概率,则对于任意正数$ \epsilon > 0 $,有
	\begin{equation}
		\lim_{n\to\infty}P\left\lbrace \left|\frac{f_A}{n}-p \right| < \epsilon \right\rbrace  = 1
	\end{equation}
	或
	\begin{equation}
		\lim_{n\to\infty}P\left\lbrace \left|\frac{f_A}{n}-p \right|\geqslant\epsilon \right\rbrace  = 0
	\end{equation}
\end{theorem}

\section{中心极限定理}

\begin{theorem}[独立同分布的中心极限定理]
	设随机变量
\end{theorem}

\begin{theorem}[李雅普诺夫\myparenthese{Lyapunov}定理]
	设随机变量
\end{theorem}

\begin{theorem}[棣莫弗---拉普拉斯\myparenthese{De Moivre---Laplace}定理]
	设随机变量
\end{theorem}