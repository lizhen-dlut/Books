\chapter{函数与极限}

初等数学的研究对象基本上是不变的量,而高等数学的研究对象则是变动的量。所谓函数关系就是变量之间的依赖关系,极限方法是研究变量的一种基本方法。本章将介绍映射、函数、极限和函数的连续性等基本概念以及他们的一些性质。

\section{映射与函数}

映射是现代数学中的一个基本概念,而函数是微积分的研究对象,也是映射的一种。本节主要介绍映射、函数及有关概念,函数的性质与运算等。

\subsection{映射}

\subsubsection{映射概念}

\begin{definition}
	设$ X $、$ Y $是两个非空集合,如果存在一个法则$ f $,使得对$ X $中每个元素$ x $,按法则$ f $,在$ Y $中有唯一确定的元素$ y $与之对应,那么称$ f $为从$ X $到$ Y $的映射,记作
	\begin{equation}
	f:X \to Y \notag
	\end{equation}
	其中$ y $称为元素$ x $的像,并记作$ f(x) $,即
	\begin{equation}
	y = f(x) \notag
	\end{equation}
	而元素$ x $称为元素$ y $的一个原像;集合$ X $称为映射$ f $的定义域,记作$ D_f $,即$ D_f = X $;$ X $中所有元素的像所组成的集合称为映射$ f $的值域,记作$ R_f $或$ f(x) $,即
	\begin{equation}
	R_f= f(X) = \left\lbrace f(x)|x \in X \right\rbrace \notag
	\end{equation}
\end{definition}

从上述映射的定义中,需要注意的是:
\begin{enumerate}[(1)]
	\item 构成一个映射必须具备以下三个要素:
		\begin{inparaenum}[(i)] 
			\item 集合$ X $,即定义域$ D_f = X $;
			\item 集合$ Y $,即值域的范围:$ R_f \in Y $;
			\item 对应法则$ f $,使对每个$ x \in X $,有唯一确定的$ y = f(x) $与之对应
		\end{inparaenum}。
	\item 对每个$ x \in X $,元素x的像y是唯一的;而对每个$ y \in Y $,元素$ y $的原像不一定是唯一的;映射f的值域$ R_f $是$ Y $的一个子集,即$ R_f \subset Y $,不一样$ R_f = Y $。
\end{enumerate}

\subsection{函数}

\subsubsection{函数的概念}

\begin{definition}
	设数集$ D \subset R $, 则称映射$ f:D \to R $ 为定义在$ D $上的函数,通常简记为
	\begin{equation}
		y =f(x), x \in D \notag
	\end{equation}
	其中$ x $称为自变量,$ y $称为因变量,$ D $称为定义域,记作$ D_f $,即$ D_f =D $。
\end{definition}

函数的定义中,对每个$ x \in D $,按对应法则$ f $,总有唯一确定的值$ y $与之对应,这个值称为函数$ f $在$ x $处的函数值,记作$ f(x) $,即$ y=f(x) $。因变量$ y $与自变量$ x $之间的这种依赖关系,通常称为函数关系。函数值$ f(x) $的全体所构成的集合称为函数$ f $的值域,记作$ R_f $或$ f(D) $, 即
\begin{equation}
	R_f = f(D) = \left\lbrace y \mid y = f(x)\text{,} x \in D \right\rbrace \notag
\end{equation}

\subsubsection{函数的几种特性}

\paragraph{函数的有界性}

\paragraph{函数的单调性}

\paragraph{函数的奇偶性}

\paragraph{函数的周期性}

\begin{definition}
	设函数$ f(x) $的定义域为$ D $,如果存在一个正数$ l $,使得对于任意$ x \in D $有$ (x\pm l) \in D $,且
	\begin{equation}
		f(x+l)=f(x)
	\end{equation}
	恒成立,那么$ f(x) $为周期函数,$ l $称为$ f(x) $的周期,通常我们说周期函数的周期是指最小正周期。
\end{definition}

\begin{definition}
	狄利克雷(Dirichlet)函数
	\begin{equation}\label{Dirichlet}
		D(x) = \left\lbrace 
			\begin{array}{rl}
			1 \text{,} & x \in Q \text{,}\\
			0 \text{,} & x \in Q^C \text{。}
			\end{array}
		\right. 
	\end{equation}
\end{definition}

很容易验证\hyperref[Dirichlet]{狄利克雷函数}(\ref{Dirichlet})是一个周期函数,任何正有理数$ r $都是它的周期,因为不存在最小的正有理数,所以他没有最小正周期。

\subsubsection{反函数与复合函数}

\subsubsection{函数的运算}

\subsubsection{初等函数}

在初等数学中已经将过下面几类函数:

幂函数:$ y = x^\mu \text{,} (\mu \in \text{R是常数}) $,

指数函数:$ y = a^x \text{,} (a>0 \text{且} a \ne 1) $,

对数函数:$ y = \log_a x \text{,} (a>0 \text{且} a \ne 1\text{,特别当} a = e \text{时,记为} y = \ln x ) $,

三角函数:如$ y = \sin x $、$ y = \cos x $、$ y = \tan x $等,

反三角函数:如$ y = \arcsin x $、$ y = \arccos x $、$ y = \arctan x $等。

以上这五类函数统称基本初等函数。

由常数和基本初等函数经过有限次的四则运算或有限次的函数复合步骤所构成的、并可用一个式子表示的函数,称为初等函数。

应用上常遇到以$ e $为底数的指数函数$ y=e^x $和$ y=e^{-x} $所产生的双曲函数以及它们的反函数---反双曲函数,它们的定义如下:

双曲正弦:$ \sinh x = \dfrac{e^x - e^{-x}}{2} $

双曲余弦:$ \cosh x = \dfrac{e^x + e^{-x}}{2} $

双曲正切:$ \tanh x = \dfrac{e^x - e^{-x}}{e^x + e^{-x}} $

\section{数列的极限}

一般地,有如下数列极限的定义:
\begin{definition}
	设$ \left\lbrace x_n \right\rbrace  $为一数列,如果存在常数a,对于任意给定的正数$ \varepsilon $(不论它多么小),总存在正整数N,使得当n>N时,不等式
	\begin{equation}
		\lvert x_n - a \rvert < \varepsilon
	\end{equation}
	都成立,那么就称常数a时数列$ \left\lbrace x_n \right\rbrace  $的极限,或者称数列$ \left\lbrace x_n \right\rbrace  $收敛于a,记为
	\begin{equation}
		\lim_{n \to \infty} x_n = a
	\end{equation}
	或
	\begin{equation}
		x_n \to a \  (n \to \infty)
	\end{equation}
\end{definition}


\subsection{数列极限的定义}

待补充

\subsection{收敛数列的性质}

待补充

\section{函数的极限}
待补充

\subsection{函数极限的定义}
待补充

\subsubsection{自变量趋于有限值时函数的极限}
待补充

\subsubsection{自变量趋于无穷大时函数的极限}
待补充

\subsection{函数极限的性质}
待补充

\section{无穷大与无穷小}
待补充

\subsection{无穷小}
待补充

\subsection{无穷大}
待补充

\section{极限运算法则}
待补充

\section{极限存在准则 两个重要极限}
待补充

\section{无穷小的比较}
待补充

\section{函数的连续性与间断点}
待补充

\subsection{函数的连续性}
待补充

\subsection{函数的间断点}

待补充

\section{连续函数的运算与初等函数的连续性}
待补充
\subsection{连续函数的和、差、积、商的连续性}
待补充
\subsection{反函数与复合函数的连续性}
待补充
\subsection{初等函数的连续性}
待补充
\section{闭区间上函数的连续性}
待补充
\subsection{有界性与最大值最小值定理}
待补充
\subsection{零点定理与介值定理}
待补充
\subsection{一致连续性}
待补充