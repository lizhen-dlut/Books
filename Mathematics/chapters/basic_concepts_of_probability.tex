\chapter{概率论的基本概念}

自然界和社会上发生的现象是多种多样的。有一类现象,在一定条件下必然发生,例如,向上抛一石子必然下落,同性电荷必相互排斥,等等。这类现象称为\myconcepts{确定性现象}。在自然界和社会上存在着另一类现象,例如,在相同条件下抛同一枚硬币,其结果可能是正面朝上,也可能是反面朝上,并且在每次抛掷之前无法肯定抛掷的结果是什么;这类现象,在一定的条件下,可能出现这样的结果,也可能出现那样的结果,而在试验或观察之前不能预知确切的结果。但人们经过长期实践并深入研究之后,发现这类现象在大量重复试验或观察下,它的结果却呈现出某种规律性。这种在大量试验或观察中所呈现出的固有规律性,就是我们以后所说的统计规律性。

这种在个别试验中其结果呈现出不确定性,在大量重复实验中其结果又具有统计规律性的现象,我们称之为随机现象。概率论与数理统计是研究和揭示随机现象统计规律性的一门数学学科。

\section{随机试验}

我们遇到过各种试验。在这里,我们把试验作为一个含义广泛的术语。它包括各种各样的科学实验,甚至对某一事物的某一特征的观察也认为是一种试验。下面举一些试验的例子:
\begin{itemize}
	\item[$ E_1 $] :抛一枚硬币,观察正面$ H $、反面$ T $出现的情况;
	\item[$ E_2 $] :将一枚硬币抛掷三次,观察正面H、反面T出现的情况;
	\item[$ E_3 $] :将一枚硬币抛掷三次,观察出现正面的次数;
	\item[$ E_4 $] :抛一颗骰子,观察出现的点数;
	\item[$ E_5 $] :记录某城市120急救电话台一昼夜接到的呼唤次数;
	\item[$ E_6 $] :在一批灯泡中任意抽取一只,测试它的寿命;
	\item[$ E_7 $] :记录某地一昼夜的最高温度和最低温度。
\end{itemize}

上面举出了七个试验的例子,它们有着共同的特点。例如,试验
